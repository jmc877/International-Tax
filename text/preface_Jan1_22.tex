\chapter{Preface}

Welcome to \textit{International Taxation: A Transactional Approach}.  This book introduces you to the contours of the field of U.S international taxation.  Before delving into the material, I'd like to briefly highlight some of the characteristics that distinguish this book from other international tax casebooks.  A quick perusal of the table of contents of each would reveal that the topic coverage is virtually identical:  residence, source of income, taxation of foreign persons investing or doing business in the United States, the taxation of U.S. persons investing or doing business abroad, related party transactions, and perhaps, foreign currency issues and international mergers and acquisitions.  So, why should you use this one?  

First, since it's free, you can use it together with any other casebook  or materials assigned by your professor.  Second, I've tried to alter the presentation to make learning and retaining the material a bit easier.  Given the complexity of the topic for neophytes, I have found that even highly motivated students can often miss the forest for the trees.  For example, a student may be able to recite the holding of a case or conclusion of a revenue ruling on the application of the portfolio debt rules but not be able to tell you how the United States generally taxes foreigners on returns on debt capital.  

Each topic begins with an overview that sketches out, sometimes in considerable detail, the contours of the subject matter.  The overview will usually walk you through the relevant code sections, perhaps highlight some regulatory guidance, and applicable tax treaty provisions.  A complaint I have had with many casebooks was understanding the larger relevance of a holding of a particular case.  The more in-depth material is addressed in the materials that follow the overview, such as cases, administrative guidance, legislative history, comments, and problems.  

This books incorporates income tax treaties as an integral part of the U.S. international tax regime.  As the United States has entered into tax treaties with almost all of its important  trading and investment partners, many of the U.S. rules for taxing foreigners are found in tax treaties rather than in the bowels of the Internal Revenue Code.  

Without fail, the positive portions of student evaluations have generally lauded the benefit of applying the materials to problems and often have suggested covering even more problems.  As most students in the class are third-years with some legal work experience, the appeal of problems probably reflects the correct view that transactional attorneys are hired to help clients solve their pressing current business problems and avoid future ones.  

Finally, there are two other novelties that readers may find useful.  I incorporate some introductory accounting treatment of transactions.  Especially for publicly traded entities, a tax advisor must be aware of the accounting treatment of a proposed transaction.  Creative tax saving ideas without a concomitant accounting benefit often fall by the wayside.  In addition,  \textit{International Taxation: A Transactional Approach} also introduces students to the foreign law treatment of certain items.  Although the book's primary focus is the U.S taxation of international income, a good tax planner must take into account all relevant effects, including foreign law.  Many structures and transactions involving U.S. based multinationals cannot be understood without an awareness of foreign law concerns.

 Over the last twenty years, many U.S.-based multinationals, especially technology companies structured their foreign operations so that they pay little or no foreign or U.S. tax on their current profits, giving rise to so-called \emph{stateless income}.  Many of these U.S. multinationals accumulated abroad vast sums of untaxed capital: Apple alone reported having \$200 billion of overseas cash in 2016. Since bringing back those untaxed earnings to the United States could have resulted in a U.S. tax of 35\% to 40\%, it was advantageous from a tax, finance, and accounting perspective to leave those earning abroad, even if the capital could be more profitably employed in the United States.  The U.S. multinationals though complained the relatively high U.S. corporate tax rate of 35\% placed them at a competitive disadvantage to multinationals based in foreign countries with lower tax rates.  During the early 2010's Congress, tax commentators, and the popular press focused much attention on these structures and the massive loss of U.S. tax revenue.  

%Another pathology of the U.S. international tax regime is that it can be more tax efficient for a U.S. multinational company to \textit{invert} its corporate structure so that the U.S. parent company becomes a subsidiary of a new foreign parent corporation, while the owners of the company remain largely the same.  These inversions can also occur when a foreign company merges with a U.S. company.  Inversions potentially remove a significant amount of future taxes from the U.S. tax base.  

In response to the continuing rise of stateless income and accumulation of untaxed profits overseas, Congress enacted  in 2017 the most sweeping changes in more than 30 years to the U.S. international tax regime in the Tax Cut and Jobs Act.  In the TCJA, Congress: enacted a territorial system (participation exemption) under which certain dividends from foreign corporations are exempt from U.S. tax; subjected to current taxation, albeit at a reduced rate, a U.S. shareholder's portion of a foreign subsidiary's global intangible low-taxed income (GILTI); taxed U.S. shareholders on the accumulated foreign earnings of their foreign subsidiaries; imposed a 10\% tax on certain deductible base erosion payments (BEAT) to related foreign persons; and provided an export subsidiary in the form of a reduced U.S. corporate tax rate on the foreign-derived intangible income of U.S. corporations (FDII).  Importantly, the TCJA reduced the U.S. corporate tax rate from 35\% to 21\%, one of the lowest rates among our trading partners.  

Surprisingly the TCJA let inact many of the e four years since the enactment of the TCJA international provisions, Treasury has issued massive and complicated regulations to provide guidance to overwhelmed taxpayers.      

Parallel with the U.S. response to the rise of \emph{stateless income}, the OECD, in its \emph{Base Erosion and Profit Shifting (BEPS) action plan}, has begun to modernize many of the long-standing international tax norms that have guided international capital flows over the last 80 years.  These initiatives have begun to be enacted into law by OECD signatories, including the United States.

It's virtually certain the current U.S. international tax regime will continued to be revised in 2022, although the changes may be more incremental than those in the TCJA.  Furthermore, the issuance of Treasury regulations show no sign of diminishing.  For students, this is a wonderful opportunity:  you will be acquiring your knowledge of the new U.S. international rules at the same time as your future bosses and will therefore know as much as they do.   

As this is a work in progress, I'd appreciate any suggestions on how to improve the book and accompanying materials.  They can be sent to the author at:  jcolon@fordham.edu.    


\begin{framed}
Last modified:  Jan 1, '22; \emph{preface\_Jan1\_22.tex}
\end{framed}