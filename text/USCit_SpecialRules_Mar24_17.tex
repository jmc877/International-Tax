\chapter{U.S. Persons Residing Abroad and Expatriates}

U.S. citizens, resident aliens, and corporations are subject to residence basis taxation.  If U.S. persons are also subject to foreign tax (after any applicable treaty), relief from double taxation is found through the foreign tax credit mechanism (sections 901--904).

Some special rules apply to U.S. individuals.  Under section 911, U.S. citizens and resident aliens residing abroad can elect to exclude from U.S. tax up to \$102,100 for 2017 of foreign earned income (and a portion of foreign housing expense).  If a U.S citizen (or long-term resident alien) renounces his citizenship (abandons his U.S. residence), he is subject to mark-to-market taxation on all of his assets and a modified U.S. gift and estate regime.  Section 877A.  Both of these special regimes are discussed below.

\section{Section 911}
	\irc{911(a), (b), (c) (skim), (d)(1)--(6)}
	

Under section 911(a), U.S. citizens and resident aliens who have established a bona fide foreign residence can elect to exclude up to \$102,100 (2017 amount) and certain housing cost reimbursements from U.S. tax.   The purported justification for the provision is that firms generally increase compensation for foreign assignments, but that this increased compensation does not fully offset the additional expenses of living abroad.
 
Section 911 applies to a \emph{qualified individual}, which is defined to be an individual with a foreign tax home, and who is: (1) a U.S. citizen and bona fide resident of a foreign country (or countries) for an uninterrupted period which includes the entire taxable year; or (2) a U.S. citizen or resident alien and present in a foreign country (or countries) during at least 330 full days of 12 consecutive months. \S911(d)(1)(A) and (B).  

Whether a person is a resident of a foreign country is determined under the superseded residence rules in Reg.\@ \S1.871-2 (and cases thereunder) for determining the U.S. residence of aliens.  Reg.\@ \S1.911-2(c).  An individual is not a bona fide resident of a country if he claims to be a nonresident in written statements to the authorities of that country and his earned income is not subject to foreign tax because of nonresidency.  \S911(d)(5); Reg.\@ \S1.911-2(c)(1) and (2).  \emph{Tax home} under section 911 has the same meaning as under section 162(a)(2) (relating to being able to deduct traveling expenses while away from home), and the regulations clarify that a person's tax home is ``considered to be located at his regular or principal place of business.''  Regs. \S1.911-2(b).  
 
The election under section 911 applies only to foreign \emph{earned income} and the \emph{housing cost amount}, and thus does not cover foreign interest or dividends.  Earned income is income from services, such as salaries and professional fees, but does not include deferred compensation such as pensions and social security benefits.  \S911(b)(1)(A).  Although gains from the sale of property are not earned income, gains arising from the sale of property produced or created by the taxpayer, such as a sculpture, can qualify as earned income.  \emph{See Cook v. US}, 599 F.2d 400 (Ct. Cl. 1979).  In addition, royalties or gains from the transfer or sale of property rights of a writer in his creations qualify as earned income.  \emph{See} Rev.\@ Rul.\@ 80-254, 1980-2 C.B. 222.

 If a taxpayer participates in a business in which both services and capital are ``material income producing factors,'' the regulations permit the taxpayer to treat a ``reasonable allowance'' as personal service compensation, but in no case can the amount treated as earned income exceed 30\% of the individual's share of the net profits.  Regs.\@ \S1.911-3(b)(2).  Whether a business is one in which service and capital are material income producing factors is a facts and circumstances determination, but it has been held that ``capital is ordinarily a material income-producing factor if the operation of the business requires substantial inventories or substantial investments in plant, machinery, or other equipment.''  \emph{Rousku v. CIR}, 56 T.C. 548 at 551 (1971).
 
 Congress recognized that foreign moves may entail duplicative housing costs and that employers must often provide housing assistance for expatriates and consequently qualified individuals can exclude the \emph{housing cost amount}, which is the amount of housing expenses (as limited below) in excess of a base amount.  Thus, if housing expenses do not exceed the base amount, no amount can be excluded.  Housing expenses include rent (and the value of employer provided housing), utilities, and repairs, but not the cost of domestic labor, interest, or taxes.  \S911(c)(3); Reg.\@ \S1.911-4(b).  These expenses are limited to 30\% of the maximum exclusion, which is \$30,630 for 2017, but the IRS can adjust this amount for geographic differences.  \S911(c)(2)(B).  For example, the annual housing expenses for Vienna and Doha, which are getting closer to each other every day, are \$35,400 and \$42,744 respectively, and for Paris, \$66,500.  The base amount is 16\% of the maximum exclusion, or \$16,336 for 2017.  Self-employed persons can deduct the housing cost amount, subject to the limitation of the excess of the person's foreign earned income over the amount excluded under section 911(a).  \S911(d)(4).
 
 If a person elects the benefits of section 911, no deduction or credit is allowed to the extent that it is allocable to excluded amounts.  \S911(d)(6).  This rule can reduce or eliminate a credit for foreign taxes paid by a person electing section 911.  The amount of taxes that are not creditable is determined by multiplying foreign taxes times a fraction, the numerator being the section 911 exclusion for the year (reduced by other deductions denied by section 911(d)(6)), and the denominator being foreign earned income plus any income in the foreign tax base that is not foreign earned income.  For example, assume a person earns \$183,000 and pays foreign taxes of \$100,000.  Under section 911(d)(6), \$54,207 of the foreign taxes, 100k*(99.2/183k), are not creditable.
 
\section{Expatriates}
	\irc{877A}
	
	
 A U.S. citizen (or resident alien), wherever resident, is generally subject to tax on his or her worldwide income, except if he is eligible for and elects the benefits of section 911.  A U.S. citizen (and resident alien) is also subject to U.S. gift tax on gratuitous transfers of property wherever located (\S2501), and U.S. estate tax on the value of his or her worldwide estate at death (\S2001).  The United States also imposes a generation skipping transfer (GST) tax on certain transfers to persons two or more generations below the transferor.  \S2611.  For 2017, the first \$5.49 million of a person's taxable estate is exempt from estate tax, and the first \$5.49 million of lifetime gifts is exempt from gift tax.   Transfers exceeding these amounts are taxed at 40\%.     

A nonresident alien is subject to U.S. income tax only on income arising in the United States, and is subject to U.S. gift tax only on real or tangible property situated in the United States at the time of transfer (\S25.2511-1(b)) and U.S. estate tax only on property situated in the United States at the time of death (\S2103).  The first \$60,000 of a nonresident's estate is exempt from tax, but there is no exemption amount for gifts of U.S. property.   

%(Note, at the time of writing (March 20, 2010), the U.S. estate and GST are no longer in effect.  The gift tax is scheduled to remain in effect but with a top rate of 35\%.  Property received from a decedent who dies in 2010 has a basis equal to the lower of cost or fair market value.  For 2011, the \emph{ancien} gift and estate regime is scheduled to be resurrected.  It is believed that the U.S. estate and gift tax regime will be reinstated retroactively some time in 2010.)    
% 
 Given the difference in income, gift, and estate tax bases for U.S. and foreign persons, for a U.S. citizen or resident alien owning significant liquid or foreign assets, expatriation (or abandoning U.S. residence) can potentially significantly lower future U.S. income and transfer tax obligations.  
 
 Prior to June 13, 2008, expatriates were subject to a modified U.S. income and transfer tax regime for 10 years following expatriation.  In general, an expatriate was taxed like a nonresident but the tax base was expanded:  certain income items were treated as originating in the United States and certain transfers of property were treated as transfers of U.S. situs property. In the Heroes Earnings Assistance and Relief Tax Act of 2008, Congress significantly revised the U.S. expatriation tax regime.  A person expatriating after June 17, 2008, is treated as selling his or her assets for their fair market value and is taxed on the net gain. In addition, bequests and gifts to a U.S. citizen or resident by an expatriate are subject to U.S. estate and gift tax in the hands of the recipient. 
 
 A U.S. citizen who renounces his or her U.S. citizenship and is a ``covered expatriate'' is treated as selling all of his or her property for its fair market value on the day before expatriation. \S877A(a)(1).  (These provisions also apply to any person who has held a green card for 8 of the last 15 years ending with the year of expatriation.)   An expatriate's net gain (the excess of gains over losses that can be taken into account) in excess of \$699,000 (for 2017) is taxable.  \S877A(a)(3)(A).  
 
 A person is a covered expatriate if: (1) the expatriate's average annual net income tax for the 5 taxable years ending before the date of expatriation was greater than \$162,000; or (2) the expatriate's net worth is \$2 million or more.   Even if these thresholds are exceeded, an expatriate will not be a covered expatriate if: (1) the expatriate was dual citizen at birth, and at the time of expatriation continues to be a citizen of the other country and is taxed as a resident of the other country; and (2) has been a U.S. resident for not more than 10 of the previous 15 taxable years ending with the taxable year of expatriation.  877A(g)(1)(B)(i).  Residence for these purposes is determined under the substantial presence test.
 
 A covered expatriate can elect on a property-by-property basis to defer payment of the mark-to-market tax.   This election requires the expatriate to post a bond for the amount of tax due plus interest (currently about 6\% per annum).  The deferral period generally ends when the property is sold. \S877A(b).  Special rules apply to interest of an expatriate in deferred compensation items and trusts (both grantor and nongrantor).  \emph{See} \S877A(d) and (f).
 
 Perhaps the most important changes in the expatriate regime were the changes to the estate and gift tax provisions:  any U.S. person who receives directly or indirectly any gift or bequest from a covered expatriate is subject to U.S. gift and estate tax on the value of the gift or bequest received. \S2801.  The rate applicable to such transfers is the highest gift or estate tax rate in effect on the date of transfer.  The IRS issued in 2015 detailed guidance on the expatriate provisions in proposed regulations under section 2801.  
 
 \begin{framed}
 Last modified: Mar. 24, '17; USCit\_SpecialRules\_Mar24\_17.tex
 \end{framed}
 