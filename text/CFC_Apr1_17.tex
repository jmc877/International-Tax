\chapter{Controlled Foreign Corporations:  CFCs}

	\section{Introduction}
	
A fundamental U.S. tax principle is the separate tax treatment of corporations and their owners, even corporations that are wholly owned by another corporation or a single individual.  As we have seen in earlier chapters, foreign persons (including foreign corporations) are not subject to U.S. income tax unless the income is U.S. source FDAP or ECI.  The combination of these two principles could render U.S. income tax on the worldwide income of U.S. persons largely voluntary, unless the rule treating corporations as separate from their U.S. owners were modified.
	
Until 1937, U.S. taxpayers could take advantage of the separate tax status of a corporation and its owners by forming a corporation in a low-tax jurisdiction and transferring to it income producing assets (business or passive) or assets with a significant amount of unrealized gain.  Three possible benefits could accrue to the corporation's U.S. owners.  First, since the income earned by the corporation would not be taxed by the United States (except to the extent it was U.S. source FDAP) until it was paid out as a dividend, U.S. tax on current income could be deferred.  Second, if the assets produced ordinary income, the income could be left in the corporation, and the stock of the foreign corporation could then be sold, with result that ordinary income could be converted to capital gain.  Third, if the stock of the corporation were bequeathed to a U.S. person, the stock's basis would be its fair market value on the date of death.  The U.S. heirs could then sell the stock for no gain and thereby eliminate any U.S. tax on the income earned and accumulated by the corporation or the built-in gains in the corporation's assets.  

In 1937, Congress enacted the foreign personal holding company (FPHC) provisions (former sections 551 through 558).\footnote{The FPHC provisions were repealed in 2004.  Congress believed that the CFC and PFIC regimes (discussed in Chapter 12) were sufficient to protect the fisc from abusive uses of foreign corporations.}  A foreign corporation was FPHC if: (1) 50\% of total combined voting power of all classes of stock or the total value of all stock was owned directly or indirectly at any time during the year by five or fewer U.S. individuals; and (2) at least 60\% of its gross income consisted of foreign personal holding company income (FPHCI).  
	
FPHCI included all passive income, such as dividends, interest, gains from stock and securities transactions, and commodities transactions, as well as certain rents, income arising from personal service contracts, and income from the use of corporate property by a shareholder.  The latter categories were intended to prevent the deflection of income to offshore entities controlled by U.S. persons.    

Once a corporation was a FPHC, its U.S. shareholders were taxed on their share of the foreign corporation's FPHCI (with some adjustments), whether or not it was actually distributed to the shareholders.  Upon including these amounts in income, a U.S. shareholder would increase its basis in the stock of the foreign corporation, and when dividends were actually paid, the shareholder would decrease its basis in the stock.  If the foreign corporation was FPHC in the year preceding the decedent's death, the basis of the shares in the hands of the decedent would be the lower of their fair market value or the adjusted basis.  Section 1014(b)(5).

After 1937 until 1962, the FPHC rules were the only statutory impediment to U.S. persons using a foreign corporation to avoid current taxation on business or passive income.  The scope of the FPHC regime, however, was quite limited.  First, because it only applied to foreign corporations owned by five or fewer \emph{individuals}, widely held U.S. corporations were never subject to the FPHC net.  Second, even for those corporations that satisfied the narrow ownership test, the FPHC's reach could be avoided if enough active business income could be earned so as to reduce the percentage of FPHCI below the statutory limit.  If the owners of a foreign corporation were U.S. persons, when they died the basis of their shares would be stepped up to fair market value, and thus deferral of the foreign corporation's income could eventually result in permanent exemption.

The IRS employed a couple of relatively ineffective weapons against such arrangements.  First, section 482, which mandates that related parties deal with each other on an arm's-length basis could be employed when the transactions involved U.S. entities, for example, transferring intellectual property to a related foreign entity at a bargain price or rate.  Section 482 was limited, however, as it did not apply to foreign-to-foreign transactions.  The IRS could also employ conduit arguments, but as seen above in \emph{Northern Indiana}, the IRS's success in this area was limited.  

This period also saw the rise of \emph{foreign base companies}, which were related foreign corporations incorporated in a low-tax jurisdiction that engaged in transactions with other related corporations in order to accumulate profits in the entity in the low-tax jurisdiction.  For example, a Cayman Islands subsidiary could purchase goods from a U.K. subsidiary and then sell them to a Spanish subsidiary.  With careful attention to the pricing of the purchases and sales, much of the economic profit from the manufacturing and distribution could remain in the Cayman Islands.  Similarly, the base company could be hired to perform services on behalf of other related corporations with the same result of accumulating profits in the tax haven.

Why should the United States care if foreign taxing authorities were getting the short end of stick by these arrangements?  By using foreign base companies, U.S. multinationals have an advantage over purely domestic corporations in that their foreign economic income is not subject to U.S. tax until it is repatriated.  Furthermore, even arm's-length transactions can result in accumulation of passive income, \emph{etc.}, in offshore entities, which results not only in deferral of U.S. tax liabilities but also in the uneconomic movement and accumulation of capital oversees.

In 1961, President Kennedy proposed doing away with deferral for all foreign corporations controlled by U.S. persons, arguing that the current rules were biased against U.S. investment.  U.S. businesses argued that the elimination of deferral would put them at a disadvantage vis-a-vis their foreign counterparts, and the Kennedy bill was scaled back in House and Senate.  The final legislation adopted in 1962, which is codified mostly in subpart F of the IRC (sections 951--965), taxes certain U.S. shareholders of a CFC currently on their share of the CFC's subpart F income--mostly passive type income, including certain types of income from foreign base company transactions--but retains deferral for active business income.  In addition, the gambit of converting ordinary income into capital gain was curtailed by requiring that gains realized upon the sale or liquidation of a CFC be treated as dividend income to the extent of untaxed E\&Ps of the CFC.  \S 1248.  In addition, certain investments in U.S. property made by a CFC, for example, a loan to the CFC's U.S. parent or purchase of the U.S. parent's stock, are treated as distributions by the CFC and taxable under section 956. 

As under the former FPHC rules, a  U.S. shareholder increases his basis in his CFC stock for subpart F inclusions (and section 956 inclusions).   Distributions traced to prior subpart F (and section 956) inclusions are treated as tax-free returns of capital and result in a reduction in a shareholder's basis in his CFC shares.  \emph{See} \S\S 961(a) and (b) and 959.  In addition, if a U.S. shareholder is a U.S. corporation, the subpart F or section 956 inclusion may also bring with it deemed paid foreign tax credits under section 902.  \S60(a)(1).

We can now appreciate why the CFC rules are so complicated.  First, the rules must distinguish between business income and the various categories of subpart F income and investment in U.S. property and the associated expenses of each.  In addition, since only shareholders owning 10\% or more of the voting stock a CFC are required to include currently in income subpart F income, it is necessary to trace and assign the earnings and exclusions to the correct shareholder.  Finally, detailed rules are needed to coordinate \emph{actual} and \emph{deemed} distributions and to coordinate actual and deemed distributions with the foreign tax credit limitation rules of section 904.
  
	
	\section {Definition of CFC and U.S. Shareholder}
		\irc{951(a)(1) and (2); 951(b); 952(a) and (c)(1)(A); 954(a) and (b) (skim); 957(a); 958(a) and (b)}
	
		If a foreign corporation is a \emph{controlled foreign corporation} (CFC) for 30 or more consecutive days during any taxable year, each \emph{U.S. shareholder (USSH)} who owns stock in the corporation on the last day that it is a CFC is required to include in income his \emph{pro rata share} of the CFC's \emph{subpart F} income.  \S951(a)(1)(A)(i).  
		
		A USSH is a U.S. person who owns (pursuant to the attribution rules of sections 958(a) and (b)) 10\% or more of the total combined voting power of all classes of the foreign corporation's  voting stock.  Thus, for example, if a USSH sells his stock to another U.S. person, only the U.S. person holding the shares at year end (or the last day of the year the corporation is CFC) is subject to the subpart F provisions.  
		
		A CFC is any foreign corporation if USSHs own \emph{more than} 50\% of: (1) the total combined voting power; \emph{or} (2) the total value of the stock of the corporation.  A U.S. person is deemed to own stock of a foreign corporation if he owns it directly, indirectly, or constructively.  \S\S951(b); 957(a); 958(a)(1) and 958(b).  
		
		
		\addcontentsline{toc}{section}{\protect\numberline{}Definition of CFC Problems} 
	\begin{center}
		\textbf{Definition of CFC Problems}
	\end{center}
	\begin{select}
	
			\begin{enumerate}
	
	\item FC, a foreign corporation is owned 50-50 by IBM and Sony (Japan).  Is it a CFC?  Is IBM a USSH?
		\begin{enumerate}
			\item What if FC is owned 50-50 by IBM (Japan), a wholly owned Japanese corporation and Sony (Japan)?
			\item What if FC is owned 50-50 by IBM (US) and Sony (US), a wholly owned U.S. subsidiary of Sony (Japan)?
		\end{enumerate}
	\item Eleven U.S. individuals each own 9.09\% of FC, a foreign corporation.  Is FC a CFC?  Are any of the individuals USSHs?
	\item FC is owned by a group of 2 U.S. and 10 foreign persons.  The U.S. persons own non-voting class B shares that represent 80\% of the value of the corporation.  Is FC a CFC?  Are the U.S. persons U.S. shareholders?
	\item The stock of FC is held by the following (unrelated) U.S. persons as follows:  A owns 50\%;  B, C, D, E, F, and G each own 8.33\%
	\item FC, a foreign corporation is owned 50-50 by IBM and Sony.  In addition, IBM owns 1 share of Sony. Is it a CFC? [\S 958(a)(1) and (2)]
	\item Why would a U.S. shareholder want to create CFC status?  [\S 904(d)(3)]
	\item Assume the stock of FC is held equally among Joan, a U.S. person, and her 10 U.S. family members--lineal descendants and ancestors--9.09\% each).  Is FC a CFC? [951(b); 957(a); 958(b); 318(a)(1)(A)]
	\item Assume that 100\% of the stock of FC is held by Joan, a U.S. person.  Is Ana, her daughter, a USSH?  [\S951(b)]  Does Ana have any subpart F inclusion for the year?
	\item Assume that 50\% of the stock of FC is held by Joan, a U.S. person.  Unrelated U.S. persons, Maria and her son, Jesse, own 9\% and 5\% respectively.  Is FC is CFC?  Is Joan, Maria, or Jesse a USSH?  
	\item Same facts as previous question and FC has 100 of subpart F income for 2008 (assume that it has no other income).  How much does each shareholder include in income under section 951(a)(2)? 
	\item FC is a foreign corporation 100\% owned by IBM and earns \$5 million of subpart F income for 2008.  How much does IBM include in income under section 951(a)(1)(A)?
	\item   IBM owns 100\% of the stock FC from Jan 1 until May 26 when it sells 60\% to Virgin.  FC earns \$100 of subpart F income for the entire year.  What's IBM's subpart F inclusion? [\S 951(a)(2)]
	\item IBM owns 40\% of the stock FC which earns \$100 of subpart F income for the entire year.  IBM buys the remaining 60\% from Virgin on May 26.  What's IBM's subpart F inclusion? [\S 951(a)(2)]  
	\item FC is owned 60\% by IBM and 40\% by Oracle.  IBM buys out Oracle on May 26.  What are IBM and Oracle's subpart F inclusions?  
	\item 	FC is owned 100\% by U.K. corporation.  IBM buys 100\% of FC on June 30.  If FC's subpart F income is \$100, what's IBM's inclusion?  What if FC distributed 25 before the sale?  What if FC distributed 75 before the sale?
	
	\end{enumerate}
		
	\end{select}
		
	\section{Overview of 951 Inclusions and Special Rules}
		\irc{951(a)(1)(A) and (B); 952(a) and (c)(1)(A); 954(a) and (b); and 961(a)}
		
Once a corporation is a CFC for an uninterrupted 30-day period, each U.S. shareholder must include in income: (1) his pro rata share of the CFC's \emph{subpart F income}, and (2) the section 956 amount, which relates to the investment of earnings by the CFC in U.S. property.  \S951(a)(1)(A) and (B).  The most important categories of subpart F income are insurance income (defined in section 953) and foreign base company income (determined under section 954).  \S 952(a)(1) and (3).  The amount of subpart F income for any given year is limited, however, to the CFC's current E\&Ps.  \S952(c)(1)(A).  We will focus solely on foreign base company income (FBCI).
		
FBCI consists of foreign personal holding company income (FPHCI), foreign base company sales income (FBCSaI), foreign base company services income (FBCSerI), and foreign base company oil related income (FBCORI).  \S 954(a).  We will focus only on the first three categories.  A couple of special rules apply to CFCs earning  FBCI.  If a CFC's FBCI and gross insurance income are less than the lesser of 5\% of gross income or \$1 million, then none of the CFC's gross income is FBCI or insurance income.  \S 954(b)(3).  If, however, the FBCI and insurance income exceed 70\% of the CFC's gross income, then all of the gross income will be FBCI or insurance income.  Also, FBCI and insurance income do not include income that was subject to a foreign tax rate greater than 90\% of the maximum in section 11 (31.5\% in 2014).  \S 954(b)(4).  The rules for computing FBCI are set out in Reg. \S 1.954-1.    
		  
When a USSH has an inclusion of income under section 951(a)--subpart F income or investment in U.S. property--his basis in his CFC shares is increased by the amount of the inclusion.  \S 961(a).  The purpose of this rule is to prevent double taxation when the USSH sells his shares.  For example, assume that a USSH forms a CFC and contributes \$100 to the CFC, which earns \$10 of subpart F.  If the CFC's basis were not increased by the \$10 subpart F inclusion, when the USSH sold the CFC shares, he would report a gain of \$10 (\$110 amount realized (the value of corporation) minus his \$100 basis).  He would thus be taxed twice on the CFC's \$10 of earnings.  If the USSH owns the stock of the CFC indirectly through a foreign entity, the USSH adjusts his basis of his ownership interest in the foreign entity.  \S 961(a).  If the intermediate foreign entity is also a CFC, section 961(c) authorizes the Treasury to promulgate regulations providing for appropriate basis adjustments in both the directly and indirectly held CFCs.  To date, no regulations have been issued.  As discussed in more detail below, when the earnings that have been taxed under subpart F or an investment in U.S. property are actually distributed to a USSH, the earnings are excluded from income under section 959, and the USSH must reduce his basis in the stock of the CFC (or intermediate foreign entity) under section 961(b).   
		  
If the USSH of a CFC is a U.S. corporation and the section 951 inclusion is with respect to a foreign corporation that is a member of the same qualified group (as defined in section 902(b)) as the U.S. corporation, the section 951 inclusion is treated as dividend for purposes of section 902.  \S 960(a)(1).  Thus, subpart F inclusions can bring with them deemed paid credits.  When the earnings that were previously taxed under section 951 are actually received, they will not bring with them any deemed paid credits.  \S 961(a)(2).
		
			\section{Foreign Base Company Income}
		\codereg{954(c), (d), and (e); 954(h) and (i) (skim) }{1.954-1 (Calculation of FBCI)--skim very, very lightly}
		
			
		\subsection{Foreign Personal Holding Company Income}
				\codereg{954(c)}{1.954-2(b)(5)(ii); -2(b)(6); -2)(d)(1), (2), and (3) (skim the examples)}
		
		
FPHCI consists of portfolio-type income--interest, dividends, rents, and royalties--arising from the investment of easily movable capital.  Congress believed that this capital could have been as easily invested directly by the U.S. owners and deferral is therefore inappropriate.  Various categories of transactions generate portfolio-type income that constitutes FPHCI, but if the income is connected with the active conduct of the CFC's business, for example \emph{rents} derived from operating a rental car business, it loses its character as FPHCI and instead constitutes business income eligible for deferral.  

Included generally in FPHCI are dividends, interest, rents, royalties, annuities, and the net gains from the sale of property that produces dividends and interest or that does not produce any income. Thus, this provision covers gains from the sale of non-dividend paying stock, options, forwards, and futures.  \S 954(c)(1)(A) and (B); Reg. \S 1.954-2(e)(3).  If the property is held for use in the CFC's business, however, gains and losses are not FPHCI.  Reg. \S1.954-2(e)(3)(ii)--(4). Thus, gains from the sale of stock of a subsidiary are FPHCI, but gains arising from the sale of the CFC's business property are not.  This distinction can exploited by taxpayers through the check-the-box regime.  \emph{See Dover Corporation v. CIR} below.  For dealers of bonds, stocks, forwards, options, and notional principal contracts, income from such property is not FPHCI.  \S 954(c)(2)(C). 

Dividends or interest are excluded from FPHCI if the CFC receives the dividends or interest from a related corporation organized in the same country and which has a substantial part of its trade or business assets located  in the same foreign country.  \S 954(c)(3)(A)(i).  This  exception is intended to permit the tax-free movement of earnings from the manufacturing activities of one CFC to another CFC located in the same country.  Consequently, this exception does not apply to interest that reduces the payor's subpart F income.  For example, if the payor CFC earns subpart F income and the interest paid is deducted against that income, it would not qualify.  \S 954(c)(3)(B).  In addition, the dividend exception does not apply to any earnings of the dividend-paying CFC that were accumulated during the period which the recipient did not directly or indirectly own the CFC.  \S 954(c)(3)(C).  

The check-the-box regime has removed some of the bite of this provision.  For example, assume that a Dutch CFC owns 100\% of the stock of a Swiss CFC.  Dividends or interest paid by the Swiss CFC to the Dutch CFC would not be eligible for the same-country exclusion.  If, however, the Swiss CFC were a disregarded entity, earnings could be transferred from the Swiss CFC to the Dutch CFC without generating any subpart F income, even though for foreign tax purposes the payments would have tax significance.  

A similar exclusion (and exception to the exclusions) applies to rents and royalties received from \emph{related} corporations organized in the same country as the recipient.  \S 954(c)(3)(A)(ii).  The exception facilitates the separation of holding companies that own title to intellectual or real property that is licensed to corporations carrying on active businesses in the same country.  Rents and royalties derived in the active conduct of a trade or business and which are \emph{not} received from related persons are also not FPHCI.  \S 954(c)(2)(A) and Reg. \S1.954-2(b)(6).  This exception encompasses such businesses as hotels, rental car companies, and intellectual property licensing companies.  

For post-2005 tax years, section 954(c)(6) provides that dividends, interest, rents and royalties are not FPHCI if they are received from a related corporation and if the income is not attributable to the payor corporation's income which is subpart F.  The rules for determining whether income is attributable to a particular category of income--subpart F or non-subpart F--are the same rules used in determining whether interest, dividends, or subpart F inclusions from a CFC are allocable to the passive category or general category for purposes of calculating the the foreign tax credit limitation under section 904.  This rule greatly facilities the tax-free movement of business earnings among CFCs but may also inappropriately lead to the deferral of income that was previously subpart F income when the payment is deductible.  For example, interest paid to a CFC in a tax haven could possible qualify under this provision, provided that none of the interest was allocable to the payor corporation's subpart F income.  This provision was originally supposed to expire in 2009, but was extended through 2013.               

FPHCI includes other similar types of income.  Net gains from commodities transactions, except hedging transactions or gains and losses from selling commodities if the commodities are inventories or used in the CFC's business, are FPHCI, as are interest equivalents, substitute dividends, and notional principal contract income.  \S\S 954(c)(1)(C), (D), (E), (F), and (G).

FPHCI also covers amounts received under a contract to furnish personal services if a person other than the CFC can designate the service provider or amounts received from the sale or disposition of such a contract when the service provider owns at least 25\% of the CFC's stock during the year.  \S 954(c)(1)(I). 

Section 954(h) expanded the exceptions to the FPHCI prong of the subpart F regime to include the FPHCI of CFCs predominantly engaged in the banking and financing business.  A similar rule applies to insurance companies.  \S 954(i).  The rationale for this rule is that banking and financing businesses earn FPHCI as part of their active business and that such income therefore is similar to active income earned by manufacturing companies and should be eligible for deferral.  Although enacted as a temporary measure in 1997, this provision, 15 years later, shows no signs of extinction.       

\addcontentsline{toc}{section}{\protect\numberline{}Dover Corporation v. CIR}
\begin{select}
\caseart{Dover Corporation v. CIR}{122 T.C. 324 (2004)}{Halpern, J.}\\
\begin{center}\textbf{Sale of H \& C}\\
\textbf{Background}\\
\end{center}
\textit{[On June 30, 1997, Dover UK, a wholly owned subsidiary of Dover Corporation, and Dover Corporation entered into an agreement with Thyssen Industrie Holdings U.K. PLC (Thyssen), a German corporation registered in England and Wales, and its German parent, Thyssen Industrie AG, for the sale by Dover UK to Thyssen of the stock of H \& C, a UK corporation.  Under UK law, beneficial title to the H \& C shares passed from Dover UK to Thyssen on July 11, 1997, when an escrow condition was satisfied.  For US tax purposes, gain from the sale of stock of a controlled foreign corporation (CFC) (a corporation owned or controlled by US persons) is foreign personal holding company income (FPHCI) and is currently taxed to the US owners.  Income from the sale of business property, however, is not FPHCI, and will only be subject to US tax when it is remitted, if ever, to the US owners.]}
\begin{center} \textbf{Retroactive Election To Treat H \& C as a Disregarded Entity}\\
\end{center}
By letter dated December 3, 1998, [Dover] requested that respondent grant an extension of time, pursuant to sections 301.9100-1(c) and 301.9100-3, for H \& C to file a retroactive election to be a disregarded entity for Federal tax purposes (the request for 9100 relief). Specifically, [Dover] requested: ``H \& C be granted an extension of time to make an election: (a) * * * to be disregarded as an entity separate from its owner for U.S. tax purposes and (b) effective immediately prior to the sale of stock in H \& C by Dover UK to Thyssen UK.'' In the request for 9100 relief, petitioner stated that the date of the sale was June 30, 1997, and, on the Form 8832, Entity
Classification Election (Form 8832), attached to the request for 9100 relief, it set forth June 30, 1997, as the proposed effective date of the election.

\ldots [The CIR] granted to H \& C ``an extension of time for making the election to be disregarded as an entity separate from its owner for
federal tax purposes, effective immediately prior to the sale on * * * [June 30, 1997], until 60 days following the date of this
letter.'' Respondent, however, added the following caveat:
\begin{quote}no inference should be drawn from this letter that any gain from the sale of * * * [H \& C's] assets immediately following its
election to be disregarded as an entity separate from its owner gives rise to gain that is not [FPHCI].
\end{quote}
On or about October 10, 1999, H \& C made an election on Form 8832 to be disregarded as a separate entity. The Form 8832 specifies that the election is to be effective beginning June 30, 1997.

\ldots
This case presents an issue of first impression and, insofar as we are aware, the first occasion that any court has had to opine
on the impact of the so-called check-the-box regulations on the application of a specific provision of the Code, in this case,
section 954(c)(1)(B)(iii) (defining, in part, FPHCI).
\ldots
\begin{center}\textbf{The Check-the-Box Regulations}\\
\end{center}
Section 301.7701-3(a) sets forth the general rule that ``[a] business entity that is not classified as a corporation * * * can elect
its classification for federal tax purposes as provided in this section.''

In pertinent part, section 301.7701-3(g)(1)(iii), provides:
\begin{quote}(iii) Association to disregarded entity. If an eligible entity classified as an association elects * * * to be disregarded as an
entity separate from its owner, the following is deemed to occur: The association distributes all of its assets and liabilities to
its single owner in liquidation of the association.
\end{quote}
Section 301.7701-2(a), states that, ``if * * * [an] entity is disregarded, its activities are treated in the same manner as a sole proprietorship, branch, or division of the owner.''

Under section 301.7701-3(c)(1)(i), a classification election, including an election to change classification, is made by filing a
Form 8832 with the IRS service center designated on that form. Under subdivision (iii), the election is effective ``on the date
specified by the entity on Form 8832'' if, as in this case, one is specified.
Under section 301.7701-3(g)(3)(i), an election to change classification ``is treated as occurring at the start of the day for which the election is effective,'' and ``[a]ny transactions that are deemed to occur * * * as a result of a change in classification [e.g., in the case of a change in classification from association to disregarded entity, the deemed liquidation] are treated as occurring immediately before the close of the day before the election is effective.'' For example, if H \& C's disregarded entity election is effective as of the start of business on June 30, 1997, the deemed liquidation of H \& C is treated as occurring immediately before the close of business on June 29, 1997.

The making of a disregarded entity election ``is considered to be the adoption of a plan of liquidation immediately before the
deemed liquidation,'' thereby qualifying the parties to the deemed liquidation for tax-free treatment under sections 332 and
337. Sec. 301.7701-3(g)(2)(ii).
Lastly, section 301.7701-3(g)(2)(i), provides:
\begin{quote}(2) Effect of elective changes.--(i) In general. The tax treatment of a change in the classification of an entity for federal tax
purposes by election under paragraph (c)(1)(i) of this section is determined under all relevant provisions of the Internal
Revenue Code and general principles of tax law, including the step transaction doctrine.
\end{quote}

The preamble to the 1997 proposed regulations, which contains the identical provision, explains the purpose of the above
quoted provision:
\begin{quote}This provision * * * is intended to ensure that the tax consequences of an elective change will be identical to the
consequences that would have occurred if the taxpayer had actually taken the steps described in the * * * regulations.
[....]
\end{quote}
\ldots
Petitioner argues that ``the check-the-box regulations * * * impose continuity of business enterprise as a consequence of * * *
[a disregarded entity] election,'' citing section 301.7701-2(a). In pertinent part, that regulation provides: ``If * * * [a business
entity with only one owner] is disregarded, its activities are treated in the same manner as a sole proprietorship, branch or
division of the owner.''
Petitioner argues: ``As a consequence [of the above-quoted regulation], there was as a matter of law and under respondent's
own check-the-box regulations * * * a continuing business use of H \& C's assets, which were deemed to be a branch or
division of Dover UK.''
\begin{center}
\textbf{C. Analysis and Application of Authorities}\\
\end{center}
Respondent specifically acknowledges that, for tax purposes, H \& C's disregarded entity election constituted a deemed section 332 liquidation of H \& C into Dover UK, whereby H \& C became a branch or division of Dover UK. Respondent refers to the disregarded entity election as a ``check-the-box liquidation'' and states that there is no difference between it and an actual section 332 liquidation.

Accordingly, the principal question before us is whether, attendant to a section 332 liquidation, the transferee parent
corporation succeeds to the business history of its liquidated subsidiary with the result that the subsidiary's assets used in its
trade or business constitute assets used in the parent's trade or business upon receipt of those assets by the parent.

Because Dover UK's disregarded entity election is characterized as an actual liquidation of H \& C for income tax purposes,
among the undisputed tax consequences are the following: (1) Dover UK recognized neither gain nor loss on its deemed
receipt of H \& C's assets, see sec. 332(a); (2) it succeeded to H \& C's basis in those assets, see sec. 334(b); and (3 ) it would
add H \& C's holding period to its own (deemed) holding period in those assets, see sec. 1223(2). Moreover, the deemed-received assets did not constitute a single, mass asset with a unitary holding period, but comprised the numerous classes of both tangible and intangible property necessary to constitute a going elevator installation and service business (e.g., tools, spare parts, fixtures, and accounts receivable). Each item deemed received by Dover UK came with a distinct, carryover basis and an existing holding period.

Agreeing, as he must, to the foregoing description of the tax consequences resulting to Dover UK from its deemed receipt of H \& C's assets, respondent, nevertheless, argues: ``Dover UK must * * * use, or hold for use, such assets for the requisite period of time in its trade or business before Dover UK is allowed to exclude from FPHCI the gain from the [deemed] sale of those assets.'' Respondent refuses to attribute H \& C's business history to Dover UK:
\begin{quote}Dover UK had a separate identity from H \& C and the business of H \& C (installing and servicing elevators) was not the
business of Dover UK (a holding company). In addition, Dover UK never intended to use the assets in an elevator business.
It acquired the assets for the purpose of selling those assets and avoiding FPHCI.
\end{quote}
The arguments of the parties concerning whether we must deem Dover UK to have succeeded to H \& C's business history center on section 381, which provides that the acquiring corporation in a section 332 liquidation succeeds to the various tax attributes of the distributing corporation described in section 381(c). While section 381(c) does not list among the carryover attributes the distributing corporation's business history, we agree with petitioner that respondent's denial that Dover UK succeeded to H \& C's business history is inconsistent with his position in Rev. Rul. 75-223, Rev. Rul. 77-376, G.C.M. 37,054 (Mar. 21, 1977), and a number of private letter rulings. \ldots The crucial finding in all of the rulings discussed above is that, in any corporate amalgamation involving the attribute carryover rules of section 381, the surviving or recipient corporation is viewed as if it had always conducted the business of the formerly separate corporation(s) whose assets are acquired by the surviving corporation. \ldots 

In Rauenhorst v. Commissioner, we refused ``to allow * * * [IRS] counsel to argue the legal principles of * * * opinions against the principles and public guidance articulated in the Commissioner's currently outstanding revenue rulings." Id. at 170-171. Consistent with our holding in Rauenhorst, we refuse to allow respondent to argue the legal principles of Acro Manufacturing Co. v. Commissioner, 39 T.C. 377, (1962), against the principles subsequently articulated in Rev. Rul. 75-223, 1975-2 C.B. 109, Rev. Rul. 77-376, 1977-2 C.B. 107, and G.C.M. 37,054 (Mar. 21, 1977). We therefore consider respondent to have conceded that, as a direct result of a section 332 liquidation of an operating subsidiary, the surviving parent corporation is considered as having been engaged in the liquidated subsidiary's preliquidation trade or business, with the
result that the assets of that trade or business are deemed assets used in the surviving parent's trade or business at the time of receipt. See Rauenhorst v. Commissioner, supra at 170-171, 173. As stated by respondent on brief, pursuant to section 301.7701-3(g)(1)(ii) and (2)(i), ``there is no difference between a check-the-box liquidation and an actual liquidation.''  Therefore, notwithstanding our holding in Acro Manufacturing Co. v. Commissioner, supra, we conclude that respondent has conceded that Dover UK's deemed sale of the H \& C assets immediately after the check-the-box liquidation of H \& C constituted a sale of property used in Dover UK's business within the meaning of section 1 .954-2(e)(3)(ii) through (iv).

Respondent's acknowledgment that the business history and activities of a subsidiary carry over to its parent in connection with a section 332 liquidation of the subsidiary is also reflected in section 301.7701-2(a), which provides that ``if the entity is disregarded, its activities are treated in the same manner as a sole proprietorship, branch, or division of the owner''. \ldots  Thus, the plainly understood import of the cited regulation's use of the terms ``branch'' and ``division'' to describe the impact of the deemed section 332 liquidation resulting from a disregarded entity election with respect to an operating subsidiary (particularly in light of respondent's ruling position, as set forth supra) is that the activities of the business operation indirectly owned by the parent through its former subsidiary become the
activities of a functional or operating business unit directly owned and conducted by the parent. It follows from the language of the regulation that the assets used in the business of the (deemed) liquidated subsidiary retain their status as assets used in the same business by the (deemed) branch or division of the parent.  

We interpret our statement in Acro Manufacturing Co. v. Commissioner, 39 T.C. at 386, that the taxpayer ``neither acquired nor used the Button assets in its business'' as tantamount to a statement that the Button business never became an operating branch or division of the taxpayer. Therefore, the Secretary and the Commissioner, in effect, rejected our position in that case by issuing section 301.7701-2(a), as well as Rev. Rul. 75-223, Rev. Rul. 77-376, and G.C.M. 37,054.

Finally, we note that, consistent with his admonition in the preamble to the final check-the-box regulations, T.D. 8697, 1997-1 C .B. at 216, that ``Treasury and the IRS will continue to monitor carefully the uses of partnerships [and, by extension, disregarded entities] in the international context and will take appropriate action when * * * [such entities] are used to achieve results that are inconsistent with the policies and rules of particular Code provisions,'' respondent was, of course, free to amend his regulations to require a minimum period of continuous operation of a foreign disregarded entity's business, prior to the disposition of that business, as a condition precedent to treating the owner as having been engaged in the trade or business for purposes of characterizing the gain or loss. But, in the absence of respondent's exercise of that authority, we must apply the regulation as written. \ldots 
\begin{center}
\textbf{Conclusion}\\
\end{center}
Dover UK's gain on the deemed sale of the H \& C assets does not constitute FPHCI\ldots
\end{select}

	\addcontentsline{toc}{section}{\protect\numberline{}FPHCI Problems} 
	\begin{center}
		\textbf{FPHCI Problems}
	\end{center}
	\begin{select}
	
			\begin{enumerate}

				\item CFC receives: 100 of bank interest; 100 of interest from a Virgin PLC bond; 50 of NPC income; 70 of substitute dividends; 100 from the sale of stock of IBM.  Which of the preceding items are FPHCI (disregard the de minimis exception and full inclusion rule)?

				\item CFC buys a building and rents it out for years 1 and 2.  In year 3, it converts the building to use in its trade or business.  A couple of years later, it sells the building for a gain.  Is the gain FPHCI?  Reg. \S\S 1.954-2(a)(3); 1.954-2(e)(3)(iii).

				\item CFC owns and rents out a building. The CFC hires outside managers to manage it.  Is the rental income FPHCI?  Reg. \S 1.954-2(c)(3), Ex. 3.

				\item CFC holds a portfolio of photographs that its photographers have taken and that it licenses.  Is the royalty income FPHCI?  What if CFC purchases a portfolio of photographs and licenses the copyrights.  Is the royalty income FPHCI?  Reg. \S 1.954-2(d)(1), (2), and (3), Examples.
				
				\item MK, a comely Hollywood actress, is hired to co-star in a new feature film.  She forms a foreign corporation that will sign the contract obliging her to act in the film and receive the compensation.  Assume that this arrangement would be respected for U.S. tax purposes, \emph{i.e.}, that the income paid to the foreign corporation would be the corporation's (and not MK's) income.  Is the income FPHCI?  \S954(c)(1)(H).

			\end{enumerate}
	\end{select}
	


	
	\subsection{Foreign Base Company Sales Income}
	\codereg{954(d)}{1.954-3(a)(2)-(a)(4) (skim lightly some of the examples
	in -3(a)(4)(iv)(d); -3(b)(4), Examples 1 and 2}\\
		
		The abusive use of foreign base companies was a primary target of the subpart F regime.  Congress was particularly concerned with transactions that resulted in separating the income of a selling (or sales) subsidiary, usually incorporated in a low-tax country, from the manufacturing activities of related corporation, usually carried on in a high-tax country.  
		
		FBCSalI consists of income---including profits, commissions, and fees---from purchases or sales of personal property where: \margit{Income from property manufactured or produced in the CFC's country of incorporation or sold for use in the CFC's country of incorporation is not FBCSalI.} (1) a related person is either the buyer or seller; (2) the purchased property is manufactured, produced, grown, or extracted outside the CFC's country of formation; and (3) the property is sold for use, consumption, or disposition outside that country.  \S 954(d)(1).  FBCSalI also includes income a CFC earns from acting as an agent---the ``on behalf of'' language of the statute---for a related person.  For purposes of section 954, \emph{related} is defined to mean persons in control of or who are controlled by the CFC (or controlled by the same persons that control the CFC), where \emph{control} is ownership of more than 50\% of the relevant interests of the controlled or controlling entity.  \S 954(d)(3).    
		
		The regulations, originally issued in 1964, provide an important exception to FBCSalI for income from the sale of property to any person if the property is manufactured or produced by the CFC.  Reg. \S 1.954-3(a)(4)(i). \margit{The manufacturing exception.} This portion of the regulations was substantially modernized in regulations finalized in December, 2008, and which are effective for tax years beginning after June 30, 2009.  The new regulations recognize multinationals often have extensive cross-border manufacturing networks and that a CFC through its employees can make significant contributions to the manufacturing process carried on more efficiently by third-party contract manufacturers.  The regulations permit the CFC to treat such property as being manufactured by the CFC for FBCSalI purposes.       
		
		Purchased property is considered to be manufactured if the CFC \emph{substantially transforms} the property, for example, converting wood pulp into paper, steel rods into screws, or tuna fish into canned tuna.  Reg. \S 1.954-3(a)(4)(ii).  In the case of component parts, if the CFC's activities are considered ``substantial'' and are generally considered to constitute manufacturing, production, or construction, the sale of the property composed of the component parts will be treated as the sale of a manufactured product rather than the sale of component parts. Reg. \S 1.954-3(a)(4)(i) and (iii).  Under a safe-harbor provision, this test is satisfied if conversion costs (direct labor and factory burden) are 20\% or greater of the total cost of goods sold, provided the activities are not packaging, repackaging, labeling, or minor assembly operations.  Packaging, repackaging, labeling, or minor assembly operations will not constitute manufacturing.  Reg. \S1.954-3(a)(4)(iii).
		
	Under the \emph{substantial contribution} test, if property is manufactured, produced, or constructed by a third party and the CFC's employees substantially contribute to the manufacture, production, or construction, the property will be deemed to have been manufactured, produced, or construction by the CFC.  Reg. \S 1.954-3(a)(4)(iv).  Factors to be considered in determining whether a CFC makes a substantial contribution to the manufacture of personal property include: (1) oversight and direction of the activities or process (including management of the risk of loss) pursuant to which the property is manufactured under the principles of Reg. \S 1.954-3(a)(4)(ii) and (iii); (2) performance of manufacturing activities that are insufficient to satisfy the tests provided in Reg. \S 1.954-3(a)(4)(ii) or (iii); (3) control of the raw materials, work-in-process and finished goods; (4) management of the manufacturing profits; (5) material selection; (6) vendor selection; (7) control of logistics; (8) quality control; and (9) direction of the development, protection, and use of trade secrets, technology, product design and design specifications, and other intellectual property used in manufacturing the product.  Reg. \S 1.954-3(a)(4)(iv)(b).  
	
	Priv. Ltr. Rul. 201206003 excerpted below addresses application of the same-country manufacturing exception of Reg. \S1.954-3(a)(2). 

	Another important part of the FBCSalI regime is the branch rule of section 954(d)(2).  \margit{The Branch rules of section 954(d)(2).} Assume that a manufacturing CFC operates in a high-tax jurisdiction with a territorial tax system.  To avoid foreign tax, the CFC could set up a sales branch in a low-tax jurisdiction.  The \emph{sales income} earned by the sales branch would be exempt from foreign tax imposed by the CFC's country of incorporation and subpart F, because under U.S. tax principles, the activities of a branch do not have separate tax significance and therefore the same person would be treated as manufacturing and selling.  The branch thus could function as a tax haven, both for U.S. and foreign purposes.  Under section 954(d)(2) and Reg. \S 1.954-3(b), in certain circumstances the branch is treated as a wholly owned subsidiary of the CFC, thereby potentially generating FBCSalI.
	
	The branch rule is triggered when use of branch has the same \emph{tax effect} as the use of a separate CFC.   Furthermore, it can apply to either a sales/purchase or a manufacturing branch located outside the CFC's country of incorporation.  
	
	In the case of a sales/purchase branch, the use of a branch will have the same tax effect as a separate CFC if the income allocated to \emph{branch} is taxed at a rate less than 90\% of and at least 5\% points less than effective tax rate of the CFC.  Reg. \S 1.954-3(b)(1)(i).  In the case of a manufacturing branch, the use of a branch will have the same tax effect as a separate CFC if income allocated to \emph{remainder of the CFC} is taxed at a rate less than 90\% of and at least 5\% points less than the effective tax rate of the branch country.  Reg. \S 1.954-3(b)(1)(ii).
	
	If the branch rule applies, the branch is treated as a separate subsidiary incorporated in the country where it's located.  Reg. \S 1.954-3(b)(2)(i)(a).  Property sold or purchased by a \emph{sales/purchase branch} is treated as purchased or sold ``on behalf'' of the CFC by the branch.  In essence, property is treated as having been transferred tax-free between the branch and the CFC after purchase and before sale, thereby \emph{potentially} generating FBCSalI.  Property sold or purchased by a CFC from the \emph{manufacturing branch} is treated as made ``on behalf'' of the branch by the CFC.  In essence, property is treated as having been transferred tax-free between the branch and the CFC after manufacture and before sale, thereby \emph{potentially} generating FBCSalI.

	After the original branch regulations were issued, the IRS issued Rev. Rul. 75-7, 1975-1 C.B. 224, which held that a CFC did not realize FBCSalI upon the purchase of ore concentrate from a related person that was converted into ferroalloy by an unrelated contract manufacturer and sold to unrelated persons.  Under the facts of the ruling, the CFC retained substantial control over the manufacturing process and accordingly was treated as having substantially transformed the property.  The ruling also held that because the contract manufacturer was located in a different country than the CFC, the branch rules were potentially applicable.
	
	Subsequent to the issuance of Rev. Rul. 75-7, the Tax Court held in two cases, \emph{Ashland Oil, Inc. v. CIR}, 95 TC 348 (1990) and \emph{Vetco, Inc. v. CIR}, 95 TC 579 (1990), that an unrelated manufacturing corporation and a wholly owned subsidiary could not be treated as branches under the branch rule of section 954(d)(2).  In response to these cases, the IRS issued Rev. Rul. 97-48, which revoked Rev. Rul. 75-7 and held that the activities of a separate contract manufacturer cannot be attributed to the CFC.  Consequently, activities of contract manufacturers could no longer be treated as being carried out by the CFC that hired the contract manufacturer.  It was clear that the manufacturing regulations would need to be revisited to take into account modern international manufacturing processes. 
	
	The regulations issued in 2008 also significantly modify and modernize the branch rules.  In particular, the regulations provide guidance for applying the branch rule when more than one branch engages in manufacturing.  In addition, the regulations incorporate a presumption that if a branch satisfies the physical manufacturing test, the remainder of the CFC will be presumed not to make a substantial contribution to the manufacture of the property.	

\addcontentsline{toc}{section}{\protect\numberline{}PLR 201206003}
\begin{select}
\revrul{Priv. Ltr. Rul. 201206003}{Feb. 10, 2012}
\ldots\\


\begin{center} \textbf{FACTS}
\end{center}


Taxpayer is a U.S. publicly traded multinational company and a leading global provider of Products. Taxpayer conducts activities directly and through domestic and foreign subsidiaries. Taxpayer, directly or indirectly, wholly owns all of the issued and outstanding shares of certain controlled foreign corporations (``CFCs�), within the meaning \S957(a), including Corporation X, which was created under the laws of Country 1.

Corporation Y is a publicly traded multinational corporation that is not related to Taxpayer, or any of Taxpayer's subsidiaries and other affiliated groups, within the meaning of \S954(d)(3). Corporation Y was created under the laws of Country 1. Corporation Y is a leading manufacturer of Products.

Pursuant to an agreement between Taxpayer affiliates and Corporation Y affiliates, Corporation Y and its affiliates perform physical manufacturing activities for Products (as described below), and sell finished Products to Taxpayer affiliates, including Corporation X, for distribution in Taxpayer's supply chains in Region. Corporation X resells Products to various Taxpayer distribution center affiliates that are related persons within the meaning of \S954(d)(3). Taxpayer distribution center affiliates generally on-sell Products to Taxpayer sales entities, which, in turn, sell Products to third party customers generally within the same jurisdiction as the applicable Taxpayer sales entity. The distribution of Products makes up a significant portion of Taxpayer's Business.

The manufacture of Products by Corporation Y and its affiliates is a multi-step process and entails several stages of manufacturing in multiple jurisdictions that involve component parts production and final assembly, with approximately b component parts embedded in each of the Products. Notwithstanding the total components, most of which are purchased as raw materials, Corporation Y manufactures several critical component parts incorporated in Products exclusively in Country 1, as described below.

In addition, Corporation Y and its affiliates conduct finishing manufacturing activities with respect to Products in countries other than Country 1, as set forth below.

Certain component parts are critical to the finished Products from both a value and cost perspective (``Critical Component Parts"). Of the approximately b component parts in each of the Products, c parts, with respect to Product 1, and d parts, with respect to Product 2, are Critical Component Parts. Of the Critical Component Parts, e parts, with respect to Product 1, and f parts, with respect to Product 2, are manufactured by Corporation Y exclusively in Country 1 (collectively, ``Country 1 Manufactured Component Parts").

The Country 1 Manufactured Component Parts are manufactured exclusively in Country 1 for certain essential competitive reasons, including quality control and protection of critical, competitively-advantaged intellectual property inherent in the manufacturing of the component parts. Manufacturing activities are performed by a significant number of employees of Corporation Y in factories located in Country 1. However, Products do not bear the moniker ``Made in Country 1," and are identified in certain reports provided by Corporation Y as non-Country 1 manufactured Products.

The finishing manufacturing activities with respect to Products are performed outside of Country 1 at finishing manufacturing plants located outside of Country 1. The activities performed at the plants include the manufacture of component parts embedded in Products, the assembly of Products and packaging, labeling and shipping of Products. Products finished in these plants are designated as ``Made in --------", with the jurisdiction of the finishing manufacturing plant determining the applicable designation. Corporation Y's finishing manufacturing activities in jurisdictions outside of Country 1 are conducted through wholly-owned subsidiaries of Corporation Y in those jurisdictions. The largest finishing manufacturing subsidiary outside of Country 1 is located in Country 2.

Taxpayer represents that the manufacturing activities performed by Corporation Y in Country 1 with respect to the Country 1 Manufactured Component Parts are substantial in nature and ``constitute the manufacture, production, or construction of property" with respect to finished Products within the meaning of Reg. \S1.954-3(a)(4)(iii), and are substantial with respect to the manufacture of the finished Products as a whole.

In addition, Taxpayer believes that the manufacturing activities performed by Corporation Y and its affiliates with respect to Products in Country 2 may ``constitute the manufacture, production, or construction of property� with respect to finished Products, within the meaning of  Reg. \S1.954-3(a)(4)(iii).


\begin{center} \textbf{RULING REQUESTED}
\end{center}


Income earned by Corporation X with respect to the sale of Products purchased from Corporation Y, or its affiliates, to a related person (within the meaning of \S954(d)(3)) is not foreign base company sales income within the meaning of \S954(d) because the income qualifies for the same country manufacturing exception under \S954(d)(1)(A).

\begin{center} \textbf{LAW}
\end{center}

Section 954(d)(1) defines foreign base company sales income (``FBCSI") to mean income (whether in the form of profits, commissions, fees, or otherwise) derived in connection with: the purchase of personal property from a related person and its sale to any person, the sale of personal property to any person on behalf of a related person, the purchase of personal property from any person and its sale to a related person, or the purchase of personal property from any person on behalf of a related person where (A) the property which is purchased (or in the case of property sold on behalf of a related person, the property which is sold) is manufactured, produced, grown, or extracted outside the country under the laws of which the CFC is created or organized, and (B) the property is sold for use, consumption, or disposition outside such foreign country, or, in the case of property purchased on behalf of a related person, is purchased for use, consumption, or disposition outside such foreign country.

Section 954(d)(3) provides that a person is a related person with respect to a CFC if: (1) such person is an individual, corporation, partnership, trust, or estate which controls, or is controlled by, the CFC; or (2) such person is a corporation, partnership, trust, or estate which is controlled by the same person or persons which control the CFC. Control is defined as the direct or indirect ownership of more than 50 percent of the total voting power of all classes of stock entitled to vote or the total value of a corporation, or more than 50 percent of the beneficial interest in a partnership.

 Reg. \S1.954-3(a)(2) provides that FBCSI does not include income derived in connection with the purchase and sale of personal property (or purchase or sale of personal property on behalf of a related person) in a transaction described in  Reg. \S1.954-3(a)(1) if the property is manufactured, produced, constructed, grown or extracted in the country under the laws of which the CFC that purchases and sells the property (or acts on behalf of a related person) is created or organized. The principles set forth in  Reg. \S1.954(a)(4)(ii) and (a)(4)(iii) apply under  Reg. \S1.954(a)(2) in determining what constitutes the manufacture, production, or construction of personal property, excluding the requirement set forth in  Reg. \S1.954(a)(4)(i) that the provisions of  Reg. \S1.954(a)(4)(ii) and (a)(4)(iii) may only be satisfied through the activities of employees of the corporation manufacturing, producing or constructing the personal property. The principles of  Reg. \S1.954(a)(4)(iv) apply under Reg. \S1.954(a)(2) in determining what constitutes the manufacture, production or construction of personal property but only when the personal property is manufactured, produced or constructed by a person related to the CFC within the meaning of  Reg. \S1.954-1(f).
 
 
\begin{center} \textbf{ANALYSIS}
\end{center}

 
Taxpayer is a U.S. Shareholder of Corporation X, which is a CFC. Accordingly, Taxpayer is required to include amounts in income under \S951(a)(1), including its pro rata share of Corporation X's subpart F income.
 
 One type of subpart F income is FBCSI. In general, income derived by a CFC from the purchase and sale of property is FBCSI if the property is sold to a person that is a related person with respect to the CFC within the meaning of \S954(d). However, pursuant to \S954(d)(1)(A), the income is not FBCSI if the property is manufactured in the country in which the CFC is organized (``same country manufacturing exception").
 
 Corporation X, which was created under the laws of Country 1, derives income from the sale of Products to related persons. However, pursuant to the same country manufacturing exception, Corporation X's sale of products will not generate FBCSI if another person physically manufactures the Products in Country 1.
 
 Corporation X purchases Products from Corporation Y and its affiliates. Corporation Y and its affiliates manufacture Products in a multi-step process, which involves component parts production and final assembly, in multiple jurisdictions. Employees of Corporation Y and its affiliates conduct manufacturing activities with respect to Products in Country 1 and outside of Country 1. Specifically, the manufacturing activities with respect to the Country 1 Manufactured Component Parts are conducted by Corporation Y exclusively in Country 1, and manufacturing activities with respect to some component parts and final assembly are conducted by Corporation Y and its affiliates outside of Country 1.
 
 Taxpayer has represented that the activities conducted by Corporation Y in Country 1 constitute manufacturing within the meaning of Reg. \S1.954-3(a)(4)(iii) and are substantial with respect to the Products as a whole.
 
 
\begin{center} \textbf{RULING }
\end{center}

 
 Based on the information submitted and the representations made, we rule as follows:
Income earned by Corporation X with respect to the sale of Products purchased from Corporation Y, or its affiliates, to a related person (within the meaning of \S954(d)(3)) is not FBCSI within the meaning of Code \S954(d) because the income qualifies for the same country manufacturing exception under \S954(d)(1)(A).
 
 \end{select}
	
	
\addcontentsline{toc}{section}{\protect\numberline{}FBCSalI Problems} 
	\begin{center}
		\textbf{FBCSalI Problems}
	\end{center}
	\begin{select}
	\emph{For the following problems, DC is a U.S. corporation that owns 100\% of CFC1, incorporated in the U.K., and 100\% of CFC2, incorporated in the Cayman Islands.}
			\begin{enumerate}
				\item CFC2 buys grapes from DC and sells them to CFC1.  Is the income FBCSalI?  [Reg. \S 1.954-3(a)(1)(ii)(a).]
				\item CFC2 buys skis manufactured by DC and sell them to unrelated distributors in the U.K.  Is the income FBCSalI? [Reg. \S 1.954-3(a)(1)(iii).]
				\item CFC1 buys skis manufactured by DC and sells them to unrelated persons in the U.K.  Is the income FBCSalI? [Reg. \S 1.954-3(a)(3).]
				\item CFC1 buys shirts manufactured in the U.K. and sells them to DC. Is the income FBCSalI? [Reg. \S 1.954-3(a)(2).]
				\item CFC2 buys skis from DC, paints them and wraps them in plastic, and sells them to CFC1.  Is the income FBCSalI? [Reg. \S 1.954-3(a)(4)(iii)]
				\item CFC1 buys engines, transmissions, and other car components from DC and assembles and sells the cars in the U.K.  Is the income FBCSalI? [Reg. \S 1.954-3(a)(4)(iii), Example 2.]
				\item CFC1 purchases raw materials from a related person. The raw materials are manufactured into Product X by CM, an unrelated corporation, pursuant to a contract manufacturing arrangement. CM physically performs the substantial transformation, assembly, or conversion outside of the U.K.  Product X is sold by CFC1 for use outside of the U.K.  Under the terms of the contract, CFC1 retains the right to control the raw materials, work-in-process, and finished goods, and the right to oversee and direct the activities or process pursuant to which Product X is manufactured by CM. CFC1 owns the intellectual property used in the manufacturing process, and through its employees, engages in product design and quality control and controls manufacturing related logistics. CFC1 employees exercise the right to oversee and direct the activities of CM in the manufacture of Product X.  Is the income FBCSalI? [Reg. \S 1.954-3(a)(4)(iv)(d), Example 2.]
				\item CFC1 manufactures bikes in the U.K. and uses a ``-stan'' branch to sell the bikes outside of the U.K.  The U.K. tax rate is 30\% and the ``-stan'' rate is 5\%.  Does CFC1 potentially have any FBCSalI?   [Reg. \S 1.954-3(b)(4), Example 1.]
				\item CFC2 has a manufacturing branch in China and sells goods produced by the branch throughout the world.  China taxes the manufacturing profit at 20\% but doesn't tax any of the income from the sales.  The Cayman Islands tax rate is 0\%.  If CFC2 were incorporated in China, all of its income would be taxed at 20\%.   Does CFC2 have any FBCSalI?   [Reg. \S 1.954-3(b)(4), Example 2.]
				\end{enumerate}
		\end{select}
		
	
		\subsection{Foreign Base Company Services Income}
				\codereg{954(e)}{1.954-4(b)(3), Examples }

Foreign base company services income (FBCSerI) is income from services performed for or on behalf of a related person and performed outside the country where the CFC is incorporated.  Services include technical, managerial, engineering, architectural, scientific, and skilled industrial commercial services.  The purpose of the provision is to deny tax deferral where the income of a service subsidiary is separated from the manufacturing or similar activities of a related corporation.  Under Reg. \S1.954-4(b), services are performed on behalf of a related person in four scenarios:

\begin{enumerate}
	\item the CFC receives compensation or any other ``substantial financial benefit from'' a related person for performing services;
	\item the CFC performs services that a  related person is obligated to perform;
	\item the CFC's performance of the services is a ``condition or material term of sale''; or
	\item the CFC receives ``substantial assistance'' from a related party.
\end{enumerate}  

In Notice 2007-13 (Jan. 9, 2007), the Service announced that it will amend the substantial assistance rules of the FBCSerI regulations by eliminating the subjective ``principal element" test and treating assistance rendered by a related person as substantial if the assistance satisfies an objective cost test.  The regulations are being revised because ``many of the U.S. multinationals that provide services outside of the United States currently have globally integrated businesses with support capabilities for unrelated customer projects in different geographic locations, largely based on factors such as expertise and cost efficiencies."  Below is an excerpt from the Notice, which describes the current rules and discusses the proposed changes.

\addcontentsline{toc}{section}{\protect\numberline{}Notice 2007-13}
\begin{select}
\revrul{Notice 2007-13}{2007-5 IRB 1}
\ldots\\
\begin{center} \textbf{SECTION 2.  SUBSTANTIAL ASSISTANCE RULES.}\\
\textbf{A. BACKGROUND}
\end{center} \ldots 

Treas. Reg. \S 1.954-4(b)(1)(iv) defines ``services which are performed for, or on behalf of, a related person'' to include substantial assistance contributing to the performance of services by a CFC that has been furnished by a related person or persons. Treas. Reg. \S 1.954-4(b)(2)(ii) sets forth the rules for the application of the substantial assistance test. Treas. Reg. \S 1.954-4(b)(2)(ii)(a) states, in general, that assistance ``shall include, but shall not be limited to, direction, supervision, services, know-how, financial assistance (other than contributions to capital), and equipment, material, or supplies.'' Treas. Reg. \S 1.954-4(b)(2)(ii)(b) and (c) then provide separate tests depending on whether the assistance provided by the related person or persons is in the form of (1) direction, supervision, services or know-how, or (2) financial assistance, equipment, material or supplies.

Treas. Reg. \S 1.954-4(b)(2)(ii)(b) provides that assistance in the form of direction, supervision, services or know-how may be substantial under either a subjective or an objective test. Under the subjective test, assistance in the form of direction, supervision, services or know-how will be considered substantial if the assistance provides the CFC with skills which are a principal element in producing the income from the performance of such services by such CFC (the principal element test). For example, a CFC enters into a contract with an unrelated person to drill an oil well. The technical and supervisory personnel who oversee the drilling of the well are employees of M, a person related to CFC. In such an instance, the services performed by CFC for the unrelated party are considered foreign base company services because the services performed by M substantially assist CFC in the performance of the contract and the services performed by M are a principal element in producing the income from the performance of the drilling contract. Cf. Treas. Reg. \S 1.954-4(b)(3), Ex. 2.

Alternatively, under the objective test, assistance in the form of direction, supervision, services or know-how may be substantial if the cost to the CFC of the assistance furnished by persons related to the CFC equals 50 percent or more of the total cost to the CFC of performing the services performed by such CFC (the cost test). For these purposes, costs are determined after taking into account adjustments (if any) made under section 482. See Treas. Reg. \S 1.954-4(b)(2)(ii)(b).

Treas. Reg. \S 1.954-4(b)(2)(ii)(c) states, in general, that financial assistance, equipment, material, or supplies furnished by a person related to the CFC shall be considered assistance only in the amount, after taking into account adjustments (if any) made under section 482, by which the consideration actually paid by the CFC to the related person for the purchase or use of such item is less than the arm's length charge for such purchase or use. The total of all such amounts from all related persons is compared with the profits derived by the CFC from the performance of the services to determine whether the related party's contributions qualify as substantial assistance.

Treas. Reg. \S 1.954-4(b)(2)(ii)(d) expands on the tests in Treas. Reg. \S 1.954-4(b)(2)(ii)(b) and (c) by providing that, even if assistance furnished by a related person or persons to a CFC is not considered substantial under paragraphs (b) or (c) in isolation, it may nevertheless constitute substantial assistance when taken together or in combination with other assistance furnished by a related person or persons to the CFC. Treas. Reg. \S 1.954-4(b)(2)(ii)(e) provides that, in applying Treas. Reg. \S 1.954-4(b)(2)(ii)(b) and (d), assistance in the form of direction, supervision, services, or know-how shall not be taken into account, unless the assistance so furnished assists the CFC directly in the performance of the services performed. Treas. Reg. \S 1.954-4(b)(3) sets forth examples, including examples addressing the application of the substantial assistance test.

\begin{center}
\textbf{B. DISCUSSION}
\end{center}

The substantial assistance rules were published as final regulations in 1968 (TD 6981). The purpose of the substantial assistance rules is to treat as foreign base company services income, income received by a CFC from rendering services to an unrelated person where in rendering those services a related person substantially contributes to the CFC's performance of such services in a manner that suggests that the CFC, rather than the related party, entered into the contract to obtain a lower rate of tax on the service income. Since the regulations were published in 1968, there has been a substantial expansion in the reach of the global economy, particularly in the provision of global services. As a result, many of the U.S. multinationals that provide services outside of the United States currently have globally integrated businesses with support capabilities for unrelated customer projects in different geographic locations, largely based on factors such as expertise and cost efficiencies.

For example, a CFC may contract with an unrelated person to provide installation and subsequent repair services. A related CFC, however, is the foreign corporation that provides the repair services. Although the foreign related CFC that is providing the support services will continue to have foreign base company services income to the extent that it performs those services outside of its country of incorporation, it does not seem appropriate in the current global economy to continue to treat the profits of the CFC contracting to furnish services to the unrelated person as foreign base company services income because of the support services provided by a related foreign person. If the substantial assistance regulations are not amended to deal with these types of businesses structures, the regulations may cause taxpayers to change the way they do business or structure their operations in light of the substantial assistance rules, even if such a structure would be less efficient from a business perspective by, for example, requiring a taxpayer to duplicate a full service infrastructure in each country.

The Treasury Department and the IRS, however, remain concerned about the ability of related United States persons to shift profits offshore to CFCs organized in low tax jurisdictions in cases where the related United States person or persons provides so much assistance to the CFC that the CFC cannot be said to be providing services on its own account and thus acting as an independent entity. Accordingly, the Treasury Department and the IRS will revise the regulations to eliminate the substantial assistance rules, except in certain limited instances in which a United States person or persons provide sufficient assistance directly or indirectly to a related CFC.

\begin{center}
\textbf{C. PROPOSED GUIDANCE}
\end{center}

The Treasury Department and the IRS will amend Treas. Reg. \S 1.954-4(b)(1)(iv) and (b)(2)(ii) and the examples thereunder. Treas. Reg. \S 1.954-4(b)(1)(iv) as amended will provide that services performed by a CFC in a case where substantial assistance by a related United States person or persons (as the term is defined in section 957(c) of the Code) contributes to the performance of such service will constitute ``services which are performed for, or on behalf of, a related person.'' Treas. Reg. \S 1.954-4(b)(2)(ii) as amended will provide that ``substantial assistance" consists of assistance furnished (directly or indirectly) by a related United States person or persons to the CFC if the assistance satisfies an objective cost test. The subjective ``principal element'' test will no longer apply to determine substantial assistance. For purposes of the objective cost test, the definition of the term ``assistance" will include, but will not be limited to, direction, supervision, services, know-how, financial assistance (other than contributions to capital), and equipment, material, or supplies provided directly or indirectly by a related United States person to a CFC.

The cost test will be satisfied if the cost to the CFC of the services furnished by the related United States person or persons equals or exceeds 80 percent of the total cost to the CFC of performing the services. The term ``cost'' will be determined after taking into account adjustments, if any, made under section 482 of the Code. Taxpayers may apply the cost test either by demonstrating that the assistance provided, directly or indirectly, by related United States persons is below the 80 percent cost threshold, or, alternatively, by demonstrating that the cost of the services provided by the CFC itself, and/or by a related CFC, is more than 20 percent of the total cost to the CFC of performing the services. For this purpose, services provided by a CFC itself are not services provided ``indirectly'' by a related United States person (or persons). However, employees, officers, or directors of the CFC who are concurrently employees, officers, or directors of a related United States person during a taxable year of the CFC will be considered employees, officers or directors solely of the related United States person for such taxable year for purposes of this Notice.

The examples under Treas. Reg. \S 1.954-4(b)(2)(ii) will be amended to reflect the amendments to the regulations. The application of the proposed cost test is illustrated by the following examples.

\emph{Example 1}: USP, a U.S. corporation, wholly owns CFC1 and CFC2, each a foreign corporation. CFC1 enters into a contract with FP, an unrelated foreign person, to design a bridge for FP in Country Y, a foreign country that is not CFC1's country of organization. CFC1 incurs a total of \$100x of costs to design the bridge for FP. USP performs supervisory services in Country Y for CFC1 with respect to the contract for which CFC1 pays USP a fee. CFC1 directly performs services related to the performance of that contract that cost CFC1 \$15x. CFC2 performs centralized support services related to the performance of that contract in Country X, its country of organization, for which CFC1 pays CFC2 \$10x. CFC1 is not treated as receiving substantial assistance in the performance of that contract because more than 20\% of the cost of that contract is attributable to services furnished directly by CFC1 or a related CFC (CFC2).

\emph{Example 2}: USP, a U.S. corporation, wholly owns CFC1 and CFC2, each a foreign corporation. CFC2 enters into a contract with FP, an unrelated person, to design a bridge in Country Y, a foreign country that is not CFC2's country of organization. With respect to the contract with FP, USP performs services in Country Y for CFC1 in the form of design and technical services for which CFC1 pays USP \$85x. CFC1, in turn contracts with CFC2 to provide those services and others to CFC2 for \$90x. CFC2 uses those services together with services it performs itself that cost CFC2 \$10x to design the bridge for FP. Pursuant to the cost test, USP provides substantial assistance to CFC2 in the performance of its contract for FP because USP indirectly furnishes services to CFC2 (through CFC1) that exceed 80 percent of the total cost to CFC2 for performing the contract.

\emph{Example 3}: USP, a U.S. corporation, wholly owns CFC1 and CFC2, each a foreign corporation. CFC2 enters into a contract with FP, an unrelated person, to design a bridge in Country Y, a foreign country that is not CFC2's country of organization. With respect to the contract with FP, USP performs services in Country Y for CFC1 in the form of design and technical services for which CFC1 pays USP \$60x. CFC1, in turn contracts with CFC2 to provide those services and others to CFC2 for \$70x. CFC2 uses those services together with services it performs itself that cost CFC2 \$30x to design the bridge for FP. CFC2 is not treated as receiving substantial assistance in the performance of that contract because more than 20\% of the cost of that contract is attributable to services furnished directly by CFC2.

\begin{center}
\textbf{D. EFFECTIVE DATE}
\end{center}
Regulations to be issued incorporating the guidance set forth in this notice will apply to taxable years of foreign corporations beginning on or after January 1, 2007 and to taxable years of United States shareholders in which or with which such taxable years of the foreign corporations end. Until such regulations are issued, taxpayers may rely on this notice.

\end{select}


\addcontentsline{toc}{section}{\protect\numberline{}FBCSerI Problems} 
	\begin{center}
		\textbf{FBCSerI Problems}
	\end{center}
	\begin{select}
	\emph{For the following problems, DC is a U.S. corporation that owns 100\% of CFC1, incorporated in the U.K., and 100\% of CFC2, incorporated in the Cayman Islands.}
			\begin{enumerate}
				\item DC pays CFC2 to install and maintain computers that DC sells to customers in Switzerland.  Does CFC2 have any FBCSerI?   [Reg. \S 1.954-4(b)(3), Example 1.]
				\item DC enters into a contract to build a dam in India and assigns the contract to CFC2.  Does CFC2 have any FBCSerI?   [Reg. \S 1.954-4(b)(3), Example 5.]
				\item CFC2 enters into a contract with FP, an unrelated person, to design a bridge in Brazil. DC contracts to perform services in Brazil for CFC1 in the form of design and technical services for which CFC1 pays DC \$85x. CFC1 contracts with CFC2 to provide those services and others to CFC2 for \$90x. CFC2 uses those services, together with services it performs itself that cost CFC2 \$10x, to design the bridge for FP.   Does CFC2 have any FBCSerI?   [Notice 2007-13, Example 2.]
			\end{enumerate}
		\end{select} 	


\section{Subpart F Income:  Basis Adjustments and Distributions}
				\codereg{959; 960; and 961}{Prop. Reg. 1.959-1, -2, -3 (for future reference only)}


When a U.S. shareholder includes in income either subpart F or an investment in U.S. property under section 956, his basis in increased by the amount included in income.  \S 961(a).  In addition, if a U.S. shareholder has an inclusion under section 951 with respect to an indirectly owned (e.g., a second tier) CFC, the basis in the upper tier CFC is increased and the first tier CFC also increases its basis in the second tier CFC.  \S 961(c).  When an amount that was previously included in income under section 951 is distributed to a U.S. shareholder, the shareholder's basis in the stock of the CFC is reduced.  \S 961(b).

To prevent double taxation of earnings that have been included in a U.S. shareholder's income under subpart F, section 959 excludes from a U.S. shareholder's income any amounts received that can be traced to such previously taxed subpart F income (``PTI'').  In addition, income from a lower tier CFC that was previously taxed under subpart F is not included in the income of the upper tier CFC when it is distributed as subpart F income.  \S 959(b).  The E\&Ps of the upper tier CFC are increased, however, by the amount of the distribution from the lower tier CFC, but the earnings will retain their character as PTI.  When a CFC makes an actual distribution, the earnings distributed are treated as first attributable to retained earnings that were previously included in income as an investment in U.S. property. Then distributions are deemed to come from earnings attributable to subpart F income, and finally, from other earnings and profits.  The earnings attributable to investments in U.S. property and subpart F PTI are referred to as the ``(c)(1)'' and ``(c)(2)'' accounts.  \S 959(c).  Only if a distribution exceeds the (c)(1) and (c)(2) amounts will it be taxable as a dividend under normal corporate income tax rules.  

On August 29, 2006, important proposed regulations under sections 959 and 961 were issued that modify current regulations to reflect statutory changes and address issues that are not addressed under current regulations, such as the treatment of cross-chain stock sales and the distribution of PTI through chains of CFCs. 

\addcontentsline{toc}{section}{\protect\numberline{}Section 959 Problems} 
	\begin{center}
		\textbf{Section 959 Problems}
	\end{center}
	\begin{select}
	\emph{For the following problems, DC is a U.S. corporation that owns 100\% of CFC1, incorporated in the U.K.}
			\begin{enumerate}
				\item In 2007, CFC1 has 100 of subpart F E\&Ps and 100 of non-subpart F E\&Ps.  CFC1 distributes 100 to DC.  What are the tax consequences to DC for 2007?
				\item Same facts as previous questions.  For 2008, CFC1 has 75 of subpart F E\&Ps and 225 of non-subpart F E\&Ps.  CFC1 distributes 250 to DC.  What are the tax consequences to DC for 2008? 
			\end{enumerate}
		\end{select} 	


\section{Investment in U.S. Property}
				\codereg{956}{1.956-1 (skim very lightly) }

If a foreign corporation is a CFC for an uninterrupted period of 30 days, under section 951(a)(1)(A) and (B), its U.S. shareholders must include in income currently their pro rata share of subpart F income \emph{and} their pro rata share of the CFC's earnings invested in U.S. property--``the amount determined under 956."  

A U.S. shareholder's 956 amount is the lesser of: (1) the CFC's average investment (measured quarterly and using the property's adjusted basis) in U.S. property, to the extent that such investment exceeds the CFC's E\&Ps that were previously taxed on that basis; or (2) the CFC's current or accumulated E\&Ps, reduced by distributions during the year and by earnings that have been taxed previously as earnings invested in U.S. property.  Furthermore, a U.S. shareholder will only have an income inclusion to the extent that the amount so calculated exceeds the amount of the CFC's earnings that have been previously taxed as subpart F income.

U.S. property for purposes of section 956 includes tangible property located in the United States, stock of a U.S. corporation, an obligation of a U.S. person, and certain intangible assets including a patent or copyright, an invention, model or design, a secret formula or process or similar property right which is acquired or developed by the CFC for use in the United States.  \S956(c)(1).  

Certain property is excluded from the definition of U.S. property, including:  (1) obligations of the United States and bank deposits; (2) export property; (3) certain trade or business obligations; (4) aircraft, railroad rolling stock, vessels, motor vehicles or containers used in transportation abroad; (5) certain insurance company reserves and unearned premiums related to insurance of foreign risks; (6) stock or debt of unrelated U.S. corporations; (7) moveable property (other than a vessel or aircraft) used for the purpose of exploring, developing, or certain other activities in connection with the ocean waters of the U.S. Continental Shelf; (8) an amount of assets equal to the CFC's accumulated E\&Ps attributable to ECI; (9) property (to the extent provided in regulations) held by a foreign sales corporation and related to its export activities; and (10) obligations of a non-corporate U.S. person that is that is not (i) a U.S. shareholder of the CFC, or (ii) a partnership, estate or trust in which the CFC or any related person is a partner, beneficiary or trustee immediately after the acquisition by the CFC of such obligation.\S956(c)(2).


\addcontentsline{toc}{section}{\protect\numberline{}Section 956 Problems} 
	\begin{center}
		\textbf{Section 956 Problems}
	\end{center}
	\begin{select}
	\emph{For the following problems, DC is a U.S. corporation that owns 100\% of CFC1, incorporated in the U.K.}
			\begin{enumerate}
				\item In 2007, CFC1 has 400 of E\&Ps, no subpart F or current distributions, and invests 200 in U.S. property.  What is DC's 956 amount?
				\item Same facts as previous question, except that CFC1 distributes 50 to DC during the year.
				\item Same facts as question 1, except that CFC1 has 50 of subpart F.
				\item In 2007, CFC1 has 200 of E\&Ps, 100 of subpart F, 20 of current distributions, and invests 50 in U.S. property.
			\end{enumerate}
		\end{select} 	


\section{Section 1248}
				\codereg{1248(a), (c)(1), and (d)(1)}{None}
			
			A U.S. shareholder of a CFC is taxed only on the CFC's subpart F income and investment in U.S. property.  U.S. tax on the business profits of the CFC is deferred until those earnings are remitted to the U.S. parent.  Upon a sale of the stock of the CFC for gain, a U.S. shareholder would generally realize a capital gain.  The untaxed business earnings of the CFC, which would have been taxed at ordinary rates if earned directly by the U.S. could thus be converted into capital gain.  
			
			Under section 1248, gain from the sale of stock of a CFC by a U.S. shareholder is taxed at ordinary rates to the extent of the E\&Ps of the CFC accumulated after 1962 and while the U.S. person held the stock.  For this purposes, a CFC's E\&Ps are calculated under the normal rules but any PTI is excluded.  Section 1248(d)(1).  Corporate USSHs generally prefer dividend treatment because the section 1248 amount would be treated as a dividend for section 902 purposes, that is, the dividend inclusion would bring with it deem paid foreign taxes.  Individual USSHs generally preferred capital gains because of the favorable rate.  Since the tax rate on dividends has been reduced to 15\% for qualified dividends, and in Notice 200-70 the IRS confirmed that section 1248 amounts are eligible to be treated as qualified dividends, section 1248 is not as much a concern as before.
			
			
			\addcontentsline{toc}{section}{\protect\numberline{}Section 1248 Problems} 
	\begin{center}
		\textbf{Section 1248 Problems}
	\end{center}
	\begin{select}
	\emph{For the following problems, DC is a U.S. corporation that owns 100\% of CFC1, incorporated in the U.K.}
			\begin{enumerate}
				\item DC purchases all the stock of CFC1 for \$100 at the beginning of year 1 and sells the stock for \$140 at the beginning of year 3.  For year 1, all of CFC's E\&P's of \$10 are subpart F income.  For year 2, CFC has E\&Ps of \$15, none of which is taxed to DC under section 951(a)(1), and it distributes \$5 to DC.  What is the amount of gain realized by DC? What is the section 1248 amount?
			\end{enumerate}
		\end{select} 	

