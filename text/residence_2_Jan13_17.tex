 
\section{Resident and Nonresident Aliens}
	\crt{2(d); 6851(d); and 7701(b)}{1.871-1(a) and (b); 1.871-2 (skim); 301.7701(b)-2(d) and (f), -3(b)(3), (4), (5), (6), and (7); -3(b)(5), -8(a)(1), and -8(d) (skim)}{Article 4}

A foreign national who is not a U.S. citizen is either taxed on a residence or source basis depending on whether he is a resident or nonresident alien.  Resident aliens are subject to residence basis taxation on their worldwide income, but nonresident aliens (``NRAs'') are subject to source basis taxation.  In particular, NRAs are taxed only on certain limited categories of U.S. source investment income and income that is effectively connected with a U.S. trade or business.  Thus, the foreign source income of nonresident aliens is not taxed by the United States.  

Until 1985, an alien was a U.S. resident if he was physically present in the United States and was ``not a mere transient or sojourner.''  Reg.\@ \S1.871-2(b).  It was not necessary to show that an alien intended to reside permanently in the United Sates--which is closer to the concept of \margit{Residence for gift and estate taxes is determined by an alien's domicile--residence and intention to remain indefinitely.  Reg.\@ \S25.2501-1(b)}domicile--but only that he did not have an actual intention to return at a definite time to another country.  The regulations state that whether an alien was transient is ``determined by his intentions with regard to the length and nature of his stay.''  Ascertaining someone's intentions is never a simple exercise because the available facts are often ambiguous.  Courts and administrators focused on such factors as the alien's length of stay, U.S. and foreign dwelling arrangements, immigration status, family ties in the United States, and U.S. civic and social activity, but these determinations required significant administrative resources.  In addition, the unique factual settings of the cases made it difficult to advise aliens with certainty whether they would be resident aliens. 

To forestall these disputes, Congress in 1984 enacted \S7701(b), which provides bright-line tests based on immigration status or physical presence to determine the tax residence of an alien.  Note, the definition of residence under the 871 regulations still applies in limited circumstances, for example, to determine whether a U.S. citizen is a bona fide resident of a foreign country under \S 911(d)(1)(A).  In addition, some sections have special residence rules that supersede the \S 7701(b) definition, \emph{e.g.}, \S 865(g) (definition of residence for sourcing gains from personal property sales).
%Not all bright line; still significant exceptions based on facts and circumstances

\textbf{Lawfully Admitted for Permanent Residence}.  An alien is a U.S. resident if he is legally entitled to reside in the United States, is physically present in the United States for more than 183 days, or has elected to be a resident alien.  \S7701(b)(1)(A).  An alien is legally entitled to reside permanently in the United States if he has been granted a Permanent Resident Card, better known as a \textit{green card}.  Once secured, permanent residence status continues until rescinded or is administratively or judicially determined to have been abandoned.  \S7701(b)(6); Reg.\@ \S301.7701(b)-1(b).  Thus, even if a green card holder spends no time in the United States, he is taxed on a residence basis.  The residency stating date for a green card holder (who does not otherwise satisfy the substantial presence test) is the first day he is present in the United States while having a green card. \S7701(b)(2)(A)(ii).

 \textbf{Substantial Presence Test}. An alien satisfies the substantial presence test if he is (1) present in the United States more than 31 days during the current calendar year; \textit{and} (2) present 183 days or more during the current and previous two years.  In determining whether the 183-day test is satisfied, each day present in the current year counts as one day; each day present in the preceding year counts as \slashfrac{1}{3} ; and each day present in the second preceding year counts as \slashfrac{1}{6}.  

	\begin{framed}
		\begin{center}
			\textbf{\textsc{Substantial Presence Example}}
		\end{center}
A, a U.K. citizen, is present in the United States for 90 days in 2013; 150 days in 2014; and 120 days in 2015.  For what years does A satisfy the substantial presence test? 

A is not a resident alien for 2013 because he is present for only 90 days.  A is also not a resident alien in 2014 because he is present for only 180 days:  150 (2014) + 90 $\times$ \slashfrac{1}{3} (2013).  A is a resident alien in 2015 because he is present for 185 days, determined as follows: 
 \begin{center}
   \begin{tabular}{l c c c}
  & (a) & (b) & (a) $\times$  (b)\\
  Year & Days in US & Weight & Counted Days\\
  \hline
  2015 & 120 & 1 & 120 \\
  2014 & 150 & \slashfrac{1}{3} & 50 \\
  2013 & 90 &\slashfrac{1}{6} & 15 \\
    \hline
  & & \textbf{Total}&\textbf{185}\\
    \end{tabular}
   \end{center}

	\end{framed}

%\textbf{First Year Election} (b4) election year (not a resident, but at least 31 days in US in election year and other minimum presence requirements) and satisfies sub. presence test in subsequent year.
%	What's the greatest number of level days that can be spent in US without triggering US residence?  Stupid Tax trick:  how much can each day potentially contribute 1.5.  So can spend 121 each year w/out being resident.  What if 122 = 183 too bad.
%	Thus, can even be resident even if never spend more than 183 days in the US.    
 
 \textbf{Closer Connection Exception}. An important goal of Congress in amending the definition of resident alien was to provide bright-line rules for determining an alien's U.S. tax residence.  Congress retained some elements of the prior regime that require a facts and circumstances determination.  An alien that otherwise satisfies the substantial presence test but is here for less than 183 days in the \emph{current year} can be treated as a nonresident, provided the alien has a  \emph{foreign tax home} and a \emph{closer connection} to the foreign country.  \S7701(b)(3)(B).  The definition of tax home for purposes of the closer connection is same as under \S162, and the regulations clarify that a tax home is ``located at an individual's regular or principal (if more than one regular) place of business.''  Reg.\@ \S301.7701(b)-2(c)(1).  An alien without a regular or principal place of business or who is not engaged in a trade or business has a tax home at his ``regular place of abode in a real and substantial sense.''  \emph{Id}.  
 
The regulations also provide a non-exhaustive list of factors to be considered in determining whether an alien has a closer connection to a foreign country.  Some of the facts and circumstances are the location of the alien's permanent home, the location of family and personal belongings, the location where the alien conducts his routine personal banking activities, the jurisdiction in which the individual votes and holds a driver's license, and the country of residence indicated on forms and documents.   Reg.\@ \S301.7701(b)-2(d)(1).  These factors are similar to those used by courts under the pre-1985 definition of resident alien.  For a well-heeled alien who has the flexibility to establish a firm economic connection to one country but who's required to spend time in another, it should not be too difficult to follow the road map of the regulations and adjust his economic arrangements to be fairly certain that he has a closer connection to a foreign country.       

	\begin{framed}	
		\begin{center}
			\textsc{\textbf{Closer Connection Exception}}
		\end{center}
Same facts as previous example.  For what years can A potentially claim a closer connection to the United Kingdom?  

Since A is not a resident alien for either 2013 or 2014, the closer connection exception does not apply.  A is a resident alien in 2015 under the substantial presence test, but as he is present for fewer than 183 days in 2015, he is potentially eligible for the closer connection example.  Whether he has a closer connection to the United Kingdom will depend on the location of his tax home and U.K. connections.
	\end{framed}
	
	\textbf{Days of Presence}.  In applying the substantial presence test, an alien must generally count each day of presence in the United States.  For certain categories of aliens, referred to in the statute as \emph{exempt individuals}, their days of presence in the United States do not count under the substantial presence test.  Therefore, provided an exempt individual does not have a green card, he will not be a resident alien.  Exempt individuals include diplomats and full-time employees of international organizations such as the Inter-American Investment Corporation, the International Committee of the Red Cross, and the International Cotton Advisory Committee.  Also covered are students, teachers, and trainees.  

To prevent a wealthy, bon vivant cafe habitu\'{e} from cloaking himself in student status, the regulations limit teacher and student status to holders of the appropriate visa, which include F, J, M, and Q visas, and require \textit{substantial compliance} with the terms of the visa.   \S7701(b)(5)(C)(ii); Reg.\@ \S301.7701(b)-3(b)(2), (3), and (4).  In addition, teachers and trainees cannot exclude days of presence if they have been exempt as teacher, trainee, or student for any part of two of the preceding six years.  \S7701(b)(5)(E); Reg.\@ \S301.7701(b)-3(b)(7)(i).  A student is limited generally to five years of exemption unless he can demonstrate that he does not intend to reside permanently in the United States.  \S7701(b)(5)(E)(ii); Reg.\@ \S301.7701(b)-3(b)(7)(iii).   

Regular commuters from Canada and Mexico are also exempt individuals.  \S7701(b)(7)(B);  Reg.\@ \S301.7701(b)-3(d).  A regular commuter is one that commutes from Canada or Mexico on more than 75\% of the \emph{workdays} during the \emph{working period}, terms that are fleshed out in the regulations.  Reg.\@ \S301.7701(b)-3(e)(1) and (2).  A person present in the United States who is in transit between two foreign countries is not treated as present, provided that he is here for less than 24 hours and does not undertake any activities connected with the United States, such as having a business meeting.  \S7701(b)(7)(C); Reg.\@ \S301.7701(b)-3(d).   

One curious exception is for professional athletes who compete in \emph{charitable sports events}. \S7701(b)(5)(A)(iv).  At first glance, it is unclear who would benefit from such an exclusion.  One potential category would be athletes that come to compete in international competitions such as the Olympics or World Cup held in the United States, but such competitions are rare and generally of such short duration that it is unlikely an athlete would otherwise even come close to satisfying the substantial presence test. Digging a bit deeper, one discovers that the intended recipients of this statutory largesse were professional golfers who compete in golf tournaments organized as charitable events.\footnote{Most PGA tournaments are set up as charities, but apparently very little of the gross receipts (about 16\%) are spent on charitable activities. \emph{See} \url{http://es.pn/1hRmzkP}.   Shocking.}  Interpreted liberally, this exception would allow professional golfers to live and compete in the United States, but not pay tax on their worldwide income.  Of course, any winnings from U.S. golf tournaments and other U.S. source income, such as fees for promotions would certainly be taxed by the United States, but their foreign source income and all investment income would be exempt.  The regulations, however, largely eviscerate this exception by limiting it to only days spent competing and not days spent preparing, promoting, or traveling.    Reg.\@ \S301.7701(b)-3(b)(5).


%resident regardless of illegal
\textbf{Beginning and Ending of Resident Alien Status}.  The day that residence alien status begins determines when an alien ceases to be taxed on a source basis taxation and begins to be taxed on a residence basis.  For the year during which an alien becomes a resident alien (or ceases to be a resident alien), the alien's taxable year is bifurcated, and he is taxed on a source basis while a nonresident and on a residence basis while a resident. Reg.\@ \S1.871-13(a)(1).  The U.S. tax consequences to a person receiving income or paying an expense are determined based ``on the status of the foreign taxpayer at the time of receipt or payment." \emph{Id.}  

An alien's residency starting date generally depends on how resident alien status is acquired.  If a resident alien has a green card, the residency starting date is the first day of presence in the United States while a green card holder.  \S7701(b)(2)(A)(ii).  If an alien satisfies the substantial presence test, the residency starting period begins on the first day of U.S. presence.  \S7701(b)(2)(A)(iii).  For an alien who had a green card and also satisfies the substantial presence test, the residency starting date is the earlier of the two dates.  Reg.\@ \S301.7701(b)-4(a).  If an alien satisfies \margit{Although certain days are excluded for purposes of the residency starting date, they count for calculating substantial presence.  Reg.\@ \S301.7701(b)-4(c)(1).} the substantial presence test, he may exclude up to 10 days of presence in the United States in determining his residency starting date if he can show a foreign tax home and closer connection to a foreign country.  \S7701(b)(2)(C).  The exception is of interest to an alien who is planning to become a U.S. resident and in anticipation of the move comes to the United States, for example, on house hunting trips.

If a resident alien is a resident alien during the current year but is not for the following year, his residency termination date is the generally the last day of the calendar year. Reg.\@ \S301.7701(b)-4(b)(1).  If, however, he can show a foreign tax home and closer connection to the foreign country than to the United States, the residence termination date is the last day of presence in the United States.  Reg.\@ \S301.7701(b)-4(b)(2). 

\addcontentsline{toc}{section}{\protect\numberline{}Comments}
	\begin{center}
			\emph{\textbf{Comments}}
	\end{center}

\begin{enumerate}
	\item
	In \textit{Topsnik v. CIR}, 143 T.C. No. 12, 2014, Topsnik, a German citizen and a resident alien (he had a green card), sold stock in 2004 on an installment basis, with the installments to be paid over the next 5 years.  Because Topsnik did not file returns for 2006-2009, the IRS filed substitute returns on his behalf.  Topsnik eventually filed returns for those years claiming that he was no longer a resident alien, and even if he were, he owned no tax because he was a German resident under the U.S.-German treaty, which prohibits the taxation of capital gains by the source country.  The Tax Court found that Topsnik continued to be a resident alien until 2010 when he filed a Form I-407 and surrendered his green card as required by Reg.\@ \S301.7701(b)-1(b)(3).  Even though U.S. immigration law permits the informal abandonment of permanent resident status, the tax law's more specific rules take precedence, and Topsnik had not satisfied those rules.  The Court also rejected Topsnik's treaty claims on the basis that Topsnik was not a resident under the German treaty because he was not subject to tax on his worldwide income and had no habitual residence or domicile in Germany, 
	
	\item In \textit{Diran Li v.\@ CIR}, T.\@ C.\@  Summ.\@ Op.\@ 2016-49 (2016), Li, a Canadian citizen and resident, attempted to claim education credits against his U.S. wage income.  In 2012, Li was present in the U.\@ S.\@ from Feb. 22 to 24 and March 15 to 17 for job interviews and eventually accepted an offer from Microsoft beginning on July 1, 2012.  Li filed a 1040 for 2012.  The Tax Court found that under section 7701(b)(2)(A)(iii) Li's starting date for his U.S. residency in 2012 was the first day he was present in the United States, Feb. 22.  Consequently, since Li was a nonresident alien for part of the year, he was not eligible for the education credits pursuant to section 25A(g)(7).  	
	
	\item
					 
		\textbf{\emph{Pre-Immigration Tax Planning}}.  For an alien with few assets and who derives most of his income as wages where he resides, there may be very little difference between source basis and residence basis taxation: if he performs services here, his service income will generally be taxed at graduated rates whether he is taxed on a source or residence basis.  For an alien owning appreciated or depreciated property, however, a change in tax status from nonresident to resident will subject gains (and losses) to U.S. residence taxation when they were hithertofore subject only to source basis taxation.  Changing tax status thus presents tax planning challenges and opportunities for the peripatetic alien.    

\begin{framed}
	\begin{center}
		\textsc{\textbf{Practice Note:  Pre-Immigration Tax Planning}} 
	\end{center}
Because an alien's tax status at the time of receipt of income generally determines how the income is taxed, it is generally advisable to accelerate any foreign source income before becoming a resident alien.  For example, if an alien is entitled to compensation that is attributable to services performed abroad and the income will \textit{not} be subject to U.S. tax if it is received before becoming a resident alien it, but will be taxed by the United States if it is received after becoming a resident alien, the income should be accelerated.  Of course, the foreign tax consequences of accelerating income will also have to been considered.  Sometimes it is possible for income not to be taxed anywhere.  For example, if ending date of foreign tax residence does not coincide with the beginning date of U.S. tax residence, income received between the two dates may not be taxed by either country.    
\end{framed}

A change in tax status from nonresident alien to resident alien generally has no effect for U.S. tax purposes on unrealized gains or losses.  Consequently, if property is sold while a foreign national is a resident but was purchased before he became a resident, the property's basis for computing gain or loss must be determined.  In general, the property's basis is determined as if the taxpayer and property had always been subject to U.S. tax.   \margit{The look on a client's face when you inform of this rule--after you inform him of the scope of residence basis taxation--is similar to the one seen on a person receiving a sharp unexpected blow to the solar plexis.}  This requires that the property's tax history be recreated under U.S. tax principles and generally using U.S. dollars.  One unpleasant consequence of this rule is that an alien can have purchased property in foreign currency that has fallen in value in terms of the foreign currency, but if the foreign currency has appreciated vis-a-vis the dollar since the property was purchased, a sale of the property can result in taxable gain.  

	\begin{framed}
		\begin{center}
			\textbf{\textsc{Historical U.S. Dollar Basis for Property}}
		\end{center}
A, a Spanish resident and citizen, purchases a house for 1 million euros when the exchange rate is \$1 = 1.17 euros.   In 2008, A moves to the U.S. and becomes a resident alien.  He sells the property for 950,000 euros on July 15, 2008, but since the euro-dollar exchange rate is now \$1 = 0.6280 euros, the dollar value of his house has increased from \$854,700 to \$1,512,738.  A will not be too pleased when you inform him that he has a taxable gain of \$658,038.  He will rightly feel that he has suffered an economic loss of 50,000 euros.
	\end{framed}
	
Furthermore, although simple to state in principle, the historical U.S. tax basis rule can be very complicated to apply in practice.  It can be straightforward to recreate the basis of property in certain cases, for example, the basis of a share of stock or piece of land.  There is no clear guidance, however, on how to take into account adjustments such as depreciation and certain elections that could have been made had the person and property been subject to U.S. tax jurisdiction.  In addition, the proper method to adjust for changes in the value of foreign currency is not clear for business property.\footnote{For a more detailed discussion of these issues, see Jeffrey M. Col\'on, \textit{Changing U.S. Tax Jurisdiction:  Expatriates, Immigrants and the Need for a Coherent Tax Policy}, 24 San Diego L. Rev. 1, 60-87 (1997), Jasper L. Cummings, \textit{Determining Basis and Other Tax Items of Foreigners}, 151 Tax Notes 479 (Ap. 25, 2016) }

 
\end{enumerate}

%question:  why not mtm?  see 877
\section{Dual Residents}
Because the definition of resident varies among countries, it is possible for a person to be a resident of more than one country.  A dual resident is subject to U.S. tax on a residence basis unless a treaty applies to treat the resident alien as a resident of the treaty country instead of a resident of the United States.  For dual residents, the specter of double taxation looms large unless one or both of the countries gives a credit for taxes levied by the other country.  The U.S. foreign tax credit regime may ameliorate but not eliminate double taxation that can arise when two countries assert residence basis taxation.  For example, if a dual resident performs services in the United States, but is also taxed by another country on those services, the U.S. foreign tax credit mechanism may be insufficient to relieve double taxation.   

	\begin{framed}
		\begin{center}
			\textbf{\textsc{Double Taxation}}
		\end{center}
A, a U.K. national, is a resident under the domestic laws of the United States and the United Kingdom.  A earns \$100,000 for services performed in the United States and is taxed at a marginal tax rate of 35\% by both countries.  Because the services are performed in the United States, A cannot credit U.K. taxes paid against his U.S. tax liability.  If A cannot credit or deduct either U.S. taxes paid against his U.K. taxes or U.K. taxes against his U.S. tax liability, he could end up being subject to a marginal tax rate of 70\%.
		\end{framed}

Tax treaties attempt to prevent double taxation on a residence basis by providing a single residence for treaty purposes.  In general, a person is a resident for purposes of the Treaty if he is liable to tax by reason of his ``domicile [or] residence\ldots.'' Article 4(1).  Thus, an alien who is resident under section 7701(b) would generally be a resident under the Treaty.  Greencard holders, like citizens, are treaty residents only if they have a substantial presence, permanent home or habitual abode in the United States \emph{and} they are not residents under any treaty between the United Kingdom and another country.  Article 4(2).  The Technical Explanation to Article 4 states that substantial presence under the Treaty has the same meaning it does under section 7701(b)(3).  If a U.S. resident alien is also a U.K. resident, the tie-breaker tests of Article 4(4) will apply to determine a single residence for Treaty purposes.  These tests are discussed above in Chapter 2.2.

If a U.S. resident alien is a dual resident, but  a U.K. resident under Article 4(4), he will be a U.K. resident for all purposes of the Treaty, including the savings clause.  Consequently, the person would  be subject to U.S. tax only as permitted by the Treaty.  Note, however, that if a dual resident is a U.K. resident under the Treaty, he is surprisingly still treated as a U.S. person for other purposes of the Code, such as reporting foreign bank accounts and foreign stock ownership requirements, which may have sometimes negative tax consequences to other U.S. persons.  \emph{See} Reg.\@ \S301.7701(b)-7(a)(3).          


\addcontentsline{toc}{section}{\protect\numberline{}Comments}
	\begin{center}
			\textbf{\emph{Comments}}
					\end{center}

	\begin{enumerate}
		\item Determining a resident's center of vital interest for treaty purposes can require a detailed factual analysis.  In, \emph{Elliott v.\@ The Queen}, Tax Ct. No. 2010-898 (IT)G (Feb. 21, 2013), the Canadian Tax Court addressed the treaty residence of three U.S. citizens who lived and worked in Canada as consultants for two years.  Relaying on the OECD Commentaries to the OECD Model Convention--the Technical Explanation wasn't helpful--the tax court found that consultants' rented apartments constituted a permanent home under the treaty, and thus they had permanent homes in both countries.  The court then addressed to which country the consultants' personal and economic relations were closer, that is, their center of vital interests.  Finding that the consultants had only lived in Canada while they were fulfilling their contractual duties and left when the work was concluded, maintained all pre-existing ties to the United States, such as bank accounts, cars, cell phones, health insurance, investments, families, driver's licenses, the tax court found that the United States was their center of vital interest.
		
		\item Congress should consider allowing or requiring foreigners who become resident aliens to adjust the basis of their foreign property to fair market value.  This would prevent the importation of unrealized losses to use against U.S. income and treat foreigners with illiquid assets, such as stock of a closely-held corporation, similarly to how foreigners holding liquid assets are treated--a foreigner holding liquid assets can purge pre-immigration gain merely by selling and repurchasing the assets.  Long-term resident aliens who give up their resident alien status and are subject to U.S. tax on their unrealized gains under \S877A may elect to step up the basis of any property held upon becoming a resident alien to its fair market value.  \S877A(h)(2).  Query why the basis of property with an unrealized loss is not required to be adjusted.  \emph{See also} \S362(e)(1) (requiring the basis of property with a built-in loss imported into U.S. tax jurisdiction by a corporation as a tax-free contribution to capital or in a reorganization to be adjusted to its fair market value).
		
		\item Section 6114(a) generally requires a taxpayer to disclose when a treaty overrides a Code provision, but certain exceptions are provided for in regulations.  The disclosure is made on Form 8833.  The position that a taxpayer's residency is determined under a treaty and not the Code specifically must be disclosed.  \emph{See also} Reg.\@ \S 301.7701(b)-7 (detailing filing requirements for dual residents asserting treaty benefits); Reg.\@ \S 301.6114-1(b)(8).  Failure to disclose can result in penalties.  \emph{See} \S6712(a).

		\item One of silliest sections of the Code that applies to foreigners is section 6851(d), which requires aliens departing from the United States to first obtain a certificate of compliance with U.S. tax law, which is known as a \emph{sailing permit}. The regulations mercifully exempt students, diplomats, and their families, but every other alien, including a resident alien, is potentially caught in the sailing permit web.  If a reader knows of any person who has complied with this rule, please let the author know.

	\end{enumerate}

\begin{framed}
	Last revised Jan. 13, 2017; residence\_2\_Jan13\_17
	\end{framed}
	
