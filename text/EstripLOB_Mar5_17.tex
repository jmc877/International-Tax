\chapter{Treaty Shopping, Conduit Financing, Limitation of Benefits and Earnings Stripping}



\section{Treaty Shopping and Conduit Financing Regulations} 
\crt{7701(l)}{1.881-3}{Articles 3(l)(n), 10(9), 11(7), and 12(5)}

Treaty shopping refers to the tax planning strategy of investing or doing business in a source country through an entity formed in a third country that is entitled to treaty benefits with the goal of reducing or eliminating source country taxation on payments of interest, dividends, or royalties to the intermediate treaty entity.  Assume a resident of a country that does not have a treaty with the United States forms a U.K. entity to own the shares of a U.S. corporation.  Depending on the U.K. entity's level of ownership of the U.S. corporation, the dividend rate on dividends paid by the U.S. corporation would be lowered from 30\% to 15\%, 5\%, or 0\%.  If the cash that is paid to the U.K. entity could then be distributed to the third-country owners with little or no U.K. tax, the tax efficiency of such a structure is self evident.  

Treaty shopping can also be tax efficient even if all parties are residents of different treaty countries and the withholding tax rates are not the same under all of the treaties.  For example, if the Country A-U.S. treaty provides for a 0\% tax on royalties, the Country B-U.S. treaty provides for 10\% rate, and the Country A-Country B treaty provides for a 0\% rate, it may be worthwhile for Country B residents who wish to invest in the United States and plan to license intellectual property to their U.S. business, to consider licensing the property through a Country A entity that will, in turn, license to the U.S. business.  This could lower the effective rate on U.S. royalties from 10\% to 0\%.

Under older treaties, one could gain treaty benefits merely by incorporating an entity in one of the treaty countries.  As long as the country of incorporation had a relatively benign tax regime applicable to foreign income, there would not be a significant domestic tax cost to incorporating.  Because older treaties did not specifically limit treaty benefits for entities formed in a treaty country but owned by residents of third countries, the U.S. initially attacked treaty shopping in the courts.  

In \emph{Aiken v.\@ CIR}, 56 T.C. 925 (1971), a Bahamian parent corporation had made a loan to its second-tier U.S. subsidiary, but because there was no treaty between the U.S. and the Bahamas, the interest would have been subject to a 30\% tax.  To avoid the tax, the Bahamain corporation transferred the note of the U.S. subsidiary to its second-tier Honduran corporation in exchange for nine notes with a total face amount equal to the U.S. note and an identical interest rate.  After the dust settled, the U.S. subsidiary paid interest on the the original note to Honduran corporation, which in turn, paid an identical amount of interest to the Bahamian parent.  At this time, the U.S. had an income tax treaty with Honduras.  Although the Honduras corporation was an Honduras Enterprise (resident) under the treaty, the Tax Court found that the interest was not ``received'' by the Honduras corporation on the grounds that the Honduras corporation was a mere conduit for the interest that went from the U.S. subsidiary to the Bahamian parent.  In addition, the court found that because the inflows and outflows were identical--the Honduras corporation made no profit--there was no valid economic or business purpose for the transaction.  The limits of Aiken decision as a tool to combat treaty shopping, however, can be seen below in \emph{Northern Indiana v.\@ CIR}, 115 F.3d 506 (7th Cir. 1997), where the court held that Aiken did not apply to a back-to-back loan structure where the intermediate entity retained a profit spread of 1\%.    

Following its victory in Aiken, the U.S. embarked on a multi-pronged attack on treaty shopping.  First, it began to insist on detailed limitation on benefits articles in U.S. treaties that incorporated base erosion and ownership requirements.  \emph{See, e.g.}, UK Treaty, Art. 23.  Second, it terminated and renegotiated treaties with tax havens, such as the Netherlands Antilles.  Third, Congress enacted section 7701(l), discussed below, which gave the Treasury authority to promulgate the conduit financing regulations under Reg.\@ \S1.881-3.  Fourth, Congress enacted section 163(j), discussed below.  And fifth, the Treasury promulgated detailed regulations that deny treaty benefits for income paid to certain hybrid entities under section 894(c).
         
\addcontentsline{toc}{section}{\protect\numberline{}Northern Indiana v.\@ CIR} \begin{select}
\caseart{Northern Indiana v.\@ CIR}{115 F.3d 506 (7th Cir. 1997)}{Judges Bauer, Wood, and Coffey} 
\ldots 
\begin{center} BACKGROUND
\end{center}
Northern Indiana Public Service Company (``Taxpayer'') is a domestic public utility company. In 1981, Taxpayer formed a foreign subsidiary corporation, Northern Indiana Public Service Finance N.V. (``Finance''), in the Netherlands Antilles. Finance was organized for the purpose of obtaining funds so that Taxpayer could construct additions to its utility properties. To accomplish this, Finance issued notes in the Eurobond market and then lent the proceeds to Taxpayer.\ldots 

Taxpayer's use of a Netherlands Antilles subsidiary to borrow funds in the European market was a financially-strategic measure. During the early 1980s, domestic interest rates hovered around twenty percent. To circumvent the high interest rates, United States companies turned to foreign investors. By using a Netherlands Antilles subsidiary to borrow funds in the European market, United States companies were able to obtain tax advantages not available through direct borrowing in that market.  \ldots  However, at the time the transactions in this case occurred, interest payments by a United States corporation to a Netherlands Antilles corporation were exempt from withholding tax pursuant to Article VIII of the United States-Netherlands [Treaty].

On October 15, 1981, Finance issued \$70 million worth of notes in the Eurobond market (``the Euronotes''), at an annual interest rate of 17.25 percent. Taxpayer unconditionally guaranteed timely payment of the interest and principal on the Euronotes. Also on October 15, 1981, Taxpayer issued to Finance a \$70 million note (``the Note''), bearing annual interest of 18.25 percent. In exchange, Finance remitted to Taxpayer \$68,525,000-the net proceeds of the Euronote offering. The Euronotes and the Note had the same maturity date of October 15, 1988 and contained the same early payment penalty provisions.

In 1982, 1983, 1984 and 1985, respectively, Finance received from Taxpayer interest payments of \$12,775,000, which Finance deposited in its corporate bank account. In each of those years, Finance made interest payments of \$12,075,000 to the Euronote holders. The spread created by this borrowing and lending yielded Finance an annual profit of \$700,000 (an aggregate of \$2,800,000 for the four years). Finance invested this income to earn additional interest income. Taxpayer did not withhold any United States tax on its payments to Finance.

On October 10, 1985, Taxpayer repaid the principal amount of the Note (\$70 million), plus accrued interest (\$12,775,000) and an early payment penalty (\$1,050,000) to Finance. On October 15, 1985, Finance redeemed the Euronotes by repaying the principal (\$70 million), together with accrued interest (\$12,075,000), and an early payment penalty (\$1,050,000). Finance was liquidated on September 22, 1986, and its assets were distributed to Taxpayer.

 \ldots

In an opinion dated November 6, 1995, the Tax Court held that Taxpayer was not liable for the alleged deficiencies. The Tax Court determined that Finance was recognizable for tax purposes because it ``engaged in the business activity of borrowing and lending money at a profit,'' and that, therefore, Taxpayer's interest payments to Finance fell within the terms of the Treaty and were exempt from United States taxation.\ldots 

\begin{center} ANALYSIS
\end{center}

Under the terms of the Treaty, interest on a note that is ``derived from'' a United States corporation by a Netherlands corporation is exempt from United States taxation. The question presented to the Tax Court was whether Finance and its transactions with Taxpayer were recognizable for tax purposes, making Taxpayer's interest payments to Finance subject to the Treaty. The Tax Court determined that Taxpayer's interest payments should be recognized as having been paid to Finance, rather than directly to the Euronote holders.  \ldots

The Commissioner has suggested that Taxpayer's tax-avoidance motive in creating Finance might provide one possible basis for disregarding the interest transactions between Taxpayer and Finance. The parties agree that Taxpayer formed Finance to access the Eurobond market because, in the early 1980s, prevailing market conditions made the overall cost of borrowing abroad less than the cost of borrowing domestically. It is also undisputed that Taxpayer structured its transactions with Finance in order to obtain a tax benefit-specifically, to avoid the thirty-percent withholding tax. What is in dispute is the legal significance of Taxpayer's tax-avoidance motive.

A tax-avoidance motive is not inherently fatal to a transaction. A taxpayer has a legal right to conduct his business so as to decrease (or altogether avoid) the amount of what otherwise would be his taxes.  \ldots However, the form the taxpayer chooses for conducting business that results in tax-avoidance ``must be a viable business entity, that is, it must have been formed for a substantial business purpose or actually engage in substantive business activity.'' \ldots This rule ensures that ``what was done, apart from the tax motive, was the thing which the [treaty] intended.'' Gregory, 293 U.S at 469, 55 S.Ct.\@ at 267.

The Tax Court relied on a line of cases for the principle that so long as a foreign subsidiary conducts substantive business activity-even minimal activity-the subsidiary will not be disregarded for federal tax purposes, notwithstanding the fact that the subsidiary was created with a view to reducing taxes.  \ldots These cases involve domestic corporations which attempted to avoid taxes by creating subsidiaries-foreign subsidiaries in the majority of the cases-which conducted some transactions solely for tax-avoidance and other transactions which were not tax-motivated.

The Commissioner insists that these cases are inapposite to the present case because they involve the issue of whether income earned by a subsidiary should be allocated to its parent company on the ground that the subsidiary was a ``sham.''  The Commissioner has abandoned any argument on appeal that Finance was a ``conduit'' or ``agent'' of Taxpayer. Those buzzwords which we generally use to describe a ``sham'' corporation are absent from the Commissioner's briefs. The Commissioner argues that it was error for the Tax Court to rely on the above-cited cases because the issue here is not whether Finance is properly characterized as a ``sham,'' but, rather, whether the transactions between Finance and Taxpayer should be disregarded for tax purposes. The Commissioner urges that we look solely at the interest transactions between Taxpayer and Finance without concerning ourselves with Finance's legitimacy or its other economic activities.

To bolster her argument, the Commissioner cites Knetsch v.\@ United States, 364 U.S. 361, 81 S.Ct.\@ 132, 5 L.Ed.2d 128 (1960), and a line of captive insurance company cases for the propositions that even legitimate corporations may engage in transactions lacking economic substance and that the Commissioner may disregard transactions between related legitimate corporations.\ldots In Knetsch, the taxpayer borrowed money at a certain interest rate and used the loan proceeds to buy an annuity bearing a lower interest rate. The transaction was unrelated to any business or other economic activity, but was designed solely to generate large interest deductions. The Supreme Court affirmed the Tax Court's denial of the taxpayer's claimed interest expense deduction for the transaction because the transaction did not engender ``indebtedness'' for federal tax purposes. In addition, in each of the captive insurance company cases cited by the Commissioner, claimed business expense deductions for purported insurance transactions between a parent corporation and its wholly-owned legitimate captive insurance subsidiary were denied on the ground that the transactions did not constitute ``insurance'' for federal tax purposes.

The Commissioner's argument is creative, but unpersuasive. Regardless of the words the Commissioner uses to make her argument, in substance, the Commissioner is asking us to disregard Finance and to deem the interest payments made by Taxpayer as having gone directly to the Euronote holders. We are looking at the interest transactions and not deciding whether Finance was a ``sham.'' However, it is unnecessary, and we think inappropriate, for us to sever a corporation from its transactions in analyzing a case, such as this one, where the corporation was formed with the intent of structuring its economic transactions to take advantage of laws that afford tax savings. Finance's existence, its interest transactions with Taxpayer and its other economic activities are all relevant to our analysis. Moreover, Knetsch and the captive insurance company cases do not dictate the outcome the Commissioner desires. Those cases allow the Commissioner to disregard transactions which are designed to manipulate the Tax Code so as to create artificial tax deductions. They do not allow the Commissioner to disregard economic transactions, such as the transactions in this case, which result in actual, non-tax-related changes in economic position.

All of this is to say that the Tax Court was entitled to rely on Moline Properties, Hospital Corp., Ross Glove, Bass and Nat Harrison. These cases engender the principle that a corporation and the form of its transactions are recognizable for tax purposes, despite any tax-avoidance motive, so long as the corporation engages in bona fide economically-based business transactions. The Commissioner insists that Taxpayer cannot seek refuge in this maxim because Taxpayer's desire to avoid the thirty-percent withholding tax was its sole purpose in transacting business with Finance and because Finance engaged in no meaningful economic activity. We disagree. ``Whether a corporation is carrying on sufficient business activity to require its recognition as a separate entity for tax purposes is a question of fact and [Taxpayer] had the burden of proof.'' Bass, 50 T.C. at 602. The Tax Court determined that Taxpayer met its burden, finding that ``Finance engaged in the business activity of borrowing and lending money at a profit....''

The Commissioner relies on two cases in its attempt to show that Finance engaged in no meaningful economic activity. The first is Gregory v.\@ Helvering, 293 U.S. 465, 55 S.Ct.\@ 266, 79 L.Ed. 596 (1935). In Gregory, the Supreme Court disregarded a corporation which was created for the sole purpose of receiving passive assets and distributing its stock in a purported reorganization. The corporation was liquidated six days after it was formed.  \ldots The Supreme Court ruled that the distribution was not made ``in pursuance of a plan of reorganization,'' as the statute required, because it was ``an operation having no business or corporate purpose....'' Id. at 469, 55 S.Ct.\@ at 267.   \ldots

The second case the Commissioner relies on is Aiken Industries, Inc. v.\@ Commissioner, 56 T.C. 925, 1971 WL 2486 (1971). In that case, a domestic corporation borrowed \$2,250,000, at an interest rate of four percent, from a Bahamian corporation. The Bahamian corporation owned 99.997 percent of the domestic corporation's parent company, also a domestic corporation. The Bahamian corporation's wholly owned Ecuadorian subsidiary incorporated a company in the Republic of Honduras. The Bahamian corporation assigned the domestic corporation's note to the Honduran corporation in exchange for nine promissory notes (\$250,000 each), which totaled \$2,250,000 and bore interest of four percent. Because of this assignment, the domestic corporation made its four-percent interest payments to the Honduran corporation, which, in turn, made its four-percent interest payments to the Bahamian corporation. Prior to the assignment, the domestic corporation's interest payments to the Bahamian corporation would have been subject to the withholding provisions of \S 1441. But after the assignment, because there was an income tax treaty between the United States and the Republic of Honduras, the domestic corporation claimed exemption from the withholding provisions. The Tax Court held that the corporate existence of the Honduran corporation could not be disregarded. It also held, however, that the interest payments in issue were not ``received by'' the Honduran corporation within the meaning of the United States-Honduras Income Tax Treaty, because the Honduran corporation lacked dominion and control over the interest payments.

From Gregory and Aiken Industries, we glean the following: Transactions involving a foreign corporation are to be disregarded for lack of meaningful economic activity if the corporation is merely transitory, engaging in absolutely no business activity for profit-in other words, it is a ``mere skeleton.'' See Bass, 50 T.C. at 602 n. 3. Transactions will also be disregarded if the foreign corporation lacks dominion and control over the interest payments it collects.

In this case, Finance was set up to obtain capital at the lowest possible interest rates. Accessing the Eurobond market through a Netherlands Antilles subsidiary was not, at the time, an uncommon practice to accomplish this end. The record demonstrates that Finance ``was managed as a viable concern, and not as simply a lifeless facade.'' See id. at 602. Finance conducted recognizable business activity-concededly minimal activity, but business activity nonetheless. Significantly, Finance derived a profit. It earned income on the spread between the interest rate it charged Taxpayer on the Note (18.25 percent) and the rate it paid to the Euronote holders (17.25 percent). The foreign corporation in Aiken Industries was held to lack dominion and control because, unlike Finance, it was literally a mere conduit, earning no profit on its borrowing and lending activities. \ldots

By contrast, Finance netted an annual \$700,000 from its borrowing and lending activities. That income stream had economic substance to both Taxpayer and Finance. Each time Taxpayer made an interest payment to Finance, Taxpayer's economic resources were diminished while Finance's economic position was enhanced. Finance also reinvested the annual \$700,000 interest income in order to generate additional interest income. Taxpayer had no control over Finance's reinvestments. Finally, the transactions in Aiken Industries were entirely between related parties. Finance, on the other hand, borrowed funds from unrelated third parties, the Euronote holders.

Relying again on Knetsch, the Commissioner argues that the income Finance earned on the transactions with Taxpayer is irrelevant; that a transaction does not necessarily have economic substance for tax purposes merely because one party profits from the arrangement. The Commissioner characterizes the one-percent profit Finance earned from the spread created by its borrowing and lending activities as a ``fee'' for accommodating Taxpayer in the Eurobond offering. The Commissioner's argument misses the mark. As we explained supra, the transaction in Knetsch was unrelated to any economic activity. The taxpayer paid money solely to obtain tax deductions and did not intend to profit in a true sense, as evidenced by the fact that the pre-tax interest outlay would be greater than the pre-tax interest received. Here, a profit motive existed from the start. Each time an interest transaction occurred, Finance made money and Taxpayer lost money. Moreover, Finance reinvested the annual \$700,000 interest income it netted on the spread in order to generate additional interest income, and none of the profits from these reinvestments are related to Taxpayer.

Looking at the record as a whole, we find that the Tax Court did not clearly err by determining that Finance carried on sufficient business activity so as to require recognition of its interest transactions with Taxpayer for tax purposes. That being so, it is unnecessary to address Taxpayer's cross-appeal. The judgment of the Tax Court is AFFIRMED.

\end{select}

\addcontentsline{toc}{section}{\protect\numberline{}Conduit Financing Regulations} 
	\begin{center}
		\textbf{Conduit Financing Regulations}
	\end{center}

A conduit arrangement generally refers to an ownership structure consisting of a parent corporation that invests indirectly in the United States by using an entity formed in a third country.  The structures in \emph{Aiken} and \emph{Northern Indiana} are conduit structures.  The goal of interposing the third-country corporation is to reduce U.S. taxation by using the presumably more favorable tax treaty of the third-country than the tax treaty (if any) of the parent corporation.  To prevent the use of conduit structures for inbound U.S. investment, Congress enacted in 1993 section 7701(l), which grants the Treasury authority to recharacterize conduit financing transactions.  If a transaction is recharacterized, the conduit entity is ignored for tax purposes, and the transaction is treated as occurring directly between the parent entity and U.S. entity.  

The regulations (Reg.\@ \S1.881-3) are quite detailed and unfortunately require at least a passing familiarity with some new terminology.  Under the regulations, the CIR can disregard for purposes of section 881 the participation of one or more \emph{intermediate entities} in a \emph{financing arrangement} where the intermediate entities are \emph{conduits}.  A financing arrangement is a transaction in which capital (money, property, or property rights) is advanced from one person (financing entity) to another (financed entity) through an intermediate entity, each linked through a \emph{financing transaction}.  Reg.\@ \S1.881-3(a)(2)(i).  

A financing transaction includes debt, leases, licenses, and in limited cases, stock.   Reg.\@ \S1.881-3(a)(2)(ii)(A).  Stock can be a financing transaction if the issuer is required to redeem or the holder has a right to sell (put) to the issuer; the issuer has right to redeem and redemption more likely than not to occur; or the holder has a right to put the stock to party related to issuer.   Reg.\@ \S1.881-3(a)(2)(ii)(A).

A \emph{conduit entity} is an intermediate entity participating in financing arrangement whose participation may be ignored, and a \emph{conduit financing arrangement} is a financing arrangement effected through one or more conduit entities. Reg.\@ \S1.881-3(a)(2)(iii) and (iv).  An \emph{intermediate entity} is a conduit entity if:  (1) participation by the intermediate entity reduces U.S. withholding tax; (2) there is a tax avoidance plan; and (3) the intermediate entity is related to the financing or financed entity, or the intermediate entity wouldn't have participated in the arrangement but for the fact that the financing entity engaged in the financing transaction with the intermediate entity. Reg.\@ \S1.881-3(a)(4)(i). 

A \emph{tax avoidance plan} is a plan one of the principal purposes of which is to reduce withholding tax, determined by examining: (1) whether there has been a significant reduction (absolute or relative) in tax under section 881; (2) the intermediate entity had sufficient resources to otherwise make the advance; (3) the time period between financing transactions; and (4) whether financing transaction occurs in the ordinary course of business of integrated trade or business. Reg.\@ \S1.881-3(b)(1)-(4).

An example.  ForCo, a resident of a country with which the United States does not have a treaty, loans \$1 million to USSub, and one year later it assigns the USSub note to ForSub, a subsidiary of ForCo, in exchange for a note from ForSub.  ForSub is a resident of a country with a U.S. treaty that exempts interest from U.S. tax.  The two notes are financing transactions and constitute a financing arrangement. Reg.\@ \S1.881-3(e), Ex. 2.  If ForSub is found to be a conduit entity, the interest payment will be treated as having been made between USSub and ForCo. 

\addcontentsline{toc}{section}{\protect\numberline{}Conduit Financing Problems} 
	\begin{center}
		\textbf{Conduit Financing Problems}
	\end{center}
	\begin{select}
	
For each of the questions below, consult Reg.\@ \S 1.881-3.  FP is a foreign corporation organized in NT, a country that does not have a treaty with the United States, DS is a wholly owned U.S. subsidiary of FP, and FS a wholly owned subsidiary of FP organized in T, a country with a tax treaty with the United States 

	\begin{enumerate}
		\item FP deposits \$1 million in Bank, an unrelated bank organized in NT.  Corp, a non-bank corporation owned by a controlling shareholder of Bank and organized in T, loans \$1 million to FS.  The transaction is undertaken to avoid the conduit financing regulations.  Is there a financing arrangement if Bank, controlling shareholder, and Corp are not treated as one entity?  [Reg.\@ \S 1.881-3(e), Ex. 5]
			\item FP loans \$1 million to FS, which in turn contributes \$990,000 to FS2, a T country corporation in exchange for stock.  FS also loans \$100,000 to FS2.  FS2 loans \$1 million to DS.  The rate on the FS1-DS loan is 10\% and the rate on the FP-FS loan is 8\%.  FS has no assets other than the stock of FS2. [Reg.\@ \S 1.881-3(e), Ex. 8]
			\item FP issues debt to foreigns persons that would be eligible for the portfolio interest exemption if issued directly by DS.  FP loans the loan proceeds to DS.  Is the debt issued by FP and the DS financing transactions?  Has there been a reduction in tax?  [Reg.\@ \S 1.881-3(e), Ex. 10]
			\item Read Reg.\@ \S 1.881-3(e), Ex. 11.  Is that a correct statement of the law?  Look back under the ``cascading royalty'' materials.
			\item FP loans \$1 million to FS at a rate of 0\%.  FS loans \$1 million to DS at a rate of 8\%. FS coordinates the FP group's financing activities, and the transaction was also intended to take advantage of the T treaty. [Reg.\@ \S 1.881-3(e), Ex. 12]
			\item  FP contributes \$1 million to FS in exchange for preferred stock.  FS loans \$1 million to FS2, a T country corporation, which in turn loans the \$1 million to DS.  If FS had loaned the money directly to DS, it would not have been entitled to treaty benefits because it would be entitled to deduct the preferred dividends paid to FP.  [Reg.\@ \S 1.881-3(e), Ex. 14]  
			\item Read the Conduit Financing Examples 1-4 in the Letter from Barbara Angus, International Tax Counsel, to Gabriel Maklouf, Director Inland Revenue, International found at the end of the Technical Explanation.
	\end{enumerate}
	\end{select}

   \section{Limitation on Benefits Article} 
\crt{}{}{Article 23; Tech. Explanation to Art. 23; Protocol, Art. IV, Exchange of Notes}

To qualify for treaty benefits, a person or entity must be a \emph{resident} under Article 4 and must also be a \emph{qualified person} under Article 23, the Limitation on Benefits (LOB) article.  Under older treaties without an LOB article, an entity incorporated in a contracting state usually was a resident for treaty purposes.  As we've seen in the treaty shopping cases above, it was relatively easy to obtain treaty benefits by incorporating an entity in a treaty country and routing income from the United States through the entity.  One potential drawback to using such a structure is tax levied by the treaty country on income received by treaty resident.  This tax, however, could be avoided by choosing to incorporate in a country with a low or zero rate on income not arising in the country or by capitalizing the treaty entity with a significant amount of debt so that income received by the treaty entity could be paid out in deductible interest, thereby eliminating treaty country taxation.  The IRS attacked these back-to-back structures through application of common law principles, e.g., \emph{Aiken v.\@ CIR}, but realized that only through a more substantive LOB article could egregious forms of treaty shopping be prevented.  

Article 23(2) lists certain persons and entities that will automatically be QPs \margit{Qualified Persons}if they are otherwise residents of one of the contracting states.  Individuals, qualified governmental entities, pensions if more than 50\% of the beneficiaries are residents of either the United States or United Kingdom, and tax-exempt entities are QPs.  

A resident \emph{company}  whose \margit{Publicly traded entities}principal class of shares are listed on certain stock exchanges and whose shares are regularly traded (at least a 6\% annual turnover) will be a QP as will a company if at least 50\% of the vote and value is owned directly or indirectly by 5 or fewer publicly traded entities.  

Under the ownership/base erosion test, \margit{Ownership and base erosion test}any legal entity that is a resident of a contracting state will be a QP if: (1) at least 50\% of the vote and value of the entity is owned for at least one-half of the taxable year by QPs; and (2) less than 50\% of the entity's gross income is paid or accrued as deductible payments to 3rd country residents (arm's-length payments in the ordinary course of business for services or tangible property are ignored as well as payments on financial obligations to U.S. and U.K. banks) 
      
If a company is not a QP, it can still claim treaty benefits if: (1) at least 95\% of the vote and value is owned by 7 or fewer persons who are \emph{equivalent beneficiaries} (``EBs''); and (2) less than 50\% of the company's gross income is paid or accrued in the form of deductible payments to persons who are not EBs.  Article 23(3).  An EB is a qualified resident of an EC, EEA, or NAFTA country that would be entitled to claim treaty benefits equivalent to those claimed by the company, including U.S. or U.K. resident individuals, qualified government entities, publicly traded entities, or tax-exempt organizations.  Article 23(7)(d), modified by Protocol, Art. IV, and the Exchange of Notes.  For dividends, interest, and royalties, the treaty rate of the EB must be at least as low as the rate under the U.K. treaty.  Article 23(7)(d), modified by Protocol, Art. IV, and Exchange of Notes.
 
Under paragraph 4, a \margit{Active trade or business test}resident that is not otherwise a QP can be entitled to treaty benefits if it is engaged in the active conduct of a trade or business in one country and income derived in the other country is derived in connection with or is incidental to that trade or business.  Furthermore, the income derived from the trade or business in the other state must be derived in a trade or business that is substantial in relation to the trade or business activity in the other state.  Article 23(4)(a) and (b).

Even if a company otherwise satisfies the QP requirements, if a company has a class of shares that entitles a holder to larger portion of the company's profit, income, or gain in the other contracting state than the holder would otherwise be entitled to; and at least 50\% of the vote and value are owned by persons who are not EBs, the treaty will apply only to the proportion of the income which those holders would have received in the absence of those terms or arrangements.  Article 23(5). 

Finally, the competent authority of either country can agree to grant treaty benefits to a resident that does not otherwise qualify as a QP.   


\addcontentsline{toc}{section}{\protect\numberline{}Limitation on Benefits Problems} 
	\begin{center}
		\textbf{Limitation on Benefits Problems}
	\end{center}
	\begin{select}
	
For each of the questions below, determine whether the relevant U.K. or U.S. entity is eligible for Treaty benefits.  Consult section Article 23, the Technical Explanation (and the  Protocol and Notes if necessary).  Assume that all U.K. and U.S. persons and entities are otherwise residents under Article 4 of the Treaty.  

	\begin{enumerate}
		\item UKCo is listed on the London Exchange and 5\% of its shares trade hands each trading day, but on a given trading day, it is estimated that 50-70\% of its shares are owned by residents of a country whose name ends in ``-stan.''  [Art. 23(2)(c)(i)]
		\item IrishCo, whose shares are listed on the London Exchange, owns 100\% of UKCo. [Art. 23(2)(c)(ii), 3(1)(a)]
		\item UKCo is owned 40\% by U.K. residents and 60\% by residents of ``-stan.'' [Art. 23(2)(f)]
			\item UKCo is owned 60\% by U.K. residents and 40\% by residents of ``-stan,'' and 60\% of UKCo gross income is paid as interest to residents of ``-stan.''  [Art. 23(2)(f)]
			\item UKCo is owned 40\% by a U.K. corporation and 60\% by a Spanish corporation that is a QP under the U.S.-Spain treaty.  A dividend is received by UKCo from its U.S. subsidiary and would qualify for the 0\% rate under Article 10, par. 3.  Assume that dividend article in the U.S.-Spain Treaty is identical to the dividend article of the Treaty but does not contain the 0\% rate.  [Art. 23(3); Article IV of the Treaty Protocol; Tech. Explanation to Art. 23]
			\item UKCo is owned 100\% by residents of ``-stan'' and owns 100\% of USCo.  UKCo manufactures bicycles and USCo distributes them in the U.S.  USCo pays a dividend to UKCo.  [Art. 23(4); Tech. Explanation to Art. 23]  
				\end{enumerate}
	\end{select}

\section{Earnings Stripping}
\crt{163(j) and 7701(l)}{}{}

In capitalizing direct investment in U.S. operations, a foreign investor usually chooses some combination of debt and equity.  The equity can be in the form of cash or property, but it may be more tax efficient to license property to the U.S. business rather than contributing the property to the U.S. business because the payments for the use of the property may be tax deductible.  With the ever-increasing importance of intellectual property, this alternative is more frequently considered.  

The portfolio interest rules make a distinction between loans made by significant shareholders and loans made by others, with the former being ineligible for the portfolio interest exemption.  It is not entirely clear why there is a distinction between the two.  One reason may be that the government may have a difficultly in determining whether the rate on related party debt is truly arm's length.  For portfolio interest, presumably market forces ensure that the rate is appropriate, especially when the debt becomes a significant portion of the borrower's capital.

Although the portfolio interest rules do not apply to interest paid to significant shareholders, most treaties do not permit the source country to tax interest payments, regardless of the ownership interest of the creditor.  Consequently, a foreign firm that can qualify for treaty benefits can capitalize its direct U.S. investment in a U.S. subsidiary with a significant percentage of debt and use the deduction for the interest paid to reduce or eliminate U.S. tax on the subsidiary's income.  This is often referred as earnings stripping--paying out a U.S. subsidiary's earnings to a significant shareholder by means of a tax deductible interest payment. 

To prevent earnings stripping, Congress enacted section 163(j) in 1989.  Section 163(j) attacks earnings stripping not by taxing the interest in the hands of the creditor, but by limiting the interest deduction of the U.S. payor corporation.  (Under proposed regulations, these rules are applicable to foreign corporations with effectively connected income.  Prop. Reg.\@ \S 1.163(j)-8.)  Section 163(j) applies to any corporation that (1) has \emph{excess interest expense}, and (2) has a debt-equity ratio in excess of 1.5 to 1.  \S163(j)(2)(B).  For these purposes, balance sheet calculations are determined using adjusted bases rather than FMVs.  \S163(j)(2)(C).  \emph{Excess interest expense} is the excess of the corporation's net interest expense (interest expense minus interest income) over 50\% of the corporation's adjusted taxable income (ATI), which is taxable income recalculated with no deduction for interest, NOLs, or depreciation or amortization.  \S163(j)(2)(B)(i) and (6)(A).  In a year in which 50\% of a corporation's ATI \emph{exceeds} its net interest expense, the difference--called \emph{excess limitation}--can be carried forward three years and will reduce excess interest expense for those future years.

Here's an example.  In Year 1, USCo has 300 of debt, 100 of equity, pays 30 of interest to its foreign parent, and has 30 of ATI.  Section 163(j) applies because: (1) USCo's D/E ratio is 3:1, and (2) USCo has excess interest expenses as its interest expense--30--is greater than 50\% of its ATI--15 (50\% of 30).  In Year 2, USCo's interest expense is 9 and its ATI is 30.   Section 163(j) does not therefore apply for Year 2, and USCo has an excess limitation carryforward of 6:  9 minus 15.  If in Year 3 USCo has 15 of interest expense and 26 of ATI, section 163(j) will not apply because USCo does not have any excess interest expense as its interest expense, 15, does not exceed  50\% of its ATI, 26, \emph{plus} its excess limitation carryforward of 6.  USCo would be treated as having used 2 of its carryforward and could carryforward to Year 4 the remaining 4 of its excess limitation.  

Once it is determined that section 163(j) applies, the corporation cannot deduct any \emph{disqualified interest}, except to the extent that it exceeds the corporation's excess interest expense.  \S163(j)(1).  Interest is \emph{disqualified interest} if:  (1) it is paid to a \emph{related person} (generally 50\% ownership) and no U.S. tax is imposed on the interest; or (2) it is paid to an unrelated taxpayer, no gross U.S. tax is imposed, and it is guaranteed by a related person that is either a tax-exempt organization or a foreign person.  \S163(j)(3), (6)(D).  Importantly, interest that is fully exempt (or to the extent it is partially exempt) pursuant to a treaty is treated as not being subject to U.S. tax.  \S163(j)(5)(B).  Any interest that is not deductible under section 163(j) is treated as \emph{disqualified interest} in the succeeding year.  \S163(j)(1)(B).
     
\addcontentsline{toc}{section}{\protect\numberline{}Earnings Stripping Problems} 
	\begin{center}
		\textbf{Earnings Stripping Problems}
	\end{center}
	\begin{select}
	
For each of the questions below, consult section 163(j).  USCo is a U.S. corporation and UKCo is a non-bank U.K. corporation eligible for treaty benefits.   

	\begin{enumerate}
		\item For 2008, USCo has interest expense of 100, 3 million of assets and 2 million of liabilities, and ATI of 150.
				\begin{enumerate}
					\item Is USCo subject to section 163(j)?
					\item What's USCo's excess interest expense?
					\item What's USCo's excess limitation?
					\item What's USCo's excess limitation carryover?
				\end{enumerate}  
			\item Same facts as previous question except that USCo has interest expense of 50 and ATI of 150 for 2008, and for 2009, 70 of interest expense and ATI of 100. 			
			\item Same facts as question 1 and all the interest is paid to USCo's parent, UKCo.  
				\begin{enumerate}
					\item What is the amount of disqualified interest for 2008?
					\item How much interest can USCo deduct in 2008?
					\item What happens to any interest that cannot be deducted in 2008?
				\end{enumerate}
			\item Same facts as previous question, except that all the interest is paid to Citibank, but UKCo guarantees USCo's liability?
		
		\end{enumerate}
	\end{select}	


     
     
     \begin{framed}
     Last modified: Oct. 16, '15; EstripLOB\_Mar5\_17
     \end{framed}
