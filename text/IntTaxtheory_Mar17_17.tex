\chapter{International Tax Theory}
\section{Overview}
How should a country tax the foreign income of its citizens and residents?  Let us first focus on the foreign income of a U.S. person, for example, a U.S. citizen, resident, or corporation, that earns income from foreign sources, either passive income or trade or business income.  Assume that a U.S. person invests \$1M abroad, earns \$100k in income, and pays some foreign income tax.  How should the U.S. tax this income?    

What are the arguments for excluding the income?  If the income is passive income, \emph{e.g.}, dividends or interest, there are not too many arguments for excluding the income as the income clearly represents an accession to wealth.  If I earn \$100 of U.S. source interest and you earn \$100 of foreign source interest, there doesn't seem to be any basis to tax us differently.  

What if the foreign income is business income?  Traditional tax policy arguments would require the income to be taxed similarly to business income arising in the U.S.  The foreign income clearly constitutes an accession to wealth.  Thus, if I earn \$100 of U.S. source business income and you earn \$100 of foreign source business income, we both have an accession to wealth of \$100 and a similar ability to pay and should therefore be taxed similarly.  Furthermore, excluding foreign source business income would favor foreign investment over U.S. investment and may lead to a misallocation of capital and underinvestment in the U.S. and overinvestment abroad.  U.S. multinationals have argued, however, that taxing foreign business income currently places them at a disadvantage vis-a-vis their foreign competitors whose home countries oftentimes do not tax foreign source business income.  

The U.S. international tax policy debate over the last 50 years has been driven by the tension between taxing U.S. and foreign business income similarly and concerns over placing U.S.-based multinationals at a competitive fiscal disadvantage vis-a-vis their foreign competitors.  This has led to the current U.S. system, which is a (very complicated) compromise of these two goals.  The current U.S. tax regime taxes currently the worldwide income of its residents, including U.S. corporations.  Worldwide taxation, however, is easily avoided by the U.S. owner forming a foreign corporation to transact business and invest capital abroad.  To prevent U.S. persons from using a foreign corporation to avoid current U.S. tax on passive income, the U.S. enacted the controlled foreign corporation (CFC) regime in 1964 under which certain \emph{passive type income} of a foreign corporation controlled by U.S. persons is taxed currently to the corporation's U.S. owners, but \emph{business income} is not taxed by the United States until it is remitted.  Thus, passive income is taxed currently to the U.S. owners whether or not received, but business income taxation is deferred until the business profits are remitted to the U.S. owners.  In 1986, the U.S. enacted the passive foreign investment company (PFIC) regime under which all earnings and profits of a foreign corporation with a significant amount of passive income or assets that generate passive income are taxed currently to all U.S. shareholders (or an interest charge is imposed for the value of deferral).  

The deferral of U.S. tax on foreign business income earned by foreign corporation clearly subsidizes foreign investment and may lead to a misallocation of resources.  Since many of our important trading partners do not tax the foreign income of their multinationals, however, Congress has felt it necessary to continue to defer U.S. tax on foreign business income.  One consequence of this hybrid system--current taxation of certain categories of passive income and deferral of taxation of business income--is the complicated statutory and regulatory regime that is needed to ensure a clear separation of passive and business income, a tracing of foreign earnings and associated foreign taxes, and a mechanism to allocate expenses such as interest and R\&D between U.S. and foreign source income.  

There are three basic possible choices for taxing foreign income.  First, the U.S. could tax currently foreign income but grant a credit for foreign taxes paid to prevent double taxation.  Second, the U.S. could tax currently foreign income but permit only a deduction for foreign taxes paid.  Finally, the U.S. could exclude permanently foreign income from tax but not permit any credit for foreign taxes paid or any deduction for expenses attributable to the foreign earning.  These three choices embody the doctrines known as Capital Export Neutrality, National Neutrality, and Capital Import Neutrality.  Let's explore each in a bit more detail to see what's at stake for the world economy, the U.S. economy, and the U.S. fisc.

			\begin{center}	
				\textbf{Capital Export Neutrality}
					\end{center}
					
Under capital export neutrality (CEN), all foreign source income is included in the tax base but a credit is granted for foreign taxes paid.  The current U.S. tax system generally reflects CEN principles--inclusion of all income, regardless of source, plus a credit for foreign taxes.  CEN reflects traditional notions of tax policy of taxing all accessions to wealth equally, and furthermore, it promotes economic efficiency by channelling investment capital to where it can get the highest returns.  CEN accomplishes this taxing currently all income, whatever its source, and by granting a credit for foreign taxes paid.  This ensures that if the pre-tax return offered by an investment in Country A is higher than the pre-tax return offered in the United States the after-tax return of the Country A investment will also be higher.  

\begin{quote}
 \textbf{EXAMPLE}:  Assume that a firm has the choice of two investment projects, one in the United States and the other in France, both of which are expected to return 10\% on an investment of 1,000.  The French investment return will be taxed at 20\% and the U.S. at 35\%.  If the firm locates the investment in France, it will earn 100 and the U.S. will levy a tax of 35\% (35), but 20 of taxes will have already been paid to France, and the United States will grant a credit for the 20 of tax paid to France.  The total tax paid is therefore 35:  20 to France and 15 to the U.S.          
\end{quote}

The net result in this example is that pre-tax returns (10\%) are equal as are after-tax returns (6.5\%), regardless of the location of the investment.  Also note that if the U.S. project offered a return of 12\% and the French 10\%, firms would invest in United States because both the pre-tax and after-tax returns would be higher in the United States.  Capital would flow to where it can get the highest return.   

The U.S. international tax system deviates from CEN principles in two important aspects.  First, roughly speaking, the U.S. does not give a credit for foreign taxes paid in excess of U.S. tax otherwise due on the foreign source income.  Why not?  Foreign countries would otherwise have no incentive to keep taxes low, and the foreign taxes could otherwise offset U.S. taxes on U.S. source income.  (Of course, higher local taxes may dissuade local investors from investing locally.)  The United States would, in essence, be sending some of its tax revenues on U.S. source income to foreign treasuries.  As a result, it is possible that U.S.-based multinationals may not invest in countries with higher tax rates than the United States and too little capital from the point of view of worldwide efficiency will end up in those countries.  It should be noted, however, that most developed countries have similar corporate tax rates, and these rates are now mostly lower than the U.S. statutory rate of 35\%. \footnotetext{For a list of the statutory corporate tax rates, \emph{see} \href{https://goo.gl/wTh1x9}{World-wide Corporate Tax Rates}}

The U.S. tax system also deviates from CEN in allowing deferral of the business income of U.S. taxpayers earned through foreign subsidiaries until it is repatriated.  If the foreign tax rate on this income is very low, the present value of the U.S. tax liabilities can be very low; deferral can therefore become exemption.  There have been periodic attempts to eliminate deferral, but they have always been beaten back.  In addition, deferral has been extended to the income of companies in the banking, finance, and insurance businesses.  \emph{See} \S954(h) and (i).



\begin{center}
				\textbf{National Neutrality}
\end{center}

Under National Neutrality (NN) principles, all income is taxed currently, but foreign taxes are treated as a business expense and can therefore only be deducted.  Proponents of NN believe that it is better to set U.S. tax policy to maximize U.S. welfare rather than worldwide welfare by encouraging U.S. investment rather than foreign investment.  In particular, under NN principles, in choosing between U.S. a foreign investments, a U.S.-based multinational would choose a foreign investment over a similar U.S. investment only if it offered a rate of return \emph{after foreign taxes} that exceeded the pre-U.S. tax return of the U.S. investment.  

\begin{quote}
\textbf{Example}:  Assume the same facts as the previous example, except that the French taxes are deductible.  The after-U.S.-and-French-taxes return is 5.2\%. This is calculated as follows:  100 gross income less 20 of French Taxes = 80 of taxable income.  The U.S. taxes would be 28 (35\%*80), leaving an after-tax amount of 52, or an after-tax return of 5.2\%. 
\end{quote}

It can be seen that any investment in the United States yielding more than 8\% will leave U.S. investors with a higher after-tax return.  Thus, U.S. projects yielding between 8-10\% will yield a higher after-tax return than French projects yielding 10\%.  This may result in the inefficient allocation of capital.  If the U.S. were to adopt NN principles, other countries would probably retaliate, thereby possibly reducing international investment.  NN proponents have not had much success on the legislative front.  		

\begin{center}
    		\textbf{Capital Import Neutrality}	
\end{center}

Under capital import neutrality (CIN), foreign income is exempt, but no credit or deduction for foreign taxes paid is permitted.  The tax systems of many of our trading partners reflect CIN principles.  CIN proponents argue that CEN places U.S. multinationals at disadvantage to foreign-based multinationals whose home countries do not tax their foreign business income.  CIN proponents argue that U.S. should not tax foreign business income at all.  Won't too much investment go abroad to low-tax countries?  Maybe, but CIN proponents say there is evidence that the foreign activities of U.S. multinational cause them to expand domestic employment and activity.  In addition, if activities can be most profitably done in low-tax areas, some firm will do them there, and it's therefore better that it be a U.S. firm.  Changing the U.S. tax system to reflect CIN principles has recently been seriously addressed by the U.S. Treasury Department in its studies of tax reform.  See the excerpt below.
  
\begin{quote}  
\textbf{Example}:  Same facts as previous example, but assume a return of 12\% for U.S. investments and a 10\% return for French investments, with any French returns being exempt from U.S. tax.  The after-tax return for French investments is 8\% (10\%*(1-.2)), whereas the after-tax return for U.S. investments is 7.8\% (12\% * (1- .35)).
\end{quote}

If the U.S. were to adopt a territorial tax system, U.S. multinationals may have incentives to invest in lower-yielding foreign projects, which may cause a misallocation of resources.  The current U.S. tax system, under which the taxation of active foreign business income of foreign subsidiaries is deferred until repatriated, reflects CIN principles and may thus be viewed as a hybrid territorial and world-wide system.  

Some have argued that these traditional views of possible approaches to the U.S. tax of international income may be outdated as they were developed at a time when the U.S. was capital exporter and net creditor; now the U.S. is a capital importer and net debtor.  In addition, these rules are concerned only with investment of multinationals and fail to account for the rise in importance of portfolio investment.  Previously, only multinationals had the resources to undertake international investment.  Since 1988, however, U.S. income from portfolio investment is greater than income from direct foreign investment.  International capital markets are now well developed, and U.S. persons can earn foreign income and allocate capital efficiently without the intermediation of multinationals.  Again, any rules should probably foster movement of capital to where it can earn the highest return.

Finally, there are being developed new views of multinational corporations.  Some argue that multinationals do not allocate capital, but invest in firm-specific assets  such as know-how and advances in technology (R\&D one example) and that these activities are most efficiently performed in one location, generally the home country.  These assets are for various reasons most useful to the firm if used in all of the activities of the firm at once.  The other activity is the production of goods it sells.  Multinationals will go to where they can most efficiently use factors of production, labor, or where the oil is.

Based on this view of multinationals, some scholars have argued that tax policy shouldn't distort the ownership of assets, but instead the tax rules should aim for \emph{capital ownership neutrality}:  the most productive owners of intangible assets should possess and exploit them.  To implement capital ownership neutrality, all countries should either tax foreign income (with a full foreign tax credit) or exempt foreign income.  Note, to implement capital ownership neutrality, all countries would have to follow the same tax policy.


\addcontentsline{toc}{section}{\protect\numberline{}Excerpt from Treasury Paper on Improving Competitiveness of U.S. Tax System} 
\begin{select}
\revrul{Approaches to Improve the Competitiveness of the U.S. Business Tax System for the 21st Century}{U.S. Dept. of Treasury, Office of Tax Policy (12/20/07)}

\begin{center}
\textbf{C. Territorial tax systems}
\end{center}  
 
The increased globalization of U.S. businesses and the decline in corporate tax 
rates abroad have focused attention on the U.S. corporate tax in an international context.  
Under current U.S. law, U.S. corporations are taxed on their worldwide income, with a 
limited tax credit for income taxes paid to foreign governments (see Box 3.1 for a more 
detailed discussion of the U.S. system for taxing international income).  However, many 
U.S. trading partners currently use a ''territorial'' system, which exempts some or all of 
the overseas earnings of their businesses from taxation in the home country. 
   
The U.S. system was developed at a time when the United States was the primary 
source of capital investment and dominated world markets.  The global landscape has 
shifted considerably over the past several decades, with other countries challenging the 
U.S. position of economic preeminence.  The United States is now a net recipient of 
foreign investment rather than the largest source.   
 
This section considers the possibility of moving to a more territorial system under 
which active income that is derived from economic activity outside the United States 
would not be subject to U.S. corporate income tax.  Similar to the practice of two-thirds 
of OECD countries, a company's active foreign income earned abroad would be excluded 
from the U.S. tax base, thus placing U.S. businesses operating abroad on a more even 
playing field relative to their foreign competitors.  
 
Underlying this approach is the notion that United States multinational corporations 
provide important benefits to the U.S. economy by creating jobs and higher real wages 
for workers in the United States.  Workers employed by firms that export earn 15 percent 
more than the average worker in the U.S. economy. Moreover, when a company 
expands overseas, jobs are created in the United States to support and manage the 
company's foreign operations.  Between 1991 and 2001, U.S. multinational enterprises 
increased employment in their domestic parents by 5.5 million, nearly twice as much as 
they increased employment in their foreign affiliates.  Moreover, the current U.S. tax 
system provides a tax disincentive to the repatriation of foreign earnings, which may 
cause U.S. multinational corporations to forgo U.S. investment. 

\begin{framed} 
\begin{quote}
\textbf{Box 3.1:  The U.S. System for Taxing International Income} 
 
Under current law, corporations formed in the United States are subject to tax on 
their worldwide income, meaning that they are subject to immediate U.S. tax on all of 
their direct earnings, whether earned in the United States or abroad.  However, U.S. 
corporations with foreign subsidiaries generally are not taxed on the foreign subsidiaries' 
active business income (such as from manufacturing operations) until the income is 
repatriated.  That is, until that active business income is returned to the United States, 
typically through a dividend to the parent corporation, U.S. tax is deferred.  Not all 
foreign subsidiary income is subject to deferral, however.  For example, U.S. tax is not 
deferred on passive or easily moveable income of foreign subsidiaries of U.S. 
corporations, under the so-called ``subpart F'' anti-deferral rules.  
  
To prevent double taxation of income by both a foreign country and the United 
States, a U.S. corporation is allowed a foreign tax credit for foreign taxes paid by it and 
by its foreign subsidiaries on earnings the foreign subsidiaries repatriate.  The foreign tax 
credit is claimed by a taxpayer on its U.S. tax return, and reduces U.S. tax liability on 
foreign source income.   
 
The foreign tax credit rules are complicated and include several significant 
limitations.  In particular, the foreign tax credit is applied separately to different 
categories of income (generally distinguishing between ``active'' and ``passive'' income).  
The total amount of foreign taxes within each category that can be credited against U.S. 
income tax cannot exceed the amount of U.S. income tax that is due on that category of 
net foreign income after deductions.  In calculating the foreign tax credit limitation, the 
U.S. parent's expenses (such as interest) are allocated to each category of income to 
determine the net foreign income on which the credit can be claimed.  The allocation of 
expenses to foreign income can increase U.S. tax by reducing the amount of foreign tax 
that can be credited that year.  
    
This foreign tax credit limitation, however, does allow active income subject to 
high foreign taxes (usually active earnings of foreign subsidiaries distributed to U.S. 
parent corporations as dividends) to be mixed with active income subject to low foreign 
taxes (usually royalties or interest).  Thus, if earnings repatriated by a foreign subsidiary 
have been taxed by the foreign country in excess of the U.S. rate, the resulting ``excess'' 
foreign tax (i.e., the amount of foreign tax on the earnings that exceeds the U.S. tax that 
would be owed on the dividend) may be used to offset U.S. tax on other, lower-taxed 
foreign source income in the appropriate category.  This method of using foreign tax 
credits arising from high-taxed foreign source income to offset U.S. tax on low-taxed 
foreign source income is known as ``cross crediting.'' 
\end{quote} 
\end{framed}

\begin{center}
\textbf{Worldwide tax systems} 
\end{center}
 
Although often described as a ``worldwide'' tax system, the U.S. system for taxing 
foreign source corporate income is more accurately described as a hybrid between a 
``pure'' worldwide system for taxing foreign source income and a so-called ``territorial'' 
system.  Under a pure worldwide system, all foreign earnings would be subject to tax by 
the home country as they are earned, even if earned by a foreign subsidiary.  To prevent 
double taxation, a foreign tax credit could be allowed for all income taxes paid to foreign 
governments.  Under a ``pure'' territorial system, on the other hand, only income earned at 
home would be subject to home-country tax. 
   
Efficiency, competitiveness, considerations of fairness and administrability, and 
revenue concerns all influence international tax policy making and are sometimes in 
conflict.  As a result, no country has a pure worldwide or pure territorial system.  Various 
standards have been proposed to guide the formulation of international tax policy, as 
discussed in Box 3.2.  None of the proposed standards, however, fits all cases and the tax 
system cannot feasibly be calibrated to have different rules for every conceivable case. 
 
Accordingly, countries with predominantly worldwide systems do not subject all 
foreign source income earned by foreign subsidiaries of multinational corporations to 
immediate home-country taxation, largely so that home-based companies are not at a 
disadvantage in investing in countries with lower tax rates.  Moreover, such countries do 
not provide an unlimited foreign tax credit, because doing so could reduce, or even 
eliminate, taxes on domestic source income. 

\begin{framed} 
\begin{quote}
\textbf{Box 3.2:  Alternative Criteria for Evaluating the Worldwide Allocation of Capital} 
 
Several standards have been proposed as guides to international tax policy, such 
as capital export neutrality, capital import neutrality, and capital ownership neutrality.  
Under the principle of capital export neutrality, foreign income should be taxed at the 
home-country tax rate so as not to distort a corporation's choice between investing at 
home or abroad.  Under the principle of capital import neutrality, foreign income should 
be taxed only at the local rate so that U.S. corporations can compete with their foreign 
rivals.  Under the principle of capital ownership neutrality, the tax system should not 
distort ownership patterns.  Each of these criteria focuses on only a portion of the 
decision margins facing corporations making cross-border investments.  For example, 
each criterion focuses on investment in tangible capital without considering the critical 
role of the location of intangible capital. 
 
Capital export neutrality and capital import neutrality make assumptions for 
which there is very little empirical evidence.  One assumption relates to the supply of 
capital available to U.S. multinational corporations.  For example, capital export 
neutrality assumes that all investment by U.S. corporations comes from domestic saving---more specifically from a fixed pool of capital available to the U.S. corporate sector.  
Capital import neutrality and capital ownership neutrality assume that capital is supplied 
at a fixed rate by the integrated world capital market.  All of these standards ignore the 
presence of intangible assets and how they affect the relationship between investments in 
different locations, or how opportunities for income shifting under alternative tax systems 
alter effective tax rates in different locations. 
 
Therefore, even if the assumption that an integrated worldwide capital market 
offers financing to corporations on the same terms regardless of where they are based is 
accepted, that alone is not a sufficient basis for choosing the optimal policy.  Consider a 
potential investment in a low-tax location.  The question is---with what other investments 
in that or other locations does it compete?  Various situations are possible.  One example might be a locational intangible, like a fast-food trademark that requires that the corporation produce locally in order to supply its customers.  In that case, all competitors 
compete in the same location and should bear the same (presumably local) tax burden.  
Another example is a mobile intangible, like the design of a computer chip that can be 
produced in various locations for the worldwide market.  In that case, the competitors for 
the potential low-tax investment may be in high-tax locations including the United States.  
Capital import neutrality and capital ownership neutrality implicitly assume the first case, 
such as where capital ownership fits the case of various bidders for an existing asset with 
a given product and a circumscribed local market that will not be altered by the 
transaction.  By contrast, capital export neutrality leans toward the second case, where all 
affiliate production substitutes for domestic U.S. production.  None of the proposed 
standards fits all cases and tax policy cannot feasibly be calibrated to have different rules 
for different cases. 
\end{quote}
\end{framed} 
   
Although a predominantly worldwide approach to the taxation of cross-border 
income was once more prevalent, it is now used by fewer than half of the OECD 
countries.  Instead, many of these countries now use predominantly territorial tax systems 
that exempt all or a portion of foreign earnings of foreign subsidiaries from home-country 
taxation.  However, to prevent tax avoidance and to maintain government revenues, 
countries with predominantly territorial systems typically do not exempt certain foreign 
earnings of foreign subsidiaries, including earnings generated from holding mobile 
financial assets, or certain payments that are deductible in the jurisdiction from which the 
payment is made, such as foreign source royalty payments.  In both predominantly 
worldwide and predominantly territorial systems, the rules that determine which types of 
foreign income are taxed, when the income is taxed, and what foreign tax credits are 
available to reduce that tax, are complex and can be the source of a great deal of tax 
planning.   
 
Under the current U.S. system, taxpayers may be able to set up their operations 
either to avoid the deemed repatriation of foreign profits under anti-deferral rules, or to 
minimize, through the use of the foreign tax credits, U.S. tax on foreign profits actually 
repatriated to the United States.  These approaches effectively can allow a corporation to 
engage in ``self-help territoriality.''  For example, creditable foreign taxes associated with 
dividends paid from high-taxed foreign profits may shield foreign source royalties from 
U.S. tax, while low-taxed foreign profits may be left abroad, thereby deferring U.S. 
taxation on those low-taxed profits indefinitely.  Depending on the type of predominantly 
territorial system chosen, current U.S. law may be more favorable to many U.S. corporate 
taxpayers than a predominantly territorial system. 
 
In part because of self-help territoriality, the current U.S. system for taxing cross-border corporate income raises little revenue from the taxation of dividends.  U.S. tax on 
all corporate foreign income was about \$18.4 billion in 2004, the most recent year for 
which data are available.  Importantly, a relatively small part of that revenue, at most 20 
percent, was derived from dividends paid by foreign subsidiaries to their U.S. parents.  
Foreign source royalties, as well as foreign source interest and income from foreign 
subsidiaries not eligible for deferral under the current system, represent a much more 
substantial source of tax revenue than dividends. 
 
In addition to raising little revenue, the present system also leads to distortions in 
economic behavior.  For example, to avoid the residual U.S. tax on repatriated earnings, 
U.S. corporations may choose not to repatriate foreign earnings and thereby forgo U.S. 
investment opportunities.  In addition, U.S. corporations engage in complex planning and 
incur significant planning costs to reduce the residual tax on repatriations.   
 
\begin{center}
\textbf{Territorial tax systems} 
\end{center}
 
More than half of OECD countries use a type of territorial system that exempts 
dividends from abroad from home-country tax.  These systems, generally referred to as 
``dividend exemption'' systems, have been proposed previously in the United States and 
could reduce some of the economic distortions imposed by the current U.S. tax system. 
   
Although the details of a dividend exemption system can vary greatly, a ``basic'' 
dividend exemption system, discussed in greater detail in the next section, is likely to 
increase U.S. corporate income tax revenues.  At the present 35-percent statutory 
corporate tax rate, the Treasury Department estimates that the revenue increase would be 
substantial, roughly \$40 billion over a 10-year period.  This revenue gain arises primarily 
from the elimination of foreign tax credits that, in effect, shield a considerable portion of 
low-taxed non-dividend foreign source income, such as certain royalties, from U.S. tax.  
In other words, because the basic dividend exemption system does not exempt many 
types of foreign source income and because the foreign tax credit otherwise arising from 
exempt dividends would be eliminated, low-taxed, non-exempt foreign income would be 
subject to U.S. tax with a much smaller available foreign tax credit, which would result in 
a large tax increase.  In addition, the allocation of expenses to exempt foreign source 
income increases U.S. tax because such expenses are effectively disallowed.  
 
One way to address the expected tax increase that would result from adopting a 
basic dividend exemption system would be to extend the exemption beyond foreign 
subsidiary dividends to include certain foreign source royalties.  However, either a full or 
partial exemption for royalties might be viewed as providing a U.S. tax exemption for 
income that may have arisen from U.S. activities, such as U.S. research and development.  
Moreover, exempting royalties might lead to a double benefit with respect to this income, 
as the United States would be providing an exemption for payments that in most foreign 
jurisdictions would give rise to a deduction.  On the other hand, moving to a dividend 
exemption system without providing some relief for royalties could exacerbate current 
issues with respect to the migration of a corporation's intangible assets and could also 
lead to the transfer of research and development activities outside the United States. 
                                                 
In any case, a dividend exemption system would reduce some of the complexity 
related to the current foreign tax credit regime, primarily because dividends would no 
longer give rise to foreign tax credits.  Other complex provisions would need to remain, 
including those related to non-exempt income, such as foreign source royalties (assuming 
foreign source royalties remain subject to U.S. tax) and interest, as well as income 
inclusions resulting under the subpart F rules.  Moreover, rules regarding the pricing of 
transactions between U.S. corporations and their foreign affiliates (the so-called ''transfer 
pricing'' rules) would come under increased pressure, as the move to a basic territorial 
system would increase the incentive to shift income and assets to low-taxed offshore 
jurisdictions.  However, extending the exemption system to include additional forms of 
business income, such as royalties, could relieve some of that pressure and in addition 
allow for further simplification.   
 
\begin{center}
\textbf{Types of territorial approaches:} 
 
\emph{Basic dividend exemption system} 
\end{center}
 
As noted above, more than half of the members of the OECD employ a dividend 
exemption system.  Many U.S. territorial tax proposals to date, including those of the 
Joint Committee on Taxation and the President's Advisory Panel on Federal Tax Reform, 
are of a dividend exemption variety.  Unlike many foreign dividend exemption systems, 
however, U.S. territorial proposals to date have generally required the allocation to (and 
therefore the disallowance of) a significant amount of expenses to exempt foreign 
income.  A system along the lines of these prior U.S. proposals is referred to here as a 
``basic'' dividend exemption system.  Several of the major features of a basic dividend 
exemption system are discussed below. 
 
\textbf{Treatment of active business income}.  Under a basic dividend exemption system, 
dividends paid by foreign subsidiaries of U.S. corporations would not be subject to U.S. 
tax, nor would foreign active business income earned directly by foreign branches of U.S. 
corporations.  Gains from the sale of assets that generate exempt income, and gains from 
sales of foreign corporation shares generating exempt dividends, would also not be 
subject to tax while losses from the sale of such assets or stock would likewise be 
disallowed.  Non-dividend payments from foreign subsidiaries to U.S. corporations, such 
as royalties and interest, would remain subject to U.S. tax.  Businesses would not receive 
foreign tax credits for foreign taxes paid (including both subsidiary-level taxes and 
dividend withholding taxes) that are attributable to earnings repatriated as dividends (or 
attributable to foreign active business income earned through foreign branches) because 
this income would not be subject to tax in the United States.  Under this system, foreign 
tax credits would continue to be available with respect to foreign taxes paid on non- 
exempt foreign income, such as royalties and interest. 
 
\textbf{Treatment of mobile income}.  Under a basic dividend exemption system, passive and 
easily moveable income---such as subpart F income under the current U.S. tax system 
(mobile income)---would continue to be subject to U.S. tax, either when earned directly 
or when earned by foreign subsidiaries (even if not repatriated).  Mobile income could 
include interest, dividends, rents, and royalties arising from passive assets.  Under this 
approach, a foreign tax credit would be available to offset foreign tax paid on mobile 
income. 
   
\textbf{Expense allocation}.  Because dividends paid by the foreign subsidiary would not be 
subject to U.S. tax when received by the U.S. corporate parent under a basic dividend 
exemption system, business expenses incurred by the U.S. parent attributable to those 
dividends would be disallowed, either in whole or in part, as a deduction against U.S. 
taxable income.  In addition, expenses attributable to exempt foreign active business 
income earned directly by a foreign branch of a U.S. corporation would be disallowed.  In 
particular, interest expense incurred by a U.S. corporation to earn exempt foreign 
earnings would be allocated to those earnings and would be nondeductible, as would an 
appropriate portion of general and administrative expenses.  Because royalties would 
continue to be subject to U.S. tax, research and experimentation expenses could continue 
to be fully deductible.  To achieve the proper allocation of expenses to taxable income, 
detailed expense allocation rules, similar to the current expense allocation rules, would be 
necessary and would inevitably introduce complexity to the system. 
  
If expenses associated with foreign exempt income were disallowed as 
deductions, a dividend exemption system would create additional incentives for tax 
planning by multinational corporations to reduce the amount of expenses allocable to 
foreign income.  Again, these pressures also exist under current law because expense 
allocations to foreign source income can restrict use of foreign tax credits.  However, 
expense allocations would have much broader effects under a dividend exemption system 
because such expense allocations could directly reduce the deductions that could be taken 
by taxpayers, not solely restrict the use of foreign tax credits 
 
Instead of disallowing deductions for expenses allocable to exempt foreign source 
income, some countries, such as France and Italy, exempt less than 100 percent of foreign 
subsidiary dividends and directly earned foreign active business income.  The benefit of 
adopting this modification to a basic dividend exemption system is that it allows for the 
elimination of some relatively complex rules associated with expense allocation. 
   
\textbf{Transfer pricing}.  A basic dividend exemption system would also create additional 
incentives for multinational corporations to use transfer pricing to minimize taxable 
income generated by domestic operations and maximize income generated by active 
foreign business operations.  These pressures also exist under current law, and a large 
body of rules has evolved to enforce ``arm's length'' transfer pricing among related 
parties.  However, the pressures are more pronounced in a basic dividend exemption 
system because shifting income and assets overseas may result in exempt foreign source 
income, rather than produce income that is eligible merely for deferral of tax, as under 
the current U.S. system.  To the extent that transfer pricing enforcement cannot withstand 
the increased pressures, significant abuse concerns might arise.   
 
\textbf{Revenue consequences of basic dividend exemption system}.  As noted above, adoption of 
a basic dividend exemption system by the United States would likely increase corporate 
income tax revenues.  The Treasury Department estimates the 10-year revenue gain 
associated with a basic dividend exemption system such as outlined above (with no other 
changes to the U.S. international tax rules) to be approximately \$40 billion at the present 
35-percent statutory corporate tax rate.  If the corporate tax rate were 28 percent, the 10-year estimate of the revenue gain would increase to \$50 billion.
 
The increase in corporate income tax revenues from adoption of a basic dividend 
exemption system is a result of two factors.  First, the relatively small revenue loss from 
eliminating the U.S. tax on dividends is more than offset by the full taxation of royalties 
and other foreign source income still subject to tax.  Approximately two-thirds of foreign 
source royalty payments are essentially exempt from U.S. tax because of cross-crediting 
with high-taxed dividends.  Under a dividend exemption system, royalties would no 
longer be shielded from U.S. tax by such cross-crediting.  Second, the allocation of 
expenses to exempt foreign source income increases U.S. tax because that allocation has 
greater negative consequences under a dividend exemption system (deduction 
disallowance) than under current law (decreased ability to use foreign tax credits).  
 
Indeed, under a basic dividend exemption system, because of the continued full 
taxation of royalties and the disallowance of deductions for expenses attributable to 
exempt foreign income, the effective tax rate on investment in a low-tax location would 
increase, not decrease.  Grubert and Mutti (2001) estimate that the effective tax rate for a 
typical investment in a low-tax affiliate would increase from about 5 percent under 
current law to about 9 percent under a basic dividend exemption.  For an investment in 
intangible assets, the effective tax rate would increase from about 26 percent to 35 
percent.  These estimates do not take income shifting into account and, thus, may 
overestimate the actual burdens corporations may face in low-tax countries. 

\begin{center} 
\textbf{Alternative territorial approaches} 
\end{center}
 
A basic dividend exemption system might reduce economic distortions by 
addressing the problem of forgone domestic investment opportunities and eliminating 
certain tax avoidance costs.  However, as discussed above, it would increase the overall 
tax burden on foreign source income of U.S. corporations primarily because of the 
greater tax on royalty income, which, in many cases, is currently shielded from U.S. tax 
through cross-crediting, would incur greater U.S. tax, and because the allocation of parent 
expenses to exempt foreign source income would result in disallowance of certain 
deductions.  This higher level of tax may well affect a variety of business decisions 
including the location of investment.    
 
The higher tax burden could be addressed by exempting other foreign source 
income in addition to dividends and active foreign source income earned through foreign 
branches, or by relaxing the current expense allocation rules.  Such a system would 
potentially allow for additional simplification but, depending on the details of the system, 
could pose abuse and revenue-loss problems, including the loss of tax revenue generated 
from what is currently U.S. source income.  For example, an approach that also exempts 
50 percent of royalties and requires no disallowance of interest and general and 
administrative expenses would cost \$75 billion over a 10-year period.   
 
\begin{center}
\textbf{Alternative territorial approaches include}: 
\end{center}
 
\textbf{Narrower definition of mobile income}.  Dividend exemption approaches generally 
assume that mobile income (which would be subject to immediate U.S. tax when earned 
either by foreign subsidiaries or directly by U.S. corporations) includes more than passive 
income earned by non-financial institutions.  While the details vary, as a general matter, 
dividend exemption proposals tend to tax currently certain types of foreign active 
business income deemed to be easily moveable.  Expanding the categories of exempt 
income could allow for simplification and would reduce the revenue gains associated 
with dividend exemption.   
 
\textbf{Extension of the exemption to certain foreign source royalties and interest}.  An 
alternative territorial approach could exempt, in whole or in part, royalties and interest 
received by U.S. corporations in the active conduct of a trade or business.  Another 
alternative would be to treat income received by U.S. corporations from foreign 
subsidiaries in the form of royalties and interest as exempt income on a look-through 
basis.  In other words, if such income were allocable to the active business income of the 
payor, it would be eligible for exemption.   
 
As discussed above, exempting royalties could be controversial, as some may 
view it as providing a U.S. tax exemption for income that may have arisen from U.S. 
activities, such as U.S. research and development.  On the other hand, moving to a 
dividend exemption system without providing some relief for royalties could exacerbate 
current issues with respect to intangible migration, and lead to the transfer of research 
and development activities outside the United States. 
 
\textbf{Reduce expense disallowance}.  Another alternative territorial system would limit exempt 
income to foreign source dividends and foreign active business income of foreign 
branches of U.S. corporations, but relax the rules disallowing a U.S. parent corporation's 
interest and general and administrative expenses attributable to exempt foreign income.  
This alternative would disallow only a fixed percentage of appropriately attributable 
expenses, thereby possibly allowing deductions of expenses attributable to foreign source 
income against U.S.-source income.  Alternatively, all expenses could be fully deductible 
but only a percentage of foreign source dividends and directly earned foreign active 
business income would be exempt, similar to what France, Italy, and other countries do.
 
If a portion of interest and general and administrative expenses are not allocated 
to exempt foreign source income (and, therefore, not disallowed), the effective tax rate on 
investment in low-tax countries could be negative (indicating that the investment has a 
higher return after taxes than before taxes) because the tax saving from the U.S. 
deduction could exceed the foreign tax on the income.  This result might encourage 
multinational corporations to shift business activities abroad that otherwise would be 
conducted in the United States but for tax motives. 
   
In sum, a basic dividend exemption system would remove the tax disincentive to 
the repatriation of foreign earnings.  It would also reduce some of the complexity related 
to the current system with respect to foreign tax credits, primarily because dividends 
would no longer give rise to foreign tax credits.  Nevertheless, other complex provisions 
would remain, for example, with respect to non-exempt income, such as foreign source 
royalties and interest as well as subpart F inclusions.  As noted above, the transfer pricing 
rules may come under increased pressure, as the move to a basic dividend exemption 
system could increase the incentive to shift income and assets to low-tax offshore 
jurisdictions.  Extending the foreign source income exemption to include other active 
business income, such as royalties, could allow for additional simplification, would 
eliminate the revenue raised by moving to a basic dividend exemption system, and could 
relieve some of the increased pressure on the transfer pricing rules, but could raise other 
issues and concerns. 

\end{select}

The following articles in the popular press highlight the distortions caused by the current U.S. tax regime for U.S.-based multinationals and the challenges legislators and tax administrators face in designing a better system.  



\addcontentsline{toc}{section}{\protect\numberline{}More U.S. Profits Parked Abroad} 
\begin{select}
\revrul{U.S. Profits Parked Abroad}{\small{Wall Street Journal, Mar. 11, 2013, p. B1}}


U.S. companies are making record profits. And more of the money is staying offshore, and lightly taxed.


A Wall Street Journal analysis of 60 big U.S. companies found that, together, they parked a total of \$166 billion offshore last year. That shielded more than 40\% of their annual profits from U.S. taxes, though it left the money off-limits for paying dividends, buying back shares or making investments in the U.S. The 60 companies were chosen for the analysis because each of them had held at least \$5 billion offshore in 2011.

The practice is a result of U.S. tax rules that create incentives for companies to maximize the earnings, and holdings, of foreign subsidiaries. The law generally allows companies to not record or pay taxes on profits earned by overseas subsidiaries if the money isn't brought back to the U.S.

Big American companies are booking more of their sales in faster-growing foreign markets. But companies also are moving more of their earnings overseas by assigning valuable patents and licenses to foreign units.

Untaxed foreign earnings are part of a contentious debate over U.S. fiscal policy and tax code. The current system attracts criticism from many points of view. Business groups want the U.S. to tax profit based on where it is generated, as many countries do, rather than globally, as the U.S. does now. Moreover, they point out, tax rates are higher in the U.S. than in many other nations, putting American companies at a disadvantage.

Others say that the growing cash hoards often are the result of sophisticated corporate maneuvers to shift profits to low-tax countries.

Within the group of 60 companies, the Journal found 10 that parked more earnings offshore last year than they generated for their bottom lines. They include Abbott Laboratories, whose store of untaxed overseas earnings rose by \$8.1 billion, to \$40 billion. The increase exceeded the pharmaceutical maker's net income of \$6 billion, which was weighed down by a \$1.4 billion charge related to early repayment of debt. Including that charge, Abbott reported a pretax loss on its U.S. operations.

An Abbott spokesman declined to comment.

Honeywell International Inc. boosted its store of untaxed earnings held by its offshore subsidiaries and earmarked for foreign investment by \$3.5 billion last year to \$11.6 billion, a rise equal to the industrial conglomerate's annual profit, excluding a pension adjustment. The company said the increase resulted from \$2.1 billion of pretax earnings from its foreign subsidiaries and changes in estimates.

Chief Financial Officer Dave Anderson says Honeywell needs to invest outside the U.S. to fuel foreign sales, which accounted for 54\% of Honeywell's revenue last year.

He also says the tax code is part of the equation. ``The anachronistic tax system that we have penalizes companies for their success outside of the U.S.,'' Mr. Anderson says.

The amount of money at stake is significant, particularly when the U.S. budget deficit is high on the political agenda. Just 19 of the 60 companies in the Journal's survey disclose the tax hit they could face if they brought the money back to their U.S. parent. Those companies say they might have to pay \$98 billion in additional tax,more than the \$85 billion in automatic-spending cuts triggered this month after the White House and Congress couldn't agree on an alternative.

The Joint Committee on Taxation estimates that changing the law to fully tax overseas earnings would generate an additional \$42 billion for the Treasury this year alone. Congress enacted a temporary tax holiday in 2004, prompting companies to repatriate \$312 billion in foreign earnings. The law was intended to stimulate the U.S. economy, but studies found that few jobs were created and most of the money was used to repurchase shares and pay dividends. Another such holiday is considered unlikely in the next few years.

The Journal's survey of new regulatory filings found that the total earnings held by the 60 companies' foreign subsidiaries rose 15\%, to \$1.3 trillion, from \$1.13 trillion a year earlier.

The trend was most pronounced among the 26 technology and health-care companies in the Journal survey. Collectively, they parked \$120 billion in foreign units last year, accounting for nearly three-quarters of the total. At some of these companies, foreign subsidiaries hold almost all the company's cash. Johnson \& Johnson says its foreign subsidiaries held \$14.8 billion in cash and cash equivalents as of Dec. 30, out of a total of \$14.9 billion.

Not all of the earnings parked offshore are in cash. Some of the money is used to build plants and buy equipment overseas. In a paper last year, Wharton School accounting professor Jennifer Blouin and two co-authors estimated that 43\% of the offshore earnings were held in cash.

A Senate committee last year found that many tech and health-care companies have shifted intellectual property, such as patent and marketing rights, to subsidiaries in low-tax countries. The companies then record sales and profits from these lower-tax countries, which reduces their tax payments.

``There are opportunities to basically wipe away your tax on your intellectual property,'' says Ms. Blouin, the Wharton professor.

Software maker Microsoft Corp. boosted the holdings of its foreign subsidiaries by \$16 billion in the fiscal year ended June 30, 2012, to \$60.8 billion, the third-largest holding in the Journal survey. The growth in Microsoft's overseas holdings nearly equaled its net income for the year of \$17 billion, in part because Microsoft said its foreign operations accounted for 93\% of its pretax profit last year.

In its report, the Senate committee said Microsoft had shifted intellectual property to subsidiaries in Singapore, Ireland and Puerto Rico, to avoid roughly \$4 billion in U.S. taxes in 2011. Licensing rights, and revenue, sometimes traveled through more than one subsidiary to minimize the tax bill.

``Microsoft complies with the tax rules in each jurisdiction in which it operates and pays billions of dollars in U.S. federal, state, local and foreign taxes each year,'' Bill Sample, Microsoft's corporate vice president for world-wide taxation, told the Senate committee in September.

A Microsoft spokesman declined to comment further.

Oracle Corp. reported holding \$20.9 billion in its foreign subsidiaries as of May 31, 2012, up 30\% from a year earlier. Oracle lowered its tax rate last year to 23\%, from 25.1\% in 2011, raising its bottom line by \$272 million.

In a securities filing, Oracle said the tax rate fell in part because it ``increased the number of foreign subsidiaries'' in low-tax countries; the filing listed four Irish subsidiaries that weren't listed the prior year. Oracle said it expects the new subsidiaries to help it maintain a lower tax rate. An Oracle spokeswoman didn't respond to requests for comment.

Abbott runs manufacturing plants in more than a dozen foreign countries, plus Puerto Rico, and generated 58\% of its \$40 billion in 2012 revenue outside the U.S. In a securities filing, Abbott estimated that lower tax rates on its foreign operations cut its U.S. tax bill by \$1.6 billion last year.

A big Abbott subsidiary in Ireland, Abbott Laboratories Vascular Enterprises Ltd., reported profit of \euro 1.1 billion for 2011 (\$1.43 billion), the latest figures available, and paid no Irish tax, because it is incorporated in Bermuda, according to an Irish corporate filing.

Some companies are accumulating large sums of earnings that they say will remain outside the U.S. General Electric Co. reported \$108 billion held offshore at the end of last year, up from \$102 billion a year earlier; GE says most of that is invested in active business operations such as plants and research centers. At Pfizer Inc., the total rose to \$73 billion, from \$63 billion.

The swelling totals have sparked friction at companies such as Apple Inc., where investors want executives to distribute more cash through dividends and share repurchases.

Overseas balances have grown in part because U.S. multinational companies are paying less tax on their overseas operations. Offshore subsidiaries of U.S. companies paid an average 14\% tax rate in 2008, according to the most recent statistics from the Internal Revenue Service, down from 16\% in 2004.

Corporate filings offer a glimpse of the low rates companies pay outside the U.S. Apple said it held \$40.4 billion in untaxed earnings outside the U.S. as of Sept. 29, 2012. Apple estimated that it would owe \$13.8 billion in tax if it brought that money back to the U.S. That is a 34\% tax rate, just shy of the federal 35\% rate. Since foreign income taxes are creditable on U.S. taxes, that means Apple has paid less than 5\% tax on those earnings to date, says Ms. Blouin, the Wharton professor.
\end{select} 


\addcontentsline{toc}{section}{\protect\numberline{}Google's 2.4\% Rate} 
\begin{select}
\revrul{Google 2.4\% Rate Show How \$60 Billion Lost to Tax Loopholes}{\small{http://www.bloomberg.com/news/print/2010-10-21/google-2-4-rate-shows-how-60-billion-u-s-revenue-lost-to-tax-loopholes.html}}

Google Inc. cut its taxes by \$3.1 billion in the last three years using a technique that moves most of its foreign profits through Ireland and the Netherlands to Bermuda.

Google's income shifting--involving strategies known to lawyers as the ``Double Irish'' and the ``Dutch Sandwich''--helped reduce its overseas tax rate to 2.4 percent, the lowest of the top five U.S. technology companies by market capitalization, according to regulatory filings in six countries.

``It's remarkable that Google's effective rate is that low,'' said Martin A. Sullivan, a tax economist who formerly worked for the U.S. Treasury Department. ``We know this company operates throughout the world mostly in high-tax countries where the average corporate rate is well over 20 percent.''

The U.S. corporate income-tax rate is 35 percent. In the U.K., Google's second-biggest market by revenue, it's 28 percent.

Google, the owner of the world's most popular search engine, uses a strategy that has gained favor among such companies as Facebook Inc. and Microsoft Corp. The method takes advantage of Irish tax law to legally shuttle profits into and out of subsidiaries there, largely escaping the country's 12.5 percent income tax. 

The earnings wind up in island havens that levy no corporate income taxes at all. Companies that use the Double Irish arrangement avoid taxes at home and abroad as the U.S. government struggles to close a projected \$1.4 trillion budget gap and European Union countries face a collective projected deficit of 868 billion euros.

\begin{center} \textbf{Countless Companies}
\end{center}

Google, the third-largest U.S. technology company by market capitalization, hasn't been accused of breaking tax laws. ``Google's practices are very similar to those at countless other global companies operating across a wide range of industries,'' said Jane Penner, a spokeswoman for the Mountain View, California-based company. Penner declined to address the particulars of its tax strategies.

Facebook, the world's biggest social network, is preparing a structure similar to Google's that will send earnings from Ireland to the Cayman Islands, according to the company's filings in Ireland and the Caymans and to a person familiar with its plans. A spokesman for the Palo Alto, California-based company declined to comment.

\begin{center} \textbf{Transfer Pricing}
\end{center}

The tactics of Google and Facebook depend on ``transfer pricing,'' paper transactions among corporate subsidiaries that allow for allocating income to tax havens while attributing expenses to higher-tax countries. Such income shifting costs the U.S. government as much as \$60 billion in annual revenue, according to Kimberly A. Clausing, an economics professor at Reed College in Portland, Oregon.

U.S. Representative Dave Camp of Michigan, the ranking Republican on the House Ways and Means Committee, and other politicians say the 35 percent U.S. statutory rate is too high relative to foreign countries. International income-shifting, which helped cut Google's overall effective tax rate to 22.2 percent last year, shows one way that loopholes undermine that top U.S. rate.

Two thousand U.S. companies paid a median effective cash rate of 28.3 percent in federal, state and foreign income taxes in a 2005 study by academics at the University of Michigan and the University of North Carolina. The combined national-local statutory rate is 34.4 percent in France, 30.2 percent in Germany and 39.5 percent in Japan, according to the Paris-based Organization for Economic Cooperation and Development.

\begin{center} \textbf{The Double Irish}
\end{center}

As a strategy for limiting taxes, the Double Irish method is ``very common at the moment, particularly with companies with intellectual property,'' said Richard Murphy, director of U.K.-based Tax Research LLP. Murphy, who has worked on similar transactions, estimates that hundreds of multinationals use some version of the method.

The high corporate tax rate in the U.S. motivates companies to move activities and related income to lower-tax countries, said Irving H. Plotkin, a senior managing director at PricewaterhouseCoopers LLP's national tax practice in Boston. He delivered a presentation in Washington, D.C. this year titled ``Transfer Pricing is Not a Four Letter Word.''

``A company's obligation to its shareholders is to try to minimize its taxes and all costs, but to do so legally,'' Plotkin said in an interview.

\begin{center} \textbf{Boosting Earnings}
\end{center}

Google's transfer pricing contributed to international tax benefits that boosted its earnings by 26 percent last year, company filings show. Based on a rough analysis, if the company paid taxes at the 35 percent rate on all its earnings, its share price might be reduced by about \$100, said Clayton Moran, an analyst at Benchmark Co. in Boca Raton, Florida. He recommends buying Google stock, which closed yesterday at \$607.98.

The company, which tells employees ``don't be evil'' in its code of conduct, has cut its effective tax rate abroad more than its peers in the technology sector: Apple Inc., the maker of the iPhone; Microsoft, the largest software company; International Business Machines Corp., the biggest computer-services provider; and Oracle Corp., the second-biggest software company. Those companies reported rates that ranged between 4.5 percent and 25.8 percent for 2007 through 2009.

Google is ``flying a banner of doing no evil, and then they're perpetrating evil under our noses,'' said Abraham J. Briloff, a professor emeritus of accounting at Baruch College in New York who has examined Google's tax disclosures.

``Who is it that paid for the underlying concept on which they built these billions of dollars of revenues?'' Briloff said. ``It was paid for by the United States citizenry.''

\begin{center} \textbf{Taxpayer Funding}
\end{center}

The U.S. National Science Foundation funded the mid-1990s research at Stanford University that helped lead to Google's creation. Taxpayers also paid for a scholarship for the company's cofounder, Sergey Brin, while he worked on that research. Google now has a stock market value of \$194.2 billion.

Google's annual reports from 2007 to 2009 ascribe a cumulative \$3.1 billion tax savings to the ``foreign rate differential.'' Such entries typically describe how much tax U.S. companies save from profits earned overseas.

In February, the Obama administration proposed measures to curb shifting profits offshore, part of a package intended to raise \$12 billion a year over the coming decade. While the key proposals largely haven't advanced in Congress, the IRS said in April it would devote additional agents and lawyers to focus on five large transfer pricing arrangements.

\begin{center} \textbf{Arm's Length}
\end{center}

Income shifting commonly begins when companies like Google sell or license the foreign rights to intellectual property developed in the U.S. to a subsidiary in a low-tax country. That means foreign profits based on the technology get attributed to the offshore unit, not the parent. Under U.S. tax rules, subsidiaries must pay ``arm's length'' prices for the rights--or the amount an unrelated company would.

Because the payments contribute to taxable income, the parent company has an incentive to set them as low as possible. Cutting the foreign subsidiary's expenses effectively shifts profits overseas.

After three years of negotiations, Google received approval from the IRS in 2006 for its transfer pricing arrangement, according to filings with the Securities and Exchange Commission.

The IRS gave its consent in a secret pact known as an advanced pricing agreement. Google wouldn't discuss the price set under the arrangement, which licensed the rights to its search and advertising technology and other intangible property for Europe, the Middle East and Africa to a unit called Google Ireland Holdings, according to a person familiar with the matter.

\begin{center} \textbf{Dublin Office}
\end{center}

That licensee in turn owns Google Ireland Limited, which employs almost 2,000 people in a silvery glass office building in central Dublin, a block from the city's Grand Canal. The Dublin subsidiary sells advertising globally and was credited by Google with 88 percent of its \$12.5 billion in non-U.S. sales in 2009.

Allocating the revenue to Ireland helps Google avoid income taxes in the U.S., where most of its technology was developed. The arrangement also reduces the company's liabilities in relatively high-tax European countries where many of its customers are located.

The profits don't stay with the Dublin subsidiary, which reported pretax income of less than 1 percent of sales in 2008, according to Irish records. That's largely because it paid \$5.4 billion in royalties to Google Ireland Holdings, which has its ``effective centre of management'' in Bermuda, according to company filings.

\begin{center} \textbf{Law Firm Directors}
\end{center}

This Bermuda-managed entity is owned by a pair of Google subsidiaries that list as their directors two attorneys and a manager at Conyers Dill \& Pearman, a Hamilton, Bermuda law firm.

Tax planners call such an arrangement a Double Irish because it relies on two Irish companies. One pays royalties to use intellectual property, generating expenses that reduce Irish taxable income. The second collects the royalties in a tax haven like Bermuda, avoiding Irish taxes.

To steer clear of an Irish withholding tax, payments from Google's Dublin unit don't go directly to Bermuda. A brief detour to the Netherlands avoids that liability, because Irish tax law exempts certain royalties to companies in other EU-member nations. The fees first go to a Dutch unit, Google Netherlands Holdings B.V., which pays out about 99.8 percent of what it collects to the Bermuda entity, company filings show. The Amsterdam-based subsidiary lists no employees.

\begin{center} \textbf{The Dutch Sandwich}
\end{center}

Inserting the Netherlands stopover between two other units gives rise to the ``Dutch Sandwich'' nickname.

``The sandwich leaves no tax behind to taste,'' said Murphy of Tax Research LLP.

Microsoft, based in Redmond, Washington, has also used a Double Irish structure, according to company filings overseas. Forest Laboratories Inc., maker of the antidepressant Lexapro, does as well, Bloomberg News reported in May. The New York-based drug manufacturer claims that most of its profits are earned overseas even though its sales are almost entirely in the U.S. Forest later disclosed that its transfer pricing was being audited by the IRS.

Since the 1960s, Ireland has pursued a strategy of offering tax incentives to attract multinationals. A lesser-appreciated aspect of Ireland's appeal is that it allows companies to shift income out of the country with minimal tax consequences, said Jim Stewart, a senior lecturer in finance at Trinity College's school of business in Dublin.

\begin{center} \textbf{Getting Profits Out}
\end{center}

``You accumulate profits within Ireland, but then you get them out of the country relatively easily,'' Stewart said. ``And you do it by using Bermuda.''

Eoin Dorgan, a spokesman for the Irish Department of Finance, declined to comment on Google's strategies specifically. ``Ireland always seeks to ensure that the profits charged in Ireland fully reflect the functions, assets and risks located here by multinational groups,'' he said.

Once Google's non-U.S. profits hit Bermuda, they become difficult to track. The subsidiary managed there changed its legal form of organization in 2006 to become a so-called unlimited liability company. Under Irish rules, that means it's not required to disclose such financial information as income statements or balance sheets.

``Sticking an unlimited company in the group structure has become more common in Ireland, largely to prevent disclosure,'' Stewart said.

\begin{center} \textbf{Deferred Indefinitely}
\end{center}

Technically, multinationals that shift profits overseas are deferring U.S. income taxes, not avoiding them permanently. The deferral lasts until companies decide to bring the earnings back to the U.S. In practice, they rarely repatriate significant portions, thus avoiding the taxes indefinitely, said Michelle Hanlon, an accounting professor at the Massachusetts Institute of Technology.

U.S. policy makers, meanwhile, have taken halting steps to address concerns about transfer pricing. In 2009, the Treasury Department proposed levying taxes on certain payments between U.S. companies' foreign subsidiaries.

Treasury officials, who estimated the policy change would raise \$86.5 billion in new revenue over the next decade, dropped it after Congress and Treasury were lobbied by companies, including manufacturing and media conglomerate General Electric Co., health-product maker Johnson \& Johnson and coffee giant Starbucks Corp., according to federal disclosures compiled by the non-profit Center for Responsive Politics.

\begin{center} \textbf{Administration Concerned}
\end{center}

While the administration ``remains concerned'' about potential abuses, officials decided ``to defer consideration of how to reform those rules until they can be studied more broadly,'' said Sandra Salstrom, a Treasury spokeswoman. The White House still proposes to tax excessive profits of offshore subsidiaries as a curb on income shifting, she said.

The rules for transfer pricing should be replaced with a system that allocates profits among countries the way most U.S. states with a corporate income tax do--based on such aspects as sales or number of employees in each jurisdiction, said Reuven S. Avi-Yonah, director of the international tax program at the University of Michigan Law School.

``The system is broken and I think it needs to be scrapped,'' said Avi-Yonah, also a special counsel at law firm Steptoe \& Johnson LLP in Washington D.C. ``Companies are getting away with murder.''  

\end{select} 

\addcontentsline{toc}{section}{\protect\numberline{}Comments}
	\begin{center}
			\emph{\textbf{Comments}}
	\end{center}

\begin{enumerate}
	\item Please skim the overview of the BEPS project prepared by the Joint Committee on Taxation on the class web site.
	\item In recent years, many of the G20 countries (the economically developed countries) have become concerned that tax planning by multinationals has seriously eroded source basis taxation and the associated tax revenues.  These countries share many of the same concerns as the United States, including the rise of stateless income, abusive transfer pricing especially with respect to intellectual property, hybrid entities and the mismatch of income, interest stripping, and abusive use of treaties. The OECD became concerned that unilateral responses by OECD members could result in double taxation  and increased tax uncertainty for cross-border investments, which led the OECD to address base erosion and profit shifting in the context of cross-border transactions.  In response to their findings, the OECD approved in 2013 the \emph{BEPS Action Plan}, which identified 15 action items that required new international standards.\footnote{OECD, Action Plan on Base Erosion and Profit Shifting, July 19, 2013, available at \href{http://www.oecd.org/tax/action-plan-on-base-erosion-and-profit-shifting-9789264202719-en.htm}{OECD BEPS Action Plan}}  Key action items were electronic commerce, hybrid mismatch arrangements, transfer pricing aspects of intellectual property, CFC rules, interest deductibility, and data collection.  Final reports on all items were finished in 2015 and endorsed by the G20 leaders.  Individual countries, including the United States, have already begun to implement some of the action items.\footnote{Recent developments and country-by-country trackers can be found in at the following links:   \href{http://www2.deloitte.com/global/en/pages/tax/articles/beps-country-scorecards.html}{Deloitte BEPS Scorecards} and \href{http://www.ey.com/GL/en/Services/Tax/OECD-base-erosion-and-profit-shifting-project}{E\&Y BEPS} } We'll visit many of these topics throughout the semester.
	\item The first BEPS item to be implemented in the United States is the country-by-country reporting as set forth in the 2015 Final Report for Action 13 (Transfer Pricing Documentation and Country-by-Country Reporting) in Reg. \S1.6038-4 (TD 9773, 29 June 2016).  Under these rules, US-parented multinational enterprises with annual revenues exceeding \$850 million must file Form 8975, ``Country-by-Country' Report''.   The report requires the ultimate parent entity to report for each \emph{constituent entity} such information as the entity's tax residence, residence of incorporation, and main business activity of the constituent entity.  Furthermore, certain financial and employee information must be reported in the Form 8975 for each tax jurisdiction in which one or more constituent entities are resident.  Among the information that must be reported is the revenues generated from transactions with other constituent entities; the profit or loss before income tax; the total income tax paid on a cash basis to all tax jurisdictions, and any taxes withheld on payments received by the constituent entities; the total number of employees on a full-time equivalent basis; and the net book value of tangible assets.
	

\end{enumerate}


\begin{framed}
Last modified: Mar. 17, '17; IntTaxtheory\_Mar17\_17.tex
\end{framed}

