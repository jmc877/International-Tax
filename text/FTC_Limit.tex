\chapter{Foreign Tax Credit}
\crt{904(a) and (c)}{1.904-2(a)-(c)}{Article 24}

\section{Foreign Tax Credit Limitation:  Sections 904}
 

Section 901(a) permits a U.S. taxpayer to elect to credit against its U.S. tax liability foreign income taxes, including in lieu of taxes under section 903.  A U.S. taxpayer's FTC is subject, however, to the limitations of section 904. Under section 904(a), a taxpayer can generally take a credit against its U.S. tax liability in the same ratio as its foreign source taxable income bears to its worldwide taxable income.
 
  \begin{equation*}
FTC Limit = US Tax (pre-credit) * \frac{FSTI}{ WWTI}
\end{equation*}


This can also be written as:

\begin{equation*}
FTC Limit = US Tax Rate * FSTI
\end{equation*}

Thus, as a rough approximation, a taxpayer can credit foreign taxes to the extent that the foreign tax rate does not exceed the taxpayer's U.S. average tax rate.  Foreign taxes paid that exceed a taxpayer's FTC limit can be carried back one year and forward ten years and possibly used if a taxpayer is in an excess limitation position for those years.  \S904(c).

An issue that has vexed tax administrators since the inception of the foreign tax credit is whether all foreign source income, including income on which no foreign tax has been levied, should be included in the numerator and an ``overall'' FTC limitation computed, or whether the FTC limitation should be applied on an item-by-item basis, a country-by-country basis, or some other division.  A taxpayer's FTC limitation can vary considerably depending on which regime applies.  

Assume that USCO, a U.S. corporation with a 35\% U.S. tax rate, earns income and pays foreign taxes to the following jurisdictions:  

\begin{center}
 
  \begin{tabular}{l c c c c c}         
  & Germany & UK & Japan & US & Total WW\\
  \hline
  Income & (100) & 100 & 100 & 100 & 200 \\
  FT Paid & 0 & 50 & 50 & 0 & 100 \\
  \hline
    \end{tabular}
   \end{center}


Under an overall limitation \margit{The overall limitation}, USCO's FTC limitation would be computed as follows: 
\begin{quotation}

FTC limit = 70 x [100 (FS income) / 200 (WW income)] = 35
\end{quotation} 
USCO would receive a credit for 35 of the 100 of foreign taxes paid.  

Under a per-country limitation method\margit{The per-country limitation}, a separate FTC limitation is computed for each country. 
USCO's FTC limit would be computed as follows: 
\begin{quotation}
UK: = 70 x [100 (UK income) / 200 (WW income)] = 35

Japan: = 70 x [100 (Jap. income) / 200 (WW income)]  = 35
 
Germany: = 100 x [0 (Germ. income) / 200 (WW income)] = 0
 
Total FT Credits =  70 
\end{quotation}
A per-country limitation limits the credit for income taxes paid in each foreign country to the 
effective rate of U.S. tax on income from sources within that country that would be paid absent the 
credit.  In this example, USCO receives a total FTC of 70 instead of the 35 under the overall limitation. 

Sometimes, however,  the overall limitation could be more beneficial.  Assume that USCO earns income 
from and pays foreign taxes to the following jurisdictions: 
\begin{center}
 
  \begin{tabular}{l c c c c c}         
  & N.A. & UK & Japan & US & Total WW\\
  \hline
  Income & 100 & 100 & 100 & 100 & 400 \\
  FT Paid & 0 & 50 & 50 & 0 & 100 \\
  \hline
    \end{tabular}
   \end{center}
Under an overall limitation, USCO's FTC limitation would be computed as follows: 

\begin{quotation} FTC limit = 140 x [300 (FS income)/400 (WW income)] = 105 
\end{quotation}
Thus, USCO would receive a credit for all foreign taxes paid, even though taxes paid to the UK and to Japan are 
much higher than U.S. taxes on the UK and Japanese income.

The overall limitation allows high foreign taxes to be averaged with low foreign taxes, and as long as the overall rate is less than the taxpayer's U.S. rate, a FTC can be taken for all foreign taxes.  Between 1921 and 1986, U.S. taxpayers computed their FTC limit using overall limitation, although between 1954 and 1976, an election could be made to use the per-country limitation.  

In 1986, Congress enacted a highly complex basket system under which income was classified into one of nine baskets and a separate limitation was computed for each basket.  The baskets included passive income, high withholding tax interest, financial services income, shipping income, dividends from each non-controlled section 902 corporation, and all other (aka the ``general limitation'' basket) income.  \S904(d)(1) (repealed).  Within each basket, high taxed income could be averaged with low taxed income, but averaging across baskets was (and is)  not permitted.  It was (and is) therefore possible to have excess credits in one basket and excess limitation in another.  Certain baskets, however, were unlikely to ever have excess foreign credits.  The passive basket, for example, included income such as interest and dividends, but excluded interest income subject to a gross withholding tax of greater than 5\%.  

Because of the extreme complexity of the multiple basket foreign tax credit regime, Congress amended section 904(d) in 2004 by eliminating all of the baskets except two, the passive basket and the residual ``general category'' basket.  These rules are effective for post-'06 tax years.

The passive basket includes income that would be foreign personal holding company income and certain PFIC inclusions, but excludes export financing interest and high-taxed income (income taxed at a rate exceeding the highest rate in section 1 or 11).  \S954(d)(2)(B)(i), (ii), and (iii).  For persons in the financial services business, income that would otherwise be passive is treated as general category.  \S954(d)(2)(C) and (D).  All other income is general category income.

For U.S.-based multinationals, the heart of the foreign tax credit rules is found in section 904(d)(3).  Assume that a U.S. corporation does business directly in a foreign country and pays foreign taxes.  The income and taxes would be classified as either passive or general category depending on the nature of the income.  Now assume that the U.S. corporation does business abroad indirectly through foreign subsidiaries.  When the foreign income is distributed to the U.S. parent, it would be a dividend and thus fall automatically into the passive basket even though all of the income earned by the subsidiary was active business (general category) income.  

To ameliorate this problem, section 904(d)(3) provides that income such as subpart F inclusions, interest, rents, and royalties received from a CFC in which the taxpayer is a U.S. shareholder is treated as passive income only to the extent that it is attributable to the passive income of the CFC; otherwise, it's general category income.  \S904(d)(3)(A), (B), and (C).  Dividends received from a CFC are treated as passive in the same proportion as the E\&Ps of each category from which the dividend is paid bears to the total E\&Ps.  \S904(d)(3)(D).   (Note, a dividend from a CFC also includes the section 78 gross up.).  Dividends from non-controlled section 902 corporations (a corporation in which the shareholder owns between 10 and 50\% of the payor corporation) are treated as passive in the same proportion as the corporation's passive E\&Ps bear to the total E\&Ps.  \S904(d)(4).  Basically, section 904(d)(3) attempts to classify the income received from a CFC in the same basket it would have been had the U.S. parent corporation directly conducted the foreign business.  

\emph{Example}  CFC, a wholly owned subsidiary of DC, a U.S. corporation, earns \$200 of income of which \$85 is FBCSalI, \$15 FPHCI, and \$100 non-subpart F income.  P's subpart F inclusion is \$100, \$15 of which would be passive income and \$85 general category income.  
 
     \addcontentsline{toc}{section}{\protect\numberline{}Foreign Tax Credit Limitation Problems} 
	\begin{center}
		\textbf{Foreign Tax Credit Limitation Problems}
	\end{center}
	\begin{select}
	
\emph{For each of the questions below, consult sections 78, 901, 902, 903, and 904.  ``DC'' refers to a US corporation; ``FC1'' to a UK corporation wholly owned by DC; and  ``FS2'' to a UK corporation wholly owned by FC1.  Assume that DC's tax rate is 35\%.}  

	\begin{enumerate}
	
	\item In 2009, DC receives a 100 dividend from FC (a non-CFC foreign corporation in which DC owns 1\%) on which it pays no foreign tax.  DC also earns  200 from services performed in Switzerland on which it pays 100 in Swiss taxes.  Calculate DC's FTC(s) and any limitation(s).
	
		\item FC1 earns 200 from services performed in Germany on behalf of DC on which it pays 75 in German taxes.  In addition, FC1 sells real  property not used in business--vacant land producing no rent (see Reg. \S1.954-2(e)(3)(iii))--located in the U.K., realizing a gain of 200 and paying 85 of U.K. tax.  Calculate DC's FTC(s) and any limitation(s). 
 
 		\item In 2009, FC1 earns 100 of interest, 100 of FBCSalI, and 100 of non-subpart F manufacturing income.  It has post-'86 taxes of 100 and post-'86 E\&Ps of 400.  It pays a 100 dividend to DC in 2009.
	
		
		\end{enumerate}
		\end{select} 
 
