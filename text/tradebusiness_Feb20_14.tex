\chapter{The Taxation of Business Income of Foreign Persons} 
\section{Trade or Business}
\crt{864(b) and (c); 864(e)(2) and (3); 871(b); 882}{1.864-4(a), (c)(1)(i), (c)(2)(i), (c)(3); 1.864-5(a); 1.864-6(a), (b)(1); 1.864-7(a)(1), (d), and (e)}{Articles 5 and 7}

If a foreign person's activities in the United States rise to level of a trade or business, the United States taxes at graduated rates the income that is \textit{effectively connected} (ECI) with the trade or business.  \S\S 871(b) and 882(a).  Apart from the performance of services in the United States, which almost always constitutes a trade or business (\emph{see}, \S864(b)(2)), a foreign person will generally be considered to have a U.S. trade or business if his activities are \textit{considerable, continuous, and regular}.  \margit{A foreign person has a U.S. trade or business if his activities are considerable, continuous, and regular.}  \emph{See Pinchot v.\@ CIR, 113 F.2d 718 (2nd Cir. 1940)}.  

The tax stakes can be very high.  A foreign person not engaged in a U.S. trade or business is only taxed at flat rates on very limited categories of U.S. source income, namely, dividends and royalties.  Crossing the trade or business threshold potentially triggers graduated rates on all U.S. source income and even some foreign source income.  Note that a foreign person engaged in a U.S. trade or business is also allowed deductions in determining ECI.  Thus, in some circumstances, it can be advantageous for a foreign person's U.S. activities to constitute a trade or business in order to avail himself of effectively connected deductions. \S\S 873(c) and 882(c).  Determining at what point the trade or business threshold is crossed can be challenging.       

Congress provides an important statutory exclusion from being engaged in a U.S. trade or business for trading in securities or commodities in section 864(b)(2).  This provision ensures that foreign persons can trade in the United States an unlimited amount of stocks, bonds, and commodities, through resident brokers without those activities constituting a U.S. trade or business.  A foreign person who is not a dealer and is trading for his own account can even have an office with employees in the United States and not be engaged in a U.S. trade or business.  This is a very important exception, especially for attorneys practicing in New York.  One trade or business issue that has received a lot of attention over the last few years is whether hedge funds that acquire loans to U.S. persons, either at origination or pursuant to a workout, run afoul of the trading safe harbor.  This issue is explored below in Lee Sheppard, \emph{Neither a Dealer nor a lender Be, Part 4---Vulture Fund Self--Help}.

The rules for determining whether income is ECI are found in sections 864(c)(2) through 864(c)(7).  U.S. source FDAP income, including compensation, is ECI if: (1) it is derived from assets used in the U.S. trade or business; or (2) the activities of the business were a material factor in the realization of the income.  \S864(c)(2).  An example of U.S. source FDAP that would be ECI is interest on a U.S. bank account of a foreign corporation with a U.S. business.  \emph{See} Reg.\@ 1.864-4(c)(2)(v), Ex. 1.  If the U.S. source FDAP is not ECI, it will be taxed under the regular FDAP rules.	  

All other U.S. source income, such as income from the sale of goods in the United States, is ECI.  \S864(c)(3).  This rule, known as the \emph{force of attraction} rule, can have some surprising consequences.  Assume that a foreign business with no U.S. trade or business sends goods by mail to U.S. customers with title passing upon delivery to the customer's house.  If the foreign business were to open a U.S. business, all of the income from the mail order business would now become ECI under the force of attraction rule.  Most treaties prohibit the application of force-of-attraction principles in taxing permanent establishments.  

Some very narrow categories of foreign source income can also be ECI if the foreign person has a U.S. office or other fixed place of business and the income is \emph{attributable} to the U.S. office.  See  \S\S864(c)(4)(A) and (B) (categories of income) and 864(c)(5) (rules for determining whether income is attributable to the U.S. office).  Before you become too concerned that the treating foreign source income as ECI is a violation of some imaginary no-extraterriatorial-taxation norm, rest assured that the income has a strong nexus to the United States--the foreign taxpayer must have a U.S. office and the income must be attributable to the office.  In most of the cases, the income could have been treated as U.S. source.  

One area of great uncertainty is to what extent the activities of U.S. agents are imputed to a foreign person in determining whether the foreign person's U.S. activities rise to the level of a trade or business.  Apart from the statutory imputation of a partnership's activities to all of its partners, including limited partners, and from a trust and estate to all of the beneficiaries (\S875), the scope of agent imputation is found solely in often contradictory case law and administrative guidance.  This issue is explored below in the case of non-treaty residents in Chief Counsel Memorandum 2009-010 and for treaty residents in Rev.\@ Rul.\@ 2004--3.

Deferred income received when a foreigner no longer has a U.S. trade or business can be ECI if it would have been ECI had it been received when the foreigner had a U.S. trade or business.  \S864(c)(6).  A similar rule applies to property that is removed from a U.S. trade or business and later sold.  \S864(c)(7).  

Section 865(e)(2) contains a special sourcing rule that treats all income from the sale of personal property (including inventory) as U.S. source if the income is attributable to an office that the foreign person has in the U.S.  This rule can be an unpleasant surprise as it can potentially draw into the U.S. tax net income from activities, such as manufacturing, that occur outside of the United States.  Fortunately, the United States has not been keen to aggressively apply the statute. 

For residents of treaty countries, the United States can tax business profits only if the profits are attributable to a permanent establishment in the United States.  Article 7.  A permanent establishment generally requires some fixed place of business in the United States.  Article 5.  Like the case of being engaged in a U.S. trade or business, the difficult issues arise with respect to the activities of agents and the imputation of their activities to their principals.  The issue of imputation of activities of agents is addressed below in the context of treaties in the \emph{Handfield} and \emph{Taisei} cases.  
     
\addcontentsline{toc}{section}{\protect\numberline{}Pinchot v.\@ CIR}
\begin{select}
\caseart{Pinchot v.\@ CIR}{113 F.2d 718 (2nd Cir. 1940)}
{ CHASE, Circuit Judge}\\
This petition to review a decision of the Board of Tax 
Appeals presents primarily the question of whether or 
not a non-resident alien was engaged in business in this 
country at the time of her death within the meaning of 
Sec. 302(e) of the Revenue Act of 1926, 44 Stat. 9, 26 
U.S.C.A. Int. Rev.\@ Acts, p. 227, which provides that bank 
deposits of a non-resident not engaged in business at the 
time of death shall not be deemed property within the 
United States; and, secondarily, whether, if the decedent 
was then engaged in business here, her net estate for taxation should be determined by deducting the full amount 
of certain liens which the Board refused to deduct in full. 

The essential facts were stipulated and, so far as now 
important, are tha[t] the decedent, Antoinett Eno Johnstone, 
died July 1, 1934, a British subject and a non-resident. Much of her property in this country consisted 
of improved real estate in the City of New York owned in 
common by her and her two [b]rothers of whom one is her 
executor and the petitioner herein. This real estate was 
made up of eleven parcels of which the decedent's share 
had a gross value of about one million dollars. The petitioner, Amos R. E. Pinchot, managed the properties for 
her and the third owner under broad powers of attorney 
which included also the management of certain personal 
property owned by the three. He bought and sold property for the co-owners in his discretion without consult[i]ng 
the decedent who did not personally take pa[r]t in the transactions.  This management ``consisted of the leasing and 
renting of the properties when they became idle, collection of rents and payment of operating expenses, taxes, 
mortgage interest and other necessary obligations." Over 
a period of eighteen years five parcels of real estate had 
been sold and five had been purchased. There were no 
sales or purchases during the last three years before the 
decedent's death. 

Though the stipulation does not show the number or 
the amount of the transactions of the petitioner in 
managing these eleven buildings in New York, it is certain that they must have been considerable in both respects as well as continuous and regular. Their maintenance required the care and attention of the owners and 
the decedent supplied her part of that by means of her 
agent and attorney in fact. What was done was 
more than the investment and re-investment of funds in 
real estate. It was the management of the real estate itself 
for profit. Whether or not that was engaging in business 
within the meaning of federal tax statutes is a federal 
question which cannot be con[t]rolled by state decisions. 
It necessarily involved alterations 
and repairs commensurate with the value and number of
buildings cared for an[d] such transactions as were necessary 
constitute a recognized form of business. The management of real estate on such a scale for income producing 
purposes required regular and continuous activity of the 
kind which is commonly concerned with the employment 
of labor; the purchase of materials; the making of contracts; and many other things which come within 
the definition of business in Flint v.\@ Stone Tracy Co., 220 
U.S. 107, 31 S.Ct.\@ 342, 55 L.Ed. 389, Ann. Cas. 1912B, 
1312, and within the commonly accepted meaning of that 
word. We think the Board was right in deciding that this 
decedent was engaged in business in this country at the 
time of her death. The bank deposits in the United States 
were, therefore, properly treated as property in this country. Our decision in Higgins v.\@ CIR, 2 Cir., 
111 F.2d 795, did not touch the question of real estate 
management as a business. 

\ldots
%Nor does it follow, as the petitioner contends, that if 
%the decedent was engaged in business here her net taxable estate must be determined by valuing the property 
%in the ordinary manner under Treasury Regulations (80 
%T.R.Art. 13(4)). That is, the value of the decedent's share 
%in the real estate for purposes of taxation is not necessrily 
%the net value of her share of the equity. Congress has provided how the value of the net American esta[t]e of a non-resident shall be determined. A non-resident's estate situated here is to be 
%computed for taxation by deducting from the gross tha[t] 
%portion of the deductions allowed the estate of a resident 
%decedent which the value of such part bears to the total 
%gross estate, wherever situated, limi[t]ed in amount, however, to a sum not to exceed ten per centum of 
%the value of the gross estate situated in this country. The 
%petitioner has already been allowed a deduction to this 
%extent and that is all to which he is entitled. Whether 
%or not bank deposits are be treated as property in this 
%country does not control as to deductions but, instead, 
%that subject is governed by a separate statute in which 
%Congress has, as it might, made a separate classification. 
%The same limitation on deductions allowed estates 
%of non-residents is applicable to all such estates and does 
%not vary as they chance to be engaged, or not engaged, in 
%business here at the time of death. 

Affrmed.
\end{select}

\addcontentsline{toc}{section}{\protect\numberline{}Neill v.\@ CIR} 
\begin{select}
\caseart{Neill v.\@ CIR}{ 46 B.T.A. 197 (1942)}{Leech, Judge}

Respondent determined a deficiency in income tax in the amount of \$481.83 for the calendar year 
1938. The issues presented are (1) whether the income 
of petitioner, a nonresident alien individual, is to be computed under section [871(a)(1) or section 871(b)]\ldots  and, if under section [871(a)(1)] whether petitioner is taxable upon net income computed by deduction of mortgage interest paid and is entitled to a personal exemption of \$1,000. 
We find the facts as stipulated. Briefly, these are that 
petitioner is a nonresident alien. Her return for the calendar year 1938 was filed with the collector of internal 
revenue for the first district of Pennsylvania. Her 
income from sources within the United States consists of 
rentals paid by a tenant occupying property owned by her 
in Philadelphia, Pennsylvania. She inherited this property upon the death of her husband on October 26, 1937. 
It is held under a long term lease by a tenant who, under
the terms of that lease, erected a building thereon and is 
obligated under the lease to pay taxes and insurance and 
maintain the property. 

The property referred to is encumbered by a mortgage 
in the sum of \$50,000 executed by petitioner and her husband in 1931, the ground lease on the property having 
been assigned at that time to the mortgagee as collateral 
security for the mortgage. For many years petitioner has 
employed a firm of attorneys with offices in Philadelphia, 
to whom the tenant pays the rentals due petitioner under 
her direction. These attorneys then pay for her the interest 
due upon the mortgage and such incidental expenses for 
which petitioner may be obligated. 
During the calendar year 1938, rental in the sum of 
\$5,000.04 was paid by the tenant to petitioner's attorneys 
for her account. Of this amount the attorneys paid out 
\$3,360.90, interest due upon the mortgage, \$15 in insurance premiums, and miscellaneous expenses of \$8.28, or 
a total of \$3,384.18. The petitioner reported upon her 
return the net rent received after deducting these items, or 
\$1,615.86. Against this income she took credit for \$1,000 
as a personal exemption as a nonresident alien engaged 
in a trade or business within the United States. In determining the deficiency, the respondent has computed a tax 
due under section [871(a)(1)] at the rate of 10 percent on her gross income of \$5,000.04, 
disallowing the deductions of \$3,384.18 and the personal 
exemption of \$1,000. 

The issues are determined by whether petitioner is a 
nonresident alien carrying on business or having a place 
of business in the United States and therefore taxable on 
net income under section [871(b)]. If she is not, it is clear 
that her tax is to be computed on gross income under 
section [871(a)(1)]. 

Although admitting that she does not operate the 
building owned by her in Philadelphia, such operation being by the tenant, petitioner contends that such ownership 
constitutes the carrying on of business. We do not agree. 
The ownership of this property by petitioner is no more a 
business activity carried on within the United States than 
her ownership of stocks or bonds of American companies held for her by an American agent. We think the rule is settled 
that the mere ownership of property from which income 
is drawn does not constitute the carrying on of business 
within the purview of the cited section.  Petitioner contends, however, that, if it should 
be held that she is not engaged in business within the 
United States within section [871(b)], then the facts 
established show that she maintains an office there for 
the transaction of business, which is the office of her attorneys in Philadelphia. \ldots

%Upon this question petitioner is 
%concluded by our decisions in Aktiebolaget Separator, 45 
%B.T.A. 243, and Recherches Industrielles, S.A., 45 B.T.A. 
%253. 

%Petitioner also argues that since the mortgage was in 
%default in 1938 the mortgagor, who held the lease as security for its payment, was entitled to collect the rents 
%and retain, at least, the interest---that petitioner thus never 
%had a right to receive in the taxable year anything more 
%than the net rents after deduction of the contested interest. 
%Any question on this point is eliminated because no such 
%default has been established. 

%We hold that petitioner is taxable under section 
%[871(a)(1)], that her tax must be computed upon 
%the gross rentals received by her through her attorneys 
%in Philadelphia, and that she is not entitled to the deduction of mortgage interest or expenses, nor to the personal 
%exemption allowed to a nonresident alien engaged in business here. 

%Decision will be entered for the respondent.
\end{select}

Why was a trade or business found in \emph{Pinchot} but not in \emph{Neill}?  What are the main factual differences?  How much work did Pinchot do himself?  His agents?  Why should the services of agents be attributed to a principal but not the services of a tenant?

Certain activities carried on in the United States by a foreign person may not necessary rise to the level of being a U.S. trade or business if they are merely a clerical, ministerial, or auxiliary part of a trade or business carried on in another country. Consider, for example, the case of a European store owner who comes to the United States and purchases goods to sell in his store.  The purchase of goods in the United States, while an important part of the business, is only a part of the business and may not constitute a U.S. trade or business.  To have a U.S. trade or business may require that the main business activity occur in the United States.  In the example, marketing and selling are the main income generating activities of the business. \emph{See}, \emph{Scottish American Inv.\@ Co. v.\@ CIR}, 12 T.C. 49 (1949)  (foreign investment trust that maintained a U.S. office performing ``useful routine and incidental services" did not have U.S. trade or business because the business activities of the office were ``merely  helpfully adjunct").  \emph{See also} Article 5(4)(d) (maintenance of fixed place of business solely for the purpose of purchasing goods is not a PE).  

In \emph{U.S. v.\@ Balanovski}, why is Balanovski (and the partnership of which he was partner) found to have a U.S. trade or business?  Wasn't Balanovski merely a purchasing agent, albeit for expensive goods?  What would be the result in the case below if title to the goods had passed in Argentina?    


\addcontentsline{toc}{section}{\protect\numberline{}U.S. v.\@ Balanovski} 
\begin{select}
\caseart{U.S. v.\@ Balanovski}{236 F. 2d 298 (2nd Cir 1956)}{Clard, Circuit Judge}

\ldots

Defendants Balanovski and Horenstein were copartners in the argentine partnership, Compania Argentina de 
Intercambio Comercial (CADIC), Balanovski having an 
80 per cent interest and Horenstein, a 20 per cent interest. 
Balanovski, an Argentinian citizen, came to the United 
States on or about December 20, 1946, and remained in 
this country for approximately ten months, except for an 
absence of a few weeks in the spring of 1947 when he 
returned to Argentina. His purpose in coming here was 
the transaction of partnership business; and while here, 
he made extensive purchases and sales of trucks and other 
equipment resulting in a profit to the partnership of some 
\$7,763,702.20. 

His usual mode of operation in the United States was 
to contact American suppliers and obtain offers for the 
sale of equipment. He then communicated the offers to 
his father-in-law, Horenstein, in Argentina. Horenstein, 
in turn, submitted them at a markup to an agency of the 
Argentine Government, Instituto Argentino de 
Promocion del Intercambio (IAPI), which was interested 
in purchasing such equipment. If IAPI accepted an offer, 
Horenstein would notify Balanovski and the latter would 
accept the corresponding original offer of the American 
supplier. In the meantime IAPI would cause a letter of 
credit in favor of Balanovski to be opened with a New 
York bank. Acting under the terms of the letter of credit 
Balanovski would assign a portion of it, equal to CADIC's 
purchase price, to the United States supplier. The supplier could then draw on the New York bank against the 
letter of credit by sight draft for 100 per cent invoice 
value accompanied by (1) a commercial invoice billing 
Balanovski, (2) an inspection certificate, (3) a nonnegotiable warehouse or dock receipt issued in the name of the New York bank for the account of IAPI's Argentine 
agent, and (4) an insurance policy covering all risks to the
merchandise up to delivery F.O.B. New York City. Then, 
if the purchase was one on which CADIC was to receive 
a so-called quantity discount or commission, the supplier 
would pay Balanovski the amount of the discount. These 
discounts, paid after delivery of the goods and full payment to the suppliers, amounted to \$858,595.90, 
constituting funds which were delivered in the United 
States. 

After the supplier had received payment, Balanovski 
would draw on the New York bank for the unassigned 
portion of the letter of credit, less 1 per cent of the face 
amount, by submitting a sight draft accompanied by (1) 
a commercial invoice billing IAPI, (2) an undertaking to 
ship before a certain date, and (3) an insurance policy covering all risks to the merchandise up to delivery F.A.S. 
United States Sea Port. The bank would then deliver 
the nonnegotiable warehouse receipt that it had received 
from the supplier to Balanovski on trust receipt and his 
undertaking to deliver a full set of shipping documents, 
including a clean on board bill of lading issued 
to the order of IAPI's Argentine agent, with instructions 
to notify IAPI. It would also notify the warehouse that 
Balanovski was authorized to withdraw the merchandise. 
Upon delivery of these shipping documents to the New 
York bank Balanovski would receive the remaining 1 per 
cent due under the terms of the letter of credit. Although 
Balanovski arranged for shipping the goods to Argentina, 
IAPI paid shipping expenses and made its own 
arrangement there for marine insurance. The New York 
bank would forward the bill of lading, Balanovski's invoice billing IAPI, and the other documents required by 
the letter of credit (not including the supplier's invoice 
billing Balanovski) to IAPI's agent in Argentina. 

Twenty-four transactions following substantially this 
pattern took place during 1947. Other transactions were 
also effected which conformed to a substantially similar 
pattern, except that CADIC engaged the services of others 
to facilitate the acquisition of goods and their shipment 
to Argentina. And other offers were sent to Argentina, 
for which no letters of credit were opened. Several letters 
of credit were opened which remained either in whole or 
in part unused. In every instance of a completed transaction Balanovski was paid American money in New York, 
and in every instance he deposited it in his own name with 
New York banks. Balanovski never ordered material from 
a supplier for which he did not have an order and letter of 
credit from IAPI. 

Balanovski's activities on behalf of CADIC in the 
United States were numerous and varied and required the 
exercise of initiative, judgment, and executive responsibility. They far transcended the routine or merely 
clerical. Thus he conferred and bargained with American 
bankers. He inspected goods and made trips out of New 
York State in order to buy and inspect the equipment in 
which he was trading. He made sure the goods were 
placed in warehouses and aboard ship. He tried to insure that CADIC would not repeat the errors in supplying 
inferior equipment that had been made by some of its 
competitors. And while here he attempted `to develop' 
`other business' for CADIC. 

Throughout his stay in the United States Balanovski 
employed a Miss Alice Devine as a secretary. She used, 
and he used, the Hotel New Weston in New York City as 
an office. His address on the documents involved in the 
transactions was given as the Hotel New Weston. His supplier contacted him there, and that was the place where his 
letters were typed and his business appointments arranged 
and kept. Later Miss Devine opened an office on Rector 
Street in New York City, which he also used. When he 
returned to Argentina for a brief time in 1947 he left a 
power of attorney with Miss Devine. This gave her wide 
latitude in arranging for shipment of goods and 
in signing his name to all sorts of documents, including 
checks. When he left for Argentina again at the end of his 
10-month stay, he left with Miss Devine the same power 
of attorney, \ldots which she used throughout the balance of 
1947 to arrange for and complete the shipment of goods 
and bank the profits. 

When Balanovski left the United States in October 
1947 he filed a departing alien income tax return, on 
which he reported no income. In March 1948 the 
CIR of Internal Revenue assessed 
\$2,122,393.91 as taxes due on income for the period 
during which Balanovski was in the United States. In 
May 1953 the CIR made a jeopardy assessment 
against Balanovski in the amount of \$3,954,422.41 and 
gave him notice of it. At the same time a similar jeopardy 
assessment, followed by a timely notice of deficiency, was 
made against Horenstein in the amount of \$1,672,209.90, 
representing his alleged share of CADIC's profits on the 
abovedescribed sales of United States goods. 

\ldots 

\begin{center} \textbf{The Merits}
 \end{center}
The district court held that CADIC was not engaged 
in a trade or business within the United States within 
the meaning of [section 864], but that each of the partners was liable for certain 
taxes because Balanovski as an individual was so engaged 
in business and therefore taxable under [section 871(b)], while 
Horenstein received ``fixed or determinable annual or periodical gains, profits, and income'' within the meaning of 
[section 871(a)(1)(A)]. We, on the contrary, hold that the 
partnership CADIC was engaged in business in the United
States and that hence the two copartners were taxable for 
their share of its profits from sources within the United 
States. \ldots 

CADIC was actively and extensively engaged in business in the United States in 1947. Its 80 per cent partner, 
Balanovski, under whose hat 80 per cent of the business 
may be thought to reside, was in this country soliciting 
orders, inspecting merchandise, making purchases, and 
(as will later appear) completing sales. While maintaining regular contact with his home office, he 
was obviously making important business decisions. He 
maintained a bank account here for partnership funds. He 
operated from a New York office through which a major 
portion of CADIC's business was transacted. 
 
We cannot accept the view of the trial judge that, 
since Balanovski was a mere purchasing agent, his presence in this country was insufficient to justify a finding 
that CADIC was doing business in the United States. We 
need not consider the question whether, if Balanovski (an 
80 per cent partner) were merely engaged in purchasing 
goods here, the partnership could be deemed to be engaged in business, since he was doing more than purchasing. Acting for CADIC he engaged in numerous 
transactions wherein he both purchased and sold goods in 
this country, earned his profits here, and participated in 
other activities, pertaining to the transaction of business. 
Cases cited in support of the proposition that CADIC was 
not engaged in business here are quite distinguishable. 

As copartners of CADIC, Balanovski and Horenstein 
are taxable for the amount of partnership profits from 
sources within the United States under the statutory provisions cited above. The district court held them taxable 
only upon the `discounts' or `commissions' paid CADIC 
by the suppliers after completion of the sales transactions, 
not upon the total profits of the sales This solution of the 
problem is in seeming conflict with the usual rule that discounts received as inducements for quality purchasing are 
considered as reducing the purchasers' cost for tax 
purposes.  
Further, isolation of the discount from the sales transaction is not in accord with preferred accounting technique. Isolation of the discount for tax purposes would be more appropriate if the court considered 
the partnership as a broker receiving commissions, rather 
than as a vendor. But we need 
not consider whether the circumstances here justified the 
segregation for tax purposes of the discounts from the 
remainder of the sales profits for we hold the total profits on these 
transactions, including the discounts, to be taxable in full.
 
Under [sections 861(a)(6) and 863(b)], a nonresident alien engaged in business here derives 
income from the sale of personal property in `the country in which (the goods are) sold.' By the overwhelming 
weight of authority, goods are deemed `sold' within the 
statutory meaning when the seller performs the last act 
demanded of him to transfer ownership, and title passes to the buyer. 

Here, by deliberate act of the parties, title, or at least 
beneficial ownership, passed to IAPI in the United States. 
Under the letters of credit, Balanovski was paid in the 
United States and CADIC's last act to complete performance was done here. When Balanovski presented evidence of shipment---the clean ocean bill of lading made 
out to the account of an Argentine bank with the directive 
`Notify IAPI'---he had completed CADIC's work and 
he received the final 1 per cent of IAPI's contract price. 

The time when title to goods passes depends, of 
course, upon the intention of the parties.  When documents of title, such as a bill of lading, are given up, the presumption is that the seller has 
given up title, together with the documents. In F.O.B. and F.A.S. contracts 
there is a presumption that title passes from the seller 
just as soon as the goods are delivered to the carrier 
`free on board' or `free alongside' the ship, as the case 
may be. Both of these presumptions, which would tend to establish that title passed 
from CADIC to IAPI in the United States, are not altered 
by the use of a letter of credit. Nor need we here consider whether 
more than `beneficial' title passed immediately to IAPI or 
whether a `security' or `legal' title rested with the intermediary bank. 
 
All the available evidence confirms, rather than rebuts, these presumptions of passage of title in the United 
States. All risk of loss passed before the ocean voyage. 
IAPI took out the marine insurance. CADIC performed 
all acts to complete the transaction, retained no control of 
the goods, and there was no possibility of withdrawal. 

Judge Palmieri apparently did not contest that title to 
the goods passed in the United States. But he applied a 
test based upon the `substance of the transaction' to hold 
that Argentina was the place where the income-producing 
contracts were negotiated and concluded, the place of the 
buyer's business, and the destination of the goods. This 
led him to conclude that Argentina, rather than 
the United States, was the place of sale. The judge further buttressed this result by observing that IAPI, rather than CADIC, had insisted upon the passing of title in the 
United States.
 
Although the `passage of title' rule may be subject to 
criticism on the grounds that it may impose inequitable 
tax burdens upon taxpayers engaged in substantially similar transactions, such as upon exporters whose customers 
require that property in the goods pass in the United 
States \ldots no suitable 
substitute test providing an adequate degree of certainty 
for taxpayers has been proposed. \ldots 

\ldots

Of course this test may present problems, as where 
passage of title is formally delayed to avoid taxes. \ldots 
Hence it is not necessary, nor is it desirable, to require 
rigid adherence to this test under all circumstances. But the rule does provide for a certainty and ease 
of application desirable in international trade. \ldots Where, 
as here, it appears to accord with the economic realities 
(since these profits flowed from transactions engineered in major part within the United States), we see 
no reason to depart from it. \ldots Hence we hold that the 
partners are liable for taxes on the entire profits of the 
partnership sales amounting to \$7,763,702.20. 

\ldots

\end{select}

In \emph{Neill}, we saw that mere ownership of property is not enough for a foreign person to be considered to be engaged in a U.S. trade or business, if the lessee assumed many of the normal obligations of property ownership, such as maintenance, upkeep and repairs, and payment of taxes.  Rev.\@ Rul.\@ 73-522 extends the holding of \emph{Neill} to property rented out under net lease agreements whereby the lessee undertakes to pay real estate taxes, operating expenses, repairs, interest and principal on existing mortgages and insurance.  Notice how the IRS refused to treat 227 leasing negotiations on the properties as sufficient activity to constitute a trade or business.   

\addcontentsline{toc}{section}{\protect\numberline{}Rev.\@ Rul.\@ 73-522}
\begin{select}
\revrul{Rev.\@ Rul.\@ 73-522}{1973-2 C.B. 226}

\ldots

Advice has been requested whether a nonresident alien individual is considered to be engaged in trade or business 
within the United States during the taxable year, within the meaning of section 871 of the Internal Revenue Code of 1954, 
under the circumstances described below. Advice has also been requested whether the term ``rents,'' as used in section 871, 
includes considerations other than the payment of a stipulated rental, i.e., payment of taxes, repairs, etc., by the lessee 
described below.
 
The taxpayer, a nonresident alien individual who has not elected to treat real property income as income effectively 
connected with the conduct of a trade or business within the United States pursuant to section 871(d) of the Code, did 
not, except as described below, engage in any activity within the United States during the taxable year ended December 
31, 1971. 

The taxpayer owned rental property situated in the United States that was subject to long-term leases each providing 
for a minimum monthly rental and the payment by the lessee of real estate taxes, operating expenses, ground rent, repairs, 
interest and principal on existing mortgages, and insurance in connection with the property leased. The leases are referred 
to as ``net leases'' and were entered into by the taxpayer on December 1, 1971. The taxpayer visited the United States for 
approximately one week during November 1971 for the purpose of supervising new leasing 227 negotiations, attending 
conferences, making phone calls, drafting documents, and making significant decisions with respect to the leases. This 
was his only visit to the United States in 1971. The leases were identical in form (net leases) to those applicable to the 
properties owned by the taxpayer prior to December 1, 1971, and were entered into with lessees unrelated to each other or to the taxpayer. 

Section 871(a)(1) of the Code imposes for each year a tax of 30 percent of the amount received from sources within 
the United States by a nonresident alien individual as income in the form of items enumerated, but only to the extent that 
the amount so received is not effectively connected with the conduct of a trade or business within the United States. 

Section 871(b)(1) of the Code provides for the imposition of a tax on a nonresident alien individual engaged in trade 
or business in the United States during the taxable year as provided in section 1 or 1201(b) on his taxable income that is 
effectively connected with the conduct of a trade or business within the United States. 

Court decisions involving nonresident alien individual owners of real estate in the United States have developed a 
test for determining when such individuals are engaged in trade or business within the United States as a result of such 
ownership. These cases hold that activity of nonresident alien individuals (or their agents) in connection with domestic real estate that is beyond the mere receipt of income from rented property, and the payment of expenses incidental to the 
collection thereof, places the owner in a trade or business within the United States, provided that such activity is 
considerable, continuous, and regular. 

In the instant case the taxpayer's only activity in the United States during the taxable year ended December 31, 1971, 
was the supervision of the negotiation of leases covering rental property that he owned during that year. No other acitvity 
was necessary on the part of the lessor in connection with the properties because of the provisions of the net leases. The 
taxpayer's supervision of the negotiation of new leases is not considered to be beyond the scope of mere ownership of real 
property or the mere receipt of income from real property since such activity was sporadic rather than continuous (that is 
a day-to-day activity), irregular rather than regular, and minimal rather than considerable. 

Accordingly, the taxpayer in the instant case is not considered to be engaged in trade or business within the 
United States during the taxable year ended December 31, 1971, within the meaning of section 871 of the Code. See 
Evelyn M. L. Neil, 46 B.T.A. 197 (1942), wherein the operation of one parcel of real estate by the lessee did not result in 
the owner being considered to be engaged in trade or business. Compare Adolph Schwarcz, 24 T.C. 733, acq. 1956-1, 
C.B. 5, wherein an owner operating one parcel of rental property in all its aspects was considered to be engaging in trade 
or business. 

With regard to the second question presented, section 1.871-7(b)(1) of the Income Tax Regulations provides that for 
purposes of section 871(a)(1) of the Code ``amounts'' received (including rents) means ``gross income.'' Section 1.61-8(c), 
to the extent pertinent, provides that if a lessee pays any of the expenses of the lessor such payments are additional rental 
income of the lessor. 

Accordingly, ``rents,'' as used in section 871 of the Code, includes considerations other than the payment of a stipulated 
rental, i.e., amounts paid by the lessee for taxes, repairs, etc., in accordance with the terms of a net lease.
\end{select}

For real estate operations like those in Rev.\@ Rul.\@ 73-522, being taxed at a flat rate on the rental income produced by the real estate can be confiscatory as the tax could easily exceed the net profits generated by the property due to expenses such as interest, taxes, and insurance.  Sections 871(d) and 882(d) provide that a foreign person can elect to treat real estate income as ECI and thereby be taxed only on the net profits of the real estate.  Rev.\@ Rul.\@ 91-7 reminds taxpayers that in order to make the ECI election, the real estate must produce income, which can be a problem if the real estate is vacant property.  What planning ideas would you suggest to avoid this result?    

\addcontentsline{toc}{section}{\protect\numberline{}Rev.\@ Rul.\@ 91-7}
\begin{select}
\revrul{Rev.\@ Rul.\@ 91-7}{1991-1 C.B. 110}

\ldots \\
(1) May a nonresident alien or a foreign corporation make an election under section 871(d) or 882(d) of the Internal 
Revenue Code for a taxable year in which the taxpayer does not derive income from U.S. real property. 

(2) May a nonresident alien or a foreign corporation make an election under section 266 of the Code to capitalize real 
estate taxes, mortgage interest, and other carrying charges attributable to unimproved and unproductive real property for 
a taxable year in which it may not deduct such expenses under section 873(a) or 882(c). 

\begin{center} \textbf{FACTS}
\end{center} 
A, a nonresident alien individual, and FC, a foreign corporation, co-own parcel P, unimproved real estate located in 
the United States. Parcel P is held for investment purposes. During the 1990 taxable year, A and FC do not derive any 
income from parcel P. A and FC annually pay real estate taxes, mortgage interest, and other carrying charges connected 
with parcel P. Neither A nor FC has made an election under section 871(d) or section 882(d) of the Code with respect 
to parcel P in any previous taxable year. 
\begin{center} \textbf{LAW AND ANALYSIS}
\end{center} 
Nonresident alien individuals and foreign corporations are subject to taxation at a rate of 30 percent on certain types 
of U.S. source gross income (including rent) which is not effectively connected with the conduct of a trade or business 
within the United States (ECI). See sections 871(a) and 881(a) of the Code. 
Nonresident alien individuals or foreign corporations that are engaged in a U.S. trade or business during the taxable 
year are subject to taxation under sections 871(b) and 882(a) of the Code on their taxable income which is ECI. Sections 
873(a) and 882(c) allow deductions to nonresident alien individuals and foreign corporations only if and to the extent that 
the deductions are connected with the ECI of such persons. 
Nonresident alien individuals and foreign corporations that, during a taxable year, derive from U.S. real property gross 
income which is not ECI may elect for such taxable year to treat such income as if it were ECI. See sections 871(d) and 
882(d) of the Code. Income subject to the election is taxable as ECI under section 871(b) or section 882(a), and is 
not taxable under section 871(a) or section 881(a). 

Because A and FC do not derive any income from parcel P during the 1990 taxable year, neither taxpayer can make 
an election for such year under section 871(d) or section 882(d) of the Code. Sections 1.871-10(a) and 1.882-2(a) of the
Income Tax Regulations. Under sections 873(a) and 882(c), A and FC are not allowed any deductions in the 1990 taxable 
year for the real estate taxes, mortgage interest, or other carrying charges paid during such taxable year, because such 
amounts are not connected with ECI. 

Section 1.1016-(c) of the regulations provides that adjustments to basis shall be made for taxes and other carrying 
charges which the taxpayer elects, under section 266 of the Code, to treat as chargeable to capital account, rather than as 
an allowable deduction. Under section 1.266-1(b)(1)(i) of the regulations, annual taxes, interest on a mortgage, and other 
carrying charges on unimproved and unproductive real property, which are otherwise expressly deductible under Subtitle 
A of the Code, may be capitalized at the election of the taxpayer. An item not otherwise deductible may not be 
capitalized under section 266. See section 1.266-1(b)(2). 

Because the real estate taxes and mortgage interest incurred during the 1990 taxable year on parcel P are not allowable deductions for A and FC for such taxable year, those expenses cannot be capitalized under section 266 of the Code by adding the amount of the expenses to the basis of parcel P. 
\begin{center} \textbf{HOLDINGS}
\end{center} 

(1) A nonresident alien individual or foreign corporation may not make an election under section 871(d) or 882(d) of 
the Code for a taxable year in which the foreign taxpayer does not derive income from U.S. real property.
 
(2) A nonresident alien individual or foreign corporation may not make an election under section 266 of the Code to 
capitalize real estate taxes, mortgage interest, and other carrying charges attributable to unimproved and unproductive 
U.S. real property if, during the taxable year in which such expenses are incurred, such expenses are not allowable 
deductions under section 873(a) or 882(c). 
\end{select}

The next article by tax journalist Lee Sheppard explores the scope of the trading exception as applied to the activities of hedge funds in directly and indirectly supplying capital to the U.S. debt and equity markets.  

\addcontentsline{toc}{section}{\protect\numberline{}Lee Sheppard, Neither a Dealer Nor a lender Be, Part 4}
\begin{select}
\revrul{Lee Sheppard, Neither a Dealer Nor a Lender Be, Part 4--Vulture Fund Self-Help}{2008 TNT 220-5 (Nov.\@ 13, 2008)}

All but a few hedge funds are experiencing their fifth straight month of 
decline, and investors are heading for the exits. Most of the funds that are 
hurting are long/short equity funds. A few vulture funds are thriving. 

The bad news for hedge funds is showing up at retail. Luxury department 
stores are reporting sales declines as rich folk have stopped their 
excessive shopping. But when the wolf is at the door, love flies out the 
window. One part of retail sales that appears to be holding up---pun fully 
intended---is the Agent Provocateur corner of Bloomingdale's famous 
lingerie department. 

Agent Provocateur is a British purveyor of trashy, high-camp intimate 
apparel that looks like Madonna's old stage clothes, as well as other 
accoutrements. Salesclerks dress in retro maid's outfits and high heels. 
The principal customers for this stuff, which is expensive, appear to be 
straight-laced couples anxious to revive their love lives. Agent 
Provocateur's basic message is puritanical---sex is forbidden and dirty.

No wonder it sells in America. 

Hedge fund managers who are finding no joy in bustiers and garter belts 
are hoping instead to revive distressed debt, which they are scooping up 
at prices so low that it is no wonder the Treasury Department's plan to buy 
up bad assets failed. Meanwhile, the Federal Reserve has lent \$2 trillion, 
outside of the bailout program, to market participants whose identities it 
refuses to disclose to Congress. (Bloomberg, Nov.\@ 10, 2008.) 

The owners of the bad assets do not want to acknowledge values that 
make Merrill Lynch's 22 cents on the dollar look rich. Hedge funds are 
paying 20 cents on the dollar for loans that are still performing, and three--quarters of a cent on the dollar for defaulted loans, according to Stephen 
Land of Linklaters, who spoke at the packed November 11 International 
Tax Institute session in New York City. 

The day's theme was cross-border treatment of distressed debt 
acquisitions and workouts, and the other speaker was David Miller of 
Cadwalader, Wickersham \& Taft. The pair explained that funds were 
taking somewhat aggressive positions on questions for which there are no 
answers, or answers that are deemed unfair.
 
\ldots
%Chief among the latter is the requirement that market discount be accrued 
%on defaulted debt. (For discussion, see Doc 2008-11209 [PDF] or 2008 
%TNT 102-13 . For the New York City Bar Association report, see Doc 
%2008-16478 [PDF] or 2008 TNT 145-58 . For the New York State Bar 
%Association Tax Section letter, see Doc 2008-18051 [PDF] or 2008 TNT 
%162-14.) 

%As the market discount rules literally apply, if some loans in a large 
%portfolio recover, the holder has an asymmetric result of ordinary income 
%on them and capital loss on the rest that never recover. Land explained 
%that some funds are arguing for open transaction treatment based on old 
%case law. Some even argue that open transaction treatment can be 
%applied to an entire pool of bad loans, rather than on a per-loan basis, by 
%analogy to section 1272(a)(6)(C)(iii), a special rule that permits pooled
%reporting for mortgages subject to prepayment. 

\begin{center}
\textbf{Securities Trading}
\end{center}
 
Under section 864(b)(2), a foreign investor is not engaged in a U.S. trade 
or business if the investor: (1) trades in stocks or securities through a 
resident broker, commission agent, custodian, or other independent 
agent, or (2) trades in stocks or securities for the investor's own account. 

For years the unanswered question has been whether explicit loan 
origination by hedge funds, which bought loans from banks before the ink 
was dry and helped negotiate loans they were about to buy, constitutes a 
U.S. trade or business. Miller was dubious about the application of the 
publicly traded partnership ruling that hedge fund advisers are fond of 
relying on by analogy to argue that making a few loans does not create a 
U.S. trade or business. (Ltr 9701006, Doc 97-406 or 97 TNT 3-64 .) 

Now that loans are going bad, the question has moved to the deemed 
origination resulting from a workout, Miller explained. When a loan is 
substantially modified so that section 1001 requires recognition, a new 
loan is deemed to have been exchanged for the original loan. (Reg.\@ 
section 1.1001-3.) 

Does that deemed origination put a hedge fund in a U.S. trade or 
business? Miller posited four commonplace scenarios. The most obvious 
scenario is the vulture fund that buys, holds, and disposes of bad 
mortgages. It renegotiates with borrowers, causing recognition events 
under section 1001, and then sells the modified loans in a bulk sale. This 
would clearly be a U.S. trade or business. 

A similarly easy case of creation of a U.S. trade or business is when a 
vulture fund buys dodgy corporate debt and renegotiates it either 
one-on-one or as a member of the creditors' committee in a bankruptcy. 
Section 1001 recognition occurs when the debt is modified and replaced 
by new debt. But when the old debt is replaced by equity, Miller noted,
that exchange would not give rise to a U.S. trade or business. 

What if the fund was a portfolio investor that did not expect the debt to go 
bad? What if the debt was purchased with no expectation of default, but 
bad things happened and the fund joined the creditors' committee to 
negotiate an exchange requiring recognition under section 1001? Or what 
if the fund, finding itself in the same position, passively stayed away from 
the negotiations but consented to the substantial modifications that were 
negotiated? 

These latter two cases should not give rise to a U.S. trade or business, in 
Miller's view, because it could be argued that they are the natural 
consequences of investing in a debt security that goes into the toilet. It 
could be argued that the concept of holding a security for appreciation 
should encompass the situation when a foreign investor's portfolio 
manager cuts a good deal on a debt restructuring. Miller saw these two 
cases as funds merely acting in ways that protected their investments. 

Miller further argued that the legislative history of section 864(b)(2) 
indicated that Congress was primarily concerned about dealers posing as 
traders to avoid effectively connected income. A workout or deemed 
origination arising from a workout does not make any participant a dealer, 
Miller noted. Moreover, the word ``note" used in the regulations' definition 
of ``security" had to encompass some sort of origination, because there 
was no secondary market when the regulation was written. (Reg.\@ section 
1.864-2(c)(2).) This argument would shield all workouts. 

The government could argue that the profits from loan origination and 
deemed origination under section 1001 arise from labor-intensive 
activities conducted in the United States. Labor--intensive activities 
conducted in the United States demonstrate the existence of a U.S. trade 
or business that is not protected by section 864(b)(2). A trader's return, by 
contrast, should be attributable to the passive receipt of income and 
market gain--all that section 864(b)(2) was intended to cover. (For 
discussion, see Doc 2008-14453 [PDF] or 2008 TNT 127-3.) But hedge 
fund managers might argue that portfolio management is also labor--intensive, Miller noted. 

Miller acknowledged other arguments that debt workouts create a U.S. 
trade or business. The statute, in section 864(b)(2)(A)(ii), clearly restricts 
the securities--trading safe harbor to trading for one's own account. And 
there would be a question of competition with U.S. banks if foreign 
investors were allowed to negotiate freely. 

Section 864(b)(2)(A)(i) is more restrictive, applying only when the investor 
has no fixed place of business in the United States. Some funds go so far 
as to put the workout manager in London, having him negotiate on the 
phone. Using the phone, however, is impractical when the bad loans are 
mortgages.
 
Hedge fund lobbyists asked Treasury to explicitly excuse their foreign 
investors from effectively connected income on acquiring and working out 
dodgy loans. This effort failed, as wiser heads at Treasury decided the 
headline ``Cayman Hedge Fund Forecloses on Grandma's Homestead'' 
was not something they wanted to see above the fold. (For discussion, 
see Doc 2008-8576 [PDF]or 2008 TNT 75-5 .) 

In London, Agent Provocateur's home base, a foreign investor can use an 
independent agent to negotiate and originate loans without incurring 
British tax provided the agent is acting in the ordinary course of its 
business, receives customary compensation, and is entitled to no more 
than 20 percent of the investor's profit from the loans. (Finance Act 2003, 
schedule 26.) 

Some have argued for a similar safe harbor for the United States. Miller 
cautions that such a rule could allow U.S. branches of foreign banks to 
avoid U.S. tax on what is clearly a U.S. trade or business of lending. 
Foreign banks are already able to keep purchased loans out of their U.S. 
tax base by fobbing them off to foreign hedge funds. Moreover, as bitter 
experience with transfer pricing has taught, it is hard to tell when an agent 
is independent, and when the income allocated to the agent is correct.

\ldots

%\begin{center}
%\textbf{REMICs}
%\end{center} 
%Normally discussion of real estate mortgage investment conduits is about 
%as exciting as marital sex. But REMICs have donned Agent Provocateur's 
%frilly knickers lately, as hedge fund advisers have figured out that they are 
%good vehicles for bad debt, given the special provisions applicable to 
%REMICs. Some very large transactions have been done with REMICs as 
%workout vehicles. 

%As ever, the impetus of these advisers is to prevent foreign investors from 
%being deemed to be in a U.S. trade or business. Advisers are using 
%REMICs to shield foreign investors from the tax ramifications of their fund 
%managers' actions. A ``bucket" structure sees the fund putting the bad 
%loans in a parallel fund that features a blocker corporation between it and 
%the foreign investors. The blocker's effectively connected income would 
%be stripped out by means of a related-party loan.
% 
%A more sophisticated structure, Miller explained, has the hedge fund 
%owning regular interests in a REMIC to which bad loans have been 
%contributed. A REMIC regular interest is considered a debt by statute, so 
%that a foreign investor cannot be deemed to be in a trade or business by 
%virtue of the REMIC renegotiating or foreclosing on a loan. The residual 
%interest in a REMIC, which would be held by a financial intermediary, is 
%considered equity by statute, and could drag its holder into a trade or 
%business. (Sections 860B(a), 860C.) 

%The REMIC rules have special provisions for dealing with loans in the 
%static mortgage pool that go bad. But the big question is whether a bad 
%loan, or one that is about to go bad, can be put into a REMIC in the first 
%place. The IRS seems to think not. Miller quoted our article reporting 
%remarks made by IRS officials at the ABA Tax Section meeting in San 
%Francisco in September to the effect that they did not want to see 
%REMICs used as workout factories. (For prior coverage, see Doc 
%2008-19876 [PDF] or 2008 TNT 182-3.)

%Miller takes this to mean that the IRS thinks bad loans should not go into 
%REMICs. That is, bad loans are not ``permitted investments'' in the 
%statutory parlance. But the statute, he argues, does not prevent it. Section 
%860G(a)(1)(A) says that the holders of REMIC regular interests must be 
%unconditionally promised a specified principal amount, with fixed interest 
%or interest passed through from mortgages. 

%But the flush language of section 860G(a) permits variance due to 
%prepayments from mortgages. The regulations allow a regular interest to 
%be subject to some contingencies, among which is disruption of payments 
%of principal or interest due to credit losses from defaults on mortgages, 
%unanticipated expenses, and lower-than-expected returns on investments. 
%(Reg.\@ section 1.860G-1(b)(3)(i).) 

%If, however, a REMIC acquires a bad loan (here assuming that a bad loan 
%is a permitted asset) and forecloses, the real estate so acquired would not 
%be a permitted investment, so it must immediately be sold to avoid a 100 
%percent excise tax. If a REMIC has more than a tiny percentage of 
%impermissible investments in its portfolio, it would lose its REMIC status. 
%So fund managers set up a corporation to purchase foreclosed houses, 
%funding it with loans, Miller explained. (Sections 860D(a)(4), 860F(a).) 

\end{select}


%\addcontentsline{toc}{section}{\protect\numberline{}Lee Sheppard, Neither a Dealer Nor a lender Be, Part 2}
%\begin{select}
%\revrul{Lee Sheppard, Neither a Dealer Nor a lender Be, Part 2:  Hedge Fund Lending}{2005 TNT 157-3 (Aug. 16, 2005)}

%A majority interest in the Manchester United Football Club was recently 
%sold to American investor Malcolm Glazer, who also owns the Tampa Bay 
%Buccaneers. For all the gnashing of teeth and rending of garments that 
%occurred in Blighty, one would think that the company being sold was a 
%piece of the national patrimony rather than a profitable, publicly traded 
%entertainment enterprise. 

%Man U's fans, who own 17 percent of the shares, are still protesting 
%outside Old Trafford, and more importantly, boycotting ticket sales. Glazer, 
%who during the acquisition process had been burned in effigy, had to 
%promise up and down that nothing would change, including the manager, 
%who gratefully explained to fans in program notes that the team is a 
%publicly traded company. 

%Man U is changing, as it must, as the two Irishmen who controlled the 
%majority of the shares appear to have recognized. Amid all the bleating,
%very little attention was paid to why the pair might be motivated to sell. 
%Our theory is that the Irishmen didn't want to have to fire manager Sir Alex 
%Ferguson, who is their buddy. The next owner will have to. Why? Because 
%the team has gone as far as it's going to go under Ferguson, whose style 
%of both football and management has gone by the boards. Martin O'Neill 
%is waiting by the phone. 

%The Man U team that racked up a decade's worth of trophies was a 
%homegrown, mostly English team that played English long-ball football 
%better than the bottom 15 teams in England's Premier League. It was a 
%team that was disciplined and consistent, and that had unshakeable 
%confidence. Most days, it was not a team that could beat a good 
%European outfit, but that was not necessary to prevail in England. So that 
%formula worked until the European invasion of English football. 

%Mostly French Arsenal, managed by a cool-headed French intellectual, 
%started playing consistently and won the domestic league twice. Then a 
%Russian billionaire poured money into Chelsea, building a European 
%powerhouse in West London that romped over the others to win the 
%domestic league handily. And a Spanish intellectual took over Liverpool, 
%recreating Valencia on the Mersey and winning the Champions League. 
%So it's been three years since Man U won the Premier League. Doesn't 
%sound so bad, except that the team's yuppie dilettante fans fully expect it 
%to win the domestic league every year. 

%There has also been much bleating about the fact that Glazer borrowed 
%the money to make the purchase. Now, we like to think of bankers as 
%sober types who don't throw money at uncertain propositions like 
%entertainment enterprises in which the talent competes with the owners 
%for the profits. Nonetheless, some banks did lend money to Glazer. Who 
%else lent money to Glazer? Three American hedge funds, Citadel 
%Investment Group, Perry Capital, and Och-Ziff Capital Management, lent 
%275 million pounds of the 790 million pounds that Glazer paid for the 
%team. 

%By so aggressively jumping into lending, hedge funds, which are running
%short of new investment and arbitrage opportunities, may be making a 
%tactical mistake that would be highly detrimental to their goal of escaping 
%U.S. taxation as nonresident investors. They should not be indulged as 
%nonresidents, and America's stupid residence rules should be changed, 
%but that is not the subject of this article. For purposes of this article, we 
%assume that hedge funds are the nonresidents that they claim to be. 

%Hedge funds may be lending too much, and taking the risks of loans too 
%much, putting themselves in the active trade or business of lending in the 
%United States, causing them to have effectively connected income taxable 
%in the United States. As this article will discuss, the threshold for being 
%considered to be in the trade or business of lending in the United States is 
%not high, and the rules that hedge funds depend on to avoid U.S. taxation 
%on their other activities will not protect their lending. 

%Basically, the law says the opposite of what hedge funds want it to say. 
%The law says that a lending business in the United States can have the 
%effect of making securities trading an additional trade or business, but not 
%that lending can be an adjunct to the securities trading safe harbor. 

%\begin{center}
%\textbf{The Ten Commandments}
%\end{center} 
%A hedge fund is an unregulated investment partnership that, though 
%managed out of Greenwich, Conn., or New York City or some other 
%domestic location, claims to be nonresident merely by virtue of having 
%been organized as a corporation in a tax and banking haven. 
%Management is carried on in the United States, while records may be 
%maintained offshore. (There are many hedge funds run out of London, but 
%we are concerned with the American ones.) Hedge funds' nonresident 
%status is tolerated because of foolish American residence rules that define 
%a partnership as resident where it is organized, rather than where it is 
%managed and controlled. \ldots

%Hedge funds typically organize the partnership that constitutes the fund
%itself in a tax haven. The management company, another related 
%partnership, operates in the United States. If the activities of the 
%management company are confined to investing and trading for the fund's 
%own account, the fund will not be considered to have a trade or business 
%in the United States because of the securities trading safe harbor. The 
%management company would be the fund's dependent agent and office in 
%the United States. The securities trading safe harbor excuses those 
%factors, but they remain highly relevant for other trade or business 
%questions, as this article will show. 

%Foreign investors are taxed on income from U.S. sources that is 
%effectively connected with a trade or business within the United States. 
%What is a "trade or business within the United States" is defined in section 
%864(b) by excluding certain activities. Otherwise, what constitutes a trade 
%or business is defined by the case law. 

%Section 864(b)(2)(A)(ii) makes an exception for trading in securities and 
%commodities for one's own account. The objective of the exception, 
%installed in 1966, was to encourage foreign portfolio investment in the 
%United States by allowing foreign investors to pool their money and hire 
%American managers. The exception's main effect is to absolve foreign 
%investors from tax on U.S.-source capital gains, since interest and 
%dividends are subject to withholding, albeit at reduced treaty rates. 

%The legislative history of the securities trading safe harbor shows that 
%Congress, rather than wanting to attract foreign capital, had simply given 
%up on collecting tax on capital gains realized by foreign investors, and had 
%decided that making brokers rich would be a good way to make up the 
%lost revenue. That is why foreigners have to transact through a resident 
%broker. Said the Senate:
% 
%\begin{quote}
%[A] nonresident alien will not be subject to the tax on capital gains, 
%including so-called gains from hedging transactions, as at present, it 
%having been found administratively impossible effectually to collect 
%this latter tax. It is believed this exemption from tax will result in
%considerable additional revenue from the transfer taxes and from the 
%income tax in the case of persons carrying on the brokerage 
%business. (Senate Report No. 2156, 74th Cong., 2d sess., p. 21.)
%\end{quote}
% 
%Hedge funds are in a state of constant argument with the tax administrator 
%over whether various forms of income they receive is effectively 
%connected. They depend utterly on the section 864(b)(2)(A)(ii) safe harbor 
%for trading for one's own account, an important part of the longstanding 
%policy of being very nice to foreign investors. 

%The standards for the securities trading safe harbor were significantly 
%loosened in 1997 to permit maintenance of a principal office in the United 
%States without tax consequences. Before its liberalization, the securities 
%trading safe harbor was called ``the Ten Commandments'' because it had 
%10 criteria for staying out of U.S. tax jurisdiction, among which was that 
%the investor could not have an office. Foreign investors found that those 
%criteria were inconvenient and cramping their activities, so the Clinton 
%administration and Congress obliged. (Doc 97-25873, 97 TNT 178-8 .) 

%The regulations were not amended to reflect the legislative change. 
%According to reg. section 1.864-2(c)(2) and proposed reg. section 
%1.864(b)-1, trading in securities encompasses all sorts of active portfolio 
%management, but it does not include the active business of lending. 
%``Securities'' are broadly defined to include loans under reg. section 
%1.864-2(c)(2)(i)(c). That means that a hedge fund can buy loans or pieces 
%of loans after someone else has made them. The question becomes 
%whether the early entry of a hedge fund into a loan syndicate or a credit 
%derivative is an extension of credit or an assumption of credit risk that is 
%tantamount to making a loan. \ldots

%Credit derivatives are part of the debate about how much lending a hedge 
%fund can do without being engaged in a lending trade or business in the 
%United States. Proposed reg. section 1.864(b)-1 would provide that 
%foreign taxpayers who enter into derivative contracts for their own account
%are not engaged in a trade or business in the United States solely by 
%reason of those contracts, provided that they are not dealers in derivatives 
%or otherwise dealers in securities or commodities. Entering into 
%derivatives contracts (including hedging) would not constitute a U.S. trade 
%or business if the taxpayer met the definition of an eligible nondealer. 
%Given the broad definition of derivatives, that would be a sweeping rule. 
%The proposed regulation tries to shoehorn derivatives into the trading safe 
%harbor by analogizing them to the underlying securities and commodities. 
%Credit derivatives, under this approach, would logically be analogous to 
%loans. Prop. reg. section 1.864(b)- 1(b)(2)(ii)(E) would define derivatives 
%to include an evidence of an interest or derivative contract in any note, 
%bond, debenture, or other evidence of indebtedness. 

%As the following discussion will show, hedge funds essentially want to 
%stretch the securities trading safe harbor to cover all of their dabbling in 
%lending, beyond credit derivatives. The law doesn't permit that, but that 
%doesn't stop tax practitioners from making the argument. The law 
%basically says that if hedge funds go into a trade or business of lending, 
%all of their securities trading income would become effectively connected 
%with a lending business or a securities trading business, with the effect 
%that the securities trading safe harbor goes away. 

%\begin{center}
%\textbf{Hedge Fund Lending}
%\end{center}
% 
%What kind of lending are hedge funds doing? 

%Hedge funds often participate in lending syndicates. ``Big name hedge 
%funds like Soros Fund Management LLC, Cerberus Capital Management, 
%Och-Ziff Capital Management Group, Black Diamond Capital 
%Management LLC, Citadel Investment Group and Satellite Capital 
%Management have now become names to reckon with in syndicated loans 
%and are playing a key role in loan market dynamics," said Investment 
%Dealers' Digest. Bankers are happy about the money hedge funds bring, 
%but other institutional investors regard them as opportunistic. A year ago,
%hedge funds accounted for 9 percent of primary loan investments, a 
%portion that is growing. (Investment Dealers' Digest, June 14, 2004.) 

%The lead bank in a syndicate negotiates the loan as the agent of all of the 
%other banks in the syndicate. When a lead bank syndicates a loan, there 
%are two tranches of money. One tranche of money is immediately 
%advanced by the other members of the syndicate; that is, the lead bank 
%never had credit risk for that amount. Another tranche is ``warehoused" by 
%the lead bank, meaning that it advances the money, and assumes credit 
%risk until it can find other participants to take that part of the loan off its 
%hands. The point is that the syndicate member banks minimize the period 
%of their exposure to the borrower, which becomes relevant to hedge fund 
%participation. 

%For a hedge fund to participate in a syndicated loan, it is not necessary to 
%join the syndicate. A hedge fund may buy a participation from a syndicate 
%member or enter into a derivative contract with a syndicate member that 
%transfers the risk of the loan. As the following discussion will show, some 
%tax practitioners want hedge funds to keep their distance from the 
%syndicate, with the aim that the tax law treat the hedge fund as a mere 
%investor in the loan, rather than as a lender. 

%Hedge funds also make cash advances on letters of credit, make debtor- 
%in-possession loans, enter into repos, and purchase public interest private 
%equity securities (PIPES). Hedge funds are making bridge loans as well, 
%competing with banks and investment banks for business, and apparently 
%preferring riskier deals. ``Precisely because of their lack of transparency, 
%there is concern about their ability to handle both the risks and obligations 
%of lenders," said The Wall Street Journal, while noting that hedge funds 
%may be more sophisticated about some kinds of risks than banks. (The 
%Wall Street Journal, May 26, 2005, p. C1.) 

%Hedge funds are engaging in forms of debtor-in-possession financing that 
%banks have eschewed. Hedge funds are thought to ``loan to own"; that is, 
%taking riskier loan bets with the view that they can take over the borrower 
%on a default. (The Wall Street Journal, July 18, 2005, p. C1.) Hedge funds
%account for 70 percent of the market for second-lien loans to borrowers of 
%questionable creditworthiness. As the name implies, second liens don't 
%have much of a claim to the borrower's collateral, but they have a high 
%yield and could be satisfied with equity in bankruptcy. They are 
%replacements for junk debt, which hedge funds also buy and lend directly 
%on. (Investment Dealers' Digest, July 25, 2005, p. 26.) Second liens have 
%not been tested in bankruptcy court. (Investment Dealers' Digest, Jan. 25, 
%2005.) 

%Hedge funds use loan participations and synthetic loans to hedge their 
%other exposures to the borrower's credit. Indeed, credit derivatives are 
%being designed with hedge funds in mind, since they make up 20 percent 
%of the market. Hedge funds are also fond of synthetic collateralized debt 
%obligations (CDOs), that is, derivatives backed by credit derivatives. A 
%CDO is a derivative backed by debt securities. A synthetic CDO is a 
%double layer of derivatives. There are also triple layers, which have the 
%effect of obscuring where the risk went. (Investment Dealers' Digest, May 
%16, 2005, p. 26.) 

%Hedge funds seem to mostly want to use synthetic CDOs for trading or 
%offsetting the risk of other investments, rather than for explicitly absorbing 
%the risk of the underlying loans. The notional amount of credit derivatives 
%can easily exceed the underlying loans. (Investment Dealers' Digest, Nov.\@ 
%15, 2004.) 

%Hedge funds would have it that everything they do should fit under the 
%securities trading safe harbor, so that they would not be deemed to have a 
%lending or financial trade or business in the United States, and would not 
%be taxed on effectively connected income. The general view of 
%practitioners is that much in the way of loan participation and credit 
%derivatives fits into the safe harbor, and that hedge funds should keep 
%their distance from negotiations. 

%Hedge funds further argue that the existing rules should be reinterpreted 
%or changed to meet their desire to avoid U.S. income tax liability. The 
%basic argument is that lending should be treated no differently from buying
%a debt security. That argument is being pressed most vigorously by Stuart 
%Leblang of Akin Gump Strauss Hauer \& Feld and Rebecca Rosenberg of 
%Caplin \& Drysdale. (Leblang and Rosenberg, ``Toward an Active Finance 
%Standard for Inbound Lenders," Caplin \& Drysdale Tax Management 
%International Journal, March 8, 2002. http://www.akingump.com 
%/docs/publication/402.pdf.) 

%\begin{center}
%\textbf{Trade or Business}
%\end{center} 

%Case law and administrative law both make it very easy to be in the trade 
%or business of regular and continuous lending in the United States, 
%regardless of whether the taxpayer has a banking license or would be 
%regulated as a bank by any government. 

%First, the taxpayer has to be in a trade or business. The relevant case law 
%is the securities trading case law that predates the securities trading safe 
%harbor. Second, the taxpayer's trade or business has to be lending or 
%banking or financing. (We use those terms interchangeably, while 
%practitioners fuss that banking can only be done by banks.) For that 
%question, reg. section 1.864-4(c)(5) has a bias in favor of finding that a 
%foreign taxpayer is engaged in the active conduct of a banking, financing, 
%or similar business in the United States. (Readers will recognize these 
%questions as similar to the questions posed by CDOs discussed in Doc 
%2001-13623 [PDF], 2001 TNT 94-3 .) 

%Back to the first question--the existence of a trade or business in the 
%United States. To have a trade or business in the United States, a 
%taxpayer would have to have an office in the United States. The trade or 
%business case law does assign relevance to facilities and actions within 
%the United States that would be excused under the securities trading safe 
%harbor. In short, if hedge funds go as far as the securities trading safe 
%harbor allows, they would have too much presence in the United States 
%when questions arise about other trades or businesses. 

%Setting aside the securities trading safe harbor, it doesn't take much more
%than hedge funds routinely do to have a trade or business in the United 
%States. All it takes, really, is extensive trading and the presence of an 
%agent in the United States. Here it is useful to remember that trading 
%securities is a trade or business and that, but for the safe harbor, income 
%effectively connected with it would be subject to U.S. taxation. This 
%becomes relevant when the taxpayer has other U.S. activities. Whether 
%the taxpayer has a trade or business in the United States is a factual 
%question. (Rev.\@ Rul.\@ 88-3, 1988-1 C.B. 268.) 

%It is well established in the case law that a taxpayer need not be acting 
%directly to be considered to be in a particular trade or business. An agent 
%acting on the taxpayer's behalf will put the taxpayer in a trade or business 
%in the United States, according to the case law. 

%In Adda v.\@ CIR, 10 T.C. 273 (1948), affirmed, 171 F.2d 457 (4th 
%Cir. 1948), cert. denied, 336 U.S. 552, the question was whether an 
%Egyptian resident of France could be considered to be engaged in the 
%commodities trading that his brother in the United States was carrying out 
%under his orders. The brother had discretionary authority over the 
%taxpayer's accounts. It was undisputed that the trading was extensive 
%enough to constitute a trade or business. If the taxpayer had used a 
%broker, he would have been exempt from taxation as a passive foreign 
%investor under the predecessor of section 864(b)(2). But by using his 
%brother as his agent, he became liable for tax as though he had 
%conducted the trading business in the United States himself. 

%CIR v.\@ Nubar, 185 F.2d 584 (4th Cir. 1950), reversing 13 T. C. 
%566, cert. denied, 341 U.S. 925, involved another Egyptian who was 
%present in the United States managing his own securities and 
%commodities trading, which he did quite successfully. That was held 
%sufficient to constitute the conduct of a trade or business. The court was 
%unsympathetic to the taxpayer's position that he should be considered a 
%nonresident eligible for the securities trading safe harbor while he lived in 
%the country, in violation of the immigration laws, and made a fortune. 
%The court believed that the purpose of the exemption ``was to exempt from
%taxation, except as to taxes which could be collected at the source, aliens 
%over whom no effective jurisdiction in enforcement of the tax laws could 
%be exercised." These investors' only contact with the United States was 
%supposed to be their resident brokers. The ``advantage of taxpayer's 
%presence in this country, which was an important element in his trading 
%here, may not be ignored and the case treated as though it involved 
%nothing more than sales and purchases by brokers for a principal across 
%the seas," said the court. 

%A Chinese investor who used a resident commission agent was somewhat 
%luckier in avoiding U.S. tax in Liang v.\@ CIR, 23 T.C. 1040 
%(1955). The taxpayer's resident agent did not work for anyone else and 
%had discretionary authority over the account. The court found that there 
%was not enough activity in the taxpayer's account to raise it from 
%investment to trading. The court believed that the agent ``did no more than 
%was required to preserve an investment account for his principal.'' Under 
%Higgins v.\@ CIR, 312 U.S. 212 (1941), continuity and regularity 
%are not enough to establish a trade or business when the activity is 
%essentially personal investment management. 

%\begin{center}
%\textbf{The Banking Rule}
%\end{center}
% 
%Once the taxpayer has been determined to be in a trade or business, the 
%special banking rule of reg. section 1.864-4(c)(5) kicks in to determine 
%whether that business is lending or financing or a similar business. This 
%rule is essentially a limited revival of the ``force of attraction" doctrine that 
%went out the door when the securities trading safe harbor was enacted in 
%1966. The regulation contains both a definition of ``active conduct of a 
%banking, financing, or similar business" and an effectively connected 
%income rule for income derived from the conduct of that business. 

%The banking rule was written on behalf of banks--no surprise there--specifically Canadian banks. Those banks were operating through 
%branches in the United States, so they were filing U.S. tax returns for 
%income that was clearly the product of a trade or business conducted in
%the United States. They were also making loans into the United States 
%from their Canadian headquarters. In the normal course of the operation 
%of the rules before most interest was exempted from withholding, interest 
%paid by an American borrower to the bank's Canadian headquarters 
%would have had tax withheld from it. 

%The banks didn't want withholding because it would endanger their profit 
%margins. So they lobbied to have the interest paid by American borrowers 
%to Canadian headquarters considered income effectively connected with 
%the conduct of a trade or business in the United States. Because separate 
%corporations are respected, that trade or business would have to be the 
%trade or business of the Canadian parent, not the U.S. branch. So there 
%had to be a very low threshold of activity that would put the parent in the 
%active conduct of a trade or business of lending in the United States by 
%virtue of the actions of the branch as agent. It is that very low threshold 
%that hedge funds are now arguing against. 

%Reg.\@ section 1.864-4(c)(5)(i) lists activities that constitute being in the 
%lending business. Engaging in any one or more of those six activities in 
%the United States, in whole or in part, will put a foreign taxpayer in a 
%lending business in the United States. The six activities listed in the 
%regulation are: 
%\begin{enumerate}
%	\item (a) receiving deposits of funds from the public;
%	\item  (b) making personal, mortgage, industrial, or other loans to the public; 
%	\item  (c) purchasing, selling, discounting, or negotiating for the public on a 
%regular basis, notes, drafts, checks, bills of exchange, acceptances, 
%or other evidences of indebtedness;
%	\item (d) issuing letters of credit to the public and negotiating drafts drawn 
%thereunder; 
%	\item  (e) providing trust services for the public; or 
%	\item  (f) financing foreign exchange transactions for the public.
%\end{enumerate}  

%All six activities require that the foreign taxpayer deal with the public. 
%Hedge funds sniff that they would never sully themselves by dealing with
%the great unwashed public. Leblang and Rosenberg complain that the 
%regulation does not define ``public." Leveraged hedge funds--as most are 
%in these days of chasing meager returns--borrow from banks and 
%investment banks. The regulation does not require that a lender accept 
%deposits from the public. The regulation specifically exempts foreign 
%finance subsidiaries from being considered to be lending to their own 
%affiliates. The necessity and the narrowness of that latter exception 
%implies a broad notion of what constitutes the public, that is, any unrelated 
%borrower. 

%The regulation goes on to state that it is not necessary that the entity be 
%subject to bank regulation. Its activities are what matters. Hedge funds are 
%not subject to bank regulation or any other kind of regulation, a result of 
%negligence and politics. Some hedge fund representatives advocate an 
%interpretation of the regulation that would require a banking license for the 
%taxpayer to be considered to have a lending trade or business in the 
%United States. That is the opposite of what the rule clearly says. 

%Banks don't lend anymore. Banks negotiate loans and then immediately 
%get rid of them, either by selling the whole loan or fobbing off the risk by 
%means of credit derivatives. (Investment Dealers' Digest, Nov.\@ 15, 2004.) 
%So what does it mean to make loans to the public? What does it mean to 
%originate a loan? 

%Hedge funds sometimes negotiate and originate the loans themselves--it's going to be their money anyway, so they want control of the process. 
%Or they may join a syndicate of banks making a big loan. Those direct 
%lending activities would clearly put them in a lending trade or business in 
%the United States. Some tax advisers strive to make those relationships 
%indirect on the view that only direct relationships constitute lending. 

%Practitioners advising hedge funds take a very narrow view of what it 
%means to originate a loan. They take the position that only the bank that 
%negotiated the loan--that'd be the lead bank in a syndicate on a big loan--originated it. So no one else in the syndicate did so, and no hedge fund 
%standing by to buy a piece of the loan from a syndicate member did so,
%practitioners believe. The view is that if the hedge fund does not join the 
%syndicate, then the lead bank in the syndicate--clearly an agent for the 
%members--cannot be an agent for the hedge fund that would put it in the 
%trade or business of lending in the United States. 

%Well, gee, aren't hedge funds then just passive investors in old and cold 
%loans? Sometimes the ink is barely dry on the loan documents. The 
%longest period any bank will wait to fob a loan off onto a hedge fund is 48 
%hours. The hedge fund's contract may contain a material adverse effect 
%clause that allows it to back out within that period. Bankers, of course, 
%would like to be rid of those clauses, but lawyers for hedge funds believe 
%it keeps them out of an origination position. More conservative advisers 
%tell hedge funds to wait a few days and then buy a piece of the loan in the 
%secondary market at market price. 

%If a hedge fund is on the bankers' speed dial, however, there could be a 
%problem with volume and regularity and continuity of loan participations. 
%The lead bank could be deemed to be the hedge fund's agent in the 
%United States. Practitioners argue that every deal is individually 
%negotiated and sui generis. But so is every loan, and uniqueness of deals 
%does not preclude lender status. Moreover, the smaller the amount of the 
%loan, the larger a hedge fund's proportionate position could be. Taking a 
%larger chunk of a loan could militate against the hedge fund's mere 
%investor posture. 

%Why don't regulated commercial banks have an interest in seeing their 
%hedge fund competitors pay U.S. tax on income from loans made into the 
%United States? Hedge funds are the best customers of commercial banks 
%and investment banks. They borrow heavily and pay a lot of fees for prime 
%brokerage as they churn their portfolios. More important, they stand ready 
%to buy loans that banks want to get rid of. All in all, there is no percentage 
%for any regulated financial intermediary in doing anything that would run 
%counter to hedge funds' well-being. 

%Sometimes hedge funds are not competitors. Many hedge funds have 
%commercial banks as investors. Seems that regulated commercial banks
%want to be able to make some loans off their books. 
%If a bank makes a risky loan like the ones hedge funds have shown a 
%taste for, it has to reserve against it in case of default. Having an equity 
%interest in the hedge fund allows some banks to get a risky loan off the 
%balance sheet, so no capital need be reserved against it. Moreover, under 
%some countries' interpretations of the Basel capital rules, that same hedge 
%fund equity interest might even be shown on the balance sheet as positive 
%capital, even though the fund holds a risky loan. The Bank for 
%International Settlements is reviewing the definition of eligible capital. 
%(http://www.bis.org/publ/bcbs107.htm.) In the meantime, it is also worried 
%about credit risk transfer. (http://www.bis.org/publ/joint13.pdf.) 

%``Our systemic fear is that our core banks and brokers are meeting their 
%capital requirements by moving their credit exposure into unregulated 
%funds that don't have capital requirements,'' Randall Dodd of the Financial 
%Policy Forum Derivatives Study Center told Investment Dealers' Digest. 
%(Investment Dealers' Digest, May 16, 2005, p. 26.) Unregulated market 
%participants that are not subject to capital requirements are able to take 
%risks that participants subject to capital requirements cannot, increasing 
%risk and vulnerability in the system. (See http://www.financialpolicy.org 
%/fpfspr8.pdf.) 

%Sometimes banks, usually foreign banks, are partners in hedge funds that 
%make loans to U.S. borrowers. Many of those foreign banks are already 
%engaged in a banking trade or business in the United States, according to 
%reg. section 1.864-4(c)(5). They would just rather park the loans in a 
%hedge fund or somewhere other than on their own balance sheets. 
%Moreover, those banks are ineligible for the securities trading safe harbor 
%because they are dealers -- there is no such thing as a universal bank that 
%is not also a dealer. So this group of hedge fund investors has no reason 
%to care whether the hedge fund is also in a lending trade or business in 
%the United States. 

%If banks doing business in the United States are partners in a hedge fund, 
%would the bank partners' lending trade or business also put the hedge
%fund in a lending trade or business; that is, would the partners' activities 
%be ascribed to the partnership? Certainly the law deems a partner, as a 
%member of partnership that is conceptually an aggregate of its partners, to 
%be conducting the business conducted by the partnership. If the partners 
%are all banks, and were using the hedge fund as a vehicle to do their 
%normal business, then the partnership would logically be viewed as an 
%extension of the banks, even though the test of the partnership having a 
%trade or business in the United States is determined at partnership level. 
%(See reg. section 1.701-2(f), examples 1 and 2, and Ruppel v.\@ 
%CIR, T.C. Memo. 1987-248.) 

%The key to interpreting the application of the banking rule to hedge funds 
%lies in a special little rule that once allowed foreign banks to avoid 
%withholding on some portfolio securities. That'd be reg. section 1.864-4(c)(5)(ii), the so-called 10 percent rule, which allows banks to piggyback a 
%small amount of securities portfolio activity on to their effectively 
%connected banking business. That rule was made to accommodate the 
%banks, which were in the habit of keeping a small percentage of their U.S. 
%assets invested in securities. The rule is confined to investing; it does not 
%cover trading. 

%Reg.\@ section 1.864-4(c)(5)(ii) states that U.S.-source income and gain 
%from securities will be considered to be effectively connected to the 
%taxpayer's lending business if the securities were acquired by the 
%taxpayer's U.S. office to satisfy reserve requirements, or are short-term or 
%government securities, or meet other stated criteria. But income and gain 
%from debt securities will not be treated as effectively connected to the 
%extent that debt securities exceed 10 percent of the total value of the 
%assets of the taxpayer's U.S. office. 

%So some income from investment securities will be kicked out of the 
%effectively connected income category under this rule. Back in the day, 
%the excess income from debt securities that was kicked out of the 10 
%percent rule would have been subject to withholding. That is, the 10 
%percent rule was a benefit for banks, allowing them to avoid withholding 
%on what was regarded as a reasonable amount of debt securities held in
%the course of their U.S. lending business. Now, with no withholding on 
%portfolio interest, the 10 percent rule is obsolete. It is clear that all of the 
%debt securities held by a taxpayer engaged in a lending business in the 
%United States should be deemed to produce effectively connected 
%income. 

%Example 1 of reg. section 1.864-5(c)(5)(vii) posits a foreign bank that has 
%a branch in the United States. The branch does enough to put the foreign 
%bank in a lending trade or business in the United States. The branch also 
%buys debt securities and Treasury securities using dollar deposits 
%transferred to it by its parent. The interest income on those securities is 
%effectively connected with the parent's banking trade or business in the 
%United States under reg. section 1.864-5(c)(5)(ii). In example 2, the 
%branch also buys listed shares of U.S. issuers. The dividends and gains 
%on those shares are not deemed to be effectively connected, again under 
%reg. section 1.864- 5(c)(5)(ii), because the purchases were not made to 
%meet reserve requirements. 

%Reg.\@ section 1.864-5(c)(5)(vi) is a further exception to the 10 percent rule, 
%which says that a taxpayer with a banking or financing business in the 
%United States could have effectively connected income from a trading 
%business if the income is not effectively connected with its lending 
%business. It states, in pertinent part: 
%\begin{quote}
%Any dividends, interest, gain, or loss from sources within the United 
%States which . . . is not effectively connected with the active conduct . 
%. . of a banking, financing, or similar business in the United States 
%may be effectively connected . . . with the conduct by such taxpayer 
%of another trade or business in the United States, such as, for 
%example, the business of . . . trading in stocks or securities for the 
%taxpayer's own account.
%\end{quote}
% 
%Basically, that says that income from securities must be tested to see if it 
%is effectively connected with a securities trading business. So if a hedge
%fund has a banking or financing trade or business in the United States, its 
%securities trading safe harbor could be forfeited. Its trading activity could 
%become effectively connected to a securities trading business in the 
%United States, or even effectively connected to its lending business. 
%Securities trading is a business. The safe harbor is premised on trading 
%being the taxpayer's only contact with the United States. 

%Well, why can't a taxpayer have a lending business on one side and a 
%securities trading safe harbor on the other? Because the safe harbor 
%assumes that trading is the taxpayer's only contact with the United States. 
%Once the taxpayer is in some other business in the United States, like 
%lending, the securities trading safe harbor goes out the window because 
%the trading would be effectively connected with some other business. And 
%that other business may well be securities trading -- only securities trading 
%standing alone is eligible for the safe harbor. 

%The connection of any securities trading to any trade or business in the 
%United States would be tested under the activities test of reg. section 
%1.864-4(c)(3) for fixed and determinable income and capital gains, which 
%asks whether the taxpayer's activities in the United States were a material 
%factor in the realization of that income. The question is whether the 
%income arose from the taxpayer's activities in the United States, even 
%though the income itself is passive. 

%Reg.\@ section 1.864-5(c)(3)(i) notes that this inquiry is pertinent when the 
%passive income is similar to the income the taxpayer earns in its business. 
%Most nonfinancial taxpayers are permitted some activity relating to 
%portfolio investments that are not effectively connected with a trade or 
%business. ``In applying the business-activities test, activities relating to the 
%management of investment portfolios shall not be treated as activities of 
%the trade or business conducted in the United States unless the 
%maintenance of the investments constitutes the principal activity of that 
%trade or business," reg. section 1.864-4(c)(3)(i) states. 

%That sentence leaves open the possibility that securities trading could rise 
%to the level of a trade or business by itself, as example 1 of reg. section
%1.864-4(c)(3)(ii) demonstrates. In that example, the foreign taxpayer is an 
%investment company that has its principal office in the United States and 
%has a trade or business in the United States by reason of that office. 
%Therefore, any U.S.- source dividends, interest, and gains are effectively 
%connected with the taxpayer's trade or business of securities trading. 

%Practitioners, however, look at that example differently. They argue that 
%because after 1997 an office will not put a taxpayer in a securities trading 
%business, the securities trading safe harbor is automatic and not elective. 
%In that view, any securities trading is automatically shoved off to the safe 
%harbor and never effectively connected with a trade or business. That 
%view holds that there is no such thing anymore as a securities trading 
%trade or business that produces effectively connected income unless the 
%taxpayer is a dealer. (Tax Notes, June 15, 1998, p. 1465, Doc 98- 19132, 
%98 TNT 114-71.) The Tax Court disagrees, as the discussion in the next 
%section will show. 

%Under reg. section 1.864-5(c)(5)(vi), there is no ability to segregate 
%activities happening in the same office in the United States. Is it possible 
%to segregate banking and trading activities into two different offices in two 
%different towns in the United States? It should not be, because the 
%question is what overall business the taxpayer is doing in the United 
%States. 

%If segregation were possible, then reg. section 1.864- 5(c)(5)(vi) would 
%never apply to anything. The government should put out a ruling stating 
%that the securities trading safe harbor is not available when the taxpayer 
%is otherwise engaged in the active conduct of any trade or business in the 
%United States. The ruling should state that the safe harbor is based on the 
%assumption that the taxpayer has no other trade or business, and that its 
%only contact with the United States is trading. 

%Practitioners would also like to be able to argue that any securities trading 
%is not effectively connected with any lending trade or business. How's that 
%again? It's a preposterous argument in the case of universal banks, and 
%even more so in the case of hedge funds. For hedge funds, the point of
%lending is to have another type of exposure to the borrower's credit to 
%complement their existing exposure, and possibly to gain information to 
%use in arbitraging their other positions in the borrower. So if a hedge fund 
%had a lending trade or business in the United States, and its securities 
%trading is to be tested separately, it is possible that the securities trading 
%would be connected with the lending business. 

%Practitioners also point out that reg. section 1.864-4 was not amended to 
%take into account the liberalization of the statutory securities trading safe 
%harbor. The regulation was last touched a year before that change, and 
%that was to tweak the asset use test, which is not relevant for our 
%purposes. So the regulation still carps about the importance of an office in 
%the United States, which is perfectly permissible under the safe harbor. 
%Practitioners believe that the safe harbor trumps the regulations to the 
%extent that it is inconsistent with them and, further, that it is not possible to 
%lose the safe harbor. 

%Some practitioners believe that segregation by means of different entities 
%using the same supply of capital is possible. The view of practitioners 
%advising hedge funds is that a separate entity to hold the offending activity 
%will prevent the taint from affecting eligibility for the securities trading safe 
%harbor. It is rare, however, for hedge funds to maintain separate entities, 
%since their main argument is that everything they do in one U.S. 
%management office belongs under the securities trading safe harbor. 

%Practitioners rest assured that no other activity producing effectively 
%connected income would deleteriously affect eligibility for the securities 
%trading safe harbor as long as hedge funds are not dealers. But some 
%hedge funds are on the verge of becoming dealers or are acquiring dealer 
%affiliates. (Investment Dealers' Digest, Apr. 11, 2005.) 

%\begin{center}
%\textbf{Case Law}
%\end{center}
% 
%InverWorld Inc. v.\@ CIR, T.C. Memo. 1996-301, holds that a 
%taxpayer need not be regulated as a bank anywhere in the world to be
%considered to be engaged in the active conduct of a banking, financing, or 
%similar trade or business in the United States according to reg. section 
%1.864-4(c)(5)(i). A significant aspect of this case is the court's expansive 
%view of what constitutes a banking, financing, or similar trade or business. 
%The court treated the Cayman Islands parent and its U.S. subsidiary 
%separately for purposes of holding that the subsidiary was acting as an 
%agent for the parent, which was engaged in a banking business in the 
%United States. 

%InverWorld involved a Mexican-controlled shell financial services 
%company, organized in the Caymans (the parent), that got investment 
%management services from its substantive second-tier U.S. subsidiary 
%(the subsidiary). The Tax Court held that the parent, acting through the 
%subsidiary, was engaged in a banking or financing trade or business 
%within the United States under section 864(b), rejecting the argument that 
%any activity of the subsidiary on behalf of the parent should be eligible for 
%either of the section 864(b)(2)(A) securities trading safe harbors. 

%The parent lent money to rich foreign individuals who were customers of 
%its ultimate Mexican parent (we're talking flight capital here, readers). The 
%loans were secured by the customers' certificates of deposit and were 
%used to purchase interests in the subsidiary's investment funds or, in a 
%back-to-back loan situation, to avoid Mexican taxes. 

%Tax Court Judge Thomas B. Wells found that those activities constituted 
%making personal loans to the public, within the purview of reg. section 
%1.864-4(c)(5)(i). Judge Wells concluded that the parent, acting through 
%subsidiary, engaged in five of the six activities listed in that regulation, 
%putting it in an active financing trade or business. He found that the 
%subsidiary received deposits of funds from the public, made personal 
%loans to the public, regularly purchased and sold evidences of 
%indebtedness to the public, issued letters of credit to the public, and 
%financed foreign exchange transactions for the public. 

%The next question was whether the parent, acting through the subsidiary 
%as its dependent agent, was engaged in any kind of trade or business
%within the United States. The group argued that its extensive securities 
%trading activity should be placed under one of the two securities trading 
%safe harbors of section 864(b)(2)(A), so that it would be excluded from the 
%determination whether the parent was engaged in a trade or business 
%within the United States. That is, the group wanted to segregate its 
%lending activities from its securities trading, putting the latter under the 
%securities trading safe harbor, in the hope of reducing the relevant activity 
%in the United States. 

%In InverWorld, the office of the subsidiary and all of the trading activity in 
%United States were taken into account in determining whether the parent 
%had a trade or business in the United States. Hence a qualitative and 
%quantitative analysis of what the group was doing put it in a trade or 
%business in the United States. (Scottish American Investment Co., Ltd. v.\@ 
%CIR, 12 T.C. 49 (1949).) That means that not only does 
%securities trading not shelter lending, but securities trading would be taken 
%into account in determining whether the taxpayer had a trade or business 
%in the United States. The Tax Court took a broad view of financial 
%intermediation in asking what counted toward a trade or business.
% 
%It may be that the court dragged in all of the U.S. investment management 
%activity on the ground that it was ineligible for the safe harbor. Judge 
%Wells concluded that the parent was not eligible for the section 864(b)(2)(A)(i) securities trading safe harbor because it requires independent 
%agents, and the group's U.S. affiliates were not independent. Moreover, 
%because the parent was engaged in a trade or business in the United 
%States through its second-tier domestic subsidiary, it further violated the 
%safe harbor by having an office in the United States. That it was largely 
%not trading for its own account also made the parent ineligible for the 
%proprietary securities trading safe harbor of section 864(b)(2)(A)(ii). 

%One practitioner argues that the holding that the parent was engaged in a 
%lending trade or business in the United States because of the activities of 
%its U.S. affiliate is dicta, because the court had already held that the 
%parent was engaged in a trade or business in the United States under the 
%general tests of reg. sections 1.864-4(c)(2) and (3). So resort to the
%banking rule of reg. section 1.864-4(c)(5) was unnecessary to the holding. 
%Practitioners also take issue with the court's interpretation of reg. section 
%1.864-4(c)(5) itself. They sniff that taking client money is not the 
%equivalent of taking deposits from the public, and that lending to clients is 
%not lending to the public. That is, they differentiate investment 
%management from lending, which the court did not do. 

%Hedge funds want to go further; they want to piggyback lending onto the 
%securities trading safe harbor, or at least segregate securities trading. 
%That is not how reg. section 1.864-4(c)(5) works. Once a foreign taxpayer 
%is found to be in a lending business in the United States, its securities 
%trading is swept into the lending business or a securities trading business, 
%and all of the income is considered effectively connected with a trade or 
%business in the United States. That is what InverWorld says. 

%Mexican flight capital also figured in Pasquel v.\@ CIR, 12 T.C.M. 
%1431 (1953), the favorite case of hedge fund advisers, who rely on it for 
%the idea that sporadic lending does not rise to the level of a banking trade 
%or business in the United States. In Pasquel, a Mexican individual lent 
%money to a troubled American corporation to complete the purchase of 
%two World War II surplus landing boats. The lender and the borrower split 
%the profits when the boats were resold at a gain three months later. Even 
%though the lender was protected against loss, the Tax Court was unsure 
%whether the relationship was a loan or was a joint venture, as the 
%government argued. 

%The court held that a single isolated transaction was not enough to put the 
%Mexican individual in any kind of trade or business in the United States. 
%Following the guidance of earlier courts, the court held that ``engaged in 
%business'' connoted continuous and sustained activity, not the 
%performance of a single disconnected business act. The court's emphasis 
%was on the fact that there was one transaction. Plus the U.S. government 
%already had its cut; the corporation had withheld on the lender's share of 
%the profit, and the government was arguing for more. (For the single-loan 
%rationale, see also Sales v.\@ CIR, 37 T.C. 576 (1961).)

%Lawyers representing hedge funds would have it that there is a vast 
%uncharted territory of uncertainty between the heavy lending activity of 
%InverWorld and occasional activity rationale of Pasquel. Maybe that would 
%be so if those were the only two relevant cases. Both cases happened to 
%involve cross-border lending. But when the lending activity does not cross 
%any border, we have a pretty solid idea what it takes to be in the active 
%conduct of a banking or financing trade or business. Indeed, we have 
%such a good idea what the lending business is that most of the decisions 
%are Tax Court memorandums, indicating settled law. And no, the standard 
%is not different when a border is crossed. 

%The posture in most of those cases is that an individual taxpayer claimed 
%to be in the trade or business of lending for purposes of taking a section 
%166 bad debt deduction. 

%Factors used by courts to determine whether a taxpayer is engaged in the 
%trade or business of lending money include: the number of loans; the time 
%period over which the loans were made; the time devoted to the lending 
%activity; whether the taxpayer advertised; whether the taxpayer 
%maintained a separate office; the maintenance of separate books; the 
%taxpayer's reputation in the community as a lender; and the relationship of 
%the debtors to the taxpayer-lender. United States v.\@ Henderson, 375 F.2d 
%36, 41 (5th Cir. 1967), cert. denied 389 U.S. 953 (1967). 

%In Imel v.\@ CIR, 61 T.C. 318 (1973), the individual taxpayer was 
%an officer and majority shareholder of a bank, who made personal loans 
%to individuals and businesses that the bank could not make. Of course 
%some of those loans went bad, and the taxpayer claimed a section 166(a) 
%ordinary business bad debt deduction. The government argued that the 
%taxpayer's claim should be restricted to a short-term capital loss, a 
%nonbusiness bad debt deduction under section 166(d). So the question 
%was whether the taxpayer was in the trade or business of lending. He had 
%made eight or nine personal loans during a four-year period. The court did 
%not believe that that was regular and continuous enough to give his 
%lending the status of a separate business.

%Conversely, in McCrackin v.\@ CIR, T.C. Memo. 1984-293, the 
%individual taxpayer, a coal broker, had a sideline of lending that was 
%commingled with the coal brokering business of his sole proprietorship. 
%The taxpayer lent money to a friend for use in his timber business. That 
%loan went bad due to fraud, and the taxpayer suffered a large loss on 
%foreclosure and inability to collect a judgment against the debtor. The 
%court noted that the taxpayer and his coal brokerage made 66 loans over 
%the course of 15 years to a dozen unrelated borrowers, and that the coal 
%brokerage held itself out as a lender. The court held that the activity rose 
%to the level of a business, regardless of the identity of the lender, so that 
%the losses were business bad debts under section 166(a). 

%Likewise, in Jessup v.\@ CIR, T.C. Memo. 1977-289, the Tax 
%Court found a lending business. The individual taxpayer owned and 
%managed the Trailways bus system, but also dabbled in the insurance 
%and banking businesses. The taxpayer personally made and guaranteed 
%loans, and he reported considerable income from that activity. Over a 
%decade, he made 31 loans to 17 unrelated persons, with the aggregate 
%amount lent ranging from the hundreds of thousands to the low millions. 
%Along the way, the taxpayer guaranteed two speculative loans to an 
%inventor who wanted to exploit a new copper recovery process. That 
%didn't work out, and the taxpayer joined the board of the borrower's 
%corporation while attempting to keep the loans current. The bank called 
%the loans, and the taxpayer paid on the guarantee. 

%The taxpayer claimed a section 166(a) business bad debt deduction for 
%the entire amount of the loans to the inventor, arguing that the latter's 
%collateral, which took the form of unrecorded mortgages, was 
%unenforceable. Relying on Imel and Barish v.\@ CIR, 31 T.C. 1280 
%(1959), the court held that the taxpayer's ``lending activities were 
%sufficiently extensive and continuous so as to constitute a trade or 
%business." The government was stuck arguing that the taxpayer did not 
%advertise his willingness to lend and had no separate lending office. The 
%court saw no need for those things. (See also Minkoff v.\@ CIR, 
%T.C. Memo. 1956-269, finding a lending business when 40 loans were 
%made over five years.)

%The court also had to decide whether the loan was proximately related to 
%that trade or business of lending under reg. section 1.166-5(b). Whether a 
%bad debt is proximately related to the taxpayer's trade or business is 
%determined by the ``dominant motivation" of the taxpayer in acquiring the 
%debt. (United States v.\@ Generes, 405 U.S. 93, 103 (1972), rehearing den. 
%405 U.S. 1033 (1972).) The proximate relationship test is satisfied when 
%the loans are the primary activity of the lending business and the 
%dominant motive for making them is to earn interest as opposed to equity 
%investment in the borrower. (Hutton v.\@ CIR, T.C. Memo. 
%1976-6.) 

%The Tax Court considered another businessman who claimed to be in the 
%business of lending in Ruppel v.\@ CIR, T.C. Memo. 1987-248. 
%That taxpayer was the majority shareholder, president, and chairman of a 
%bank, and he sat on the boards of other local banks. He had extensive 
%investments in local businesses, in one of which he was actively involved. 
%He made loans to those businesses from his own cash and by being able 
%to borrow at low rates. He did not advertise and had no separate lending 
%office. He had records but no separate bank account for lending. He used 
%employees of his bank and other businesses to service the loans. Over a 
%decade, he made 124 loans, mostly through partnerships in which he was 
%a partner. 

%The taxpayer made 13 loans totaling roughly one million dollars to a group 
%of closely held companies whose principal asset was a limestone mine, 
%taking back a security interest in some of the owners' personal assets and 
%company leases. Of course, some of the borrowers defaulted on the 
%loans, the taxpayer foreclosed, and the borrowers became insolvent. The 
%taxpayer claimed a business bad debt deduction for the amount of the 
%loans in excess of the collateral. The government quibbled that the 
%taxpayer's partnerships, and not the taxpayer himself, made most of the 
%loans. 

%Relying on Imel, the Tax Court concluded that the taxpayer had a trade or 
%business of lending even without considering the partnership loans, 
%having made 27 loans to 13 borrowers during the two years at issue. (See
%Cushman v.\@ United States, 148 F. Supp. 880 (D. Ariz. 1956), in which a 
%lending business was found when 21 loans were made over eight years.) 
%The court excused the taxpayer's failure to report his lending on Schedule 
%C or maintain a separate bank account for it. 

%Lest these decisions be regarded as ancient history, they were reaffirmed 
%in Carraway v.\@ CIR, T.C. Memo. 1994-295. The individual 
%taxpayer, a clothing retailer, wanted a business bad debt deduction for 
%nine advances to a corporation in which he was an officer and held shares 
%and a loan to another shareholder in that corporation. The taxpayer was a 
%partner in a factoring business, but the corporate borrower wasn't 
%borrowing against receivables. So the court found no lending business 
%and no proximate relationship. Despite agreeing that the advances were 
%debt, the government asked for negligence penalties, which it got. 
%Leblang and Rosenberg dispute the relevance of the bad debt deduction 
%precedent. ``The domestic bad debt cases can be very difficult to reconcile 
%with each other, and do not provide a logical, predictable template that 
%can be used to make the trade or business of lending determination for 
%purposes of section 882 or section 871,'' the pair say. Because there is 
%little cross-border precedent, they argue, we cannot know how factors like 
%the number of loans should be applied when the lender is foreign. 

%How's that again? Lending is lending, regardless of whether a real or 
%artificial national border stands between borrower and lender. If the case 
%law is so difficult to decipher, why do practitioners worry about the foreign 
%investor's level of involvement in the lending process, how many loans it 
%makes, and whether those loans are negotiated in the United States? 
%Because those are all relevant factors under the case law, and the case 
%law is pretty clear on what facts and circumstances put a foreign investor 
%in the trade or business of lending in the United States. 

%Moreover, some of the tax advice that is being given to hedge funds is 
%more formal than substantive. A bank can initiate a loan, then securitize or 
%swap it to a hedge fund within hours after it is signed. The hedge fund 
%takes the position that the bank made the loan, that it is a mere investor,
%and that it need not report effectively connected income. 

%Of course, the IRS has not helped matters by giving ill-advised private 
%rulings on the lending business question. That would be LTR 9701006, 
%issued by the financial institutions and products group at the IRS. The 
%issue was whether a publicly traded limited liability company satisfied the 
%passive income requirements to be treated as a partnership under section 
%7704. The conduct of a financial business would disqualify the LLC from 
%partnership treatment. The LLC had previously been a real estate 
%investment trust and had converted to LLC status so it could originate 
%some mortgages. It promised to originate five or fewer mortgages each 
%year. 

%The IRS ruled that five or fewer loans a year did not put the LLC in a 
%financial business for purposes of section 7704. An important factor in the 
%IRS's ruling was that the LLC was owned by state labor union pension 
%plans, and it planned to lend on projects that employed union labor. (The 
%IRS also had to rule that the LLC was not a taxable mortgage pool, which 
%was even more of a stretch.) Ever since that ruling was released, tax 
%advisers have been telling hedge funds that five or fewer loans a year will 
%not cause them to have effectively connected income from a banking 
%business in the United States. 

%Trouble is, some hedge funds are making five loans a month. Of course, if 
%the hedge funds making debtor-in-possession loans really were lending 
%with the idea of owning, or making loans in conjunction with taking equity 
%interests in borrowers, then they might be able to argue that their 
%dominant motive in lending was investment rather than a banker's motive 
%of earning interest and fees. 

%True to their names, hedge funds often take several offsetting positions in 
%the same borrower -- convertible debt, a credit default swap, a long equity 
%position, or a synthetic CDO that may pay off in preferred shares. They 
%often buy PIPES, which are discount-priced variety packs of the securities 
%of the same issuer, including shares, convertible debt, warrants, and debt. 
%PIPES usually include convertibles. Buying PIPES by itself would not put
%a hedge fund in a lending business. 

%Where do privately negotiated loans that are convertible into the 
%borrower's shares fall? They are not securities, and if the hedge fund did 
%the negotiating, they would point to a lending relationship since that tax 
%law treats convertible debt as debt. Leblang and Rosenberg argue that 
%even though they are clearly loans, it is somehow inappropriate to treat a 
%hedge fund that makes them as having a lending trade or business in the 
%United States if it did not get the funds from customer deposits. They 
%acknowledge that even if such a hedge fund did not take customer 
%deposits, and the convertible loans were converted, the hedge fund could 
%nonetheless be in a lending trade or business in the United States under 
%current law. 

%\ldots

%Wouldn't do to argue that hedge funds are not lending because they are 
%more interested in information than in yield. Buying a loan participation to 
%get nonpublic information would not prevent a hedge fund from being in a 
%lending trade or business. Banks with trading desks have the same 
%internal conflicts with use of information from loans they originate. 
%(Chinese walls are being put to the test as borrowers worry that 
%information furnished to lenders is being used in trading.) 

%The hedge funds' tax argument would be that the loan participation bits of 
%a hedged position in the borrower were just pieces of a larger hedged 
%exposure, rather than a relationship primarily defined by lending. That 
%argument might not get far under current rules. All that would happen is 
%that the income from the other related investment might be effectively 
%connected with the lending trade or business. 

%Hedge funds that are lending aggressively are not believed to be filing 
%protective returns admitting to being engaged in a banking or financing 
%trade or business in the United States. That is a risky strategy because if 
%the position that there is no trade or business in the United States is not 
%sustained, the hedge fund will not be able to deduct expenses against its 
%effectively connected income. (That is what happened in InverWorld.) If 
%the hedge fund is leveraged, then its chief expense will be interest, and 
%that will not be permitted to offset interest income from lending. 

%\begin{center}
%\textbf{The Policy}
%\end{center}
% 
%Well, gee, if we're making rules to satisfy Canadian banks, and have a 
%longstanding practice of making foreign investors happy, why shouldn't we 
%rewrite the rules again to make hedge funds happy?

%Because hedge funds, unlike Canadian banks, don't want to submit to 
%U.S. tax jurisdiction. They want it both ways. They want to earn money in 
%the United States, manage their portfolios from the United States, and not 
%pay a dime in tax to the United States--the sort of attitude that clearly set 
%the judge's jaw in Nubar. Hedge funds' claims to nonresident status are 
%no more credible than Mr. Nubar's, and it is a bit much they ask that the 
%rules be turned upside down so that they can pay no tax whatsoever. 
%Hedge funds have willingly submitted to regulation when it suits them, 
%such as when they want to become broker-dealers or have the ability to 
%clear their own trades, both of which require regulatory approval. 

%Before asking whether current rules should be administratively or 
%legislatively changed to accommodate hedge funds, we should step back 
%and ask whether the rules make sense. There are two sets of rules we are 
%concerned with here. 

%First, there is the American policy of making life very comfortable for 
%foreign investors, which consists of exemptions from source withholding 
%for most interest, reduced treaty withholding rates for dividends, and the 
%securities trading safe harbor for capital gains. Hedge funds want to 
%expand that generous list to include their active lending in the United 
%States. 

%Second, there are the stupid American residence rules. Hedge funds are 
%not the only reason that the place of organization test should be changed 
%to a principal place of management test. (For discussion, see Doc 
%2005-14577 [PDF], 2005 TNT 132-4 .) 

%The questions of residence and investor treatment feed into each other. If 
%the United States is going to continue to allow businesses to declare 
%themselves resident wherever they want, then it ought to be taxing on the 
%basis of source and withholding on all cash outflows. 

%In the long run, again as to truly foreign investors, the law ought to require 
%source withholding on all outflows, with a further requirement that the 
%foreign investor file a tax return in the United States to recover any
%amounts withheld. Plenty of foreign investors invest in dividend-producing 
%equity in the United States on that basis already because they would 
%rather endure withholding than disclose their identities. Countries like 
%India have sensibly decided that source withholding is an assured, 
%administrable way to make sure that tax is collected. But even India has 
%exemptions from source withholding. 

%If, as is more likely, the United States continues to tax on the basis of 
%residence and doing business within the United States, then the definition 
%of how much activity constitutes nexus ought to be expansive. That is, a 
%relatively small amount of activity within the United States should bring the 
%business into U.S. tax jurisdiction. The banking rule should be kept as it 
%is, and the securities trading safe harbor should not be expanded any 
%further than it has been already. Indeed, there are good arguments for 
%restricting the safe harbor to individuals. 

%Even if the law were to be changed to acknowledge that hedge funds 
%really are domestic businesses, there would still be plenty of truly foreign 
%investors whose financial activities in the United States raise questions 
%about whether those activities constitute some kind of trade or business. 
%So the question becomes whether the current law's criteria for what 
%constitutes a trade or business are appropriate. Like the financial 
%intermediaries that got section 954(h) enacted, hedge funds are 
%inadvertently making the point that it takes very little activity to run a 
%financial business. 

%That means that if a business, even if the business is an investment fund, 
%has borrowers or insurance policyholders in the United States, it should 
%be considered to be doing business in the United States and be required 
%to pay tax and file a return accordingly. An insurance analogue of the 
%banking rule should be adopted to bring foreign insurers selling policies in 
%the United States into U.S. tax jurisdiction. It is absurd that New 
%York-based AIG has to answer to the SEC, but not to the IRS because it 
%claims residence in Bermuda. Insurance would have to be broadly defined 
%to encompass derivatives that transfer risk. Hedge fund managers know 
%that; they have been buying so much catastrophe risk in the form of
%derivatives that some have decided to form insurance companies 
%themselves. 

%Congress thought about derivatives and economic equivalents in the 
%recently enacted American Jobs Creation Act of 2004. New section 864(c) 
%(4)(B) expands each category of foreign-source income that is treated as 
%effectively connected to include its economic equivalents. That means 
%that foreign-source interest equivalents -- like the payments on a synthetic 
%loan -- would be taxed as effectively connected income if the taxpayer 
%was engaged in the active conduct of a lending trade or business in the 
%United States. 

%The same thinking about economic equivalents should be applied to 
%inbound loan equivalents. Credit derivatives and syndication participation 
%and synthetic loans should put a hedge fund into a lending trade or 
%business. Reg.\@ section 1.864-4(c)(5)(i) should be amended and expanded 
%to include loan equivalents. 

%Going in the other direction, Leblang and Rosenberg propose that the 
%amount of activity required to establish a trade or business of lending in 
%the United States be increased significantly from what it is now, with the 
%effect that hedge funds could continue to lend without having effectively 
%connected income. Alternatively, they ask for a lending safe harbor for 
%lenders that are not regulated as banks, on the ground that regulated 
%banks borrow cheaper. Such an exception would not be available to 
%hedge funds formed by banks. 

%Specifically, Leblang and Rosenberg argue that a standard like section 
%954(h), the subpart F exception for ``active conduct of a finance or similar 
%business," should govern, despite its wholly different purpose. The 
%subpart F active financing exception requires, among other things, that 
%the taxpayer be dealing with the local public in its place of organization 
%and that it carry out its business directly, rather than through agents. The 
%purpose of those requirements is to ensure that the business is sufficiently 
%active to justify being exempted from current taxation under subpart F 
%because it is an active business. The purpose of the current rules of
%section 864, in contrast, is to determine when the taxpayer's lending to 
%American borrowers is sufficient to bring it into U.S. tax jurisdiction. 

%Leblang has expanded that argument to include an argument that hedge 
%funds, as opposed to their American managers, should not be considered 
%lenders under reg. section 1.864-4(c)(5) because the goodwill from 
%lending accrues to the managers and not to the funds and their investors. 
%(That's probably true of everything hedge funds do, given the exorbitant 
%fees their managers charge.) Leblang's argument is that because hedge 
%fund investors are entitled only to net asset value for their shares, the 
%goodwill developed in the lending business accrues to the managers 
%instead. Thus the managers might have a lending business, but the 
%investors would not have effectively connected income. That argument is 
%premised on the idea that the managers are independent agents of the 
%investors. (For development of this argument, see Tax Notes, June 16, 
%2003, p. 1663, Doc 2003-14516 [PDF], 2003 TNT 116-37 .) 

%The IRS has been forced to think about hedge fund lending questions. 
%The IRS is looking at the question. 

%But what about the policy of excusing foreign investors from tax to 
%encourage them to allow the United States to continue to borrow 80 
%percent of the world's capital? 

%First an administrative point. If they do not already, tax policymakers must 
%view withholding exemption as tantamount to tax exemption. A tax 
%withheld is a tax collected. A tax not withheld will probably never be 
%collected. So the present law's withholding exemptions could be just as 
%easily redrafted as tax exemptions. The same holds for the law's failure to 
%require reporting of items like capital gains. If withholding is not going to 
%be required, but there is still a misplaced desire to collect tax, then there 
%should be reporting of all items of investment income or gain. And there 
%should be money spent on the IRS so that it can analyze the reports. 

%Well, gee, why don't we explicitly give up on trying to collect tax on 
%investment income and gain? Isn't the object of the Bush administration's
%tax reform project to ensure that no one pays tax on any income from 
%capital? However ideologically appealing exemption of income from 
%capital from tax is to some people, the U.S. government--aptly 
%characterized by economist Paul Krugman as a pension plan with an 
%army--cannot afford it. 

%Moreover, in the postindustrial society in which the largest manufacturers 
%have become voluntary employees' beneficiary associations, there are 
%only two types of income--wage income and financial income. Both have 
%to be taxed. The question becomes to what extent, not whether, financial 
%income should be taxed. Congress has not adopted a policy of 
%nontaxation for every foreign investment in the United States or, in the 
%case of hedge funds, every investment masquerading as foreign. 

%A short while back, we discussed Britain's deliberate decision to become 
%a haven for rich investors and financial intermediaries through a 
%combination of light taxation and lax regulation. (Doc 2005-13052 [PDF], 
%2005 TNT 120-8 .) Those policies have had the effect of attracting 
%foreign resident nondomiciliaries and making London a financial capital. 
%But so what? The policies have hastened Britain's descent into a rentier 
%society. When politicians decide to make life easy for rich foreigners, they 
%should ask what domestic benefit there is from that policy. 

%Should the United States go further down the road to a rentier society? 
%``The goal of attracting foreign capital should be weighted heavily in 
%relation to concerns about the competitiveness of U.S. lenders," Leblang 
%and Rosenberg argue. But there are few genuine capital importation 
%policy questions here. 

%Hedge funds aren't foreign investors who have somewhere else to go to 
%invest. If it were worthwhile for them to have money invested in other 
%markets, it would already be there. Leblang raises the specter of hedge 
%fund lending being conducted out of London, which is more 
%accommodating to investors-cum-lenders than the United States appears 
%to be. Section 151 of schedule 26 of the Finance Act of 2003 excuses 
%nonresident companies from British tax if they use an independent
%resident broker or investment manager to carry out investment 
%transactions that include ``the placing of money at interest." 

%Hedge funds are American businesses with American managers and 
%mostly American investors that have most of their investments in 
%American issuers. Yes, there are plenty of hedge funds in London, and 
%plenty of genuine foreign investors in them. But to the extent that hedge 
%funds represent a recycling of American investment capital, it is absurd to 
%argue, as Leblang and Rosenberg do, that hedge funds should be 
%coddled because they represent foreign financial resources coming into 
%the United States. Chinese investors buying mortgage- backed securities 
%are foreign investors. Hedge funds, by and large, are not. 

%So what good are hedge funds doing the United States? Aren't hedge 
%funds providing liquidity, as their chief apologist, Federal Reserve Board 
%Chair Alan Greenspan, argues? 

%To believe that a large group of secretive investors are helping the 
%securities markets function in a world awash in capital, one would have to 
%believe that the markets are more liquid when no one knows what the hell 
%is going on. Liquidity exists with or without hedge funds; that large chunks 
%of capital have migrated to hedge funds does not make them special. In 
%fact, the world is now in a global liquidity bubble thanks to central bankers 
%keeping interest rates low to prevent necessary deflation. The Long Term 
%Capital Management collapse showed that hedge funds are as vulnerable 
%to liquidity shocks as any other financial intermediary. 

%Proper tax treatment of hedge funds is not, as Leblang and Rosenberg 
%would have it, a policy question of how to make foreign investors 
%comfortable. Truly foreign investors who confine their activities to pure 
%investing are already quite comfortable with the generous securities 
%trading safe harbor. 

%Indeed, if one were going to be nice to a particular set of mobile investors, 
%it ought to be the Chinese, who don't owe U.S. tax. ``It is ironic that 
%western capitalists can thank the world's biggest communist country for
%their good fortune,'' intoned The Economist, explaining how China driving 
%down the price of labor has increased the returns to capital. ``Global 
%inflation, interest rates, bond yields, house prices, wages, profits and 
%commodity prices are now being increasingly driven by decisions in 
%China." (The Economist, July 28, 2005, p. 61.) 
%\end{select}

%
In Chief Counsel Memorandum 2009-010, the IRS addresses a foreign corporation's creative effort to avoid having a U.S. trade or business by using an independent agent to solicit and negotiate loans with U.S. persons, but leaving final approval to be made by the foreign corporation.  The CCM  covers attribution of an agent's activity to the principal and the office requirement of Reg.\@ \S1.864-4(c)(5).   


\addcontentsline{toc}{section}{\protect\numberline{}Chief Counsel Memorandum 2009-005}
\begin{select}
\revrul{Chief Counsel Memorandum 2009-010}{9/22/09 (release date)}
\ldots\\

\begin{center} \textbf{Issue}
\end{center}

%Lending in the United States by Foreign Person Giving Rise to Effectively Connected Income
%This memorandum sets forth the legal analysis with respect to certain lending activities undertaken by foreign corporations. This advice may not be used or cited as precedent.
%ISSUE
%Whether interest income earned by a foreign corporation with respect to loans originated by an agent, whether dependent or independent, in the United States is attributable to �the U.S. office� through which the foreign corporation�s banking, financing or similar business activity is carried on, such that the interest income is �effectively connected income�?
%CONCLUSION
%The interest income received by a foreign corporation with respect to loans that it originated to U.S. borrowers constitutes income effectively connected with such foreign corporation�s banking, financing or similar business when an agent, whether dependent or independent, performs origination activities described in the facts below on the foreign corporation�s behalf with respect to such loans in the United States.

\begin{center} \textbf{FACTS}
\end{center}

A corporation is organized in Country X (``Foreign Corporation'') and 100 percent of the shares in Foreign Corporation are held by shareholders who are not U.S. Persons as defined by Section 7701(a)(30). Country X does not have a bilateral income tax treaty with the United States. Foreign Corporation makes loans to U.S. persons (the ``U.S. Borrowers'') within the United States.

Foreign Corporation has no office or employees located in the United States. To originate loans to the U.S. Borrowers, Foreign Corporation outsources the origination activities to a United States corporation (``Origination Co.''). Under a service agreement between Foreign Corporation and Origination Co., the activities performed by Origination Co. include the solicitation of U.S. Borrowers, the negotiation of the terms of the loans, the performance of the credit analyses with respect to U.S. Borrowers, and all other activities relating to loan origination other than the final approval and signing of the loan documents. Origination Co. conducts these activities on a considerable, continuous, and regular basis. Under the terms of the service agreement, Foreign Corporation pays Origination Co. an arm's length fee for its services. Origination Co. performs the origination activities from an office located in the United States, and Origination Co. is subject to U.S. federal income taxation. Although Origination Co. performs all of the origination activities on behalf of Foreign Corporation, Origination Co. is not authorized to conclude contracts on behalf of Foreign Corporation. Foreign Corporation�s employees, who work in an office located outside of the United States, give final approval for the loans and physically sign the loan documents on behalf of Foreign Corporation.

\begin{center} \textbf{LAW}
\end{center}
\ldots
%Section 882
%Pursuant to section 882(a)(1) of the Internal Revenue Code, a foreign corporation engaged in a trade or business within the United States during the taxable year is subject to U.S. federal income tax on its taxable income that is effectively connected with the conduct of a trade or business within the United States.

\begin{center}
\textbf{Definition of a ``Trade or Business Within the United States''}
\end{center}

To be subject to tax under section 882, a foreign corporation must be engaged in a``trade or business within the United States.'' A ```trade or business within the United States' includes the performance of personal services within the United States at any time within the taxable year . . . .'' Section 864(b). The term ``trade or business within the United States'' does not include ``[t]rading in stocks or securities through a resident broker, commission agent, custodian, or other independent agent.'' Section 864(b)(2)(A)(i). This safe harbor does not apply if the taxpayer has an office or other fixed place of business in the United States at any time during the taxable year through which the transactions in stocks or securities are effected. Section 864(b)(2)(C). In addition, the term ``trade or business within the United States'' does not include ``[t]rading in stocks or securities for the taxpayer's own account, whether by the taxpayer or his employees or through a resident broker, commission agent, custodian, or other independent agent, and whether or not any such employee or agent has discretionary authority to make decisions in effecting the transaction.'' Section 864(b)(2)(A)(ii). This clause does not apply in the case of a dealer in stocks or securities.	\emph{Id.}

If a foreign corporation does not qualify for the section 864(b) safe harbors, the unavailability of such safe harbors is not a determination that such foreign corporation is engaged in a trade or business within the United States. Treas. Reg.\@ \S 1.864-2(e). Rather, whether a foreign corporation is treated as engaged in a trade or business within the United States ``shall be determined on the basis of the facts or circumstances in each case.'' Treas. Reg.\@ \S 1.864-2(e).

\begin{center}
\textbf{Definition of ``Effectively Connected Income''---U.S. Source Income}
\end{center}

%Once a foreign corporation is found to be engaged in a trade or business within the United States, the foreign corporation's income must be ``effectively connected'' with the U.S. trade or business to be taxable under section 882(a). Section 864(c) defines when such foreign corporation's income, gain or loss will be treated as effectively connected with the conduct of a United States trade or business. With respect to U.S. source interest income, when determining that such income is effectively connected with the conduct of a trade or business within the United States, the factors taken into account include whether (A) the income is derived from assets used in or held for use in the conduct of such trade or business, or (B) the activities of such trade or business were a material factor in the realization of the income. Section 864(c)(2).

Notwithstanding the ``asset use test'' and the ``business--activities test'' articulated in section 864(c)(2) and the regulations thereunder, Treas. Reg.\@ \S 1.864-4(c)(5) provides a special rule for determining whether income is effectively connected with a ``banking, financing or similar business activity.'' Specifically, any U.S. source interest received by a foreign corporation during the taxable year in the active conduct of a banking, financing, or similar business in the United States is treated as effectively connected to the conduct of that business ``only if the stock or securities giving rise to such income, gain, or loss are attributable to the U.S. office through which such business is carried on'' and the securities were acquired in one of the specified manners enumerated in the regulations, which includes making loans to the public. Treas. Reg.\@ \S1.864-4(c)(5)(ii). A stock or security is deemed to be attributable to a U.S. office ``only if such office actively and materially participates in soliciting, negotiating, or performing other activities required to arrange the acquisition of the stock or security.'' Treas. Reg.\@ \S1.864-4(c)(5)(iii). Treas. Reg.\@ \S 1.864-4(c)(5)(iv) provides rules for determining when a stock or security was acquired in the course of making loans to the public. Even when U.S. source income from stocks and securities is not effectively connected with the active conduct of a foreign corporation's banking, financing or similar business in the United States, such income may be effectively connected with the conduct of another U.S. trade or business under the ``asset-use test,'' as provided in Treas. Reg.\@ \S 1.864-4(c)(2), or the ``business-activities test,'' as provided in Treas. Reg.\@ \S1.864-4(c)(3). Treas. Reg.\@ \S 1.864-4(c)(5)(vi).

\begin{center}
\textbf{Foreign Source Effectively Connected Income}
\end{center}

Generally, foreign source interest income is not treated as effectively connected with the conduct of a United States trade or business. Section 864(c)(4)(A). Foreign source interest income of a foreign corporation derived from the active conduct of a banking, financing, or similar business within the United States, however, is treated as effectively connected with the conduct of a United States trade or business ``if such person has an office or other fixed place of business within the United States to which such income, gain, or loss is attributable.'' Section 864(c)(4)(B). For purposes of section 864(c)(4)(B), when determining whether a foreign corporation has an office or other fixed place of business, the office or other fixed place of business of an agent will be disregarded unless the agent (i) has the authority to negotiate and conclude contracts in the name of the foreign corporation and regularly exercises such authority and (ii) is not a general commission agent, broker or other independent agent acting in the ordinary course of business. Section 864(c)(5)(A). In addition, a foreign corporation's income, gain or loss will not be attributable to an office or fixed place of business in the United States unless such office or fixed place of business ``is a material factor in the production of such income, gain, or loss'' and the office or fixed place of business regularly carries on the type of activities from which such income, gain or loss was derived. Section 864(c)(5)(B).

Treas. Reg.\@ \S 1.864-5 provides rules for determining when a foreign corporation's foreign source income will be treated as effectively connected with a United States trade or business. Treas. Reg.\@ \S1.864-6 provides rules for determining when a foreign corporation that is engaged in a United States trade or business has an office or fixed place of business in the United States.

With respect to a foreign corporation that is engaged in a U.S. trade or business, Treas. Reg.\@ \S 1.864-7 defines the term ``office or other fixed place of business'' for the purposes of Section 864(c)(4)(B), Treas. Reg.\@ \S1.864-6 and Treas. Reg.\@ \S1.864-5(b), all of which are provisions relating to foreign source effectively connected income. Treas. Reg.\@ \S 1.8[64]-7(a)(1) [SIC]. When determining whether a foreign corporation has an office or other fixed place of business with regard to foreign source income, the office of a dependent agent is disregarded unless such agent has the authority to negotiate and conclude contracts in the name of the foreign corporation and regularly exercises that authority. Treas. Reg.\@ \S 1.864-7(d)(1)(i).

\ldots
%Source of Interest Income
%The source of interest income as foreign or domestic depends upon the borrower. In general, interest income from loans made to U.S. borrowers will be sourced as income from sources within the United States. Section 861(a)(1).

\begin{center}
\textbf{ANALYSIS}
\end{center}

%Foreign Corporation is engaged in a ``trade or business within the United States'' pursuant to Section 864(b)(2)
Based on the facts and circumstances described above, Foreign Corporation is engaged in a trade or business within the United States.

\begin{center}
\textbf{Attribution of an Agent's Activities}
\end{center}

Although Origination Co. acts on behalf of Foreign Corporation pursuant to a service contract and does not have authority to conclude contracts, Origination Co. performs activities that are a component of Foreign Corporation's lending activities, such as the solicitation of customers, the negotiation of contractual terms and the performance of credit analyses. In similar circumstances, courts have found an agency relationship to exist in fact and have attributed the activities of the U.S. agent to the foreign principal in determining whether the foreign principal conducted considerable, continuous, and regular activity within the United States. \emph{See Inverworld, Inc. v.\@ CIR}, T.C. Memo. 1996-301 (finding that the activities of a U.S. corporation, although nominally an independent contractor and not an agent, were attributed to a foreign corporation where the activities of the U.S. corporation were in fact those of an agent); I.R.S. Tech. Adv.\@ Mem. 80-29-005 (March 27, 1980) (``In resolving the issue of whether the A Trusts are engaged in a trade or business within the United States for purposes of Section 864(b) of the Code, it is irrelevant whether [the company operating the A Trusts' oil leases] is an independent contractor of the A Trusts or the actual agent of the trusts.'' (citing \emph{Lewenhaupt v.\@ CIR}, 20 T.C. 151 (1953), \emph{aff'd}, 221 F.2d 227 (9th Cir. 1955))). The activities performed by Origination Co., therefore, are attributable to Foreign Corporation for purposes of determining whether Foreign Corporation engages in a trade or business within the United States.

Courts have found a U.S. trade or business where a taxpayer's U.S. activities, either directly or through an agent, are considerable, continuous, and regular. \emph{De Amodio v.\@ CIR}, 34 T.C. 894, 905-06 (1960), \emph{aff'd}, 299 F.2d 623 (3rd Cir. 1962) (concluding that the taxpayer had engaged in a U.S. business because the activities of taxpayer's agent were considerable, continuous and regular, and that those activities, which constituted more than the mere ownership of real property or receipt of income from real property, were attributable to the taxpayer); \emph{Lewenhaupt v.\@ CIR}, 20 T.C. 151 (1953), \emph{aff'd}, 221 F.2d 227 (9th Cir. 1955) (concluding that the taxpayer had engaged in a U.S. business because taxpayer's activities through an agent were considerable, continuous and regular even though the agent received the taxpayer's approval prior to taking any important action); \emph{Handfield v.\@ CIR}, 23 T.C. 633, 637-38 (1955) (concluding that the taxpayer was engaged in a trade or business within the United States because an agent made substantial sales in the United States on behalf of the taxpayer pursuant to a distribution agreement); \emph{Adda v.\@ CIR}, 10 T.C. 273, 277 (1948) (concluding that the taxpayer engaged in a trade or business within the United States through the activities undertaken by the taxpayer's agent). With respect to Foreign Corporation's lending business, Origination Co. undertakes activities on behalf of Foreign Corporation that are more than ministerial and clerical in nature. \emph{See Spermacet Whaling \& Shipping Co. v.\@ CIR}, 30 T.C. 618, 634 (1958), \emph{aff'd}, 281 F.2d 646 (6th Cir. 1960) (holding that the taxpayer was not engaged in a trade or business within the United States where the U.S. activities of its agent were ministerial and clerical activities that involved ``very little exercise of discretion or business judgment necessary to the production of the income''). Because the lending activities of Foreign Corporation, which were carried on by Origination Co., were considerable, continuous, and regular, Foreign Corporation is engaged in a U.S. trade or business.

\begin{center}
\textbf{Lending Trade or Business within the United States}
\end{center}

The activities with respect to Foreign Corporation's loans to U.S. Borrowers constitute a trade or business because Foreign Corporation lends money to customers on a considerable, regular and continuous basis with the intention of earning a profit. \emph{Compare Inverworld, Inc. v.\@ CIR}, T.C. Memo. 1996-301 (finding that a foreign corporation was engaged in a trade or business within the United States when its activities in the United States, including lending money to clients, were regular and continuous enough to constitute `` `a banking, financing or similar business in the United States' '') with \emph{Pasquel v.\@ CIR}, 12 T.C.M. 1431 (1953) (finding that a taxpayer was not engaged a trade or business within the United States when the taxpayer entered into a ``single and isolated'' financing transaction in the United States). Such trade or business is treated as being within the United States because Foreign Corporation's loan origination activities conducted through Origination Co. occur within the United States. \emph{Adda v.\@ CIR}, 10 T.C. at 277-78 (finding that the taxpayer engaged in a trade or business within the United States through the activities of an agent even though the agent did not have an office in the United States); \emph{Inverworld, Inc. v.\@ CIR}, T.C. Memo. 1996-301; \emph{see also, e.g., Pinchot v.\@ CIR}, 113 F.2d 718, 719-720 (2d. Cir. 1940) (finding that the taxpayer was engaged in a U.S. business because the activities in the United States were considerable, regular and continuous).

\begin{center}
\textbf{Section 864(b)(2) Safe Harbors}
\end{center}

Further, because Foreign Corporation regularly and continuously originates loans to customers, such activities constitute a lending trade or business and not trading or investing activities for the purpose of section 864. \emph{Compare Inverworld}, T.C. Memo. 1996-301 with \emph{Higgins v.\@ CIR}, 312 U.S. 212 (1941) (holding that investing, no matter how extensive the activity, is not a trade or business) and \emph{Yaeger v.\@ CIR}, 889 F.2d 29, 33-34 (2d Cir. 1989) (holding that the taxpayer was an investor rather than a trader because the management of personal securities investment is not the trade or business of a trader and noting that the fundamental criteria that distinguishes a trader from an investor is the length of the holding period and the source of profit).	As the Foreign Corporation's lending activities do not constitute ``trading'' in stock and securities, Foreign Corporation does not qualify for the trading safe harbors under section 864(b)(2). Rather, based upon the facts and circumstances, Foreign Corporation is engaged in a U.S. trade or business. Treas. Reg.\@ \S 1.864-2(e).

\begin{center}
\textbf{Foreign Corporation has income effectively connected with a banking or financing business within the United States}
\end{center}

The interest income that Foreign Corporation receives with respect to the loans originated in the United States is effectively connected with the conduct of a trade or business within the United States because Foreign Corporation is engaged in a banking business and such interest income is attributable to an office in the United States.

\begin{center}
\textbf{Banking, Financing or Similar Business Activity}
\end{center}

Foreign Corporation is treated as engaged in a banking, financing or similar business activity within the United States as described by Treas. Reg.\@ \S 1.864-4(c)(5)(i) because its business, through the activities of Origination Co., includes making loans to the public. Because Foreign Corporation is engaged in a banking, financing, or similar business activity, its income from that business may be effectively connected with a trade or business in the United States, notwithstanding the ``asset use test'' or the ``business activity test.'' Treas. Reg.\@ \S 1.864-4(c)(5)(i).

\begin{center}
\textbf{The Office Requirement of Treas. Reg.\@ \S 1.864-4(c)(5)}
\end{center}

Foreign Corporation's U.S. source interest income from a banking, financing or similar business will be treated as effectively connected income if the securities giving rise to such income are ``attributable to the U.S. office through which such business is carried on'' and the securities were acquired as a result of making loans to the public. Treas. Reg.\@ \S1.864-4(c)(5)(ii). The regulation requires that the income be attributable to ``the U.S. office through which such business is carried on . . . .'' Treas. Reg.\@ \S 1.864- 4(c)(5)(ii) (emphasis added). The regulation does not specify or imply that the U.S. office belong to or be attributable to the taxpayer.

The Service is aware that some taxpayers may have taken the position that the interest income is not effectively connected with banking, financing or similar business activity because the income is not attributable to a U.S. office of the Foreign Corporation and that the office of Foreign Corporation's agent is not attributable to the Foreign Corporation under Treas. Reg.\@ \S 1.864-7(d). Because Treas. Reg.\@ \S 1.864-4(c)(5) does not provide guidance defining the phrase ``the U.S. office,'' a taxpayer may argue that the definition of the phrase ``office or other fixed place of business'' provided in Treas. Reg.\@ \S 1.864-7 should apply to interpret the phrase ``the U.S. office.'' Under the taxpayer's analysis, because Origination Co. is either an independent agent or does not have the authority to conclude loans on behalf of Foreign Corporation, the office of Origination Co. is not attributable to Foreign Corporation.

This argument misapplies both the statute and the regulations. Unlike section 864(c)(4)(B), section 864(c)(2) contains no ``office or other fixed place of business'' requirement. Section 864(c)(4)(B) and Treas. Reg.\@ \S 1.864-7 apply only for the purpose of foreign source effectively connected income described in section 864(c)(4)(B) and the regulations thereunder. Treas. Reg.\@ \S 1.864-7(a)(1). Because the interest income received by Foreign Corporation with respect to loans made to U.S. Borrowers is U.S. source income, the definition contained in Treas. Reg.\@ \S 1.864-7 does not apply.

Notwithstanding the court�s reliance upon Treas. Reg.\@ \S 1.864-7 in \emph{Inverworld}, the framework of Treas. Reg.\@ \S 1.864-7 is not relevant to the application of Treas. Reg.\@ \S 1.864-4(c)(5) in this case. In \emph{Inverworld}, the court used Treas. Reg.\@ \S 1.864-7 as a framework for interpreting section 864(b)(2)(C) ``because those regulations construe the phrase `office or other fixed place of business in the United States', which is also found in section 864(b)(2)(C)'' even though Treas. Reg.\@ \S 1.864-7 does not expressly apply to Section 864(b)(2)(C). T.C. Memo. 1996-301. As previously stated, unlike section 864(b)(2)(C) and Treas. Reg.\@ \S 1.864-7, which are concerned with whether or not the taxpayer has a U.S. office (either directly or by attribution), Treas. Reg.\@ \S 1.864-4(c)(5)(ii) does not require that the taxpayer have a U.S. office.

The rule in Treas. Reg.\@ \S 1.864-4(c)(5)(ii) elaborates a statutory provision that does not contain the same ``office or other fixed place of business'' requirements found in other sections. As a result, the regulation cannot be read to import the same ``office or other fixed place of business'' rule of section 864(c)(4)(B). It does not, for example, require by its terms that the office be the office of the taxpayer. A U.S. office of an agent of the taxpayer is sufficient. If the regulations intended that interest income must be attributable to the taxpayer's office to be treated as effectively connected with a banking, financing or other similar business, the regulation would have explicitly stated that the income must be attributable to ``the taxpayer's office.'' Alternatively, the text of the regulation would have used a possessive pronoun to indicate that the office must be the taxpayer's office. Because the regulation requires only that the interest income be attributable to ``the U.S. office,'' a U.S. office of a person other than the taxpayer may satisfy the requirement. Origination Co.'s office satisfies the office requirement articulated in Treas. Reg.\@ \S 1.864-4(c)(5)(ii) because Origination Co. has an office in the United States and the day-to-day activities required of Foreign Corporation�s lending business take place from the office of Origination Co.

\begin{center}
\textbf{Interest Income Attributable to the U.S. Office}
\end{center}

In order to have effectively connected income, the loans originated by Foreign Corporation must be attributable to a U.S. office. A loan will be attributable to a U.S. office ``only if such office actively and materially participated in soliciting, negotiating or performing other activities required'' for the acquisition of such loan. Treas. Reg.\@ \S 1.864-4(c)(5)(iii). Foreign Corporation has engaged Origination Co. to perform its origination activities in the United States, including the solicitation of borrowers and the negotiation of contractual terms. For this purposes, it is enough that Origination Co. is a dependent or independent agent of the taxpayer performing activities described above. To perform loan origination activities on behalf of Foreign Corporation, Origination Co. operates from an office in the United States. Origination Co.'s U.S. office actively and materially participates in the origination of Foreign Corporation's loans to U.S. Borrowers because the activities required to originate such loans occur through that U.S. office. The income from Foreign Corporation's loans to U.S. Borrowers, therefore, is attributable to ``the U.S. office'' of Origination Co. through which Foreign Corporation carries on its lending business.

Because Origination Co.'s U.S. office actively and materially participated in the day-to-day origination activities, Foreign Corporation's U.S. source interest income is attributable to Origination Co.'s U.S. office, even though Foreign Corporation concluded the loans outside of the United States. Foreign Corporation's interest income with respect to loans made to U.S. Borrowers, therefore, is effectively connected with a trade or business within the United States pursuant to section 864(c)(2).

%CASE DEVELOPMENT, HAZARDS AND OTHER CONSIDERATIONS
%We understand that foreign corporations and non-resident aliens may have used other strategies to originate loans in the United States giving rise to effectively connected income. We encourage you to develop these cases, and we stand ready to assist you in the legal analysis.
%Please call Peter Merkel of the Office of the Associate Chief Counsel (International) at (202) 622-3870 (not a toll-free number) if you have any further questions.

\end{select}

	\section{Permanent Establishments}

 \addcontentsline{toc}{section}{\protect\numberline{}Handfield v.\@ CIR} 
\begin{select}
\caseart{Handfield v.\@ CIR}{23 T.C. 633 (1955)}{ARUNDELL, Judge}

\ldots\\

\begin{center} \textbf{FINDINGS OF FACT}
\end{center}
[Handfield, a Canadian NRA operating as a sole proprietorship, manufactured in Canada post cards sold under the trade name `Folkards.' The business operated under the style of Folkard Company of America.  Handfield visited the U.S. for 24 days in four trips and also had a U.S. employee whose duties were to check the vendors of the American News Company (ANC) to insure that the cards were being properly displayed.  A letter setting forth the business arrangements between Handfield and ANC read as follows:
	\begin{quote}

This letter will confirm arrangements recently discussed for the exclusive distribution through our Company of Folkards, in any United States city in which it is mutually agreed to put these out. It is understood that each rack will contain 300 Folkards, and will be similar to those now being distributed in Canada.

Folkards will be billed to The ANC, Inc., at \$2.40 per rack; trade price, \$3.60 per rack; retail, \$6.00 per rack or 2 cents per card; fully returnable. It is understood that transportation, both on shipments to branches and return shipments to you, is to be assumed by the manufacturer. It is also understood that you will accept for credit all unsold Folkards, regardless of condition.

Payments will be made to you on the basis of actual check-ups of dealers' stock sixty days after distribution, and every thirty days thereafter. It is further understood as the distribution is extended, The ANC will have exclusive rights to distribute Folkards in the United States. If, however, the sale in any city should be unsatisfactory, we will pick up stock from dealers and return it to you within sixty days after it is mutually agreed to discontinue the distribution.
	\end{quote}

	Hanfield claimed a net income of \$883.70, but the CIR disallowed certain expenses, including salary paid to Handfield.]
	
	\begin{center} \textbf{OPINION}
	\end{center}

\ldots 

The petitioner manufactures a novelty item called Folkards which is a kind of postal card. He had a contract with the ANC by which the latter distributed his cards to newsstands in the United States where they were sold to the public. The petitioner contends that the ANC purchased the cards from him for resale. He further contends that the sale occurred in Canada when the cards were placed in transportation and at that time he surrendered all his right, title, and interest in the cards to the ANC.

The respondent contends that the arrangement between the petitioner and the ANC provided for an agency relationship, and that the ANC was petitioner's exclusive distributor in the United States.

\ldots 

It will be observed that the agreement between the petitioner and the ANC nowhere says that the ANC buys or will buy the Petitioner's cards or that the company is or will be obligated for any definite number of cards or in any definite amount. The contract uses the word `sale' twice. In each instance it is clear that the word refers to transactions with the public, not between the petitioner and the News Company. Thus, the contract states, `If * * * the sale in any city should be unsatisfactory, we will pick up stock from dealers, and return it to you, * * *.' And, also, that the ANC `reserves the right to withdraw them (the cards) from sale without notice' when copyright or patent infringement is threatened. The contract speaks of its purpose as confirmation of `arrangements recently discussed for the exclusive distribution through our Company' in the United States where it is `mutually agreed to put these (cards) out.' The contracts specifies the rate at which the ANC will be billed for the cards, the rate at which the cards will be billed to the `trade,' and the retail price at which the cards will be sold. But, payments were to be made `on the basis of actual check-ups of dealers' stocks sixty days after distribution, and every thirty days thereafter.' The contract stated that all cards were `fully returnable' and that transportation on shipments to and from the United States was to be paid by the petitioner and that he would allow credit on all unsold cards, regardless of condition.

The contract gave exclusive rights to the ANC `to distribute Folkards in the United States' and, as noted above, the ANC could `pick up stock from dealers and return it' after it `mutually agreed to discontinue the distribution' in any city.

The foregoing language raises some doubt whether the ANC actually sells the cards to the public or whether it acts as a distributor to newsdealers who sell to the public. We do not have enough information in the record to make any findings concerning the relationship between the ANC and the dealers. In our view of the case, it is immaterial precisely what that relationship may be because, as will appear below, the important relationship is that between the petitioner and the News Company.

Petitioner visited the United States occasionally to check on his arrangement with the ANC and during the period in issue, he was in the country for a total of 24 days on four different visits. However, he had an employee in the United States whom he paid to visit the various outlets of the ANC checking to insure that the cards were properly displayed and retailed.

From all the provisions of the contract and all the information on the operations of the petitioner in relation to it that are in this record, we think that the arrangement between the petitioner and the ANC was one in which the ANC was his agent in the United States. We think that the cards were shipped on consignment to the ANC for sale to the public. All the aspects of the agreement point to this interpretation of the contract and none are inconsistent with this interpretation.

The features of the contract which are particularly persuasive in bringing us to the interpretation we have placed on it are: The ANC does not obligate itself to buy any definite amount of merchandise from petitioner and it is obligated only to account for the merchandise which has been sold; all merchandise unsold may be returned; the petitioner will pay the transportation on the cards to and from Canada and give full credit for all cards unsold regardless of their condition; the agreement controls the retail price; and it gives the ANC the right to discontinue merchandising the cards when they move slowly or when they infringe copyright or patent provisions. All these, taken together, we think indicate that the arrangement was an agency relationship in the form of a contract of consignment. 

\ldots 

Article III of the [U.S.-Canada Treaty] subjects the industrial and commercial profits as a Canadian enterprise derived through a `permanent establishment' within the United States to the income taxes of this country. The Protocol implementing the Convention defines an `enterprise' as `every form of undertaking, whether carried on by an individual, partnership, corporation or any other entity,' and a `permanent establishment' as follows, in part:
\begin{quote}
When an enterprise of one of the contracting States carries on business in the other contracting State through an employee or agent established there, who has general authority to contract for his employer or principal or has a stock of merchandise from which he regularly fills orders which he receives, such enterprise shall be deemed to have a permanent establishment in the latter State. (Paragraph 3(f).)
\end{quote}
The ANC, under its contract with petitioner, was an `agent' in the United States with a `stock of merchandise' from which it regularly filled orders for the public. Therefore, within the meaning of the above Protocol, we think the petitioner had a `permanent establishment' in the United States under his arrangement with the ANC. It follows, then, that he was engaged in business within the United States in the year in issue and the income from his operations in this country is subject to taxation \ldots 

\end{select}

 \addcontentsline{toc}{section}{\protect\numberline{}Rev.\@ Rul.\@ 2004-3}
		\begin{select}
		\revrul{Rev.\@ Rul.\@ 2004-3}{2004-1 C.B. 486}

	\ldots

\begin{center} \textbf{FACTS}
\end{center} 

P is a service partnership that is organized under the laws of Germany. P has offices in Germany and the United States. Its U.S. office is a fixed base under Article 14 of the Treaty. P is comprised of two partners: A, a nonresident alien individual who is a resident of Germany under Article 4 of the Treaty, and B, a U.S. resident. A performs services solely at P's office in Germany and B performs services solely at P's office in the United States. A and B agree to divide the profits of the partnership equally.

\begin{center} \textbf{LAW AND ANALYSIS}
\end{center} 

\ldots

Under section 875(1) of the Code, a nonresident alien individual who is a partner in a partnership that is engaged in a U.S. trade or business is himself considered to be so engaged. Section 871(b)(1) of the Code provides that a nonresident alien individual is taxable under Code sections 1 or 55 on his taxable income that is effectively connected with the conduct of a U.S. trade or business.

Section 894(a)(1) states that the provisions of the Code shall be applied to any taxpayer with due regard to any U.S. treaty obligation that applies to such taxpayer. In Donroy, Ltd. v.\@ United States, 301 F.2d 200 (9th Cir. 1962), the court held that the U.S. permanent establishment of a partnership was attributable to a foreign person that was a limited partner under the 1942 U.S.-Canada income tax treaty. In Unger v.\@ CIR, 936 F.2d 1316, 1319 (D.C. Cir. 1991), the court followed the holding in Donroy, noting that it stood for the proposition that the office or permanent establishment of a partnership is, as a matter of law, the office of each of the partners--whether general or limited. See also Johnston v.\@ CIR, 24 T.C. 920 (1955) (holding that a partnership�s permanent establishment is deemed to be a permanent establishment of its partners); Rev.\@ Rul.\@ 90-80, 1990-2 C.B. 170 (same).

Article 14 of the Treaty provides:

\begin{quote} Income derived by an individual who is a resident of a Contracting State from the performance of personal services in an independent capacity shall be taxable only in that State, unless such services are performed in the other Contracting State and the income is attributable to a fixed base regularly available to the individual in that other State for the purpose of performing his activities.

The term ``personal services in an independent capacity'' includes but is not limited to independent scientific, literary, artistic, educational, or teaching activities as well as the independent activities of physicians, lawyers, engineers, economists, architects, dentists, and accountants.

\end{quote}

Applying Article 14 in the partnership context requires a determination of whether an individual partner in a service partnership who derives income attributable to the fixed base of the service partnership in the other Contracting State is taxable on that income even though the partner does not perform any services in the other Contracting State. Consistent with section 875 and the case law discussed above, the fixed base of a partnership is attributed to its partners for purposes of applying Article 14 of the Treaty. Accordingly, A is treated as having a fixed base regularly available to him in the United States. A is subject to U.S. net income taxation on his allocable share of income from P to the extent that such income is attributable to the fixed base in the United States without regard to whether A performs services in the United States.

\begin{center} \textbf{HOLDING}
\end{center} 

A is treated as having a fixed base regularly available to him in the United States and is subject to U.S. net income taxation on his allocable share of income from P to the extent that such income is attributable to P�s fixed base in the United States, without regard to whether A performs services in the United States. This holding also is applicable in interpreting other U.S. income tax treaties that contain provisions that are the same or similar to Article 14 of the Treaty.
\end{select}

Article 5(6) and 5(7) of the Treaty deal with the issue of agents.  In general, the activities of a dependent agent are imputed to the principal in determining whether the principal has a permanent establishment in the treaty country.  The activities of an independent agent, however, are not imputed to the principal.  The Taisei case explores the distinction between the two. 


\addcontentsline{toc}{section}{\protect\numberline{}The Taisei Fire and Marine Insurance Co., Ltd. v.\@ CIR} 
\begin{select}
\caseart{The Taisei Fire and Marine Insurance Co., Ltd. v.\@ CIR}{ 104 T.C. 535 (1995)}{TANNENWALD, Judge}

\ldots\\

The principal issue in these consolidated cases is whether, during the years at issue, petitioners had a U.S. permanent establishment by virtue of the activities of a U.S. agent in accepting reinsurance on behalf of each petitioner. \ldots 
\begin{center} \textbf{FINDINGS OF FACT}
\end{center}
\ldots 

Each petitioner is a Japanese property and casualty insurance company with its principal place of business in Japan. 
The stock of each petitioner is publicly traded on a Japanese exchange. There is no stock ownership relationship among 
petitioners. 

The primary business of each petitioner is writing direct insurance in Japan. Each petitioner also assumes reinsurance 
ceded\footnote[3]{ As used herein, the term ``cede'' refers to the act whereby an insurer (the ceding company) passes on all or part of the insurance it has written to another insurer (the reinsurer), with the object of reducing the net liability of 
the ceding company. Where the reinsurer cedes to another reinsurer all or part of the reinsurance it has previously 
assumed, the term ``retrocede'' or ``retrocession'' applies.} to it by insurers and reinsurers, including U.S. insurers and reinsurers, through a reinsurance department located 
in Tokyo. Each petitioner obtains foreign reinsurance through foreign brokers that bring reinsurance proposals to it, and 
from foreign insurers and reinsurers with which each petitioner has a direct relationship.

Each petitioner has at least one representative office in the United States that provides information on the U.S. market 
to it and assists its clients in the United States, but which does not have authority to write any form of insurance. Taisei 
and Fuji do not have U.S. branches and do not have licenses to engage in the insurance business in the United States. 

\ldots

In addition, each petitioner grants authority to two or three different U.S. agents, including Fortress Re, Inc., to
underwrite reinsurance on its behalf and to perform certain activities in connection therewith. 

Fortress Re, Inc. [is owned by Maurice Sabbah, who is CEO, his family, and Kenneth Kornfeld, who is president and chief underwriter.] \ldots 
 
Mr. Sabbah handles contacts with insurance companies Fortress represents and has responsibility for reports provided 
to those companies, in addition to certain administrative responsibilities. Mr. Kornfeld's duties include underwriting the 
reinsurance entered into on behalf of the companies Fortress represents, establishing retrocession programs with respect to 
its reinsurance treaties, managing claims with respect to those treaties, and managing the daily affairs of  Fortress. 
Mr. Sabbah and Mr. Kornfeld have total control over the daily operations of Fortress, including the hiring and firing of 
employees and the assigning of responsibilities to them. Fortress has approximately 20 employees, whose duties include 
assisting underwriting, handling claims, data processing and computer operations, secretarial support, and accounting 
services. 

Fortress maintains leased offices in Burlington, North Carolina, for which it pays the rent. Fortress purchases property 
and liability insurance in connection with its business. The operating costs of Fortress, including rent and salaries, 
are borne by Fortress. 

Fortress is a reinsurance underwriting manager, which involves acting as an agent for insurance companies in 
underwriting and managing reinsurance on behalf of such companies. Fortress is not licensed to conduct insurance or 
reinsurance business in any jurisdiction. Fortress underwrites reinsurance and places retrocessions only on behalf of the 
companies with which it enters into management agreements. Fortress enters into reinsurance and retrocession contracts 
on behalf of the companies it represents only through brokers; Fortress itself does not act as a broker. 

Fortress enters into a separate management agreement with each insurance company it represents. The agreements 
with petitioners are identical except for the net acceptance limit \ldots. Since its inception, Fortress has been 
involved in as many as 10 management agreements in a management year. A management year is defined as the annual 
period from July 1 to June 30. From inception through June 30, 1989, the management years have been designated as 
years I through XVI, respectively. 

 \ldots. In the agreements, Fortress is referred to as the 
``manager'', and the insurance company is referred to as a ``member''. Each agreement authorizes Fortress, among other 
things, to act as agent of each company to underwrite and retrocede reinsurance on behalf of each company. Under the 
agreement, the liability of the member with respect to each reinsurance contract, underwritten by Fortress on the 
member's behalf, is several and not joint with any other member. 

Under the agreement, it is contemplated that Fortress may enter into similar, or substantially similar, management 
agreements with other insurance or reinsurance companies or other insurers. Fortress does not need permission of, or even 
to consult, the companies with which it has agreements, before entering into a new agreement. Although in practice, when 
a member terminated a management agreement, Fortress offered to increase the participation of the companies it already 
represented, it was not obligated to do so. \ldots.

\ldots. Fortress is responsible for the handling and disposition of all claims against the companies it represents. In many cases, claims relating to the reinsurance underwritten by Fortress on behalf of companies it represents are not fully settled for 
many years. Fortress has total control over the handling and disposition of claims on behalf of petitioners. 

Pursuant to each agreement, Fortress regularly exercises the authority to conclude original reinsurance 
contracts and to cede reinsurance on behalf of each petitioner. Each agreement provides Fortress with underwriting 
authority on a continuous basis until the agreement is terminated. The agreements can be terminated by either party, but 
only with 6 months' notice, although in practice the notice period has been waived. After termination of an agreement, 
Fortress continues to have obligations with respect to reinsurance previously underwritten. During the years in issue, 
Fortress had continuing duties to 13 insurance companies, excluding petitioners, for contracts underwritten in 
prior management years. 

The only material limitation on Fortress' authority under the agreement is a ``net acceptance limit'', which is the 
maximum amount of net liability in respect of any one original reinsurance contract that Fortress can accept on behalf of 
a member. There is no gross acceptance limit in the agreements, so that Fortress can underwrite reinsurance contracts 
on behalf of a member that are greater than the net acceptance limit, provided that Fortress arranges for retrocessions of 
the excess over the net acceptance limit. In practice, Fortress sets its own gross acceptance limit, as to which it 
voluntarily advises petitioners. When approached by Chiyoda with regard to inserting a gross acceptance limit into its 
agreement, Fortress refused, and Chiyoda dropped its request. 

Before each management year, Fortress provided each petitioner with an estimate of net premium income for the 
upcoming year, based on the gross and net participations of that petitioner. Net premium income equals the gross premiums 
received for all reinsurance contracts less retrocession premiums. Under the terms of the management agreement, Fortress 
is not limited on how many contracts it writes, only that no one contract can exceed the net acceptance limit, so in effect 
the total net premium income from reinsurance handled by Fortress is unlimited. When any of petitioners approached 
Fortress about lowering its net premium income, Fortress' advice was to increase the quota share cession to its affiliated 
quota share reinsurer, Carolina Reinsurance Ltd. (hereinafter referred to as Carolina Re).  In 1986, Fortress anticipated a very favorable market and sought to increase its capacity. It did so by offering to 
increase the underwriting done for its existing members and by soliciting four other Japanese insurance companies 
about entering into management agreements. 

Each reinsurance contract underwritten by Fortress is executed with a ``security stamp'' that identifies each member 
and the percentage of the total liability assumed by each member under the contract. The percentage to be assumed by 
each member on each reinsurance contract entered into during the management year is determined before the start of the 
management year, after Fortress comes to terms with each member regarding the net acceptance limit for the 
year. \ldots  The percentage of each contract assumed is subject to that limit. 

Mr. Kornfeld is the chief underwriter and, as such, decides what business Fortress will underwrite and 
retrocede on behalf of the members. The retrocession program for a management year was presented to each petitioner in advance, during an annual trip by Mr. Kornfeld to each petitioner's offices. However, Fortress does not need approval, and 
did not seek or receive input, from petitioners.
 
All original reinsurance contracts, as defined in the management agreements, are ``excess of loss" contracts. The lines 
of reinsurance that were the subject of original reinsurance contracts and ceded reinsurance were aviation excess of loss, 
nonmarine catastrophic excess of loss (e.g., land-based risks such as hurricanes, tornadoes, earthquakes, and fires), and 
marine excess of loss (e.g., water-based risks involving oil rigs and ocean liners). With excess of loss reinsurance, the 
reinsured pays a premium to a reinsurer for a layer of protection, whereby the reinsurer agrees to pay all losses above a 
certain amount (the retention), but only up to a certain limit. There may be several layers of protection where a different 
reinsurer ``writes" each layer, the lowest layer being the first to bear a loss. Generally, Fortress wrote a percentage of a single layer of loss. In choosing which layer to write, Fortress tries to pick the first ``true" catastrophic layer; i.e., a 
layer that would not bear ordinary losses but would be the first to bear a loss from a catastrophe or extraordinary loss. The 
rationale behind this strategy is that the reinsurer of lower layers receives a higher premium, and Fortress feels 
a true catastrophe would cause losses at all layers so that Fortress is getting a higher premium than reinsurers of higher 
layers, yet bears similar risk. In respect of each reinsurance contract, Fortress independently decides which layer it should 
write on behalf of petitioners. 

As part of its retrocession program, Fortress cedes a part of the liability it writes, so that no reinsurance contract 
exposes its members to a direct liability greater than their net acceptance limit. In such a situation, each petitioner is liable 
in the event the reinsurers should default. The benefit of writing more reinsurance than it accepts on behalf of petitioners 
is that a commission is earned on the ceded reinsurance. Fortress transacted retrocessions through brokers, most of which 
are in London and the rest in New York. 

\ldots.

Fortress was compensated for its services pursuant to compensation schedules set forth in each agreement. During 
1986 to 1988, Fortress' income was derived from management fees, contingent commissions, and override commissions 
payable under management agreements entered into for management years I through XVI. Also, Fortress earned 
investment income on its own funds, which was not related to the management agreements. Fortress' compensation 
structure is the same as other reinsurance underwriting managers, although its management fees are slightly lower and its 
profit commissions slightly higher than the norm. 

\ldots

Fortress has the authority, under the management agreements and its agreement with old Fortress, to control the 
investment of funds withheld pursuant to such agreements in its sole discretion. Distributions are made in accordance 
with the management agreements and are made over a period of years following the management year. 
 
\ldots In 1984, the owners of Fortress caused the formation of Carolina Re, under the laws of Bermuda. In 1984, the stockers of Carloina Re were [Fortress (99\%) and various other shareholders.]

In 1986, Fortress sold its shares in Carolina Re for \$1 each [to the Sabbah's and Kornfeld.]

Carolina Re acts as a quota share reinsurer of reinsurance underwritten by Fortress, meaning that it assumes a certain 
share of the reinsurance ceded by petitioners, for which it is paid a premium. Before forming Carolina Re, Fortress 
notified petitioners of its intentions although it did not need their approval. 

Fortress requires that Carolina Re be retroceded a minimum percentage of the reinsurance contracts accepted 
on behalf of each petitioner, although each petitioner may increase the percentage. In 1988, at the insistence of Fortress, 
the minimum percentage ceded to Carolina Re was increased, despite the objections of three of four petitioners. 

\ldots. Each petitioner filed protective Federal income tax returns for the years in issue with 
the Internal Revenue Service Center at Philadelphia, Pennsylvania. \ldots. 

\begin{center} \textbf{OPINION}
\end{center} 
Under the [US-Japan Treaty], the commercial profits of a Japanese resident are exempt from 
U.S. Federal income tax, unless such profits are attributable to a U.S. permanent establishment. Convention, Art. 8(1). The 
relevant provisions of the convention whereby a Japanese resident will be deemed to have a U.S. permanent establishment 
due to the activities of an agent are as follows: 
\begin{quote}
(4) A person acting in a Contracting State on behalf of a resident of the other Contracting State, other than 
an agent of an independent status to whom paragraph (5) of this article applies, shall be deemed to be a permanent 
establishment in the first-mentioned Contracting State if such person has, and habitually exercises in the first-mentioned 
Contracting State, an authority to conclude contracts in the name of that resident, unless the exercise of such authority is 
limited to the purchase of goods or merchandise for that resident. 

(5) A resident of a Contracting State shall not be deemed to have a permanent establishment in the other Contracting 
State merely because such resident engages in industrial or commercial activity in that other Contracting State 
through a broker, general commission agent, or any other agent of an independent status, where such broker or agent is 
acting in the ordinary course of his business. 
[Convention, Art. 9.]
\end{quote} 
Initially, it is undisputed that Fortress had the authority, which it exercised, to conclude contracts on behalf of 
petitioners, so that unless Fortress is ``a broker, general commission agent, or any other agent of an independent status'' 
within the meaning of Article 9(5), petitioners will be deemed to have U.S. permanent establishments. The parties are in 
agreement that Fortress was not a ``broker" or ``general commission agent", and respondent concedes that Fortress was 
acting in the ordinary course of its business when acting on behalf of petitioners. Thus, the issue before us is whether, 
during the years at issue, Fortress was an "agent of an independent status" in respect of each petitioner. In this connection, 
we note that neither petitioners nor respondent has argued that any petitioner should be treated differently from any other 
petitioner in resolving this issue. 

\begin{center} \textbf{Background}
\end{center} 
The U.S.--Japan convention itself does not define an ``agent of an independent status". In applying a treaty definition, ``Our role is limited to giving effect to the intent of the Treaty parties." Beyond the literal language, we must examine the treaty's ``purpose, history and context." 

Our examination shows that the relevant provisions of the convention are not only based upon, but are duplicative 
of, Article 5, comments 4 and 5, of the 1963 O.E.C.D. Draft [model] Convention (hereinafter referred to as the 1963 
model).\footnote[5]{Art. 5 of the 1963 model provides in part:
 
4. A person acting in a Contracting State on behalf of an enterprise of the other Contracting State--other than an 
agent of an independent status to whom paragraph 5 applies--shall be deemed to be a permanent establishment in 
the first-mentioned State if he has, and habitually exercises in that State, an authority to conclude contracts in the 
name of the enterprise, unless his activities are limited to the purchase of goods or merchandise for the enterprise. 

5. An enterprise of a Contracting State shall not be deemed to have a permanent establishment in the other 
Contracting State merely because it carries on business in that other State through a broker, general commission 
agent or any other agent of an independent status, where such persons are acting in the ordinary course of their 
business. }\ldots  While the 1963 model 
itself provides no more definition than the convention, the model is explained in part by a commentary, which states in 
pertinent part: 
\begin{quote}
15. Persons who may be deemed to be permanent establishments must be strictly limited to those who are dependent, 
both from the legal and economic points of view, upon the enterprise for which they carry on business dealings of the Fiscal Committee of the League of Nations, 1928, page 12). Where an enterprise has business dealings with an 
independent agent, this cannot be held to mean that the enterprise itself carries on a business in the other State. In such a 
case, there are two separate enterprises. 

* * * * 

19. Under paragraph 4 of the Article, only one category of dependent agents, who meet specific conditions, is deemed to 
be permanent establishments. All independent agents and the remaining dependent ones are not deemed to be 
permanent establishment. Mention should be made of the fact that the Mexico and London Drafts * * * and a number of 
Conventions, do not enumerate exhaustively such dependent agents as are deemed to be permanent establishments, but 
merely give examples. In the interest of preventing differences of interpretation and of furthering international economic 
relations, it appeared advisable to define, as exhaustively as possible, the cases where agents are deemed to be 
``permanent establishments".
 
* * * * 

20. * * * In the Mexico and London Drafts and in the Conventions, brokers and commission agents are stated to be agents 
of an independent status. Similarly, business dealings carried on with the co-operation of any other independent person 
carrying on a trade or business (e.g. a forwarding agent) do not constitute a permanent establishment. Such independent 
agents must, however, be acting in the ordinary course of their business. * * * 

* * * * 
\end{quote}

The special problems which can arise in the case of insurance companies dealing by means of intermediaries or variously qualified representatives shall be further studied. [Commentary to Art. 5 of the 1963 model.] 
 
Based on the above, petitioners argue that the test of independent status is one of both legal and economic dependence 
and that, if we find that Fortress was either legally or economically independent of petitioners, it will necessarily 
follow that Fortress was not a permanent establishment. Respondent argues that comment 15 erroneously phrased the 
standard in terms of ``dependence" and the conjunctive ``and'' instead of the disjunctive ``or'', thus allowing either legal 
or economic independence to satisfy the requirement for independent status. The basis for this argument is that comment 15 of the 1963 model expressly refers to the Report of the Fiscal Committee 
of the League of Nations 12 (1928), the commentary to which states: ``The words `bona-fide agent of independent status' 
are intended to imply absolute \textit{independence, both from the legal  and economic points of view"} (emphasis 
supplied); Indeed, the commentary to the OECD model was changed in the 1977 revision so that both legal and economic independence is necessary.  Generally, we would have reservations 
about interpreting a convention, ratified in 1971, on the basis of a commentary, adopted in 1977, that contradicts the literal language of the commentary in effect at the time of ratification. However, in light of the extensive analysis by the 
previously cited commentators and the confirmation of such analysis by our own research, we are persuaded that the 
criteria in the later commentary reflects the original intention of the commentary to the 1963 model and that the 1963 
model should be interpreted as having a disjunctive (``or") meaning. 
 
We note, however, that if we focus, as the parties have ultimately done, on the test for legal and economic independence 
set forth in comment 37 to Article 5 of the 1977 model, as applied to the facts herein, the issue of disjunctive versus 
conjunctive reading of the 1963 model fades into the background. That comment provides: 

\begin{quote}37. Whether a person is independent of the enterprise represented depends on the extent of the obligations which this 
person has vis-a-vis the enterprise. Where the person's commercial activities for the enterprise are subject to 
detailed instructions or to comprehensive control by it, such person cannot be regarded as independent of the enterprise. 
Another important criterion will be whether the entrepreneurial risk has to be borne by the person or by the enterprise the 
person represents. * * * [Comment 37 to Art. 5 of the 1977 model.] 
\end{quote}
It is obvious that the tests of ``comprehensive control'' and ``entrepreneurial risk'', as the determinants of legal and 
economic independence, involve an intensely factual inquiry, which does not lend itself to the articulation of a ``definitive 
statement that would produce a talisman for the solution of concrete cases." 

Petitioners suggest that guidance can be found in the factors used in distinguishing employees from independent
contractors.  We think 
the employee versus independent contractor analogy is of limited use. The fact that petitioners herein are clearly not 
employees (indeed, respondent does not contend that they are) and therefore would be considered independent contractors 
does not answer the question before us, namely, whether they are the kind of independent contractors who should be held 
to be ``agent(s) of an independent status''. Nor are we prepared to accept respondent's argument that the quoted phrase 
should be given a narrow scope by virtue of the ejusdem generis rule in that it was intended to encompass only those 
agents who exhibited characteristics associated with a ``broker" or ``commission agent". We think that the generality of 
the phrase ``agent of an independent status" was intended to have an expansive rather than a confining scope, particularly 
since the words ``broker'' and ``commission agent'' themselves lack specificity. Respondent's reliance on Fleming 
(H.M. Inspector of Taxes) v.\@ London Produce Co., 1 W.L.R. 1013, 2 All E.R. 975 (Ch. Div.\@ 1968), is misplaced. In that 
case the language, to which the doctrine of ejusdem generis was applied, was totally different (``In this subsection, `broker' 
\textit{includes} a general commission agent'' (emphasis added)). 
Against the foregoing background, we turn to the determination of Fortress' legal and economic independence.
 
\begin{center} \textbf{Legal Independence}
\end{center} 

The relationship between Fortress and petitioners is defined by the management agreement that Fortress 
entered into separately with each petitioner. Petitioners have no interest in Fortress, and no representative of any of 
petitioners is a director, officer, or employee of Fortress.\footnote[10] {Comment 37 to Art. 5 of the 1977 model states: ``A subsidiary is not to be considered dependent on 
its parent company solely because of the parent's ownership of the share capital." Nor do we consider Fortress 
independent solely on the basis of the opposite situation; i.e., lack of ownership, although it is a factor to consider.}The agreements grant complete discretion to Fortress to conduct the reinsurance business on behalf of petitioners. 

Respondent agrees that Fortress had independence with respect to day-to-day operations, but then argues that its 
actions were restricted by gross acceptance limits and limits on net premium income. However, even if there were such 
restrictions, they would not necessarily constitute control. The gross acceptance limit and net premium income both relate 
to the total exposure of petitioners, and even an independent agent only has authority to perform specific duties for the 
principal. It is freedom in the manner by which the agent performs such duties that distinguishes him as independent. 

In any event, the record is clear that the gross acceptance limits were set by Fortress as part of its strategy to limit 
risk through diversification. Fortress advised petitioners of the gross acceptance limits for informational purposes and 
changed the limits without the advice or consent of petitioners. Fortress refused to put gross acceptance limits in the 
management agreements in order to retain flexibility. Respondent implies that the limit forced Fortress to enter into many 
small contracts instead of being able to enter into a few large contracts, but the pattern is consistent with Fortress'  
strategy of limiting risk through diversification, a strategy which Fortress was clearly in a position to implement through 
a plethora of available contracts. 

As to net premium income, there were no limits under the terms of the management agreements. If one of petitioners 
sought to lower its net premium income from U.S. sources, Fortress' advice was to cede a greater share to Carolina Re, 
operating in Bermuda. Petitioners could also terminate agreements with their other U.S. agents. Respondent places  
great weight on the estimates of net premium income Fortress provided to petitioners before each management year, but 
the estimates are clearly that, and nothing more, and were greatly exceeded for at least one of the years at issue. While 
Fortress was aware that at times petitioners wanted only to absorb a certain amount of net premium income, Fortress did 
not change its business to accommodate their concerns. 

Respondent further argues there were restrictions on Fortress' corporate affairs not reflected in the agreements that gave 
petitioners comprehensive control of Fortress. As evidence, respondent relies on Fortress' consultations with petitioners 
in regard to the request of Dai Tokyo to become a client of Fortress, and to Fortress' intent to include Carolina Re 
in the reinsurance program. Respondent also points out that Fortress reported to petitioners more regularly than required 
by the agreements. However, these are actions of a company seeking to maintain good relations with longstanding clients, 
rather than one seeking approval. With respect to the Dai Tokyo and Carolina Re situations, Fortress had already made 
its decision before consulting with petitioners.\footnote[11]{In another instance, in 1986, Fortress sought to increase its underwriting capacity by soliciting four new 
Japanese insurance companies to enter into management agreements. It made petitioners aware of its actions by 
sending them copies of the solicitation letter, but did not have or seek prior approval.} 
 Lewenhaupt v.\@ CIR, 20 T.C. 151, 162--163 (1953), affd. per
curiam 221 F.2d 227 (9th Cir. 1955), cited by respondent, not only involved a different test, i.e., whether the taxpayer was 
engaged in business in the United States through an agent, but involved continuous activity in managing U.S. real estate 
owned by the taxpayer which went beyond mere ownership or receipt of income. It is clearly distinguishable. 

Respondent further argues that petitioners exercised ``comprehensive control" over Fortress by acting as a ``pool". 
However, there is no evidence that petitioners acted in concert to control Fortress. In only rare and isolated instances 
did petitioners communicate with one another regarding Fortress. Further, there are references to a ``pool" throughout 
the history of Fortress, which period covers relationships with 17 separate U.S. and Japanese insurance companies. The 
inferences respondent would have us draw from the fact that petitioners are all from Japan and that petitioners 
are among the participants in regular industry conferences in Japan are simply insufficient to establish the existence of 
control by a ``pool". 

In a similar vein, we reject respondent's attempt to construct control from the fact that, during the years at issue, 
Fortress' activities were confined to the reinsurance it underwrote on behalf of petitioners. Pointing to Article 2(2) of 
the U.S.-Japan convention, \ldots respondent attempts to support her position by drawing upon the phrase ``other agent of 
independent status" in section 864(c)(5)(A) and the regulation thereunder, section 1.864-7(d)(3). Obviously, the statute simply repeats the phrase used 
in the convention. The regulations suggest two elements to be considered. The first is ownership or control, section 1.864-7(d)(3)(ii), which the regulation specifically states is not determinative. The second, section 1.864-7(d)(3)(iii), is whether the agent acts ``exclusively, or almost exclusively, for one principal" (emphasis 
added), in which event ``the facts and circumstances of a particular case shall be taken into account in determining whether 
the agent, while acting in that capacity, may be classified as an independent agent." Assuming without deciding that these 
regulations, implementing a particular statute, should be accorded interpretative effect in respect of a treaty provision, it 
has no application herein where we have concluded that Fortress acted separately in respect of each of four petitioners 
and where respondent concedes that Fortress was acting in the ordinary course of its business, a position that seems 
inconsistent with both the ``pool" and ``exclusively" concepts. Moreover, we note that the number of principals 
for whom Fortress acted varied over the years and that, even during the years before us, Fortress carried on a substantial 
amount of activity in handling claims, etc., for several other insurance companies. 

Finally, we note that all four petitioners, while not their primary business, did have reinsurance departments. 
Thus, petitioners had the ability to give detailed instructions to Fortress, yet they did not. 

As an agent, Fortress had complete discretion over the details of its work. As an entity, Fortress was subject to no 
external control. In sum, Fortress was legally independent of petitioners. 

\begin{center} \textbf{Economic Independence} 
\end{center}
Fortress is owned solely by Mr. Sabbah and his family and Mr. Kornfeld. There was no guarantee of revenue to 
Fortress, nor was Fortress protected from loss in the event it had been unable to generate sufficient revenue. Fortress has 
management agreements with four separate clients, whereby any one of them can leave on 6 months' notice. If one of 
petitioners did end its relationship, Fortress would bear the burden of finding a replacement to subscribe to that client's 
share of reinsurance contracts.\footnote[17]{We note there is no evidence that Fortress would have been unable to find such a replacement. In 1988, 
for example, Fortress rejected the overtures of Dai Tokyo to become a member. Since 1972, Fortress has had 
management agreements with 17 separate insurance companies.}
 
Respondent argues that Fortress bore no entrepreneurial risk because its operating expenses were covered by a 
management fee, and because it was guaranteed business due to the creditworthiness of the reinsurers on whose 
behalf it acted, petitioners. 

While the management agreements provided that Fortress earned a percentage of the gross premiums written which 
effectively covered Fortress' operating expenses, this did not mean that Fortress bore no risk. Fortress had to acquire 
sufficient business to produce the gross premiums. Further, it appears that this provision of the agreements is normal 
for an underwriting manager. That respondent's argument on this point misses the mark is illustrated, for example, by a 
large mutual fund that charges an annual management fee to cover operating expenses. Clearly, the mutual fund company 
would not be considered dependent on its thousands of investors. Under these circumstances, even with as few as four 
investors, Fortress cannot be considered dependent on petitioners to pay its operating expenses. 

Nor do we agree with respondent's argument that Fortress is able to secure profitable reinsurance contracts 
only because its clients are petitioners. Although Fortress needs clients with a certain minimum capital to conduct its 
business, any of hundreds of other insurance companies worldwide would be adequate substitutes. Also, it cannot 
be denied that Fortress had access to the reinsurance contracts it considered good, in part because of Fortress' relationships 
and reputation in the industry. In fact, it appears that Fortress' access to profitable reinsurance contracts, as well as its 
experience and ability to choose profitable reinsurance contracts, attracted petitioners to Fortress, and would attract other 
insurance companies if Fortress needed another client to take a share of the contracts. 

Finally, we think that the amount of Fortress' profits is significant. For the 3 years in issue, Fortress was paid 
over \$27 million. This is not the kind of sum paid to a subservient company. In addition, petitioners were in effect forced 
to share reinsurance profits with Carolina Re, an entity owned by the same people who owned Fortress, by permitting 
Fortress to cede reinsurance to Carolina Re even though Carolina Re was not as well known or financially secure as other 
potential quota share reinsurers.

\begin{center} \textbf{Conclusion} 
\end{center}
In sum, during the years at issue, Fortress was both legally and economically independent of petitioners, thus 
satisfying the definition of an agent of an independent status under Article 9 of the U.S.-Japan convention. 
Two further items deserve comment. First, petitioners point to a decision of the Federal Republic of Germany Tax 
Court at Bremen, FG I 4/73, 10 EFG 467049 (1973), affd. Decision of the FRG Bundesfinanzhof, BFH IR 152/73, BStBl 
II (24) 626-629 (1975), regarding the application of the independent agent provision of the Germany-Netherlands Treaty 
to a German insurance agent. A Dutch company had engaged a German firm as its representative and principal agent and 
signed a standard form granting power of attorney to the German firm enabling the German firm to conclude 
contracts on behalf of the Dutch company. The German firm acted as an independent insurance agent for numerous 
domestic and foreign insurance companies. The court held the Dutch company did not have a permanent establishment 
in Germany by virtue of the performance of the German insurance agent. While the result reached in the German case is 
consistent with that which we reach herein, we think that its utility herein is limited by the clearly distinguishable 
facts. Nor does De Amodio v.\@ CIR, 34 T.C. 894 (1960), affd. on other grounds 299 F.2d 623 (3d Cir. 1962), 
provide petitioners with any sustenance. There, a resident of Switzerland owned U.S. rental property which was managed 
and operated through local real estate agents. We determined that the agents fell within the term ``broker" or ``independent 
agent", in the income tax convention between the United States and Switzerland, but the discussion is limited. 

Second, we note that, in the commentary to the OECD's 1977 model, it is stated that an insurance company could do 
``large-scale business in a State without being taxed in that State on their profits arising from such business." Comment 38 to Art. 5 of 1977 model; see also comment 21 to Art. 5 of 1963 model. The commentary goes on to suggest that 
contracting states may want to contemplate that an insurance company will be ``deemed to have a permanent establishment 
in the other State if they collect premiums in that other State through an agent established there", other than a dependent 
agent. Comment 38 to Art. 5 of 1977 model. However, the commentary notes that such a provision is not in the
model and its inclusion should depend upon the factual and legal situation involved. Comment 38 to Art. 5 of 1977 model[.]
 
The [U.S.-Belgium convention of 1970], does include such an insurance provision. It provides that the independent 
agent provision ``shall not apply with respect to a broker or agent acting on behalf of an insurance company if such broker or agent has, and habitually exercises, an authority to conclude contracts in the name of that company." 
U.S.-Belgium convention, Art. 5(6). Finally, we note that it was decided [as described in the Technical Explanation] not to include reinsurance within the coverage 
of this provision. \ldots.From the foregoing it appears that the resolution of the issue 
of the existence of an agent of independent status in the insurance arena turns, at least in part, upon the presence of a 
specific treaty provision. 

Given the absence of any provision dealing with insurance or reinsurance in the U.S.-Japan convention, our holding 
herein that Fortress is not a permanent establishment of petitioners is consistent with the approach suggested by the OECD 
model \ldots and the application thereof in the U.S.-Belgium convention. 

\ldots
\end{select}

As commercial applications of the internet exploded in the 1990s, commentators began to question how virtual businesses would be taxed.  In particular, it was unclear whether conducting business--selling goods and services--over the internet would constitute a trade or business or a permanent establishment in the country in which the transaction was completed.  The Commentary to Article 5 of the OECD Model Treaty below clarifies that a website is not a PE, a web site ``hosting arrangement'' does not constitute a PE, and that a service provider will not constitute a dependent agent. 

\addcontentsline{toc}{section}{\protect\numberline{}OECD:  PEs and E-Commerce} 
	\begin{center}
		\textbf{  OECD Committee on Fiscal Affairs:  Changes to the Commentary of Art. 5 of the OECD Model Treaty}
	\end{center}
	\begin{select}
\begin{center} \textbf{Introduction}
\end{center}
1. This document contains the changes to the Commentary on the OECD Model Tax Convention 
adopted by the Committee on Fiscal Affairs on 22 December 2000 concerning the issue of the application 
of the current definition of permanent establishment in the context of e-commerce. It follows two previous 
drafts which were released for comments by Working Party No. 11 in October 1999 and March 2000. 

\ldots

6. As this document shows, the Committee has been able to reach a consensus on the various  issues 
concerning the application of the current definition of permanent establishment in the context of e--commerce (subject to the two dissenting views described at the end of this paragraph and of paragraph 14 
below).  This consensus includes the important views that a web site cannot, in itself, constitute a 
permanent establishment, that a web site hosting arrangement typically does not result in a permanent 
establishment for the enterprise that carries on business through that web site and that an ISP will not, 
except in very unusual circumstances, constitute a dependent agent of another enterprise so as to constitute 
a permanent establishment of that enterprise. However, Spain and Portugal do not consider that physical 
presence is a requirement for a permanent establishment to exist in the context of e-commerce, and 
therefore, they also consider that, in some circumstances, an enterprise carrying on business in a State 
through a web site could be treated as having a permanent establishment in that State. That is the reason 
why Spain and Portugal look forward to the results of the work of the TAG on Monitoring the Application 
of Existing Treaty Norms for the Taxation of Business Profits in the Context of Electronic Commerce (see 
paragraph 4) as regards the issue of whether changes to the definition of permanent establishment should 
be made to deal with e-commerce. 

7. As a number of commentators and delegates have noted, it is unlikely that much tax revenues 
depend on the issue of whether or not computer equipment at a given location constitutes a permanent 
establishment.  In many cases, the ability to relocate computer equipment should reduce the risks that 
taxpayers in e-commerce operations be found to have permanent establishments where they did not intend 
to.  Also, in circumstances where a taxpayer would want to have income attributed to a country where its 
computer equipment is located, that result can be achieved through the use of a subsidiary even if no 
permanent establishment is considered to exist.  It is crucial, however, that taxpayers and tax authorities 
know where the borderlines are and that taxpayers not be put in a position to have a permanent 
establishment in a country without knowing that they have a business presence in that country (a result that 
is avoided by the conclusion that a web site cannot, in itself, constitute a permanent establishment). 

8. Since a large part of the draft released in March 2000 discussed a minority view that some human 
intervention was required for a permanent establishment to exist and since many commentators have 
argued that this was the case, the Committee wishes to explain the position reached on that issue and 
reflected in the changes that have been adopted. 

9. Having further examination of the issue, the conclusion has been reached that human intervention 
is not a requirement for the existence of a permanent establishment. 

10. There is no specific reference to human intervention in paragraph 1 of Article 5 but it has been 
argued that the Commentary on Article 5, in particular paragraphs 2 and 10 thereof, imply that there is a 
requirement of human intervention for a permanent establishment to exist.  The Committee concluded, 
however, that the Commentary does not support this view. 

11. The relevant part of paragraph 2 reads as follows: 
\begin{quote}``The definition, therefore, contains the following conditions: 

[...]
 
the carrying on of the business of the enterprise through this fixed place of business.  This means 
usually that persons who, in one way or another, are dependent on the enterprise (personnel) 
conduct the business of the enterprise in the  State in which the fixed place is situated.''
\end{quote}
 
12. Although electronic commerce is developing rapidly, this statement is still accurate, i.e. usually, 
enterprises that have fixed places of business carry on their business through personnel.  This, however,
does not, and was not intended to, rule out that a business may be at least partly carried on without 
personnel. 

13. The same applies as regards to paragraph 10. According to the Committee, the example provided 
in that paragraph clearly supports the conclusion that no human intervention is required for a permanent 
establishment to exist.  Also, the first sentence (``The business of an enterprise is carried on mainly by the 
entrepreneur or persons who are in a paid-employment relationship with the enterprise (personnel)'') is still an 
accurate statement of how business operates but, again, does not rule out that a business may be at least 
partly carried on without personnel.  Finally, the Committee believes that a requirement of human 
intervention could mean that, outside the e-commerce environment, important and essential business 
functions could be performed through fixed automated equipment located permanently at a given location 
without a permanent establishment being found to exist, a result that would be contrary to the object and 
purpose of Article 5. 

14. The changes to the Commentary on Article 5 which appear below make it clear that, in many 
cases, the issue of whether computer equipment at a given location constitutes a permanent establishment 
will depend on whether the functions performed through that equipment exceed the preparatory or 
auxiliary threshold, something that can only be decided on a case-by-case analysis. Some countries did not 
like that outcome and the uncertainty that may result from it.  They suggested that, in the case of e-tailers, 
it would have been better to simply conclude that a server cannot, by itself, constitute a permanent 
establishment.  In order to reach a consensus, however, most of these countries have accepted the view 
expressed above, noting that they will take into account the need to provide a clear and certain rule in their 
own appreciation of what are preparatory or auxiliary activities for an e-tailer. The United Kingdom, 
however, has taken the view that in no circumstances do servers, of themselves or together with web sites, 
constitute permanent establishments of e-tailers and intends to make an observation to that effect when the 
changes to the Commentary on Article 5 are included in the Model Tax Convention. 

15. In order to illustrate that it is possible for functions performed through computer equipment to go 
beyond what is preparatory or auxiliary, an example has been included in the last sentence of paragraph 
42.9. It was noted during the discussion that this example is merely illustrative and should not be 
considered to determine the point at which the preparatory or auxiliary threshold is exceeded since many 
countries consider that this could be the case even if only some of the functions described in that example 
are performed through the equipment.

\begin{center} \textbf{CHANGES TO THE COMMENTARY ON ARTICLE 5}
\end{center}

Add the following heading and  paragraphs 42.1 to 42.10 immediately after paragraph 42 of the 
Commentary on Article 5 

``\textbf{Electronic commerce} 

42.1 There has been some discussion as to whether the mere use in electronic commerce 
operations of computer equipment in a country could constitute a permanent establishment. 
That question raises a number of issues in relation to the provisions of the Article. 

42.2 Whilst a location where automated equipment is operated by an enterprise may 
constitute a permanent establishment in the country where it is situated (see below), a 
distinction needs to be made between computer equipment, which may be set up at a location 
so as to constitute a permanent establishment under certain circumstances, and the data and 
software which is used by, or stored on, that equipment. For instance, an Internet web site, 
which is a combination of software and electronic data, does not in itself constitute tangible 
property. It therefore does not have a location that can constitute a ``place of business'' as there 
is no ``facility such as premises or, in certain instances, machinery or equipment'' (see 
paragraph 2 above) as far as the software and data constituting that web site is concerned. On 
the other hand, the server on which the web site is stored and through which it is accessible is 
a piece of equipment having a physical location and such location may thus constitute a ``fixed 
place of business'' of the enterprise that operates that server. 

42.3 The distinction between a web site and the server on which the web site is stored and 
used is important since the enterprise that operates the server may be different from the 
enterprise that carries on business through the web site. For example, it is common for the 
web site through which an enterprise carries on its business to be hosted on the server of an 
Internet Service Provider (ISP). Although the fees paid to the ISP under such arrangements 
may be based on the amount of disk space used to store the software and data required by the 
web site, these contracts typically do not result in the server and its location being at the 
disposal of the enterprise (see paragraph 4 above), even if the enterprise has been able to 
determine that its web site should be hosted on a particular server at a particular location. In 
such a case, the enterprise does not even have a physical presence at that location since the 
web site is not tangible. In these cases, the enterprise cannot be considered to have acquired a 
place of business by virtue of that hosting arrangement. However, if the enterprise carrying on 
business through a web site has the server at its own disposal, for example it owns (or leases) 
and operates the server on which the web site is stored and used, the place where that server is 
located could constitute a permanent establishment of the enterprise if the other requirements 
of the Article are met. 

42.4 Computer equipment at a given location may only constitute a permanent establishment 
if it meets the requirement of being fixed. In the case of a server, what is relevant is not the 
possibility of the server being moved, but whether it is in fact moved. In order to constitute a 
fixed place of business, a server will need to be located at a certain place for a sufficient 
period of time so as to become fixed within the meaning of paragraph 1. 

42.5. Another issue is whether the business of an enterprise may be said to be wholly or 
partly carried on at a location where the enterprise has equipment such as a server at its
disposal. The question of whether the business of an enterprise is wholly or partly carried on 
through such equipment needs to be examined on a case-by-case basis, having regard to 
whether it can be said that, because of such equipment, the enterprise has facilities at its 
disposal where business functions of the enterprise are performed. 

42.6 Where an enterprise operates computer equipment at a particular location, a permanent 
establishment may exist even though no personnel of that enterprise is required at that location 
for the operation of the equipment. The presence of personnel is not necessary to consider that 
an enterprise wholly or partly carries on its business at a location when no personnel are in 
fact required to carry on business activities at that location. This conclusion applies to 
electronic commerce to the same extent that it applies with respect to other activities in which 
equipment operates automatically, e.g. automatic pumping equipment used in the exploitation 
of natural resources. 

42.7 Another issue relates to the fact that no permanent establishment may be considered to 
exist where the electronic commerce operations carried on through computer equipment at a 
given location in a country are restricted to the preparatory or auxiliary activities covered by 
paragraph 4. The question of whether particular activities performed at such a location fall 
within paragraph 4 needs to be examined on a case-by-case basis having regard to the various 
functions performed by the enterprise through that equipment. Examples of activities which 
would generally be regarded as preparatory or auxiliary include: 
- providing a communications link � much like a telephone line � between suppliers and 
customers; 
- advertising of goods or services; 
- relaying information through a mirror server for security and efficiency purposes; 
- gathering market data for the enterprise; 
- supplying information. 

42.8  Where, however, such functions form in themselves an essential and significant part of 
the business activity of the enterprise as a whole, or where other core functions of the 
enterprise are carried on through the computer equipment, these would go beyond the 
activities covered by paragraph 4 and if the equipment constituted a fixed place of business of 
the enterprise (as discussed in paragraphs 42.2 to 42.6 above), there would be a permanent 
establishment. 

42.9 What constitutes core functions for a particular enterprise clearly depends on the nature 
of the business carried on by that enterprise. For instance, some ISPs are in the business of 
operating their own servers for the purpose of hosting web sites or other applications for other 
enterprises. For these ISPs, the operation of their servers in order to provide services to 
customers is an essential part of their commercial activity and cannot be considered 
preparatory or auxiliary. A different example is that of an enterprise (sometimes referred to as 
an ``e-tailer'') that carries on the business of selling products through the Internet. In that case, 
the enterprise is not in the business of operating servers and the mere fact that it may do so at 
a given location is not enough to conclude that activities performed at that location are more 
than preparatory and auxiliary. What needs to be done in such a case is to examine the nature 
of the activities performed at that location in light of the business carried on by the enterprise. 
If these activities are merely preparatory or auxiliary to the business of selling products on the 
Internet (for example, the location is used to operate a server that hosts a web site which, as is
often the case, is used exclusively for advertising, displaying a catalogue of products or 
providing information to potential customers), paragraph 4 will apply and the location will not 
constitute a permanent establishment. If, however, the typical functions related to a sale are 
performed at that location (for example, the conclusion of the contract with the customer, the 
processing of the payment and the delivery of the products are performed automatically 
through the equipment located there), these activities cannot be considered to be merely 
preparatory or auxiliary. 

42.10 A last issue is whether paragraph 5 may apply to deem an ISP to constitute a 
permanent establishment. As already noted, it is common for ISPs to provide the service of 
hosting the web sites of other enterprises on their own servers. The issue may then arise as to 
whether paragraph 5 may apply to deem such ISPs to constitute permanent establishments of 
the enterprises that carry on electronic commerce through web sites operated through the 
servers owned and operated by these ISP. While this could be the case in very unusual 
circumstances, paragraph 5 will generally not be applicable because the ISPs will not 
constitute an agent of the enterprises to which the web sites belong, because they will not have 
authority to conclude contracts in the name of these enterprises and will not regularly 
conclude such contracts or because they will constitute independent agents acting in the 
ordinary course of their business, as evidenced by the fact that they host the web sites of many 
different enterprises. It is also clear that since the web site through which an enterprise carries 
on its business is not itself a ``person'' as defined in Article 3, paragraph 5 cannot apply to 
deem a permanent establishment to exist by virtue of the web site being an agent of the 
enterprise for purposes of that paragraph.''

\end{select}

\addcontentsline{toc}{section}{\protect\numberline{}Trade/Business PE Problems} 
	\begin{center}
		\textbf{Trade/Business PE Problems}
	\end{center}
	\begin{select}
	
	\begin{enumerate}
	
	\item UKCo, a U.K. corporation, is interested in developing its presence in the U.S. market.  It is considering a variety of possible strategies, including purchasing products manufactured in the United States and  selling them in the United States and abroad, manufacturing in the United States and selling in the United States and abroad, or solely selling products in the United States that were manufactured outside of the United States, as in, for example, the U.K. by one of UKCo's subsidiaries or by an unrelated company.  Under which of the following scenarios will UKCo have a U.S. T/B or PE?  If there is a U.S. T/B or PE, briefly describe how much of the income, if any, earned by UKCo will be effectively connected with its U.S. T/B or PE.
			\begin{enumerate}
				\item UKCo purchases goods in the U.K. that are sold to independent U.S. distributors.  UKCo has no presence in the U.S.
				\item UKCo purchases goods in the U.K. that are sold in the U.S. by UKCo's U.S. branch office with title passing both in the U.S. and abroad  [\S 863(b); 865(e)(2)(A) and (3); 864(c)(4)(B)(iii); 864(c)(5)(C)]
						\item UKCo purchases goods that have been manufactured by third parties in the U.S.--(yes, I know how unlikely this scenario is)--and sells them in the U.K. with title passing in the U.K. [Balanovski]
						\item UKCo manufactures goods in the U.K. that are sold to independent U.S. distributors.  UKCo has no presence in the U.S.
				\item UKCo manufactures goods in the U.K. that are sold in the U.S. by UKCo's U.S. branch office (U.S. office, U.S. employees, etc.) with title passing both in the U.S. and abroad  [\S 863(b); 865(e)(2)(A) and (3); 864(c)(4)(B)(iii); 864(c)(5)(C)]
								\begin{enumerate}
							\item does section 865(e)(2)(A) (read literally) seem to allocate too much income to the United States?  Why?
								\end{enumerate}
				\item Same as previous question except that  UKCo contracts with USCo, an unrelated U.S. corporation, to distribute UKCo's products in the United States.  USCo will be UKCo's exclusive agent in the United States, and UKCo will pay USCo a commission for each sale made.  USCo is probihited from selling competing products not manufactured by UKCo. [See Hanfield, Rev.\@ Rul.\@ 70-424, 1970-2 CB 150 and also look at the Technical Explanation to Art. 5 and review Taisei.  Read Reg.\@ 1.864-7(d).  Is that regulation necessarily relevant?]
				\item UKCo manufactures goods in the U.K. that are sold in the U.S. to UKCo's U.S. subsidiary, which in turn distributes (sells) the goods to U.S. consumers.  Does section 865(e)(2)(A) apply?
											\end{enumerate}
	\item HF, is a successful UK hedge fund organized as UK general partnership.  It is considering expanding its investment activities to U.S. stocks and securities.  It opens a brokerage account with JPMorgan and begins to execute thousands of trades earning millions of dollars.
			\begin{enumerate}
				\item Does HF have a U.S. T/B?
				\item What if HF opens a U.S. office to scout out investment opportunities?
				\item What if HF opts to trade in swaps rather than the actual underlying securities?  [Prop. Reg.\@ 1.864(b)-1(a)]
				\item What if HF purchases distressed loans on the secondary market from an insolvent financial institution formally known as a bank?	 
			\end{enumerate} 
						
	\item Why would a taxpayer want to make the section 871(d) or 882(d) election?	
	\item Heidi, a citizen and resident of the UK, is a successful partner in a U.K. law firm who occasionally comes to NY in 2008 on client business.  In 2010, Heidi worked two months in the United States and earned \$120,000.  Is Heidi's distributive share (her share of the partnerships' income), or any part of it, subject to U.S. tax? If so, what is the rate of tax?  Would your answer change if Heidi's UK law firm had a branch office in NY? [Rev.\@ Rul.\@ 2004-3] 
%Do T/b first with special allocation/ no special allocation and then PE
	\item UKCo has licensed the rights to sell digital copies of all of the greatest hits of Welsh singers.  It posts on U.K. and U.S. servers copies of the songs that potential buyers can purchase.  To purchase a song, a buyer goes to the website, fills out a form--including credit card information--and can download a copy of the purchased music.  Does UKCo have a PE in the U.S.?  [See the OECD PE Report on PE in E-Commerce.]

	\end{enumerate}
	\end{select}
	
\section{Branch Profits Tax}
\crt{884}{1.884-1(a) and (b); 1.884-2T(a)(1) and (2); 1.884-4(a)(1)-(2) and (4), Example 1}{Article 10}
 
A foreign corporation engaged in a U.S. trade or business is subject to tax 
at graduated rates on its income that is effectively connected with its U.S. 
trade or business. \S 882(a)(1).  Since 1986, a foreign corporation engaged in a U.S. trade or business has been subject to an additional tax, the branch profits tax, on 
profits of the U.S. business that are deemed repatriated or dis-invested. 
\S884. A foreign corporation doing business in the U.S. is subject to two layers of U.S. tax on its profits:  once when they are earned, and once when they are dis-invested or repatriated. The goal of the BPT is to tax branch operations similarly to the operations of U.S. subsidiaries of foreign corporations, which also have the honor to contribute twice to the U.S. fisc:  once when the profits are earned by the U.S. subsidiary, and again when the after-tax profits are distributed to the foreign parent as a dividend. In essence, branches are treated conceptually as shadow or deemed U.S. subsidiaries.
 
A bit of background on the reasons for adopting the BPT.  Consider three ways for a foreign corporation to structure a U.S. business: (1) the use of a U.S. subsidiary; (2) the use of a special purpose foreign subsidiary; or (3) the use of a direct branch of a foreign corporation. If a U.S. subsidiary were used, the subsidiary will pay U.S. corporate tax on its profits, and a gross tax will be levied on the profits distributed as dividends. If a foreign special purpose subsidiary 
is used, \emph{and there were no BPT}, the corporation would be subject to U.S. 
tax on its effectively connected income, and \emph{prior to 2004}, U.S. tax on its 
dividends when distributed to shareholders under sections 861(a)(2)(B), 871, and 
881 (prior to the amendment in 2004 of section 871(i)(2)(D) to exempt U.S. 
source dividends paid by foreign corporations from FDAP taxation). If a 
foreign corporation with other non-U.S. businesses used a direct branch, the 
foreign corporation would be taxed on its effectively connected income, but 
there would not have been any U.S. tax on any dividends, provided that 
the portion of the branch's effectively connected income did not exceed the 25\% 
threshold in section 861(a)(2)(B). Note, there are no U.S. tax consequences on 
the remittance of profits of a U.S. branch to the foreign parent because the 
transfer is treated as merely an intra-corporate transfer and not a transfer 
between separate tax entities.
 
One aim of section 861(a)(2)(B), which treats dividends from foreign corporations as U.S. source and therefore (formerly) subject to a 30\% withholding 
tax (the so-called second level withholding tax), was to equalize the taxation of 
U.S. branches and U.S. subsidiaries. It only roughly succeeded. First, section 
861(a)(2)(B) only treats as U.S. source income a proportionate share of the foreign corporation's dividends over the preceding three years. In addition, most 
treaties prohibited the imposition of the second level withholding tax. Second, 
section 861(a)(2)(B) was easy to avoid by establishing a direct U.S. branch that 
was a small part of much larger non-US businesses. Consequently, none of the 
dividends would be U.S. source. Finally, it is unlikely that the corporations 
subject to the second level withholding tax actually paid it.

The BPT is intended to replicate the U.S. tax on dividends and interest 
that would be collected if the branch were a separate U.S. corporation. The
administrative difficulty is to find a branch proxy for the dividends and interest 
that would be paid by a U.S. corporation. Since dividends are measured by 
what a shareholder receives from a corporation, one choice could be the actual 
receipts  by the head office from the branch. This was rejected because it is not 
feasible to determine where branch activities begin and end. Congress then 
looked to withdrawals from the U.S. business activities, but believed that there 
would be problems with measuring withdrawals directly. As a surrogate for 
withdrawals, however, Congress eventually decided that the BPT would be 
levied on the branch's current earnings less any of these earnings reinvested in 
branch operations. In a year in which a branch has current earnings but investment in the branch doesn't increase, 
the current earnings are deemed distributed. In addition, if branch investment declines, the branch is deemed 
to have made a further distribution from accumulated profits equal to the 
amount of any decline.
 
Section 884(a) imposes a 30\% tax on a foreign corporation's annual \emph{divi- 
dend equivalent amount} (DEA). The DEA is defined to mean the corporation's 
\emph{effectively connected earnings and profits} (ECE\&Ps) for the year, \emph{reduced} by 
the \emph{excess} of the corporation's U.S. net equity at the end of taxable year over 
the U.S. net equity at beginning of year, and \emph{increased} by excess of the corporation's U.S. net equity at beginning of year over its U.S. net equity at the close of year. For example, assume that (1) FC has 1,000 of U.S. net equity as of close of 
2008; (2) 100 of ECE\&Ps for 2009; (3) and FC acquires 100 of additional U.S. assets during 
2009, and its USNE is 1100 as of the close of 2009. FC's DEA would be 0: 100 of 
ECE\&Ps reduced by the 100 increase in USNE between 2008 and 2009. In essence, the ECE\&Ps are deemed distributed unless they are reinvested in U.S. assets. 

ECE\&Ps are the E\&Ps attributable to income that is effectively connected 
with a U.S. trade or business. \S884(d). Under general U.S. corporate tax 
law, when a corporation makes a distribution to its shareholders--money or 
property--it is taxable to shareholders as a dividend if the corporation has 
E\&Ps either for that year or from previous years. If not, the distribution is 
a return of capital and generally not taxable. The E\&Ps of a corporation 
are basically a measure of its ability to make distributions to its shareholders. 

The starting point to determine E\&Ps is taxable income, which is then adjusted. For example, tax exempt income is added back to taxable income as 
are dividends excluded under the dividends received deductions. Important 
subtractions include taxes and capital losses otherwise limited in calculating 
taxable income. Section 884(d) also provides for some specific exceptions, 
including gains from the disposition of stock of  U.S. real property holding company.  (This rule ensures that gains attributable to U.S. real estate are taxed only twice and not three times.)  

U.S. net equity is U.S. assets less U.S. liabilities. In determining a branch's U.S. assets, 
the adjusted basis of property rather than FMV is used. The reason for using adjusted basis is that E\&Ps are computed using the adjusted basis of property. \S884(c). 

Under the section 884 regulations, U.S. assets are assets that produce ECI. Reg.\@ \S1.884-1(d)(1).  Thus, stocks and bonds are generally \emph{not} U.S. assets, and investments by a branch in these instruments constitute disinvestment of  the U.S. business and exposure to the BPT. U.S. liabilities are those treated as connected with the U.S. trade or business and are generally equal to same proportion of U.S. assets as worldwide liabilities bear to worldwide assets, except that a taxpayer can elect to use a fixed ratio (50\% for non-banks and 95\% for banks).   Reg.\@ \S1.884-1(e)(1) and Reg.\@ \S 1.882-5 (determination of allocable interest for computing ECI).

Consistent with one of the policy goals of the BPT, namely to equalize 
the tax burdens of U.S. branches and subsidiaries, the regulations provide that 
the complete termination of a U.S. business--either by selling the business or 
repatriating the assets--will not give rise to BPT liability. This rule is the 
creation of tax administrators and not found in the statute. The basis 
for this rule is the general corporate U.S. tax rule that treats liquidating 
distributions of subsidiaries not as dividends but as the exchange of assets of the subsidiary for stock of the subsidiary. In the case of a foreign parent owning stock of a U.S. corporation, the sale of stock of the U.S. corporation is 
not subject to tax. Consequently, the termination of a branch should not be 
a taxable event. Reg.\@ \S1.884-2T(a)(1).  These regulations also provide rules for transfers of branch assets in liquidations, incorporations, and reorganizations.  To qualify for this rule, the foreign corporation must hold no U.S. assets after the termination, the assets of the U.S. branch cannot be used in a U.S. business for the three succeeding tax years, and the foreign corporation cannot have any ECI for the three succeeding tax years.  Reg.\@ \S1.884-2T(a)(2)(i). 

The BPT also affects interest actually or constructively paid by the U.S. 
branch. Under section 884(f)(1)(A), interest paid by a U.S. branch is treated 
as if it were paid by a U.S. corporation. As a result, the interest will be subject 
to a flat 30\% tax unless an exemption, such as portfolio interest or a treaty, applies. Also, under section 884(f)(1)(B), if the interest payments of the branch are less than 
the interest allocated to the branch under regulations (and therefore deductible 
in computing ECI), the \emph{excess interest} is treated as though it were paid to the 
foreign corporation by a wholly owned subsidiary on the last day of the year. 
Consequently, the 30\% tax will apply unless a treaty exemption is available. 
Reg.\@ \S1.884-4 (branch level interest) and Reg.\@ \S 1.882-5 (determination of 
allocable interest for computing ECI).
 
In enacting the BPT, Congress did not intend to override treaties that prohibited the imposition of a BPT, but it imposed some restrictions on treaty 
benefits. Under section 884(e)(1), no relief is available under a treaty unless the 
treaty is an income tax treaty and the foreign corporation is a \emph{qualified resident} of the treaty counterparty.  \emph{See} Reg.\@ 1.884-5.  Assuming those requirements are satisfied, section 884(e)(2) provides that the BPT can be levied but only at the rate 
specified in the treaty on branch profits or if a rate is not specified, at the rate 
applicable to dividends from a corporation in one country paid to a resident of 
the other. Almost all modern treaties permit the imposition of a BPT. Under 
the UK Treaty, the BPT is permitted. See Article 10(7) and (8).  The rate is generally 5\%, but reduced to 0\% for U.S. branches operating prior to 10/1/98, U.S. branches of companies that qualify as QRs under the publicly traded rule of Art. 23(2)(c), and U.S. branches of a company that satisfies the derivative benefits test of Art. 23(3).  Please see the Treaty technical explanation.  The limitation of treaty benefits to qualified residents is now part of all recent U.S. treaties. \emph{See} Article 23 of the UK Treaty. 

\addcontentsline{toc}{section}{\protect\numberline{}Branch Profit Tax Problems} 
	\begin{center}
		\textbf{BPT Problems}
	\end{center}
	\begin{select}
	
For each of the questions below, consult section 884(b)-(d) and Reg.\@ \S 1.884-1(b)(4), Examples.

	\begin{enumerate}
	
	\item FC, a UK corporation owned by UK residents, forms a U.S. branch at the end of 2006 with a capital contribution of \$1 million.  For 2007, the branch earns \$150,000 and pays \$50,000 in U.S. tax.  It acquires \$100,000 in U.S. assets in 2007.  What's its DEA for 2007.
	\item Same facts as previous question except that FC acquires only \$40,000 of U.S. assets in 2007.  What's its DEA for 2007?
	\item Same facts as question 1, except that in \emph{2008}, it earns \$150,000 and pays \$50,000 in U.S. tax.  Its U.S. net equity at the end of 2008 is \$1,050,0000.  
	\item Same facts as previous question, except that its U.S. net equity at the end of 2008 is \$900,000.
	\item Same facts as question 2, except that FC completely terminates its U.S. branch.  What requirements must be satisfied to qualify for the branch termination rule of Reg.\@ \S 1.884-2T(a)(2)?  
		
		\end{enumerate}
	\end{select}	


\begin{framed}
Last Revised Feb. 20, 2014; tradebusiness\_Feb20\_14
\end{framed}

	