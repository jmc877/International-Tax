\chapter{Gains from the sale of U.S. Real Estate:  FIRPTA}
\crt{861(a)(5); 897(a)-(c), (h); 1445 (skim)}{1.897-1(b), (d), (o)(1), (o)(4); 1.897-2(b)(2)}{Article 13}

\begin{center}
\textbf{Overview}
\end{center}

Prior to 1980, sales of U.S. real estate by a foreign person had the same consequences as sales of other property:  although the gains were U.S. source, they did generally not constitute FDAP, so there was no U.S. tax unless they were effectively connected to a U.S. trade or business.  \S 864(c)(3).  As we discussed in the cases addressing whether ownership of rental U.S. real property constituted a trade or business, ownership of even a single piece of real estate can constitute a trade or business provided that the owner does not transfer by contract to the tenant too many the traditional activities of an owner, such as repairs, paying taxes, and insurance.  If a foreign person's U.S. real estate holdings did not constitute a trade or business, there was no gain upon disposition of the property, but any rental income was U.S. source FDAP subject to a flat 30\% tax.  Given the high expenses  normally associated with owning real estate, such as interest, property taxes, repairs, and insurance, a flat 30\% tax could be confiscatory.  

If a foreign person was engaged in a trade or business, however, he could deduct against rental income such related expenses as insurance, mortgage interest, and taxes, but gain realized upon the sale of the property would be ECI.  To avoid tax upon the disposition of the property but still receive the benefit of the deductions, an owner could sell the property on an installment basis and collect the proceeds in a year in which the seller was not engaged in trade or business and thereby prevent the income from being effectively connected.  In addition, even if the trade/business election under section 871(d) were made, some treaties permitted the election to be made annually, thereby mitigating the adverse tax consequences of making the 871(d) election.  Note, section 864(c)(6) now generally prevents these gambits for all income related to a U.S. trade or business. 

A traditional argument for not taxing gains of foreign persons arising from the disposition of real estate (or any property such as stock or bonds) is that the collection of tax on capital gains by means of a withholding tax is virtually impossible because  it is difficult to know how much gain the foreign person actually realized without taking into account the property's basis.  (A withholding tax is probably necessary, because once the sale proceeds are transferred abroad, it is impossible to collect any tax due.)  A withholding tax generally only works well with items of income for which cost basis does not matter, \emph{e.g.}, dividends, interest, rents, and royalties.

In the late 1970's, there was a significant increase in foreign investment in the United States, especially in U.S. farmland.  Concerned that foreigners had an unfair advantage over U.S. purchasers--the questionable rationale was that because foreigners paid no tax, they could offer a higher purchase price than taxable purchasers--which thereby caused an increase in the price of U.S. farmland to rise beyond the reach of young American families, Congress enacted the Foreign Investment in Real Property Tax Act (FIRPTA) in 1980.  As discussed below, significant changes were made to FIRPTA in the Protecting Americans from Tax Hikes (``Path'') Act of 2015.

\begin{center}
\textbf{URSRPIs and USRPHCs}
\end{center}

Under FIRPTA, which is codified in sections 897 and 1445, gains and losses of a foreign person arising from the sale of a \emph{U.S. real property interest} (USRPI) are treated as ECI, regardless of whether the foreigner otherwise has a trade or business.  \margit{FIRPTA taxes foreigners on the sale of USRPIs.}A USRPI includes both direct interests in real property--land, leases, options on land, improvements such as buildings, mines, railroad tracks--and indirect interests, such as stock in a U.S. corporation or interest in a partnership whose assets include a significant amount of USRPIs.  Section 1445 generally requires that the \textbf{buyer} withhold 15\% of the amount realized--the amount paid for the property, including any borrowed proceeds.  If the gain is less than the amount withheld, the seller can apply for a refund.  A seller or buyer can request from the IRS a certificate to withhold a lesser amount if the amount of the gain and tax liability can be shown.     

FIRPTA taxes foreign persons on the gains from the disposition of direct USRPIs and indirect USRPIs.  In particular, gain arising from a foreign person's disposition of any interest (other than solely as a creditor) in \textbf{any U.S. corporation} is subject to FIRPTA, unless the taxpayer demonstrates that the corporation is not a \emph{U.S. Real Property Holding Corporation} (USRPHC).\margit{USRPIs include direct interests in US real estate and interests in USRPHCs.}  Note, gains from the disposition of stock of a \emph{foreign} corporation are \emph{never} subject to section 897, even if the corporation's only asset is U.S. real property. But when the foreign corporation disposes of the U.S. real property, it will be taxed under 897.  A knowledgeable buyer will thereby discount the price of the shares to reflect the future tax.

A corporation is a \margit{Definition of USRPHC.}USRPHC if it holds USRPIs having an aggregate FMV equal to or exceeding 50\% of [1] the FMV of the corporation's RPIs and business assets, including its USRPIs; [2] any interest in foreign real property; and [3] any other trade or business assets.  \S 897(c)(2).  Equivalently, a US corporation will be a USRPHC if the value of its USRPIs is greater than the value of its foreign real property and trade or business asssets.  Note, investment assets are not counted in making this determination.  Thus, a US corporation can't purchase stock to avoid USRPHC status.  Remember, a foreign corporation can satisfy the definition of a USRPHC, but any gain realized upon a sale of the stock of the foreign corporation will \emph{not} be subject to FIRPTA.  

In addition, if a corporation was a USRPHC any time during the preceding five years, it will be a USRPHC even though it does \emph{not} satisfy the 50\% test on the date of disposition, unless it has disposed of all of its USRPIs in taxable transactions.  \S 897(c)(1)(A)(ii) and (B).  Publicly traded stock is not subject to section 897, provided the seller held 5\% or less of the stock during the previous five years.  \S 897(c)(3).  What kinds of corporations would be USRPHCs?  Why is the exemption for publicly traded 5\%-or-less-interests needed?    

Valuation is generally done using the assets' FMV, and asset values are reduced by debt only if the debt is secured by the property and is used to purchase or improve the property (or is a refinancing of such debt).  Reg. \S 1.897-1(o)(2)(iii). If, however, the FMV of a corporation's USRPIs are 25\% or less of the \emph{book value} of the corporation's assets, the FMV of the corporation's USRPIs is presumed to be less than 50\%, and hence the corporation won't be a USRPHC.  The regulations provide that USRPHC status must generally be determined on the last day of the corporation's taxable year, the date any USRPI is purchased, or the date on which any foreign real property interest or trade or business assets is disposed of. Reg. \S 1.897-2(c)(1)(i)--(iii).  The regulations provide some exceptions from these requirements.  \emph{See} Reg. \S\S 1.897-2(c)(2) (ordinary business transactions); 1.897-2(c)(3) (monthly determinations and transactional determinations)

\begin{center}
\textbf{Indirect Interests}
\end{center}


In determining whether a \margit{The treatment of controlling interests.} corporation is a USRPHC, if it owns 50\% or more of the FMV of all classes of the stock of another corporation, the upper-tier corporation is treated as owning a proportionate amount of the lower-tier corporation's assets.  Thus, the interposition of a foreign corporation is ineffective to break USRPI taint.  \S 897(c)(5).  A similar look-through applies for assets held by partnerships.  \S 897(c)(4)(B).

In determining whether a corporation is a USRPHC, the treatment of interests in less-than-50\% held corporations is a bit convoluted under the statute.  Under section 897(c)(4)(A), in determining whether \emph{any} corporation is a USRPHC, section 897(c)(1)(A)(ii), which provides that a USRPI is any interest in any \emph{domestic} corporation that was USRPHC during the last five years, is to be applied as if ``domestic'' read ``any corporation (whether foreign or domestic).''  This is certainly not a model of clear statutory drafting.  This provision basically says that minority interests in a \emph{foreign} corporation can be USRPIs for the purpose of determining whether the \emph{parent} corporation is a USRPHC.  Some examples may be helpful.

Assume that FC owns 100\% of USCO, which holds only USRPIs, for example buildings in N.Y.  USCO would be a USRPHC (and a USRPI), and when FC sells USCO, FIRPTA would apply.  

Now assume that USCO holds 100\% of the stock of FC1, which holds only USRPIs, such as  buildings in N.Y.  Under section 897(c)(5), USCO would be treated as owning all of the USRPIs of FC1, and USCO would be a USRPHC.  Thus, when FC sells the stock of USCO, FIRPTA would apply.  Note, the sale by USCO of the stock of FC1 would be subject to U.S. tax because USCO is a U.S. corporation taxed on its worldwide income and not because FC1 holds U.S. real estate. 

If USCO owns only 40\% of the stock of FC1.  \margit{The treatment of non-controlling interests.} Generally, stock is not a trade or business asset.  Thus, in determining whether USCO is a USRPHC, the stock of FC1 would be disregarded, and FC could sell USCO without tax.  This would be an easy end run around FIRPTA.  Section 897(c)(4)(A), however, treats an interest in \emph{any} corporation as a USRPI if it would be a USRPHC.  Because FC1 holds only U.S. real estate, it would be a USRPHC and a USRPI.  Thus, when determining whether USCO is a USRPHC, the stock of FC1 would count as a USRPI.  Remember though that section 897(c)(4)(A) does not mean that the sale of the stock of a foreign corporation that satisfies the definition of a USRPHC is subject to FIRPTA, but rather that the value of the stock of a foreign corporation can be treated as a USRPI for purposes of determining whether another corporation is a USRPHC.

\begin{center}
\textbf{REITs and Foreign Pension Funds}
\end{center}

A Real Estate Investment Trust (``REIT'') is an entity otherwise treated as a corporation that earns a significant amount of its gross income from real estate related activities (rents, interest on mortgages, and gains from the sale of real estate), invests a significant portion of its assets in real estate assets, and elects to be treated as a REIT under section 856(c)(1).  Under the REIT tax provisions, a REIT may deduct distributions of its taxable income and capital gains, and thus a REIT generally avoids entity-level tax. \S 857.  Thus, a distribution from a REIT may include capital gains from the sale of U.S. real estate.  

FIRTPA contains special rules for distributions from REITs and sales of REIT shares by foreign persons. In the case of the sale of a REIT interest, if the REIT is publicly traded, the REIT interest won't be treated as a USRPI if the seller owned 10\% or less of the REIT during the preceding five years.  \S897(k)(1)(A).  In addition, the sale of any interest in a domestically controlled \emph{qualified investment entity} (``QIE'') is not treated as a USRPI, regardless of the ownership percentage of the foreign seller.  \S897(h)(2).  A qualified investment entity includes REITs and mutual funds (RICs) that would be USRPHCs.  \S897(h)(4)(A).  A QIE is domestically controlled if less than 50\% of the value of the stock is held directly or indirectly by foreign persons over the preceding 5 years.  \S897(h)(4)(B).

In the case of a distribution from a REIT attributable to capital gains from USRPIs, a distribution from a publicly traded QIE is not treated as gain from the sale of a USRPI if the recipient owned 10\% or less of the stock during the preceding one year.  \S897(k)(1)(B).  In such case, the distribution is treated as an ordinary dividend from a U.S. corporation.  If this exception is not available, the distribution will be subject to FIRPTA. 

In 2015, FIRTPA was amended to exempt from FIRPTA gains from the sale of USRPIs by foreign pension funds. \margit{Sales of U.S. real estate by foreign pension plans.}  \S897(l)  Note that if the U.S. real estate constitutes a trade or business, the real estate gains can be taxed as ECI under section 864.  Ownership through a REIT, regardless whether the REIT is a QIE or the percentage of the REIT owned by the foreign pension, would ensure that gains are not ECI.

\begin{center}
\textbf{Withholding and Treaties}
\end{center}

When Congress enacted FIRPTA, it provided that after 1985, no treaty would prevent the application of FIRPTA to sale of USRPIs.  Consequently, FIRPTA overrides any contrary treaty provision.  

FIRPTA itself extends some benefits to treaty residents that own REITs.  Under section 897(k)(2)(A)(i) and (ii), stock of a REIT held by a \emph{qualified shareholder} is not a USRPI, and any distribution to a qualified shareholder is not treated as a gain from the sale of a USRPI.    A qualified shareholder is [1] a treaty resident under a treaty that has a exchange of information program \emph{and} the principal class of interest is publicly traded; or [2] a foreign limited partnership that has a class of limited partnership units publicly traded in the United States and such class is greater than 50\% of the value of the partnership units.  \S 897(k)(3)(A)(i)(I) and (II).  Any entity under [1] or [2] must be a \emph{collective investment vehicle} (``CIV'') and maintain records on the identity of each owner that owns 5\% or more of the entity. \S 897(k)(3)(A)(ii) and (iii).   A CIV includes a [1] foreign person that under a treaty is eligible for a reduced rate of withholding with respect to dividends paid by a REIT even if the person owns more than 10\% of the stock of the REIT, or [2] certain publicly traded partnerships. \S 897(k)(3)(B).

Dividends paid by a REIT that are not otherwise covered by the 2015 statutory changes are addressed in Art. 10, par. 4.

A \emph{purchaser} of a USRPI must generally withhold 15\% of the \emph{amount realized}. \S 1445(a).  Prior to 2015, the rate was 10\%.   Withholding is not required if the transferor furnishes an affidavit that the transferor is not a foreign person, or in the case of the transfer of a non-publicly traded U.S. corporation, the corporation provides an affidavit stating that it is not (and has not been) a USRPHC. \S 1445(b).  

There are a couple of special rules for the transfers of personal residences.  There is no withholding required if the amount realized does not exceed \$300,000 if the property is acquired by the transferee for use by him as a residence. \S 1445(b)(5).  In most large U.S. cities, you would probably advise your client to not set foot in a neighborhood where one can buy a personal residence for \$300,000.  The 10\% withholding rate is retained for sales of personal residences for under \$1,000,000.  \S 1445(c)(4).  


The following revenue ruling addresses the treatment under FIRPTA of gains from derivative instruments that track the value of residential and commercial real estate in certain areas of the United States.  

\addcontentsline{toc}{section}{\protect\numberline{}Rev. Rul. 2008-31}
\begin{select}
\revrul{Rev. Rul. 2008-31}{2008-1 C.B. 1180}

\begin{center} \textbf{ISSUE}
\end{center}  
Is an interest in a notional principal contract, the return on which is calculated by 
reference to an index described below referencing data from a geographically and 
numerically broad range of United States real estate a United States real property 
interest (``USRPI'') under section 897(c)(1) of the Code?  
\begin{center} \textbf{FACTS}
\end{center} 
X maintains and widely publishes an index (the ``Index'') that seeks to measure 
the appreciation and depreciation of residential or commercial real estate values within 
a metropolitan statistical area (``MSA''), a combined statistical area (``CSA'') (both as 
defined by the United States Office of Management and Budget), or a similarly large 
geographic area within the United States. The MSA, CSA or similarly large geographic 
area has a population exceeding one million people. The Index is calculated by 
reference to (1) sales prices (obtained from various public records), (2) appraisals and 
reported income, or (3) similar objective financial information, each with respect to a 
broad range of real property holdings of unrelated owners within the relevant geographic area during a relevant testing period.  Using proprietary methods, this information is weighted, aggregated, and mathematically translated into the Index.  

Because of the broad-based nature of the Index, an investor cannot, as a 
practical matter, directly or indirectly, own or lease a material percentage of the real 
estate, the values of which are reflected by the Index. 

On January 1, Year 1, FC, a foreign corporation, enters into a notional principal 
contract (``NPC''), within the meaning of sections 1.446-3(c)(1) and 1.863-7(a)(1) of the 
Income Tax regulations, with unrelated counterparty DC, a domestic corporation. 
Neither FC nor DC is related to X. Pursuant to the NPC, FC profits if the Index 
appreciates (that is, to the extent the underlying United States real property in the 
particular geographic region appreciates in value) over certain levels. Conversely, FC 
suffers a loss if the Index depreciates (or fails to appreciate more than at a specified 
rate). During the term of the NPC, DC does not, directly or indirectly, own or lease a 
material percentage of the real property, the values of which are reflected by the Index.  

\begin{center} \textbf{LAW}
\end{center} 
   \ldots

Section 1.897-1(c)(1) of the regulations generally defines USRPIs to include any 
interest, other than an interest solely as a creditor, in real property located in the United 
States or the Virgin Islands. Section 1.897-1(d)(2)(i) provides that an interest in real 
property other than solely as a creditor includes a fee ownership, co-ownership, or 
leasehold interest in real property, a time sharing interest in real property, and a life 
estate, remainder, or reversionary interest in such real property. The term also includes 
any direct or indirect right to share in the appreciation in the value, or in the gross or net 
proceeds or profits generated by, the real property.  
\begin{center} \textbf{HOLDING}
\end{center} 

Because of the broad-based nature of the Index, the NPC does not represent a 
``direct or indirect right to share in the appreciation in the value ... [of] the real property'' 
within the meaning of Treas. Reg. \S1.897-1(d)(2). Accordingly, FC's interest in the NPC 
calculated by reference to the Index is not a USRPI under section 897(c)(1). 

\end{select}
 
\addcontentsline{toc}{section}{\protect\numberline{}FIRPTA Problems} 
	\begin{center}
		\textbf{FIRPTA Problems}
	\end{center}
	\begin{select}
	
	\begin{enumerate}
	
	\item Which of the are following are USRPIs? [\S 897(c)(1)(A); Reg. \S 1.897-1(b), 1(d)]
			\begin{enumerate}
				\item A building in New York
				\item Undeveloped land in South Dakota
				\item Farm land in South Dakota with crops on it and combines used to harvest the crop
				\item Hotel in NYC and beds, tvs, and refrigerator
				\item Lease of a floor of a New York City building
				\item Mortgage on New York City building
				\item Option to buy New York City building
				\item Mortgage on NYC building entitling mortgage holder to fixed 6\% interest and 25\% of any gain upon sale of the building [Reg. \S 1.897-1(d)(2)(i); 871(h)(4)]
			\end{enumerate}
		\item William, a UK resident and citizen, is the sole shareholder of HoldCo, a corporation that manufactures in China and sells in Europe trendy sun glasses.  HoldCo also owns 70\% of the  stock of Sub1, which owns an office building in NYC and stocks and bonds.  Assume that the FMV (and book value) of the stocks and bonds is \$5 million (\$1 million), the office building, \$20 million (\$15 million), and the assets of the sun glass business \$20 million (\$5 million). [\S 897(c)(5)]
					\begin{enumerate}
						\item  If HoldCo is a UK corporation and William sells HoldCo stock, does FIRPTA apply?
						\item  If HoldCo is a UK corporation, Sub1 a UK corporation, and HoldCo sells the stock of Sub1, does FIRPTA apply?
									\item If HoldCo is a UK corporation, Sub1 a US corporation, and HoldCo sells the stock of Sub1, does FIRPTA apply?
									\item If HoldCo is a UK corporation and held the assets of Sub1 directly, would HoldCo be taxed on the sale of the stocks and bonds?
						\item If HoldCo is a UK corporation and owns only 30\% of Sub1, a US corporation, and HoldCo sells the stock of Sub1, does FIRPTA apply?
						\item If HoldCo is a US corporation, does FIRPTA ever apply to the sale of Sub1?
									\item If HoldCo is a US corporation and owns only 30\% of Sub1, and William sells HoldCo stock, does FIRPTA apply?
									\item If HoldCo is a US corporation and owns 70\% of Sub1, and William sells HoldCo stock, does FIRPTA apply?
					\end{enumerate}
\item Same facts as previous question (2(h)) except that Sub1 now also owns some foreign real estate with a FMV (book value) of \$20 million (\$15 million).  [\S 897(c)(4)(A)]

\item In calculating the value of assets for FIRPTA purposes, is book value relevant? [Reg. 1.897-1(o)(1), -1(o)(4); 1.897-2(b)(2)]
\item In calculating the value of assets for FIRPTA purposes, how is debt taken into account? [Reg. 1.897-1(o)(2)(iii)]   

				\item How does the treaty change any of your answers above?		
		\end{enumerate}
	\end{select}	
	
	
	

\begin{framed}
Last modified: Feb. 22, '17; Firpta\_22\_17
\end{framed}