 \section{Compensation for Services}
\crt{861(a)(3); 863(b)(1); 864(b)(1); 864(c)(6); 871(a)(1); 881(a)(1); 1441(b)(1) and (c)(1); and 1442(b)(2)}{1.861-4(b)(1), (2)(i), (2)(i)(A), (2)(E), and (F); 1.864-4(c)(6); Prop. 1.861-4(b)(2)(ii)(G); 1.1441-1(a) and (b); and 1.441-2(b)(2)(i)}{Articles 7, 14, 15, 16, 17-19 (skim lightly), and 20}

Compensation for services performed in the United States is generally U.S. source income.  \S861(a)(3).  If services are performed solely in the United States or abroad, determining the source of compensation for those services is straightforward.  There is a \emph{de minimis} exception for compensation not exceeding \$3,000 performed by a nonresident alien who is present for 90 days or less in the United States and works for either a foreign person not engaged in a U.S. trade or business or for a U.S. person if the services are performed in connection with the U.S. person's foreign place of business.  \S\S 861(a)(3)(A)-(C); 864(b)(1)(A)-(B).  As the \$3,000 limit \margit{\$3,000 in 1936 is equivalent to \$51,046 in 2014.}has not been adjusted for inflation since the provision was enacted over 60 years ago, the rule probably affects very few persons.  

If a person receives compensation for performing services both in the United States and abroad, the compensation must be allocated between U.S. and foreign sources. \S863(b)(1).  Since an employee who performs services inside and outside of the United States will often not receive separate compensation for the U.S. and foreign services, is it necessary to allocate the income between U.S. and foreign sources by examining the employee's employment contract or another method.  Regulations under section 861 provide that the allocation is to be made ``on the basis that most correctly reflects the proper source of that income under the facts and circumstances of the particular case.  In many cases, the facts and circumstances will be such that an apportionment on a time basis\ldots will be acceptable."  Reg.\@ \S1.861-4(b)(2)(i).  The same regulation also provides rules for sourcing certain benefits, such as housing, education, local transportation, tax reimbursements, hazardous pay, and moving expenses.    Reg.\@ \S1.861-4(b)(2)(ii)(D)(1)--(6).

The \emph{Stemkowski} case below illustrates how this determination is made in the absence of a specific allocation in an employment contract.  Compensation received by athletes and artists presents particular challenges.  The event(s) for which an athlete or artist is ultimately compensated may represent the culmination of much preparation, rehearsals, and training performed in locations different than the final performance(s).  Some had argued that an athlete or artist could allocate compensation between the countries where he performed and trained or rehearsed.  In 2007, the IRS issued proposed regulations under section 864 that provide that compensation received for performing services at a specific event should be allocated entirely to where the event occurs.  The preamble to the proposed regulations states that it is probably improper to allocate any of the compensation for services received to the place where the artist or athlete prepares for the performance.  \emph{See} Prop. Reg.\@ 1.861-4(b)(2)(ii)(G), (4)(c), Ex. 10 (player contract compensation not allocated to preseason or postseason unless athlete receives additional compensation).  The proposed regulations thus would overturn the portion of \emph{Stemkowski} that allocated a portion of the contract income to the preseason and postseason.  

Although compensation is listed as FDAP under section 871(a), if U.S. source compensation is received in a year in which the service provider is engaged in a U.S. trade or business, it will not be taxed at a flat 30\%, but instead is taxed as effectively connected income at graduated rates.  Performing services in the United States for even one day constitutes a U.S. trade or business, thereby subjecting the U.S. source income to graduated rates.  \S 864(b); Reg.\@ \S 1.864-4(c)(6). There is a narrow exception in \S 861(b)(1)(A) that parallels the exception in \S861(a)(3) for minor amounts of compensation paid to nonresidents temporarily present in the United States.  

Prior to 1986, deferred compensation paid in a year in which a nonresident was not engaged in a U.S. trade or business could be subject to tax under section 871 if it were U.S. source.  If the recipient resided in a treaty country, however, the deferred compensation may have escaped U.S. tax entirely.  To prevent this gambit, Congress, in 1986, enacted section 864(c)(6), which treats deferred compensation of a nonresident received in a year in which he is not engaged in a trade or business, but which is attributable to services performed in the U.S. in another year, as if it had been received in the other year.  Thus, only in very rare circumstances will compensation be taxed as FDAP.    

Treaties take a much more nuanced approach to compensation than the Code.  Compensation performed as an independent contractor, \emph{e.g.}, an attorney, accountant, or engineer, is governed by Article 7.  Compensation for services rendered as an employee working for an employer of the other treaty country is not subject to source basis taxation, provided the employee is present in the source country for not more than 183 days.  Article 14.  On the other hand, directors's fees for services performed for a company resident in one country can be taxed by the source country, regardless how many days the director is present in the source country.  Article 15.  Separate articles apply to income of sportsmen and entertainers, income from pensions, income from government services, and income of students and teachers.  \emph{See} Articles 16-20.    

\addcontentsline{toc}{section}{\protect\numberline{}Stemkowski v. CIR} 
\begin{select}
\caseart{Stemkowski v. CIR}{ 690 F.2d 40 (2nd. Cir. 1982)}{Oakes, Circuit Judge}
\ldots
\begin{center} \textbf{FACTS}
\end{center} 
Taxpayer was traded prior to the beginning of taxable year 1971 to the New York Rangers, who play their home games at Madison Square Garden in New York City. He had previously signed a two-year NHL Standard Player's Contract with the Detroit Red Wings, and this contract was assigned to and assumed by the Rangers. The contract provided for compensation of \$31,500 in the 1970-71 season and \$35,000 in the 1971-72 season plus various NHL bonuses, including a \$1500 bonus for each round won in the play-offs. The player agreed to give his services in all ``league championship'' (i.e., regular season), exhibition, and play-off games, to report in good physical condition to the club training camp at the time and place fixed by the club, to keep himself in good physical condition at all times during the season, and to participate in any and all promotional activities of the club and the league that in the opinion of the club promoted the welfare of the club or professional hockey.

In addition to their rights under this contract, NHL players in 1971 were entitled under the NHL's Owner-Player Council Minutes and Agreements to receive \$25 per exhibition game plus \$25 per week of training camp unless they had played fifty or more games in the previous season, in which case they received \$600 in lieu of payments for exhibition games and training camp allowances other than transportation, food, and lodging. The players were also provided with medical and disability coverage, per diem expenses while traveling during the regular season, and various other benefits.

An NHL player's year is divided into four periods: (1) training camp, including exhibition games, beginning in September and lasting approximately thirty days; (2) the ``league championship'' or regular season of games beginning in October and lasting until April of the following year; (3) the play-off competition, which ends in May; and (4) the off-season, which runs from the end of the regular season for clubs that do not make the play-offs, or from a club's last play-off game, to the first day of training camp. Stemkowski lived in Canada during all of the off-season and most of the training camp period and played in Canada fifteen days out of 179 during the regular season and five out of twenty-eight days during the play-offs. When he was not living in Canada or travelling to games elsewhere, he lived in Long Beach, New York, near New York City, where he shared a rented house with other professional hockey players.

On his tax return, Stemkowski reported \$44,271 in income, of which he initially excluded \$10,625 as earned in Canada\ldots

The less time Stemkowski was in the United States during the period covered by his contract, the less United States tax he owes. Thus, Stemkowski could reduce his tax liability either by showing that he was in Canada for a longer period during the time covered by the contract or, as is at issue here, that the contract covered a time during which he was in Canada. The Tax Court held that the total number of days for which Stemkowski was compensated under his contract was not 234 (all but the off-season) as he had claimed on his tax return, or 365 as he had claimed before the Tax Court, but only 179, the number of days in the regular season. The Tax Court held that Stemkowski could not use days spent in Canada during training camp, the play-offs, or the off-season in calculating his foreign-source exclusion from income. The Tax Court further found that Stemkowski's off-season physical conditioning expenses, because incurred solely in connection with his contractual obligation to show up in good condition at training camp in Canada, were not connected to income from the conduct of a trade or business within the United States and thus were not deductible. \ldots

\begin{center} \textbf{DISCUSSION}\\
\textbf{1.  Allocation of Income}
\end{center} 
The first issue is the Tax Court's determination of the portion of Stemkowski's compensation under the NHL Standard Player's Contract that was drawn from United States sources. As a nonresident alien, Stemkowski was taxable on income connected with the conduct of a trade or business, including the performance of personal services, within the United States. I.R.C. \S\S 871(b), 864(b). Where services are performed partly within and partly outside the United States, but compensation is not separately allocated, Treas. Reg.\@ \S 1.861-4(b) (1975) allocates income to United States sources on a ``time basis'':
\begin{quote} 
(T)he amount to be included in gross income will be that amount which bears the same relation to the total compensation as the number of days of performance of the labor or services within the United States bears to the total number of days of performance of labor or services for which the payment is made.
\end{quote} 
This regulation applies to Stemkowski because the NHL Standard Player's Contract does not distinguish between payments for services performed within and outside the United States.

The parties disagree on what components of a hockey player's year are covered by the basic compensation in the NHL Standard Player's Contract, and therefore on how to compute the time-basis ratio. The taxpayer contends here as he did before the Tax Court that the contract salary compensates him for training camp, play-off, and even off-season services. The Commissioner argues and the Tax Court held that the contract salary covers only the regular season, and therefore that contract salary should be allocated to United States income in the same proportion that the number of days played in the United States during the regular season (164) bears to the total number of days in the regular season (179).\footnote[4]{Although the Tax Court did not discuss separately the component of Stemkowski's total reported income representing play-off bonuses, the Commissioner concedes that Stemkowski's play-off compensation should be allocated separately from his contract salary, according to a ratio whose numerator contains the number of Ranger play-off days in the United States (23), and whose denominator contains the total number of Ranger play-off days in 1971 (28).} We agree with the Commissioner and the Tax Court that the contract does not cover off-season services, but we hold that the Tax Court's finding that the contract does not compensate for training camp and the play-offs as well as the regular season is clearly erroneous. 

The Tax Court's holding was premised on provisions in the NHL contract and other players' agreements, and on the testimony of league and club officials. The first paragraph of the NHL Standard Player's Contract provides that if a player is ``not in the employ of the Club for the whole period of the Club's games in the National Hockey League Championship Schedule,'' i.e., for the entire regular season, then he receives only part of his salary, in the same ratio to his total salary as the ``ratio of the number of days of actual employment to the number of days of the League Championship Schedule of games.'' Paragraph 15 provides that if a player is suspended, he will not receive that portion of his salary equal to the ratio of ``the number of days (of) suspension'' to the ``total number of days of the League Championship Schedule of games.'' The Tax Court concluded from these two paragraphs, and from the NHL's further agreements to pay players separate bonuses for participating in the play-offs and flat fees plus travel, room, and board for participating in training camp and pre-season exhibition games, that the basic contract salary did not cover play-off or training camp services. \ldots

We cannot uphold that finding, as we believe it clearly erroneous. The formulas for docking salary given in the contract's first and fifteenth paragraphs are not persuasive evidence that the salary compensates only for the regular season. These formulas may well use the number of days in the regular season in their denominators for administrative convenience (e.g., because the number of days to be spent in the play-offs cannot be known in advance) or to maximize the salary penalty per day lost. As to the testimony relied upon by the Tax Court, to a certain extent the owners and league officials have an interest in having the contract cover the shortest possible timespan so as to maximize loss to suspended or striking players. Furthermore, two of the league and club officials, Leader and McFarland, testified that at least training camp time was included in the contract. The contract's plain language, moreover, requires in Paragraph 2(a) that a player ``report to the Club training camp ... in good physical condition,'' and a player who fails to report to training camp and participate in exhibition games is subject under Paragraph 3 to a \$500 fine, deductible from his basic salary. True, experienced players were paid \$600 plus room and board for training camp under the Owner-Player Council minutes and Agreements, but we read those Agreements as providing that amount merely to cover the additional expenses of being away from home at training camp.

Paragraph 2 of the contract also plainly requires a player's participation in play-off games in exchange for basic contract salary. While it is true that bonuses are provided for play-off games won, these are simply added incentives, above and beyond salary, to get into and win the play-offs. In this respect, they are just like other incentive bonuses the contract provides to influence conduct during even the regular season,\emph{e.g.}, bonuses for the club's finishing in third place or better (\$2500 in this case), or for the number of goals a player scores per season above certain minimums (at least \$100 per goal over 20). Furthermore, players are required to participate in all play-off games for which they are eligible. Players may be terminated for failure to participate in the play-offs, but players receive nothing for the play-off games that they lose. Thus, we hold that the basic contract salary covered both play-off and training camp services.

We agree, however, that the off-season is not covered by the contract. During the off-season, the contract imposes no specific obligations on a player. Stemkowski argues that the obligation to appear at training camp ``in good condition'' makes off-season conditioning a contractual obligation. Fitness is not a service performed in fulfillment of the contract but a condition of employment. There was no evidence that Stemkowski was required to follow any mandatory conditioning program or was under any club supervision during the off-season. He was required to observe, if anything, only general obligations, applicable as well throughout the year, to conduct himself with loyalty to the club and the league and to participate only in approved promotional activities. 

\ldots
\end{select}


%  \addcontentsline{toc}{section}{\protect\numberline{}Excerpt from the Proposed Regulations on the Source of Compensation of Athletes and Artists} 
%\begin{select}
%\revrul{Excerpt from the Proposed Regulations on the Source of Compensation of Athletes and Artists Regulations (Prop. Reg.\@ \S 1.861-4(b)(2)(ii)(G)(g))}{October 11, 2007}       
%
%\begin{center}\emph{Explanation of Provisions}
%\end{center}
%
%\ldots
%
%The amount of income received by a person, including an individual who is an artist or an athlete, that is properly treated as compensation from the performance of labor or personal services is determined based on all of the facts and circumstances of the particular case. Proposed section 1.861-4(b)(2)(ii)(G) specifies that the amount of compensation for labor or personal services determined on an event basis is the amount of the person's compensation which, based on the facts and circumstances, is attributable to the labor or personal services performed at the location of a specific event.
%
%The IRS and the Treasury Department have determined that the proper source of compensation received by a person, including an individual who is an artist or athlete, specifically for performing labor or personal services at an event is the location of the event. A basis that purports to determine the source of compensation from the performance of labor or personal services at a specific event, whether on a time basis or otherwise, by taking into account the location of labor or personal services performed in preparation for the performance of labor or personal services at the specific event will generally not be the basis that most correctly determines the source of the compensation. This rule applies to situations covered by section 1.861-4(a) and (b).
%
%Under section 1.861-4(a), the source of compensation for labor or personal services performed wholly within the United States is generally from sources within the United States. Therefore, if a person, including an individual who is an artist or an athlete, is specifically compensated for performing labor or personal services at an event in the United States, the source of such compensation is wholly within the United States because the labor or personal services were performed wholly at an event within the United States. The proposed regulations state that a basis that purports to determine the source of such income on a time basis by taking into account the location of labor or personal services performed in preparation for the performance of labor or personal services at the specific event will generally not be a more reasonable basis for determining source of the compensation. The proposed regulations add an example to section 1.861-4(c) to illustrate the application of this rule.
%
%Section 1.861-4(b) applies to instances in which a person is compensated for performing labor or personal services at multiple events, only some of which are within the United States, and at least a portion of the person's compensation cannot be specifically attributed to the person's performance of labor or personal services at a specific location. If the person is not an individual who is compensated as an employee, the source of compensation for labor or personal services is determined on the basis that most correctly reflects the proper source of that income under the facts and circumstances of the particular case. See section 1.861-4(b)(1) and (2)(i). If a person is compensated specifically for labor or personal services performed at multiple events, the basis that most correctly reflects the proper source of that income under the facts and circumstances of the particular case will generally be the location of the events. In addition, a basis that purports to determine the source of such income on a time basis by taking into account the location of labor or personal services performed in preparation for the performance of labor or personal services at the specific event will generally not be the basis that most correctly reflects the proper source of the compensation under proposed section 1.861-4(b)(2)(ii)(G).
%
%The Commissioner may, under the facts and circumstances of the particular case, determine the source of compensation that is received by an individual as an employee under an alternative basis if such compensation is not for a specific time period, provided that the Commissioner's alternative basis determines the source of compensation in a more reasonable manner than the basis used by the individual. Compensation specifically for labor or personal services performed at a specific event is not compensation for a specific time period. The basis that most correctly reflects the proper source of that income will generally be the location of the event under proposed section 1.861-4(b)(2)(ii)(G). In addition, a basis that purports to determine the source of such income on a facts and circumstances basis by taking into account the location of labor or personal services performed in preparation for the performance of labor or personal services at the specific event will generally not more properly determine the source of the compensation under proposed section 1.861-4(b)(2)(ii)(G).
%
%\end{select}

The source of income received for \emph{not} performing services, for example as pursuant to a non-compete contract, is addressed in \emph{The Korfund Company, Inc. v. CIR}.  The \emph{Korfund} court concludes that payments to a nonresident for agreeing \emph{not} to compete in the United States are U.S. source because had the nonresident violated the contract not to compete, the place of performance would have been the United States.  Is this conclusion sound, especially today?  Would the nonresident have had to have been present in the United States to violate the non-compete agreement?  Also, why is the IRS going after Korfund instead of the nonresidents?  Can U.S. persons use the holding in \emph{Korfund} to generate untaxed foreign source income?

\addcontentsline{toc}{section}{\protect\numberline{}The Korfund Company, Inc. v. CIR} 
\begin{select}
\caseart{The Korfund Company, Inc. v. CIR}{ 1 T.C. 1180 (1943)}{Disney, Judge}
\ldots
\begin{center} \textbf{FINDINGS OF FACT}
\end{center}
[Korfund] is a New York corporation organized in 1924, \ldots [and it manufactures and sells] foundation material, such as cork plates and vibration absorbers. [The shareholders at formation were] Hugo Stoessel, a nonresident alien and citizen of Germany [925 shares] and Siegfried Rosenzweig [75 shares] \ldots

The Emil Zorn Aktiengesellschaft [Zorn] is a nonresident foreign corporation engaged in the same business as petitioner, with its principal office in Berlin, Germany. In 1928 its stock was held equally by Stoessel and Werner Genest, a nonresident alien and citizen of Germany. In 1932 or 1933 Stoessel became the sole owner of stock of Zorn.

On October 22, 1926, petitioner entered into a written contract in the United States with Zorn whereby Zorn agreed (a) not to compete with petitioner in this country and Canada or to form, or give any data for the purpose of forming, a competitive company in that territory until the end of 1945, and (b) to give technical and business advice to petitioner upon its request, and petitioner agreed (a) not to furnish material, for the isolation of noise and vibration, outside of the United States and Canada prior to December 31, 1945, except specified territory outside of European countries. Each party agreed to turn over inquiries received from territory of the other and to exchange without charge improvements, inventions, and patents involving isolation against noise and vibration. Zorn was to receive from petitioner quarterly ``a royalty of 1 1/2\% for the year 1926 and 2\% thereafter of the sale of all cork plates with iron frames and of 4\% of the sale of all vibration absorbers,'' computed in a specified manner, with a minimum payment of \$400 for 1926, \$1,000 for 1927, and \$1,250 thereafter through 1940. Zorn did not own any patents at that time. One of the purposes of the contract was to eliminate competition.

On September 21, 1928, the stock of petitioner was held as follows: Stoessel and Genest each 250 shares, Herman Hoevel 299 shares, and Siegfried Rosenzweig 201 shares. \dots

\ldots

On September 21, 1928, petitioner entered into a written agreement with Stoessel in the United States whereby Stoessel undertook to act as consultant and adviser of petitioner in matters relating to the business of petitioner and to communicate to it information of value to petitioner's business until December 31, 1939, for 10 percent of the net earnings of petitioner payable at the end of each year. He also agreed not to act in a similar capacity for any other person, association, or corporation in the United States engaged in the same or a similar business. One of the purposes of the agreement was to eliminate competition.

\ldots

Zorn and Stoessel faithfully performed their agreements not to compete with petitioner and not to give advice to its competitors. On about January 1, 1933, petitioner canceled the contracts of September 21, 1928, and October 22, 1926, with Stoessel and Zorn, and refused to make further payments to them. The contract with Stoessel was canceled on account of his failure to communicate technical information relating to petitioner's business as required by the agreement.

\ldots

On July 30, 1934, Zorn assigned to Bernard Voges, New York City, all sums due it from petitioner under the agreement of October 22, 1926, and Stoessel assigned to the same individual salary in the amount of \$1,984.04 alleged to be payable by petitioner for services rendered prior to October 1, 1932, and the balance of \$2,227.60 payable to him from surplus account, plus interest on the claims of each, with power to recover the amount, plus interest on the claims of each, with power to recover the amounts for the account of the assignors. Voges instituted suit against petitioner in August 1934 under the assignments. An understanding was reached in 1934 to settle the claims by a payment of \$2,750 to Stoessel and \$3,250 to Zorn. The claim of Zorn for \$3,250 and the claim of Stoessel for \$2,227.60 were allowed in full. The remaining amount allowed Stoessel was for demands made under the agreement of September 21, 1928. The total amount was placed at interest and earned interest of \$80, pending approval of the settlement by the German Government. Final settlement was made in 1938 when \$2,508 was paid to Stoessel and \$2,964 to Zorn and \$608 was withheld for payment of withholding taxes.

\begin{center} \textbf{OPINION}
\end{center}

In his determination of the deficiency the respondent held that the allowance of \$2,786.67 to Stoessel and \$3,293.33 to Zorn, which amounts include the proportionate share of each in the interest of \$80, constituted income from sources within the United States on which petitioner, as withholding agent, should have paid a tax equal to 10 percent of the former amount and 15 percent of the latter amount in accordance with the provisions of sections 143 and 144 of the Revenue Act of 1938. The item of \$2,786.67 includes the principal sum of \$2,227.60 representing Stoessel's share of petitioner's old surplus of \$24,910.40. Respondent admits that, of the total amount paid to Stoessel in 1938, \$2,227.60 represented the dividend and the remainder compensation under the contract. The parties differ only on whether this item of \$2,227.60 was received by Stoessel in the taxable year. \ldots

\ldots

Under their contracts Zorn and Stoessel agreed, in general, to act as consultants to petitioner. In addition Zorn agreed not to compete with petitioner or give any information for the formation of a competitive company and Stoessel agreed not to act as consultant to a competitor of petitioner. All of the amount paid to Zorn and the amount paid to Stoessel in excess of the surplus item were paid for these two general classifications of undertakings without any segregation of the amount paid for each. The respondent subjected the entire amounts to withholding tax, presumably in the absence of any basis of segregation, for he does not contend that the income from services performed as consultants is subject to the tax. Not only was no evidence offered on which to make an apportionment, but petitioner does not, upon brief, suggest or request an allocation. Under the circumstances, no apportionment is possible and we will regard all of the amounts in question under this point as having been earned by the nonresident aliens for obligations under the contracts other than service as consultants. See Estate of Alexander Marton, 47 B.T.A. 184.

The sole point of difference between the parties as to this income is whether it was earned from sources within the United States within the meaning of section 119 of the Revenue Act of 1938, and that, as already indicated, turns upon the source of the income derived from agreements not to compete with petitioner in the United States and Canada or give advice for the organization of, or to, a competitor.

The petitioner's contention is based upon the theory that the income was paid for agreements to refrain from doing specific things--negative acts. No defaults occurred and during the period of compliance the promisors were residents of Germany. Petitioner's contention is that negative performance is based upon a continuous exercise of will, which has its source at the place of location of the individual, and that, as the mental exertion involved herein occurred in Germany, the source of the income was in that country, not in the United States where the promise was given. The respondent's view of the question is, in short, that, as the place of performance would be in the United States if Zorn and Stoessel had violated their contractual obligations, abstinence of performance occurs in the same place. Petitioner relies upon Piedras Negras Broadcasting Co., 43 B.T.A. 297; affd., 127 Fed.(2d) 260.

In the Piedras Negras Broadcasting Co. case the taxpayer, a Mexican corporation, owned and operated a radio broadcasting station in Mexico, from which it broadcast programs primarily for listeners in the United States, for which it received compensation in the United States from citizens thereof. In holding that the source of such income was not within the United States, we pointed out that the studio and broadcasting plant were located, and operated by the employment of capital and labor, in Mexico; that the source of the income was, accordingly, in such studio and power plant, and that the reception of the radio impulses in receiving sets in this country was secondary, not the primary source. The court in affirming the decision said that ``the source of income is the situs of the income-producing service'' and that the source of the income was ``the act of transmission.'' This reasoning is said to be equally applicable to the situation here.

In Sabatini v. Commissioner, 98 Fed.(2d) 753, the taxpayer was an author and a subject of Great Britain. He was not in the United States before, nor during, the taxable years. By contract executed outside the United States he gave to a publisher in this country, among other rights, the right to publish certain books, as to some of which copyrights were not obtainable. As to these the taxpayer, by the contract above mentioned, agreed not to authorize any other publisher to publish the books in the United States so long as the publisher left in print its editions of the books. The taxpayer was to receive under the contracts amounts determinable from the number of volumes sold. In holding that the income paid based upon the sale of these books was derived by the taxpayer from sources within the United States, the court said:
\begin{quote}The payments were received in consideration of his granting the publisher the exclusive right to publish here. To be sure, that may not have been of great value but the parties did value it and the author received the payments as agreed. We are not now concerned with the quality of the consideration he gave but only with the taxability of that which he received. The payments were made to him for foregoing his right to authorize others for a time to publish the works here. Though others may, perhaps, lawfully have published them they could not do so under his express authority. The rights he granted were an interest in property in the United States, in the one instance the statutory copyrights obtainable and in the other the exclusive right to publish with his permission.
\end{quote}

In Ingram v. Bowers, 47 Fed.(2d) 925; affd., 57 Fed.(2d) 65, Enrico Caruso, a nonresident alien, entered into a contract in the United Stated to sing for the Victor Talking Machine Co. for the purpose of making phonograph records of selections rendered by him. The agreement contained a provision that Caruso would not permit any records of his voice to be made by any other concern. He was to receive under the contract a specified amount of the selling price of records sold, with a minimum yearly payment. In holding that the income received by Caruso under the contract from foreign sales constituted income from sources within the United States, the court pointed out that the decisive feature was the fact that the services were rendered in the United States and that those services were the source of all income derived from the contracts. No point appears to have been made of the fact that some part of the income was paid for the promise to refrain from singing for others, as it is not discussed in the opinion. Under petitioner's theory here, such part would not have been taxable.

%In Commissioner v. Ferro-Enamel Corporation, 34 Fed.(2d) 564 (Apr. 7, 1943), the taxpayer agreed to purchase the entire output of the mines of a Canadian corporation for a period of three years and as a part of the contract purchased shares of the corporation's stock, for the sole purpose of obtaining raw material for its domestic business. Later in the year the corporation went out of existence and its stock became worthless. In holding that the loss occurred in Canada, the court said:
%\begin{quote}The statute in question undertakes to classify the sources of income within the United States and without the United States by the nature and location of the activities of the taxpayer or his property which produces the income. * * *

%* * * The loss grows out of an activity or use of property and the situs of the loss is not transferred to the home of respondent because respondent wished to obtain a source of raw material.
%\end{quote}

We think the question here is governed by the principles laid down in the Sabatini, Ingram and Ferro-Enamel Corporation cases. Zorn had a right to compete with petitioner in the United States and Canada and for that purpose to form a competitive company or to assist others in forming one. Likewise, Stoessel had a right to serve other corporations or individuals in the United States engaged in a business similar to petitioner's as a consultant and to furnish them information of value to their business. They were willing to and did give up these rights in this country for a limited time for a consideration payable in the United States, just as did Sabatini in ``foregoing his right to authorize others for a time to publish the works here.'' The Circuit Court in that case calls the exclusive right to publish an interest in property in the United States; so here, in our opinion, the rights of Stoessel and Zorn to do business in this country, in competition with the petitioner, were interests in property in this country. They might have received amounts here for services or information, but were willing to forego that right and possibility for a limited period for a consideration. What they received was in lieu of what they might have received. The situs of the right was in the United States, not elsewhere, and the income that flowed from the privileges was necessarily earned and produced here. Petitioner is merely using it, so to speak, for a specified time, subject to periodical payments to the owners of the rights. Upon the termination of the contracts the rights reverted to Zorn and Stoessel, and they were then free to exercise them independent of the agreements entered into with petitioner. These rights were property of value and the income in question was derived from the use thereof in the United States.

The Piedras Negras Broadcasting Co. case is distinguishable. It involved employment of capital and labor in a foreign country in connection with the rendition of service--not the foregoing, for a consideration, of a right to transact business in the United States.

We find and hold that the source of all of the income in question was in the United States and is subject to withholding tax in the taxable year. Accordingly,

Decision will be entered for the respondent.
\end{select}

In \emph{CIR v. Piedras Negras Broadcasting}, the court was faced with determining the source of income from a Mexican radio station transmitting into the United States in exchange for payments from U.S. advertisers, which constituted 95\% of its income.  What were the U.S. advertisers paying for: broadcasting, or broadcasting to the U.S. audience?  Fast forward 50 years.  For what medium is this case potentially relevant?      
   
\addcontentsline{toc}{section}{\protect\numberline{}CIR v. Piedras Negras Broadcasting Co.} 
\begin{select}
\caseart{CIR v. Piedras Negras Broadcasting Co.}{ 127 F.2d 260 (5th Cir. 1942)}{Holmes, Circuit Judge}
The respondent is a corporation organized under the laws of the State of Coahuila, Republic of Mexico, with its principal office and place of business at Piedras Negras, Mexico. Its business is the operation of a radio broadcasting station located at Piedras Negras, just across the Rio Grande from Eagle Pass, Texas. The decisive question presented by this petition for review is whether the respondent, from the operation of its business in 1936 and 1937, derived any income from sources within the United States subject to taxation by the United States.

The taxpayer conducted its affairs in the familiar manner. Its income was derived from the dissemination of advertising over the radio and from the rental of its facilities to customers, referred to as the sale of ``radio time.''  All of its income-producing contracts were executed in Mexico, and all services required of the taxpayer under the contracts were rendered in Mexico. The company maintained a mailing address at Eagle Pass, Texas, and used a hotel room there in which it counted and allocated the funds received in the mails each day.

Contracts with advertisers in the United States were handled through an advertising agent, an independent contractor. The majority of the taxpayer's responses from listeners came from the United States, and ninety-five per cent of its income was from advertisers within the United States. Bank accounts were maintained in Texas and in Mexico. The books and records of the corporation were in Mexico, its only studio was there, and all of the broadcasts by the station originated in Piedras Negras. The broadcasts were equal in volume in all directions, and were heard by listeners in this country and elsewhere.

Section 231(d) of the Revenue Act of 1936 provides that the gross income of a foreign corporation includes only the gross income from sources within the United States. If this taxpayer, a foreign corporation, had no income from sources within the United States, no income tax was levied upon it. The Board of Tax Appeals concluded that none of the respondent's income was derived from sources within the United States, and we agree with that decision.
In Section 119 of the Revenue Act of 1936, Congress classified income, as to the source thereof, under six heads. \ldots Since the taxpayer's income was derived exclusively from the operation of its broadcasting facilities located in Mexico, or from the rental of those facilities in Mexico, its income therefrom was either compensation for personal labor or services, or rentals or royalties from property, or both, under the statutory classification. Section 119(a)(3) provides that compensation for personal services performed in the United States shall be treated as income from sources within the United States. By Section 119(c)(3), income from such services performed without the United States is not from sources within the United States. Likewise, rentals from property located without the United States, including rentals or royalties for the use of or for the privilege of using without the United States franchises and other like properties, are considered items of income from sources without the United States. Section 119(c)(4) of the Revenue Act of 1936.

We think the language of the statutes clearly demonstrates the intendment of Congress that the source of income is the situs of the income-producing service. The repeated use of the words within and without the United States denotes a concept of some physical presence, some tangible and visible activity. If income is produced by the transmission of electromagnetic waves that cover a radius of several thousand miles, free of control or regulation by the sender from the moment of generation, the source of that income is the act of transmission. All of respondent's broadcasting facilities were situated without the United States, and all of the services it rendered in connection with its business were performed in Mexico. None of its income was derived from sources within the United States. \ldots

The order of the Board of Tax Appeals is affirmed.

\textbf{McCORD, Circuit Judge, (dissenting).}

I am unable to agree with the majority opinion.

Prior to March, 1935, many programs broadcast over the Mexican station originated in a remote control studio located in Eagle Pass, Texas. After the Communications Commission denied application for continuance of the studio, programs no longer originated in the United States, but the broadcasting company continued its business operations in much the same way that it always had. While the mere broadcasting of electromagnetic waves into this country may not constitute the doing of business which produces income derived from sources within the United States, I do not think the case is as simple as that. The actual broadcasting of messages was not the only act, and the facts should be viewed as a whole, not singly, to see what was actually being done.

Various advertising contracts provided that the service to be rendered was to be from the station at Piedras Negras, but these contract provisions do not establish that the company was not taxable in this country. The programs of the Piedras Negras Broadcasting Company were primarily designed for listeners in the United States. Ninety per cent of its listener response came from this country, and ninety-five per cent of its income came from American advertisers. Through agents the broadcasting company solicited advertising contracts in this country, and it is shown that contracts were entered into by the company in the name of the Radio Service Co., an assumed name which for reasons beneficial to the company had been registered in Texas. The contracts also contained a provision that venue of any suit on such contracts would be Maverick County, Texas. Moreover, the company used Eagle Pass, Texas, as its mailing address, and its constant use of the United States mails was most beneficial to the company if not absolutely essential to the success of its operation. Money was deposited in American banks, obviously for convenience and to avoid payment of foreign exchange. Agents of the broadcasting company made daily trips to Eagle Pass where they met in a hotel room with advertising representatives and opened the mail and divided the enclosed money according to their percentage contracts with advertisers, and it is shown that the company received much of its income in this manner. It was, therefore, receiving income by broadcasting operations coupled with personal contact in this country.

I am of opinion that all the facts taken together establish that Piedras Negras Broadcasting Company was doing business in the United States, was deriving income from sources within this country, and was taxable. I think the decision of the Board should be reversed. I respectfully dissent.
\end{select}

\addcontentsline{toc}{section}{\protect\numberline{}Comments}
	\begin{center}
			\textbf{Comments}
		\end{center}


\textbf{\emph{Athletes and Entertainers}}  The compensation of athletes and entertainers presents many challenges.  When performances occur in different countries and the contract does not separately break out the fees for U.S. and foreign performances, it is necessary to allocate the fees between U.S. and foreign sources.  In addition, athletes can receive many types of compensation. In \emph{Stemkowski}, we saw an example of compensation received pursuant to a standard player's contract.  In Rev.\@ Rul.\@ 74-108, the IRS addressed the tax issues relating to a sign-on fee received by a foreign soccer player--widely believed to be Pele.  According to the ruling, ``a sign-on fee is paid to induce the player to sign and become bound by the provisions of the agreement. The agreement does not require the player actually to play for the club; it is merely a preliminary agreement that is separate and distinct from a `uniform player'' contract which binds a player to play soccer for a salary. When a player enters into an agreement, the taxpayer places him on its reserve list thereby protecting such player from recruiting efforts of any other club and preventing him from negotiating to play or playing for any other professional soccer club. No part of the sign on fee is attributable to future services, but the team anticipates the agreement and fee will induce the player to sign and become bound by the uniform player contract if the club wishes to use his services and a separate employment contract is negotiated for this purpose."  

Based on Rev.\@ Rul.\@ 58-145, which had held that a baseball signing bonus was not compensation for employment withholding tax purposes, Rev.\@ Rul.\@ 74-108 concluded that sign-on bonus was not service income under \S861(a)(3), but rather a payment for a covenant not to compete.  In determining how to allocate the payment between U.S. and foreign sources, the IRS stated: ``in some cases it may be reasonable to make the allocation on the basis of the relative value of the taxpayer's services within and without the United States, or on the basis of the portion of the year during which soccer is played within and without the United States."  How administratively feasible is the basis on which the IRS suggests allocating the sign on fee? 

Rev.\@ Rul.\@ 74-108 was revoked by Rev.\@ Rul.\@ 2004-109, 2004-2 C.B. 958, on the grounds that Rev.\@ Rul.\@ 58-145 was incorrectly decided.  Consequently, sign-on fees are now to be sourced as compensation under \S861(a)(3), but the ruling did not give any guidance on how the fee should be allocated between U.S. and foreign sources.     

Article 16 of the Treaty addresses the compensation of entertainers and sportsmen.  Once an entertainer's compensation gross receipts exceed \$20,000, the source country may tax the entertainer's income.  In the absence of Article 16, the income of many entertainers and sportsmen would be exempt from source basis taxation because either they (1) would not have a permanent establishment (in the case of an independent contractor), or (2) would be paid by a foreign employer and not present in the source country for more than 183 days (in the case of an employee).  Field Service Advice 199947027, below, addresses the treatment under Article 16 of the income of models who are hired to promote a corporation's products and services.    

 Some athletes are able to exploit the goodwill associated with their status by entering into endorsement contracts under which an athlete permits a company to use his name or likeness in advertising and agrees to perform personal services, such as appearance.  The issue of whether payments made under such contracts are payments for services or royalties is addressed below in \emph{Goosen v. CIR} and \emph{Garcia v. CIR} in Chapter 3.3.
 
 
% FSA 199947028, below, addresses the treatment of a lump sum payment received to sign an endorsement contract that permitted a company to use the athlete's name, image, etc. Compare the holding in \emph{Linseman v. CIR}  discussed in the FSA and the basis on which allocation of the lump sum payment is made with the holding in Rev.\@ Rul.\@ 74-108.  

\textbf{\emph{Withholding}} Payments to a nonresident of U.S. source compensation that is effectively connected are generally not subject to withholding under section 1441 if the the income is subject to normal wage withholding rules or specifically exempt from wage withholding or exempt under a treaty.  \S1441(c)(1); Reg.\@ \S 1.1441-4(b)(1)(i), (ii), and (iv).  To claim exemption under a treaty, the nonresident must file Form 8233 with the U.S. withholding agent.  Reg.\@ \S 1.1441-4(b)(2).

\addcontentsline{toc}{section}{\protect\numberline{}Field Service Advice 199947028}
\begin{select}
\revrul{Field Service Advice 199947027}{Sept. 30, 1999}
\ldots\\
\begin{center} \textbf{FACTS}
\end{center}


For the years in issue, Taxpayer A is a nonresident alien individual who is a
citizen and resident of Country X. Taxpayer A is a model and actor who comes to
the United States for assignments as such. According to Form 2106 (Employee
Business Expense), attached to Taxpayer A's return, Taxpayer A's occupation is
acting. For Year 1 and Year 2, Taxpayer A filed a 1040NR. On each return,
Taxpayer A indicated that Taxpayer A was claiming the benefit of the Royalties
Article of the U.S.--Country X Treaty. Specifically, the returns indicate that income
of X Dollars for Year 1 and Y Dollars for Year 2, while effectively connected with the
conduct of a trade or business within the United States, is nevertheless exempt
from U.S. income tax because such income qualifies as royalties under the U.S.--Country X Treaty.

Taxpayer A entered into a contract with Corp B on Date C (“the Agreement”). 
The Agreement is a contract between Taxpayer A and Corp B with respect to
Taxpayer A's services:
\begin{quotation}
as a model and performer in connection with the \emph{advertising, marketing,
promotion, publicizing, merchandising, and distribution for [Corp B] products and services} manufactured, sold, offered, furnished, licensed, or distributed,
now or in the future, under the [Corp B] trade name (hereinafter collectively
referred to as the ``Products"). [Emphasis added.]
\end{quotation} 

As a part of such services required by the Agreement, Taxpayer A was
required to render services as a spokesperson to Corp B, including appearing at
press conferences and granting interviews. Taxpayer A was also required to render
services:
\begin{quotation}
as a performer and model in the production of materials advertising and
promoting Corp B and its Products in all forms of media, electronic or
otherwise, whether now or later developed, including but not limited to,
television (free-t.v., basic cable, premium, pay-per-view, and closed circuit)
and radio commercials, consumer and trade print, magazines, newspapers,
point of purchase, mailers and mailing inserts, theatrical and cinema
advertising, interactive and multimedia programming, home shopping, video
for in-store use, video trailers, infomercials, how-to videos, outdoor,
collateral, catalogs, packaging, in-store, direct mail, internal company
materials, and public relations/press interview kits (hereinafter collectively
referred to as the ``materials"). Without limiting the foregoing, however, we
agree that [Taxpayer A] will not be required to sell or deliver copy offering the
sale of any Products on home shopping or in any infomercials, although
[Taxpayer A] may be required to discuss the Products in a favorable fashion. 
In addition, \emph{we shall not have the right to separately sell video tapes or
cassettes embodying [Taxpayer A's] performance, our rights in such video
tapes and cassettes being limited to broadcast uses and uses as free
giveaways or as premium items accompanying Product offers}. [Emphasis
added.]
\end{quotation}
Additionally, the Agreement required Taxpayer A to attend an orientation
session with Corp B's senior management to acquaint Taxpayer A with Corp B's
products and philosophy. Taxpayer A was further required to grant interviews and
make appearances at public relations events each year of the Agreement. The
Agreement also required Taxpayer A to use best efforts to promote and endorse
Corp B and its products at all Corp B functions attended by Taxpayer A, and to
consider promoting and endorsing Corp B and its products in all public and
professional appearances attended by Taxpayer A. The Agreement required
Taxpayer A to perform the services required thereunder in a competent and
``artistic" manner to the best of Taxpayer A's ability. 

In addition to the foregoing, the Agreement required that Taxpayer A only
use Corp B products for Taxpayer A's Type B Product needs, unless Corp B did not
manufacture or distribute such a product required by Taxpayer A. The Agreement
further required that, during the term of the Agreement, Taxpayer A use reasonable 
efforts not to publicly handle any Type B Product other than those manufactured or
distributed by Corp B.

In addition to Taxpayer A's services, the Agreement provides that:
\begin{quotation}
During the term of this agreement, [Taxpayer A] hereby grant to [Corp B] the
right to use and to license the use of your performance, name, signature,
photograph, voice, picture, likeness, or other indicia of your identity in
connection with the materials produced hereunder in such advertising,
merchandising, publicizing, promotional and marketing medium as permitted
pursuant to this agreement....
\end{quotation}

\begin{center}\textbf{LAW AND ANALYSIS}
\end{center}
The issue involved in this case is whether Taxpayer A, a model and actor, is
an ``entertainer," for purposes of the Artistes and Athletes Article of the U.S.--Country X Treaty, with respect to services performed under the Agreement. 
Paragraph 1 of the Artistes and Athletes Article of the U.S.--Country X Treaty
provides, in part, that:
\begin{quotation}
[I]ncome derived by entertainers, such as theatre, motion picture, radio or
television artistes, and musicians, and by athletes, from their personal
activities as such may be taxed in the Contracting State in which these
activities are exercised....
\end{quotation}
While the Artistes and Athletes Article of the U.S.-Country X Treaty sets forth
examples of ``entertainers" falling within its provisions, it does not define the term
``entertainer". Further, Taxpayer A's activity as a model under the Agreement is not
an enumerated activity under the language of the treaty. Accordingly, it is not clear
on the face of the Artistes and Athletes Article of the U.S.--Country X Treaty whether
Taxpayer A is an ``entertainer" with respect to Taxpayer A's activities under the
Agreement. 

The Treasury Department Technical Explanation of the Artistes and Athletes
Article of the U.S.--Country X Treaty also does not define the term ``entertainer," nor
does it further describe the types of individuals that would be considered
``entertainers" for purposes of the U.S.--Country X Treaty. Where a U.S. treaty and
the technical explanations thereto are ambiguous or silent on a point, it may be
appropriate to consider comparable provisions of the Organization for Economic
Co-operation and Development Model Double Taxation Convention on Income and
on Capital (the ``OECD Model Convention"), and the official commentaries thereto,
in interpreting the U.S. treaty, provided the language of the OECD Model
Convention is in substance substantially similar to that of the U.S. Treaty at issue. 

The provisions of the OECD Model Convention and the official commentaries
thereto are relevant because the United States is an OECD member-country and
has incorporated provisions of OECD Model Conventions into its treaties, including
the U.S.--Country X Treaty. 

Paragraph 1 of the Artistes and Sportsmen Article of the 1998 OECD Model
Convention is substantially similar to the language of the Artiste and Athletes
provision of the U.S.--Country X Treaty. Paragraph 1 of the Artistes and Sportsmen
Article of the 1998 OECD Model Convention provides:
\begin{quotation}
Notwithstanding the provisions of Articles 14 and 15, income derived by a
resident of a Contracting State as an entertainer, such as a theatre, motion
picture, radio or television artiste, or a musician, or as a sportsman, from his
personal activities as such exercised in the other Contracting State, may be
taxed in that other State. 
\end{quotation}
Paragraph 3 of the Commentary to the Artistes and Sportsmen Article of the 1998
OECD Model Convention, which explains the meaning and purpose of paragraph 1
of the Artistes and Sportsmen Article, provides:
\begin{quotation}
Paragraph 1 refers to artiste and sportsmen. It is not possible to give a
precise definition of ``artiste", but paragraph 1 includes examples of persons
who would be regarded as such. These examples should not be considered
as exhaustive. On the one hand, the term ``artiste" clearly includes the stage
performer, film actor, actor (including for instance a former sportsman) in a
television commercial. The Article may also apply to income received from
activities which involve a political, social, religious or charitable nature, \emph{if an
entertainment character is present}. On the other hand, it does not extend to
a visiting conference speaker or to administrative or support staff (e.g.
cameramen for a film, producer, film director, choreographers, technical staff,
road crew for a pop group etc.). \emph{In between there is a grey area where it is
necessary to review the overall balance of the activities of the person
concerned}. [Emphasis added.]
\end{quotation}

Paragraph 6 of the Commentary to the Artistes and Sportsmen Article of the 1998
OECD Model Convention further provides that:
\begin{quotation}

The Article also applies to income from other activities which are usually
regarded as of an entertainment character, such as those deriving from
billiards and snooker, chess and bridge tournaments.
 \end{quotation}
 
Therefore, the commentaries that address the scope of the Artistes and Athletes
Article focus on whether there is an entertainment character to the activity
performed by the individual and on whether such activity is ``usually" regarded as of
an entertainment character. 

Based on the foregoing, we believe that, in determining whether Taxpayer A
is an ``entertainer" with respect to Taxpayer A's activities under the Agreement, for
purposes of the Artistes and Athletes Article of the U.S.--Country X Treaty, the focus
should be on whether the primary purpose of the specific activity being performed
by Taxpayer A under the Agreement is entertainment. 

The Agreement provides that Taxpayer A's would provide services:

\begin{quotation}

as a model and performer in connection with the advertising, marketing,
promotion, publicizing, merchandising, and distribution for [Corp B] products
and services manufactured, sold, offered, furnished, licensed, or distributed,
now or in the future, under the [Corp B] trade name (hereinafter collectively
referred to as the ``Products").
\end{quotation}
The foregoing language indicates that, generally, the primary purpose of Taxpayer
A's activities under the Agreement is the promotion, marketing and sale of Corp B
Products, not entertainment. This is further supported by the fact that the
provisions of the Agreement setting forth the services to be performed by Taxpayer
A also focus on the promotion, marketing and sale of Corp B Products. The fact
that the Agreement refers to Taxpayer A rendering services as a model and
``performer", or that the Agreement requires Taxpayer A to perform the services
required thereunder in an ``artistic" manner, does not, in itself, change the primary
purpose of Taxpayer A's activities under the Agreement from promotion, marketing
and sale or Corp B Products to entertainment, because entertainment is generally
not the end sought to be accomplished by the activities required under the
Agreement.

Accordingly, based on the foregoing, since the primary purpose of Taxpayer
A's activities under the Agreement is generally not entertainment, Taxpayer A is
generally not an ``entertainer" for purposes of the Artistes and Athletes Article of the
U.S.--Country X Treaty, with respect to Taxpayer's activities under the Agreement,
despite the fact that Taxpayer A may also be an actor outside of the Agreement. 
However, if Taxpayer A did in fact perform an activity pursuant to the Agreement,
and the primary purpose of such activity was entertainment, then, with respect to
such activity, Taxpayer A could be an ``entertainer" for purposes of the Artistes and
Athletes Article of the U.S.--Country X Treaty.

If Taxpayer A is not an ``entertainer" within the meaning of the Artistes and
Athlete Article of the U.S.--Country X Treaty, with respect to Taxpayer A's activities 
under the Agreement, Taxpayer A's income from such activities is not taxable by
the United States under that article of the U.S.--Country X Treaty. However, such
income may be taxable by the United States under other articles of the U.S.--Country X Treaty. For example, the Independent Personal Services Article of the
U.S.--Country X Treaty may apply to the portion of such income attributable to
Taxpayer A's personal services under the Agreement if Taxpayer A either had a
fixed base regularly available in the United States for the purpose of performing
Taxpayer A's services, or was present in the United States for an aggregate of
more than 183 days in the respective years at issue. Further, the Independent
Personal Services Article of the U.S.--Country X Treaty may apply to royalty income
derived by Taxpayer A under the Agreement if Taxpayer A performed independent
personal services within the United States from a fixed base and the right or
property with respect to which the royalties are paid is effectively connected with
such fixed base. If Taxpayer A did not have a fixed base within the United States,
taxation of royalties, as defined under the U.S.--Country X Treaty, derived by
Taxpayer A under the Agreement would be governed by the Royalties Article of the
U.S.--Country X Treaty and, therefore, would be taxable only by County X, Taxpayer
A’s country of residence.

\end{select}

%\addcontentsline{toc}{section}{\protect\numberline{}Rev.\@ Rul.\@\@ 74-108}
%\begin{select}
%\revrul{Rev.\@ Rul.\@\@ 74-108}{1974-1 C.B. 248}
%\ldots\\
%The taxpayer, a domestic corporation, operates a professional soccer club in the United States. The club is a member of a professional league affiliated with the governing body of world-wide professional soccer. The taxpayer entered into an agreement with a nonresident alien individual during the current taxable year. In order to induce the nonresident alien individual to sign the agreement, the taxpayer paid him a sign on fee. The agreement was executed by the taxpayer and the nonresident alien individual outside the United States.
%
%The sign on fee is paid to induce the player to sign and become bound by the provisions of the agreement. The agreement does not require the player actually to play for the club; it is merely a preliminary agreement that is separate and distinct from a ``uniform player'' contract which binds a player to play soccer for a salary. When a player enters into an agreement, the taxpayer places him on its reserve list thereby protecting such player from recruiting efforts of any other club and preventing him from negotiating to play or playing for any other professional soccer club. No part of the sign on fee is attributable to future services, but the team anticipates the agreement and fee will induce the player to sign and become bound by the uniform player contract if the club wishes to use his services and a separate employment contract is negotiated for this purpose.
%
%The professional soccer league with which the taxpayer's team is affiliated includes eleven members. Seven of the member teams, including the taxpayer, are located within the United States and the remaining four are located without the United States. The taxpayer's team schedule provides for some of its soccer games to be played in the United States and the remainder to be played in foreign countries.
%
%Section 1441(a) of the Code provides, in part, that except as otherwise provided in section 1441(c), all persons in whatever capacity acting, having control or payment of any of the items of income specified in section 1441(b), to the extent that any of such items constitutes gross income from sources within the United States, of any nonresident alien individual shall deduct and withhold from such items a tax equal to 30 percent thereof.
%
%The items of income described in section 1441(b) of the Code include wages, compensation, and other fixed or determinable annual or periodical income. Section 1.1441-3(a) of the Income Tax Regulations provides that the sources of income shall be determined in accordance with sections 861 to 864 and regulations thereunder.
%
%Rev.\@ Rul.\@ 58-145, 1958-1 C.B. 360, provides that bonuses, which are not predicated on continuing employment, made to new baseball players solely for signing their first contracts, do not represent remuneration for services performed. Such bonuses are taxable to the baseball player as ordinary income in the taxable year received. Accordingly, in the instant case the sign on fee, which is similar to a bonus, paid to the nonresident alien individual is not compensation for labor or personal services for purposes of the source of income rules in section 861(a)(3) or section 862(a)(3) of the Code.
%
%The sign on fee, or bonus, was paid to insure that if the nonresident alien individual did play professional soccer, he would provide his services for the taxpayer only and to no other professional soccer club. See Richard A. Allen, 50 T.C. 466 (1968). The bonus was paid as compensation for the promises made by the nonresident alien individual in the sign on agreement which in essence amounted to a covenant not to compete.
%
%Compensation received for a promise not to compete is taxable as ordinary income and does not constitute income from the sale of property either real or personal. See John D. Beals, Jr., 31 B.T.A. 966, aff'd, 82 F.2d 268 (2d Cir.), XV-2 C.B. 227 (1936). Such compensation is fixed or determinable annual or periodical income and its source is the place where the promisor forfeited his right to act. Korfund Co., 1 T.C. 1180 (1943). Therefore, amounts paid to a nonresident alien for his promise not to compete in the United States are subject to withholding under section 1441(a) of the Code.
%
%In the instant case the sign on fee is paid for the nonresident alien individual's promise not to compete both within and without the United States. Therefore, the sign on fee is attributable to sources both within and without the United States and the income must be apportioned appropriately. See section 863(b) of the Code pertaining to the reporting of income partly from within and partly from without the United States.
%
%Accordingly, a portion of the sign on fee in the instant case is income from sources within the United States and is subject to withholding under the provisions of section 1441(a) of the Code. The basis upon which the sign on fee is allocated as income from sources within and sources without the United States must be reasonable and based on the facts and circumstances in each case. For example, in some cases it may be reasonable to make the allocation on the basis of the relative value of the taxpayer's services within and without the United States, or on the basis of the portion of the year during which soccer is played within and without the United States. Where a reasonable basis for allocation does not exist, the entire sign on fee is income from sources within the United States and is subject to section 1441(a).
%\end{select}

%\addcontentsline{toc}{section}{\protect\numberline{}Field Service Advice 199947028}
%\begin{select}
%\revrul{Field Service Advice 199947028}{Sept. 30, 1999}
%\ldots\\
%\begin{center} \textbf{FACTS}
%\end{center}
%For the years in issue, Taxpayer A is a nonresident alien professional athlete who is a resident and citizen of Country A. Corp X is a domestic corporation producing athletic products. As discussed below, Corp X entered into an Endorsement Contract on Date A for Taxpayer A's services and endorsement of Corp X's products. However, because Taxpayer A had by contract transferred Taxpayer A's rights to provide endorsement and services exclusively to Corp B, a Country B corporation wholly owned by Taxpayer A, the Endorsement Contract at issue is between Corp B and Corp X. Taxpayer A, however, reports all the income received from Corp X directly on Taxpayer A's income tax return.
%
%The Endorsement Contract provides that Corp X desired to acquire, on an exclusive basis, the right to use the name, image, endorsement and athletic renown of Taxpayer A in connection with the advertisement, promotion and sale of Corp X's products and to obtain certain services of Taxpayer A. The Endorsement Contract gave Corp X the right to use Taxpayer A's name, nickname, initials, autograph, voice, video or film portrayals, facsimile signature, photograph, likeness and image or facsimile image, and any other means of endorsement by Taxpayer A, in connection with the development, manufacture, promotion and sale of Corp X's products. Because Taxpayer A had contracted away these rights to Corp B, along with the right to sublicense such rights to third parties, the Endorsement Contract further provides that Corp B agrees to grant such rights to Corp X.
%
%The Endorsement Contract also provides that, throughout the term of the contract, Corp B agrees to cause Taxpayer A to wear and use exclusively Corp X's products while participating in athletic and sports related activities in public. The Endorsement Contract also provides that Corp B agrees to provide Taxpayer A's services, upon Corp X's request, for up to Z appearances each contract year in connection with the promotion, advertisement and sale of Corp X's products. The Endorsement Contract is for a period of Y Years.
%
%The Endorsement Contract provides for a fixed annual base remuneration, for the rights and services provided for under the contract, for each year during the Y Year contract period. In addition to annual base remuneration, the Endorsement Contract provides that Corp X shall make a one-time non-refundable payment (``Lump Sum Payment'') to Corporation B in the amount of X Dollars, within thirty (30) days following full execution of the Endorsement Contract. Such payment was made on Date B in accordance with the terms of the Endorsement Contract. The Endorsement Contract also provides that Corp B represents that neither Corp B nor Taxpayer A is a party to any agreement that would prevent or hinder performance of the obligations provided for in the Endorsement Contract.
%
%Taxpayer A asserts that the Lump Sum Payment was an inducement to leave Corp Y, a competitor corporation to Corp X, and sign the Endorsement Contract with Corp X. Accordingly, Taxpayer A asserts that the Lump Sum Payment was not predicated on any prior, current or future services to Corp X, nor did it relate to the grant of rights provided to Corp X in the Endorsement Contract. Taxpayer A, therefore, argues that the Lump Sum Payment would be of a type of income that would fall within the ``Other Income'' articles of the OECD and U.S. Model Tax Treaties, and, thus, should only be taxable by the Taxpayer's country of residence. Taxpayer A's contract with Corp Y expired on Date A, the same day as he entered into the Endorsement Contract with Corp X.
%
%There is no income tax treaty between the United States and either Country A or Country B.
%
%\begin{center} \textbf{LAW AND ANALYSIS}
%\end{center}
%Taxpayer A asserts that the Lump Sum Payment is not subject to U.S. taxation because it is of a type of income that would fall within the "Other Income" articles of the OECD and U.S. Model Treaties, which generally assign taxing jurisdiction over income not dealt with in other articles to a taxpayer's country of residence. There is, however, no income tax treaty between the United States and Taxpayer A's country of residence. Therefore, Taxpayer A's tax liability is determined under U.S. domestic law concepts. The issue, therefore, is the character and source of the Lump Sum Payment received by Taxpayer A as a part of the Endorsement Contract, under U.S. domestic law.
%
%Although there are no cases or public guidance relating to sign on bonuses paid under endorsement contracts, there is some legal authority relating to the character of sign on bonuses paid to an athlete by a professional team. One such case is, Allen v. Commissioner, 50 T.C. 466 (1968), which involved a \$70,000 sign on bonus paid by a baseball team to a baseball player as a part of a contract wherein the player agreed to provide services as a baseball player. The \$70,000 sign on bonus in Allen was paid over a five year period beginning on the date the contract was signed. The player in Allen argued that the sign on bonus was not compensation ``in respect of services'' within the meaning of IRC section 73 because it was non-refundable, even if the player played no games for the team, and was in addition to stated compensation provided in the contract for the player's services as a baseball player.
%
%For the tax year in issue, the player in Allen was considered a ``child'' for purposes of section 73. Section 73 provides:
%\begin{quote}
%Amounts received IN RESPECT OF THE SERVICES of a child shall be
%included in his gross income and not in the gross income of the
%parent, even though such amounts are not received by the child.
%I.R.C. section 73.
%\end{quote}
%The Allen court found that:
%\begin{quote}
%[T]he bonus payments were paid by the Phillies as an inducement
%to obtain his services as a professional baseball player and to
%preclude him from rendering those services to other professional
%baseball teams; they thus certainly constituted amounts received
%`in respect of' his services.
%\end{quote}
%Accordingly, the Allen court found that, while the \$70,000 bonus was ``an indirect rather than a direct payment for services,'' it was compensation ``in respect of'' the player's services and was, therefore, includible in the player's income under section 73. The Allen decision therefore supports the general position that a sign on bonus paid after a contract has been entered into should be treated as paid in respect of that which the contract requires (i.e., in Allen, services as a baseball player).
%
%In the instant case, Taxpayer A received the Lump Sum Payment for signing the Endorsement Contract with Corp X, an athletic wear company. The amount was nonrefundable and in addition to stated compensation. The Lump Sum Payment was consideration for having entered into the Endorsement Contract, pursuant to which Corp X was granted various rights and services, including the grant of rights to use both Taxpayer A's endorsement and services in promoting Corp X's products. Taxpayer A was entitled to the Lump Sum Payment only after Taxpayer A signed the Endorsement Contract. In view of Allen, the character of the Lump Sum Payment received by Taxpayer A should be based on the different rights and services provided for by the Endorsement Contract because the Lump Sum Payment was paid by Corp X in order to obtain those various rights and services. Therefore, because the Endorsement Contract encompassed both future endorsement rights and future services, the Lump Sum Payment should be characterized partly as payment in respect of endorsement rights and partly as compensation in respect of personal services.
%
%This case is distinguishable from Revenue Ruling 74-108, 1974-1 C.B. 248, which held that a lump sum amount bonus paid, prior to any employment contract, by a domestic soccer team to a nonresident alien player was, in its entirety, paid in consideration of a covenant not to compete. In Rev.\@ Rul.\@ 74-108, the agreement was not an actual contract for the player's services, but merely prohibited the player from negotiating a player contract with any other team. In the instant case the Lump Sum Payment was paid pursuant to the contract actually granting Corp X the rights to Taxpayer A's endorsement and services.
%
%In Ken Linseman v. Commissioner, 82 T.C. 514, 522 (1984), the Tax Court held that the most reasonable method of sourcing a sign on bonus paid by a domestic hockey team to a nonresident alien hockey player as an inducement to enter into a 6-year employment contract between U.S. and foreign sources was based on the number of games the team contemplated playing within and without the United States during the first year of the contract. Although the instant case is factually distinguishable from Linseman because the sign on bonus in Linseman was paid before the actual employment contract was entered into, it is our view that the same rationale used by the Tax Court in Linseman to allocate the source of a sign on bonus paid as an inducement to enter into a multiple-year contract should be used to allocate the Lump Sum Payment in this case, which was also paid pursuant to a multiple-year contract. Accordingly, we believe it is appropriate to allocate the source and character of the Lump Sum Payment paid during the first year of the contract based on the rights exercised and services performed during the first contract year.
%
%Based on the foregoing, the allocation of the Lump Sum Payment to each component of the contract should be in the same proportion as the allocation of the First Contract Year's annual base remuneration to each component, based on the different rights exercised and services performed pursuant to the terms of the Endorsement Contract during the First Contract Year. The source of the Lump Sum Payment should also be in proportion to the allocation of the First Contract Year's annual base remuneration to U.S. versus foreign sourced income. Linseman, 82 T.C. 514. Each component of the annual base remuneration should be sourced using the appropriate sourcing rules, in view of the character allocated to that portion of the annual base remuneration. For example, any portion characterized as royalties should be sourced based on the place of use, while portions characterized as compensation for personal services should be sourced based on the place of performance of the services. I.R.C. sections 861(a)(4), 861(a)(3)
%\end{select}

\addcontentsline{toc}{section}{\protect\numberline{}Compensation for Services Problems} 
	\begin{center}
		\textbf{Compensation for Services Problems}
	\end{center}
	\begin{select}
\begin{enumerate}
	
			
	\item John, a U.S. citizen, works in a NYC law firm.  As part of a tax controversy matter, he is sent to the U.K. to assist in document review.  He works a total of 2 months in the U.K.  His total salary paid by his firm is \$120,000, but \$5,000 is directly deposited into his U.K. bank account.  
		\begin{enumerate}
			\item What is the source of his income?
			\item Assume that he receives a \$20,000 bonus for his excellent work in reviewing the documents? (He was able to stay awake.)  What's the source of the bonus?
			\item Assume that the firm receives a performance bonus of \$1 million for its work on the case.  What's the source of the bonus?
			\item Assume that under U.K. law, he is also taxed by the U.K. on a portion of his salary.  Is there any argument under the Treaty that the income is not subject to U.K. tax? [Article 14]
		\end{enumerate}
		
	\item Elizabeth, a U.K. resident and citizen, is CFO of BritCo, a U.K. corporation.  She comes here one month per year to supervise the U.S. subsidiary operations of BritCo.  Her U.K. salary is \$240,000.
		     \begin{enumerate}
				\item How is she taxed by the U.S.? [\S\S 861(a)(3)(C)(ii); 864(b)(1), (c)(2)(B); 871(a), (b); Reg.\@ \S 1.864-4(c)(6)(ii)] 
				\item How does the U.K. treaty change your conclusion? [Article 14]
			\end{enumerate}
			
			\item Elizabeth, a U.K. resident and citizen, is CFO of Citigroup's U.K. branch operations.  Her duties require her presence in the U.S. for one month per year. Her U.K. salary is \$240,000.
		     \begin{enumerate}
				\item How is she taxed by the U.S.?
				\item How does the U.K. treaty change your conclusion? [Article 14]
			\end{enumerate}
			
		\item Richard is a U.K. citizen who works 360 days in 2008 in the United States for USCO, a U.S. corporation.  He leaves on December 31, 2008, and is not present in the United States in 2009.   Prior to coming to the United States, he negotiates with his employer to defer 80\% of his salary (\$80,000) until 2009, when he will be back in the U.K.  (Under U.S. tax principles, this agreement executed before the services are rendered should be sufficient to ensure he isn't taxed on the income until he receives it.)  USCO pays him in 2009, when he is a resident of the U.K.
				\begin{enumerate}
					\item How is he taxed under the Code?  [\S\S 1; 864(c)(6) (read slowly and carefully and follow the referenced sections); Reg.\@ \S 1.864-4(c)(6)(ii); and 871(b)]
					\item Does the Treaty change your answer to the previous problem? [Article 14]
				\end{enumerate}
				
			\item Richard is a U.K. citizen and resident who works for Google in the U.K. from 2005 until the end of 2007.  Each year during this period he spent two months in the U.S. working on projects.  During this period, he was granted compensatory options to purchase 1,000 shares of Google at \$100 per share.  The options vest at the end of 2007, when the stock is worth \$200 per share, and he exercises the options at the end of 2008 when the stock is worth \$500 per share.  In 2008, he doesn't spend any time in the U.S.  (Under U.S. law, the grant of options is generally not a taxable event, but the exercise of the options generates ordinary compensation income equal to the difference between the exercise price and fair market value.  See \S 83(a) and (e).)  				
			\begin{enumerate}
					\item What is the source of the option income?   [Reg.\@ \S 1.861-4(b)(2)(ii)(F), (ii)(G), Example 6]
					\item How is it taxed under the Code?  [\S\S 864(c)(6); Reg.\@ \S 1.861-4(b)(2)(ii)(F); Reg.\@ \S 1.864-4(c)(6)(ii); and 871(b)]
					\item Does the Treaty change your answer to the previous problem? [Exchange of Notes to Article 14 (found after the Treaty Protocol, which is found at the end of the Treaty).]
				\end{enumerate}

				\item Richard is a U.K. citizen and resident who works for IBM in the U.S. from 1990 until the end of 2007.  He retires and moves back to the U.K. at the end of 2007.  Beginning in 2008, he collects a monthly pension from IBM and a monthly social security payment.   Assume that 50\% of the pension's earnings are U.S. source income.			
				\begin{enumerate}
					\item What is the source of the pension and social security payments?   [\S 871(a)(3); Reg.\@ \S 1.864-4(c)(6)(ii)]
					\item How are they taxed under the Code?  [\S\S 864(c)(6); Reg.\@ \S 1.864-4(c)(6)(ii); and 871(a)(3) and (b)]
					\item Does the Treaty change your answer to the previous problem? [Article 17]
				\end{enumerate}
				
				\item Mary, a U.K. citizen and resident, is a professor at Nothingwhich University in the U.K.  Because she has some friends on the faculty, she scores a visiting gig at the FLS for 2009 for which she is paid \$100,000.  At the end of 2009, she returns home to Nothingwhich.  
				\begin{enumerate}
					\item What is the source of Mary's compensation?   
					\item How is it taxed under the Code? 
					\item Does the Treaty change your answer to the previous problem? [Article 20A--find it!]
				\end{enumerate}

			\item Mr. David Spice is a U.K. citizen and resident who is employed as a footballer (soccer player) by B(ad) F(ood) United, a U.K. corporation.  He receives a salary for \$12 million per year from BFUnited.  Under his contract, he is required to play all of the BFU league matches and a series of other ``goodwill" games in other countries.  In 2008, BFU signs a contract to play six games in the U.S. and receives a fee of \$1 million per game.  For 2008, Spice plays 30 games in the U.K. and 6 in the U.S.
		\begin{enumerate}
			\item What is the source of BFU's and Spice's income? [Stemkowski and Prop. Reg.\@ \S 1.861-4(b)(2)(ii)(G)]
			\item How is it taxed under the Code? 
			\item Does the Treaty change your answers above? [Articles 7, 14, and 16]
			\item When Spice signed with BFU in 2007, he received a \$1 million signing bonus.  What is the source and how it taxed by the U.S.  Does the Treaty change your answer?
		\end{enumerate}	
				
			\end{enumerate}
		\end{select}

\begin{framed}
Last Revised Aug. 30, 2015; source\_Services\_Aug30\_15
\end{framed}
