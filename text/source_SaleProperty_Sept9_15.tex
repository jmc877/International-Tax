 
\section{Sale of Property}
\crt{861(a)(5) and (6); 863(b); 865(a), (b), (d), and (g); 871(a)(1)(D); 871(a)(2); 1441(a) and (c)(5); and 1442(b)(2)}{1.863-1(b); 1.863-3; 1.865-2(a)(1) and (2); 1.1441-2(b)(2)(i)}{Article 13}

The source of income arising from the sale of property depends on the type of property and whether the property was manufactured by the seller.  Income from the sale of a \emph{U.S. real property interest} (USRPI)---U.S. real property or shares of a U.S. corporation holding substantial amounts of USRPIs---are U.S. source.  \S861(a)(5).  As we'll see below in Chapter 5, such gains are taxed as effectively connected income.  

Section 865 addresses the source of income from the sale of personal property.  Section 865(a) states the general rule that income from the sale of personal property is sourced by the residence of the \emph{seller.}  Scrolling down through section 865, an observant reader quickly discovers that the important exceptions for inventory property, depreciable personal property, and intangible property, swallow the bright-line rule.  It takes a few minutes of reflection to think of personal property that is covered by the residence rule.  One important category of property covered is securities.  

Section 865(g) contains a modified definition of ``resident" under which U.S. citizens, resident aliens, and nonresident aliens with a U.S. \emph{tax home} are U.S. residents for purposes of section 865. \S865(g)(1).  \margit{Section 865 has a separate definition of residence.} A citizen or resident alien with a foreign tax home is only a foreign resident if he pays a foreign tax of 10\% or more on any gain from the sale of personal property. \S865(g)(2).  U.S. corporations are U.S. residents, but the sale of property by a partnership is sourced at the partner level.  \S865(g)(1)(ii) and (i)(5).  

The most important exception to the general section 865 residence-of-the-seller rule is the exception for purchased inventory property in section 865(b), which takes income from the sale of inventory out of section 865 and instead sources it by where the property is sold.  \S 861(a)(6). \margit{Inventory is sourced where title passes thus making it easy to for US taxpayers to generate low taxed foreign source income.} Many activities can be an integral part of a sales transaction--meetings and negotiations, entertainment, contract signing, and product delivery--and these activities may occur in different countries.  The regulations under section 861 provide that the situs of the sale of property is where the ``rights, title, and interest of the seller in the property are transferred to the buyer''---the title passage rule.  Reg.\@ \S 1.861-7(c). Title generally passes where the sales contract states that it passes.  Consequently, it is relatively easy for a U.S. taxpayer to generate foreign source income (and an increased foreign tax credit limit) on the sale of purchased inventory by merely providing that title will pass at delivery to a foreign port.  Although the seller has increased risk by maintaining formal title until delivery, this risk can be mitigated by purchasing insurance.  

Income from the sale of intangibles is sourced under the residence-of-the seller rule if the consideration is not contingent of the productivity, use, or disposition of the intangible.  \S865(d)(1)(A).  Contingent payments, however, are sourced like royalties:  that is, they are sourced where the property is used.  \S865(d)(1)(B).  Gain from the sale of goodwill is sourced where the goodwill was generated.  \S865(d)(2).  

Income from the sale of inventory that is manufactured in the U.S. and sold abroad or \emph{vice versa} is mixed source under section 863(b).  \margit{Inventory manufactured in the US and sold abroad is mixed source.} The regulations that source income from the sale of manufactured inventory provide two rules depending on the type of property.  Income from the sale outside of the United States of property derived from natural resources within the United States such as a farm, mine or oil or gas well, is allocated between U.S. and foreign sources based on the FMV of the property at the U.S. export terminal.  Thus, the gross income realized from the sale of corn grown in South Dakota but exported to China with title passing abroad will be U.S. source to the extent of the FMV of the corn at the U.S. export terminal.  Reg.\@ \S1.863-1(b)(1).  The excess of the FMV over the export terminal price can be allocated between U.S. and foreign sources based on the rules in Reg.\@ \S 1.863-3 only if the taxpayer engages in additional production activities subsequent to shipment from export terminal and outside the country of sale.  If no additional production or additional production occurs in the country of sale, the excess will be sourced by country of sale.\footnote{Under prior regulations, income from natural resources was sourced based solely on the location of the land, mine, well,\emph{etc.} This long-standing regulation was held to be invalid in \emph{Phillips Petroleum v.\@ CIR}, 97 T.C. 30 (1991), \emph{aff'd without opinion}, 70 F.3d 1282 (10th Cir. 1995).}

For example, a U.S. company extracts rocks from a foreign mine, transports them to a processing facility where they are processed into copper concentrate and then transported to a port.  The processed rocks are shipped to purchasers in the United States.  Since these activities do not constitute ``additional production activities'' prior to arrival at the port, and because title passes in the United States, gross receipts equal to the FMV of the processed rocks will be foreign source and any excess will be U.S. source.  Reg.\@ \S 1.863-1(b)(7), Example 1.  If, for instance, the copper concentrate were transformed into smelted copper prior to shipment to the United States, the transformation would be considered to be ``additional production activities'' and gross receipts equal to the FMV of copper concentrate will be foreign source and any excess receipts will be sourced based on the rules of Reg.\@ \S1.863-3. Reg.\@ \S 1.863-1(b)(7), Example 5.  

For income from the sale of manufactured inventory that is not derived from natural resources, Reg.\@ \S 1.863-3(a) provides that the gross income must generally be allocated 50-50 between sales and production activities. \margit{The 50-50 rule for manufactured inventory.}The income from the sales activities is allocated generally based on where title passes, and production where the production assets are located.  Reg.\@ \S 1.863-3(b)(1)(i) and (c).  A taxpayer may instead elect to allocate gross income based on the independent factory price method (IFP), which allocates gross income based on the price at which the taxpayer would sell the property to an independent distributor.  In order to use the IFP method, the taxpayer must regularly sell part of its output to wholly independent distributors.  Reg.\@ \S 1.863-3(b)(2).        

For example, P, a U.S. taxpayer produces products in the United States and sells them to a foreign person with title passing abroad.  P's gross income is \$50.  Under the 50--50 method, one-half of P's gross income (\$25) is attributable to the production activities and one-half is attributable to the sales activity.  Reg.\@ \S 1.863-3(b)(1)(ii), Example.

To be subject to tax under section 871 or 881, the income must be U.S. source \emph{and} FDAP.  Consequently, any income of a foreign person that is treated as foreign source under section 865 is not subject to U.S. tax unless the foreign person is engaged in a U.S. trade or business and the income is treated as effectively connected income under section 864(c)(4). \margit{Gains from the sale of personal property are not FDAP.} Furthermore, even if gains arising from the sale of property are U.S. source, they will generally not be FDAP.  Reg.\@ \S1.1441-2(b)(2)(i).  

One exception to this rule is section 871(a)(1)(D) (and 881(a)(1)(4)), which taxes U.S. source gains from the sale of intangible property to the extent that the payments are contingent on the productivity, use or disposition of the property sold.  Another incredibly minor and mean--spirited rule is section 871(a)(2), which taxes the U.S. source capital gains of aliens present in the United States for 183 days or more.  Remember, though, that an alien present in the U.S. for 183 days or more is a resident alien and thus, not covered by this provision.  The only persons to whom this provision could apply are those persons present in the United States whose days of presence don't count under the day-count test, (\emph{e.g.}, students and teachers).\footnote{This provision may also potentially apply to a resident alien who is treated as a nonresident under a treaty tie-breaker provision.  Under Reg.\@ \S310.7701(b)-7, such persons are treated as nonresidents for all purposes of the Code.  If the person is a resident of a treaty that permits taxation of capital gains, \emph{e,g,}, the U.S.--India Treaty, then section 871(a)(2) could apply to gains realized by the dual resident.}   Rest assured that the revenues from this provision do not contribute significantly to the U.S. fiscal needs.

Under Article 13, pars. 1 and 5, gains from the sale of real estate can be taxed by the country in which the real estate is located, but other capital gains may not be taxed by the source country unless the gains are attributable to a permanent establishment.\margit{Treaties generally don't allow source based taxation of income from personal property sales, but permit source based taxation of gains from real property.}  Since the treaty treats contingent gains from the sale of intangible property as royalties, the source country may not tax them.  Article 12, par. 2(b).      
    
Rev.\@\@ Rul.\@ 64-56 explores when property is deemed transferred under section 351, thus constituting a sale of property.  Note how that same intellectual property can be sold to one or more foreign countries.  
\addcontentsline{toc}{section}{\protect\numberline{}Rev.\@\@ Rul.\@ 64-56}
\begin{select}
\revrul{Rev.\@\@ Rul.\@ 64-56}{1964-1 C.B. 133}

\ldots
The issue has been drawn to the attention of the Service, particularly in cases in which a manufacturer agrees to assist a newly organized foreign corporation to enter upon a business abroad of making and selling the same kind of product as it makes. The transferor typically grants to the transferee rights to use manufacturing processes in which the transferor has exclusive rights by virtue of process patents or the protection otherwise extended by law to the owner of a process. The transferor also often agrees to furnish technical assistance in the construction and operation of the plant and to provide on a continuing basis technical information as to new developments in the field.

Some of this consideration is commonly called `know-how.' In exchange, the transferee typically issues to the transferor all or part of its stock.

\ldots

The term `property' for purposes of section 351 of the Code will be held to include anything qualifying as `secret processes and formulas' within the meaning of sections 861(a)(4) and 862(a)(4) of the Code and any other secret information as to a device, process, etc., in the general nature of a patentable invention without regard to whether a patent has been applied for \ldots, and without regard to whether it is patentable in the patent law sense \ldots. Other information which is secret will be given consideration as `property' on a case-by-case basis.

The fact that information is recorded on paper or some other physical material is not itself an indication that the information is property. See, for example, Harold L. Regenstein, et ux. v.\@ Commissioner, 35 T.C. 183 (1960), where the fact that a program for providing group life insurance to Federal Government employees was transmitted in the form of a written plan did not preclude a finding that the payment for the plan was a payment for personal services.

It is assumed for the purpose of this Revenue Ruling that the country in which the transferee is to operate affords to the transferor substantial legal protection against the unauthorized disclosure and use of the process, formula, or other secret information involved.

Once it is established that `property' has been transferred, the transfer will be tax-free under  section 351 even though services were used to produce the property. Such is generally the case where the transferor developed the property primarily for use in its own manufacturing business. However, where the information transferred has been developed specially for the transferee, the stock received in exchange for it may be treated as payment for services rendered. See Regenstein, supra, where the taxpayer developed a plan for selling insurance which he ultimately sold to certain insurance companies. The court held that the consideration received was payment for services.

Where the transferor agrees to perform services in connection with a transfer of property, tax-free treatment will be accorded if the services are merely ancillary and subsidiary to the property transfer. Whether or not services are merely ancillary and subsidiary to a property transfer is a question of fact. Ancillary and subsidiary services could be performed, for example, in promoting the transaction by demonstrating and explaining the use of the property, or by assisting in the effective `starting-up' of the property transferred, or by performing under a guarantee relating to such effective starting-up. \ldots. Where both property and services are furnished as consideration, and the services are not merely ancillary and subsidiary to the property transfer, a reasonable allocation is to be made.

Training the transferee's employees in skills of any grade through expertness, for example, in a recognized profession, craft, or trade is to be distinguished as essentially educational and, like any other teaching services, is taxable when compensated in stock or otherwise, without being affected by section 351 of the Code. However, where the transferee's employees concerned already have the particular skills in question, it will ordinarily follow as a matter of fact that other consideration alone and not training in those skills is being furnished for the transferor's stock.

Continuing technical assistance after the starting-up phase will not be regarded as performance under a guarantee, and the consideration therefor will ordinarily be treated as compensation for professional services, taxable without regard to section 351 of the Code. \ldots

Assistance in the construction of a plant building to house machinery transferred, or to house machinery to be used in applying a patented or other process or formula which qualifies as property transferred, will ordinarily be considered to be in the nature of an architect's or construction engineer's services rendered to the transferee and not merely rendered on behalf of the transferor in producing, or promoting the sale or exchange of, the things transferred. Similarly, advice as to the lay-out of plant machinery and equipment may be so unrelated to the particular property transferred as to constitute no more than a rendering of advisory services to the transferee.

The transfer of all substantial rights in property of the kind hereinbefore specified will be treated as a transfer of property for purposes of section 351 of the Code. The transfer will also qualify under section 351 of the Code if the transferred rights extend to all of the territory of one or more countries and consist of all substantial rights therein, the transfer being clearly limited to such territory, notwithstanding that rights are retained as to some other country's territory. \ldots

The property right in a formula may consist of the method of making a composition and the composition itself, namely the proportions of its ingredients, or it may consist of only the method of making the composition. Where the property right in the secret formula consists of both the composition and the method of making it, the unqualified transfer in perpetuity of the exclusive right to use the formula, including the right to use and sell the products made from and representing the formula, within all the territory of the country will be treated as the transfer of all substantial rights in the property in that country.

The unqualified transfer in perpetuity of the exclusive right to use a secret process or other similar secret information qualifying as property within all the territory of a country, or the unqualified transfer in perpetuity of the exclusive right to make, use and sell an unpatented but secret product within all the territory of a country, will be treated as the transfer of all substantial rights in the property in that country.

\ldots
\end{select}

As we saw in the last section, it is sometimes difficult to distinguish between royalties and services when the payment is for services that result in the creation of intellectual property.  When intellectual property is transferred to and used by another person, the payment may be characterized as a payment for services, sale of property, or license, depending on the scope of the rights transferred.  Each of these characterizations, however, results in a different sourcing rule.  Rev.\@\@ Rul.\@ 84-78 address whether a payment to transmit a live boxing match constitutes payment for services, royalty, or the sale of property.



\addcontentsline{toc}{section}{\protect\numberline{}Rev.\@\@ Rul.\@ 84-78}
\begin{select}
\revrul{Rev.\@\@ Rul.\@ 84-78}{1984-1 C.B. 173}
\ldots
\begin{center}\textbf{ISSUE}
\end{center}
Whether the amount that a domestic corporation receives from a foreign corporation for the right to broadcast a live
boxing match taking place in the United States via closed circuit television only in the country in which the foreign
corporation is incorporated is foreign source income under the circumstances described below.

\begin{center}\textbf{FACTS}
\end{center}

\textit{Situation 1}. A domestic corporation, Y, obtained from the contestants in a prize fight, which will take place in the United States, the exclusive rights to broadcast the fight live and to record the broadcast for subsequent viewing. Y
entered into a contract with FX, a foreign corporation incorporated in foreign country FC. The contract provides that for a stipulated lump-sum payments to be paid by FX to Y, FX will have the right to broadcast the prize fight via closed
circuit television only to an audience in FC. The payment that Y receives from FX under the contract is to be refunded
to FX if the fight is cancelled for any reason. The broadcast and the simultaneous recording of the broadcast will be
protected under the copyright laws of Title 17 of the United States Code (1976 and Supp. 1979). The broadcast right that
Y transfers to FX is nonexclusive, and the duration of such right is only for the live showing of the fight. FX's right to
broadcast the prize fight does not include recording rights for subsequent viewing. The contract is negotiated, executed
and the consideration is paid in the United States.

\textit{Situation 2}. The facts are the same as in Situation 1, except that Y transfers to a foreign corporation, FXB, incorporated
in foreign country FCB, a broadcasting right in the specified prize fight that is exclusive and exercisable only in FCB.

\begin{center}\textbf{LAW AND ANALYSIS}
\end{center}

Section 861(a)(3) of the Internal Revenue Code provides that, subject to certain exceptions not relevant here,
compensation for labor or personal services performed in the United States will be treated as income from sources
within the United States.

Section 861(a)(6) of the Code provides that gains, profits, and income derived from the purchase of personal property
without the United States and its sale or exchange within the United States is income from sources within the United
States.

Section 1.861-7 of the Income Tax Regulations provides that gains, profits, and income derived from the purchase
and sale of personal property shall be treated as derived entirely from the country in which the property is sold.
Section 862(a)(4) of the Code provides that rentals or royalties for the use of or for the privilege of using without the
United States copyrights and other like properties shall be treated as income from sources without the United States.

In Rev.\@\@ Rul.\@ 74-555, 1974-2 C.B. 202, a nonresident alien taxpayer executed a contract with a domestic corporation
which gave the corporation the exclusive right to publish in the United States all books, and long and short stories written by the taxpayer. The revenue ruling holds that the payments received by the taxpayer under the contract are royalties for the use of, or for the privilege of using, copyrights in the United States and are not compensation for labor or personal services because the contract did not give the corporation any control over what the taxpayer was to write or when it was to be written, but merely the right to publish any books or stories that were written.

Rev.\@\@ Rul.\@ 54-409, 1954-2 C.B. 174, holds that a copyright is divisible into separate properties, and that if the owner
of a copyright granted to another the exclusive right to exploit the copyrighted work in a particular medium throughout
the life of the copyright, then the consideration received for the use of the copyright would be treated as proceeds from
the sale of property as long as this consideration was not received in certain periodic forms which were regarded as
characteristic of royalty payments. Rev.\@\@ Rul.\@ 60-226 modified this position by providing that the sale result reached in
Rev.\@\@ Rul.\@ 54-409 would hold regardless of the form of the consideration paid for the right to use the copyright. Although
the holding of Rev.\@\@ Rul.\@ 60-226 has been overridden in part by statute in the case of certain forms of consideration (see
section 871(e) of the Code), the ruling remains applicable in the present case.

The source of the payment received by Y in exchange for the grant of the right to broadcast the prize fight as United
States or foreign income is dependent upon whether the characterization of the income is compensation for labor or
personal services, income derived from the sale of personal property, or royalties for the use of or for the privilege of
using a copyright or other like property, or some other type of income.

\textit{Situation 1.} The contract entered into between Y and FX does not give FX any control over when or where the prize
fight will take place or how the arrangements for the fight will be made, nor does it confer any legal rights over the
contestants in the fight; it merely gives FX the right to broadcast the fight if it occurs. Further, the activities of Y are not exclusively performed for the benefit of FX, such that FX would own the product of Y's labor. \textit{See Ingram v.\@ Bowers}, 57 F.2d 65 (2d Cir. 1932) \textit{aff'g.} 47 F.2d 925 (S.D.N.Y. 1931). Accordingly, the payment received by Y is not compensation for labor or personal services.

The broadcasting right that Y transfers to FX is not exclusive, and the duration of such right is not for the
remaining life of Y's copyright, but is only for the live broadcast of the specified prize fight. FX cannot exploit the
broadcast for the life of the copyright since it has no recording rights. The payment that Y receives from FX for such
right, therefore, is not income derived from the sale of personal property. Rev.\@\@ Ruls. 54-509 and 60-226. The payment
that Y receives from FX for such right is for the use of, or for the privilege of using, a copyright without the United States.

\textit{Situation 2.} Although the broadcasting right that Y transfers to FXB is exclusive, the duration of such right is not for the remaining life of Y's copyright, but is limited only to the live broadcast of the specified prize fight. Because the broadcasting right that Y transfers is for less than the remaining life of Y's copyright, the payment that Y receives from FXB for such right is not income derived from the sale of personal property, even though the right is for the exclusive use of FXB. \textit{See} Rev.\@\@ Ruls. 54-409 and 60-226, and \textit{Pickren v.\@ United States,} 249 F. Supp. 560 (M.D. Fla. 1965), \textit{aff'd,} 378 F.2d 595 (5th Cir. 1967), in which the court held that the grant of exclusive rights in secret formulas and trade names for less than the remaining lives of such properties did not constitute a sale. The payment that Y receives from FXB for the broadcasting right is for the use of, or for the privilege of using, a copyright without the United States. \textit{See also Oak Manufacturing Co. v.\@ United States,} 301 F.2d 259 (7th Cir. 1962).

\begin{center} \textbf{Holdings}
\end{center}
\textit{Situation 1.} The payment that Y receives from FX is foreign source income under section 862(a)(4).\\
\textit{Situation 2.} The payment that Y receives from FXB is foreign source income under section 862(a)(4).
\end{select}

In \emph{International Multifoods Corp. v.\@ CIR}, Multifoods, a U.S. corporation, sold its Asian operations for a gain of \$2 million and allocated the sales proceeds among its trademarks (\$120K), non-compete clause (\$820K), and goodwill (\$1.1 million).  If this allocation were respected, what would be the source of the various items of income?  (See section 865(d) and \emph{Korfund}).  Now, calculate Multifoods's foreign tax credit limitation assuming that this is the only income it earned, and it is subject to a 35\% U.S. tax rate.  (The entire amount is gain.)  Finally, assume that Multifoods has \$750,000 of unused foreign tax credits from prior years.  What is its residual U.S. tax liability (after the FTC you calculated)?  What happens to its residual U.S. tax liability to the extent that the gain is U.S. source?  Although this case anticipates a bit our study of the U.S. foreign tax credit, it shows the importance of the basic source rules in determining the credit.  Finally, has this case written the source rule for goodwill under section 865(d)(3) out of the Code?  In the excerpted case below, I have left in much of the Court's discussion of the role of the tax attorneys in documenting the transaction. 

\addcontentsline{toc}{section}{\protect\numberline{}International Multifoods Corp. v.\@ CIR} 
\begin{select}
\caseart{International Multifoods Corp. v.\@ CIR}{ 108 T.C. 25 (1997)}{Ruwe, Judge}
\ldots 

\ldots We must decide what portion, if any, of the gain realized by petitioner on the sale of Asian and Pacific operations of Mister Donut of America, Inc. (Mister Donut), petitioner's wholly owned subsidiary, to Duskin Co. (Duskin) on January 31, 1989,
constitutes foreign source income for purposes of computing petitioner's foreign tax credit limitation pursuant to section
904(a). \ldots

\begin{center} \textbf{Findings of Fact}
\end{center}
[International Multifoods (IM)] was involved primarily in the  the manufacture, processing, and distribution of food products.

Mister Donut franchised Mister Donut pastry shops in the United States and abroad. As of January 1989, there were
approximately 500 Mister Donut shops in the United States, 78 shops in Asia and the Pacific, and approximately 35 to 40
shops in Europe, the Middle East, and Latin America. Mister Donut joined in the filing of petitioner's consolidated returns.

\ldots
\begin{center} \textbf{Petitioner's Asian and Pacific Mister Donut Operations}
\end{center}
[As of January 1989, IM had registered Mister Donut trademarks in many Southeast Asian countries.  The agreements were similar except for franchise fees, royalties, development schedules, and the length of the agreement.]

Mister Donut had perfected a system that utilized franchisees to prepare and merchandise distinctive quality 
doughnuts, pastries, and other food products. The franchise agreements refer to this system as the ``Mister Donut System'',
which is described as:
\begin{quote} the name ``Mister Donut,'' a unique and readily recognizable design, color scheme and layout for the premises wherein such business is conducted (herein called a ``Mister Donut Shop'') and for its furnishings, signs,
emblems, trade names, trademarks, certification marks and service marks * * *, all of which may be changed,
improved and further developed from time to time * * *
\end{quote}
The Mister Donut System also included methods of preparation, serving and merchandising doughnuts, pastries, and
other food products, and the use of specially prepared doughnut, pastry, and other food product mixes as may be changed,
improved, and disclosed to persons franchised by petitioner to operate a Mister Donut shop.

\ldots

\begin{center} \textbf{Petitioner's Sale of Its Asian and Pacific Mister Donut Operations to Duskin}
\end{center}
Duskin is a Japanese corporation which markets a variety of goods and services, primarily through franchise
operations. On November 19, 1983, petitioner and Duskin entered into an agreement for the sale of petitioner's assets,
rights, and interests in Mister Donut in Japan (the Japan Agreement). The Japan Agreement also included a covenant by
petitioner not to compete in the donut business in Japan for a period of 20 years, as well as a covenant by Duskin
not to conduct any business similar to the Mister Donut business anywhere outside Japan for a period of 10 years. By the end of 1986, petitioner had decided to sell its food distribution and franchise business. Petitioner was having difficulty providing adequate service to its Mister Donut operations in Asia and the Pacific. Duskin was seeking
to expand into new territories as it had nearly saturated the Japanese market. Given its organization, financing, and
experience, Duskin appeared the logical buyer for petitioner's franchisor's interest in Mister Donut in Asia and the Pacific.

On January 31, 1989, following 2 years of negotiations, petitioner and Duskin entered into an agreement for the sale
of petitioner's entire interest in Mister Donut in designated Asian and Pacific nations for \$2,050,000. Pursuant to the
agreement, petitioner sold its existing franchise agreements, trademarks, Mister Donut System, and goodwill for each of
the operating countries, and its trademarks \ldots and Mister Donut System in the nonoperating countries. Joseph Dubanoski, formerly a division vice president with petitioner whose primary responsibilities involved the development and
implementation of international franchises, determined petitioner's sale price. In arriving at this amount, Mr. Dubanoski
considered: (1) The royalty income generated in the operating countries; (2) the growth potential in the operating
countries; (3) the development potential in the nonoperating countries; and (4) the value of the trademarks in the operating and nonoperating countries.

\ldots 

\ldots The purchase agreement also contained a covenant by petitioner not to compete in the operating and
nonoperating countries for a period of 20 years. \ldots 
%Article XIV, paragraph 1, of the agreement stated:
%\begin{quote}
%MDAI [Mister Donut] covenants and agrees with Duskin that, for a period of twenty (20) years
%commencing on the Post--Closing Date, MDAI will not, either directly or indirectly:

%(a) carry on in any of the Non--Operating Countries or in any of the Duskin Operating Countries any business
%similar to the Mister Donut shop business being sold and transferred by MDAI to Duskin on the Post--Closing
%Date;

%(b) otherwise sell doughnuts in any of the Non--Operating Countries or any of the Duskin Operating Countries;
%or

%(c) disclose all or any part of the Mister Donut System or any of the bakery mix formulae, with or without
%the payment of consideration, to any person for use in any of the Non--Operating Countries or the Duskin
%Operating Countries. * * *
%\end{quote}

The agreement similarly contained a covenant by Duskin not to compete in any business similar to the Mister Donut
business in the United States, Canada, and 38 European, Mideastern, Caribbean, and Latin American countries
for a period of 5 years. The countries included in the Duskin covenant were nations where petitioner had Mister Donut
franchise operations or registered trademarks.\footnote[11]{The purchase agreement also amended Duskin's covenant not to compete contained in the Japan Agreement to conform with Duskin's covenant under the purchase agreement. As a result, Duskin was no longer precluded from competing in the donut business outside Japan; rather, Duskin could compete anywhere in the world outside of 41 enumerated countries, none of which were located in Asia.}
\begin{center} \textbf{Petitioner's Allocation and Reporting of the Proceeds From the Sale}
\end{center}
\ldots
The first draft [of the purchase agreement], which was dated January 20, 1988, and prepared by Bruce M. Bakerman of petitioner's legal
department, contained a provision allocating the purchase price between the existing franchises, goodwill, trademarks,
and pending trademark applications. The actual percentage to be allocated to these assets was left blank. Mr. Suess
reviewed this draft and handwrote the following on the document:
Approve subject to:
\begin{quote}
1) Review of foreign tax consequences associated with each country covered by the agreement;

2) Review of foreign source income rules to determine best way to maximize foreign source income.
Initial review indicates goodwill and noncompete covenants may give rise to such income.

3) Allocation of proceeds will be critical aspects of 1 \& 2 above, therefore flexibility in this area should
be a major negotiating point.
\end{quote}

In a memorandum dated May 24, 1988, from Michael S. Munro to Paul Quinn, Mr. Munro recommended that the
purchase agreement should not contain an allocation of the sale price. \ldots In response to this suggestion, petitioner's legal department removed the allocation from the subsequent draft dated May 25, 1988. However, in a memorandum dated May 27, 1988, Mr. Schaefer expressed concern regarding the absence of such an allocation:

\begin{quote}The lack of any purchase price allocation in the Agreement is not particularly helpful from a U.S. tax
viewpoint. However, the fact that the purchaser is a Japanese entity and the current lack of distinction in the
amount of tax on capital gains and ordinary income minimizes this concern.

It could be advantageous to have a portion of the purchase price allocated to ``goodwill'' in the four Far East
countries where Mister Donut already has franchisees.

My main concern, though, is with uncertain tax consequences surrounding the transfer of trademarks in the
Peoples Republic of China, Taiwan, Indonesia, Malaysia, Singapore, and Hong Kong. It is possible that
the trademark transfers could generate a tax in these countries. Therefore, if amounts are to be allocated to
the trademarks associated with these countries, the purchase price allocated to them should be as little as
possible. If this is not practical as negotiations continue, I would appreciate it if you could keep me advised
so that I can get some outside professional help with respect to the tax consequences of the trademark sale in
these countries.
\end{quote}
In a memorandum dated September 8, 1988, Mr. Suess provided draft language for a provision allocating the purchase
price between goodwill, trademarks, and petitioner's covenant not to compete. In his memorandum, Mr. Suess
stated:
\begin{quote}In negotiating the allocation it is important to note that the amounts allocated to goodwill and the
noncompete covenant, to the extent upheld upon IRS audit, will be tax-free to Multifoods. The amount allocated to the trademarks and pending trademark applications will be subject to a tax of approximately 38\% in the U.S. and potentially additional taxes in the countries in which such trademarks are registered. Therefore, to the extent that we can maximize the allocation to the goodwill and non--compete covenant, we will maximize Multifoods' after-tax gain on the sale.

You requested that I advise you of the potential tax consequences to Duskin of the purchase price
allocation. As previously discussed, both goodwill and trademarks are generally amortizable for tax purposes
in Japan. Non-compete covenants are also generally amortizable for tax purposes in Japan. Therefore, it is
possible that Duskin may be indifferent to the specific amounts allocated to each type of asset. * * *
\end{quote}
On or about January 27, 1989, petitioner obtained a draft of an appraisal from the Valuation Engineering Associates
Division of Touche Ross (Touche Ross), allocating the sale price among the assets to be sold. Duskin was not involved in
the selection of Touche Ross, nor did it indicate to petitioner its preferred allocation.
On January 31, 1989, Touche Ross submitted its final report, which [allocated the \$2,050,000 purchase, as
follows: Trademarks \$120,000, 6\%; Non-competition \$820,000, 40\%; and Goodwill \$1,110,000, 54\%.]

Article IV, paragraph 3, of the purchase agreement contained the same allocation.

In reporting its foreign and domestic source income for its taxable year ended February 28, 1989, petitioner followed
the allocation contained in article IV of the purchase agreement. After allocating its selling expenses among the goodwill
and trademarks sold to Duskin, petitioner reported \$1,016,643.\footnote[13] {The parties have stipulated that petitioner should have allocated selling expenses of \$97,398 to goodwill, which would have produced income in the amount of \$1,012,602.} of foreign source income from the sale of
goodwill, \$820,000 of foreign source income from the covenant not to compete, and \$109,907 of U.S. source
income from the sale of the trademarks. Petitioner did not allocate any of its selling expenses to the sale of the covenant not to compete. 

\begin{center} \textbf{OPINION}
\end{center}
We must determine what portion, if any, of the gain on petitioner's sale of its Asian and Pacific Mister Donut operations
constitutes foreign source income for purposes of computing petitioner's foreign tax credit limitation under section 904(a).

We begin with the sourcing of income rules under section 865. Section 865(a)(1) provides that income from the
sale of personal property by a U.S. resident \ldots is generally sourced in the United States. Section 865(d) provides that
in the case of any sale of an intangible, the general rule applies only to the extent that the payments in consideration of such sale are not contingent on the productivity, use, or disposition of the intangible. Sec. 865(d)(1)(A). Section
865(d)(2) defines ``intangible'' to mean any patent, copyright, secret process or formula, goodwill, trademark, trade brand,
franchise, or other like property. Section 865(d)(3) carves out a special sourcing rule for goodwill. Payments received
in consideration of the sale of goodwill are treated as received from sources in the country in which the goodwill was
generated.

Petitioner allocated \$1,110,000 of the sale price to goodwill. On brief, petitioner maintains that the franchisor's interest
it conveyed to Duskin consisted exclusively of intangible assets in the nature of goodwill; i.e., franchises, trademarks, and
the Mister Donut System. Petitioner contends that the income attributable to the sale of this goodwill constitutes foreign
source income pursuant to section 865(d)(3).\footnote[15]{On brief, petitioner appears to concede that no goodwill existed with respect to its trademarks in the nonoperating countries, since it had no franchises in those countries or customers who could ``return'' to Mister Donut stores.}

This argument mistakes goodwill for the intangible assets which embody it. Goodwill represents an expectancy that
``old customers will resort to the old place'' of business. The essence of goodwill exists in a preexisting business relationship founded upon a continuous course of dealing that can be expected to continue
indefinitely.  The Supreme Court has explained that ``The value of every intangible asset is related, to a greater or lesser degree,
to the expectation that customers will continue their patronage [i.e., to goodwill].'' An asset does not constitute goodwill, however, simply because it contributes to this expectancy of continued patronage.

%Section 865(d)(1) provides that income from the sale of an intangible asset by a U.S. resident will generally
%be sourced in the United States. Section 865(d)(2) defines ``intangible'' to include, among other things, secret processes
%or formulas, goodwill, trademarks, and franchises. Section 865(d)(3) then provides a special rule for goodwill, sourcing
%it in the country in which it was generated.

Petitioner's argument equates goodwill with the other assets listed in the definition of ``intangible'' in section 865(d)(2). This Court has recognized that intangible assets such as trademarks and franchises are ``inextricably related'' to
goodwill. However, we believe that Congress' enumeration of goodwill in section 865(d)(2) as a separate intangible asset necessarily indicates that the special sourcing rule contained in 865(d)(3) is applicable only where goodwill is separate from the other intangible assets that are specifically listed in section 865(d)(2). If the sourcing provision contained in section 865(d)(3) also extended to the goodwill element embodied in the other intangible assets enumerated in section 865(d)(2), the exception would swallow the rule. Such an interpretation would nullify the general rule that income from the sale of an intangible asset by a U.S. resident is to be sourced in the United States.\footnote[16]{Indeed, in the purchase agreement, petitioner failed to allocate any portion of the sale price to the franchise
agreements. Instead, petitioner allocated \$1,930,000 to goodwill and the covenant not to compete and later reported
this amount as foreign source income on its 1989 Federal income tax return. Petitioner allocated the remaining
\$120,000 of the sale price to the trademarks and reported this amount as U.S. source income on its 1989 return.}

Respondent contends that, although not denominated as such, what Duskin acquired from petitioner was a territorial
franchise for the operating and nonoperating countries. Petitioner, on the other hand, argues that it did not sell Duskin
a franchise, but, rather, the entire Mister Donut franchising business in Asia and the Pacific. Petitioner maintains that
the sale of a franchise requires the franchisor to retain an interest in the business and that petitioner failed to retain the requisite interest in this case following the sale to Duskin. Petitioner contends that section 1253(a) and our opinion in Jefferson-Pilot Corp. v.\@ Commissioner, 98 T.C. 435 (1992), affd. 995 F.2d 530 (4th Cir. 1993), support its interpretation of ``franchise''. 
Although section 865 does not provide a definition of franchise, section 1253(b)(1) defines it for purposes of section
1253(a) to include ``an agreement which gives one of the parties to the agreement the right to distribute, sell, or provide
goods, services, or facilities, within a specified area.'' We have found this definition to be consistent with the common
understanding of the term. When Congress uses a term that has
accumulated a settled meaning under equity or the common law, courts must infer that Congress intended to incorporate
the established meaning of the term, unless the statute otherwise dictates. \ldots

\ldots

Neither the language of section 1253(a) nor our opinion in Jefferson-Pilot supports petitioner's position. Section
1253(a) provides that the transfer of a franchise will not be treated as the sale or exchange of a capital asset so long as
the transferor retains a significant power, right, or continuing interest with respect to the subject matter of the franchise.
The necessary implication is that a franchise can be transferred without the retention by the transferor of
any significant degree of control. In such a case, the transfer will be treated as the sale or exchange of a capital asset, and the transferee will not be permitted to amortize any portion of the purchase price. \ldots

Petitioner's sale of its Mister Donut operations to Duskin constituted the sale of a ``franchise'' for purposes of section
865(d)(2). Petitioner transferred to Duskin its existing franchise agreements, trademarks, and Mister Donut System
in each of the operating countries, as well as its trademarks and Mister Donut System in the nonoperating countries.
Petitioner's Mister Donut operation utilized franchisees to prepare and merchandise distinctive quality doughnuts. This
system included methods of preparation, serving, and merchandising doughnuts. In the purchase agreement, petitioner not
only sold Duskin petitioner's rights as franchisor in the existing franchise agreements in the operating countries, but also all its rights to exclusive use in the designated Asian and Pacific territories of its secret formulas, processes, trademarks, and supplier agreements; i.e., its entire Mister Donut System. Duskin received petitioner's existing rights as franchisor, as well as the right to enter franchise agreements in the nonoperating countries.

Respondent argues that any goodwill associated with the Asian and Pacific franchise business was part of, and inseverable from, the franchisor's rights and trademarks acquired by Duskin. Respondent maintains that any gain attributable to the sale of franchises or the trademarks produces U.S. source income, as section 865 generally sources income in the residence of the seller. See sec. 865(a), (d)(1).

While there are no cases on point under section 865, case law interpreting other provisions of the Code supports
respondent's position. In Canterbury v.\@ Commissioner, 99 T.C. 223 (1992), we considered whether the excess of a
franchisee's purchase price of an existing McDonald's franchise over the value of the franchise's tangible assets was
allocable to the franchise or to goodwill for purposes of amortization pursuant to section 1253(d)(2)(A). We recognized
that McDonald's franchises encompass attributes that have traditionally been viewed as goodwill. The issue, therefore,
was whether these attributes were embodied in the McDonald's franchise, trademarks, and trade name, which would
make their cost amortizable pursuant to section 1253(d)(2)(A), or whether the franchisee acquired intangible assets,
such as goodwill, which were not encompassed by, or otherwise attributable to, the franchise and which were
nonamortizable.

We found that the expectancy of continued patronage which McDonald's enjoys ``is created by and flows from the
implementation of the McDonald's system and association with the McDonald's name and trademark." Id. at 248 (fn. ref.
omitted). In addition, we stated:
\begin{quote}
The right to use the McDonald's system, trade name, and trademarks is the essence of the McDonald's
franchise. * * * Respondent did not identify, and we cannot discern, any quantifiable goodwill that is not
attributable to the franchise. We find that petitioners acquired no goodwill that was separate and apart from
the goodwill inherent in the McDonald's franchise.\\
\textit{The franchise acts as the repository for goodwill} * * * [ Id. at 249; fn. ref. omitted; emphasis added.]
\end{quote}
We concluded that the goodwill produced by the McDonald's system was embodied in, and inseverable from, the
McDonald's franchise that the taxpayer received. \ldots

\ldots

It is also well established that trademarks embody goodwill. Consumers associate the Mister Donut trademark with their pleasurable
experience at Mister Donut shops. As a result, goodwill is also embodied in the trademarks, which Duskin acquired and
which cause customers to return to Mister Donut shops in the future and patronize them.

Petitioner's business in the operating countries was conducted by granting Mister Donut franchises. Under the purchase
agreement, Duskin received petitioner's rights as franchisor under the existing franchise agreements in the operating
countries. The franchisees in the operating countries possessed the exclusive right to open stores pursuant to established
conditions and at locations approved by the franchisor. In order to ensure that the distinguishing characteristics of
Mister Donut were uniformly maintained, the franchise agreements had established standards for furnishings, equipment,
product mixes, and supplies, which the franchisees were required to meet. The franchise agreements also required that
franchisees operate their shops in accordance with uniform standards of quality, preparation, appearance, cleanliness, and
service. The agreements provided that the franchisor could not open, or authorize others to open, any Mister Donut shops
in the franchisee's country until the franchise agreement expired, or was terminated, or unless the franchisee did not meet
its development schedule by failing to open the requisite number of Mister Donut shops.

Mister Donut's success resulted from the Mister Donut System and the high standards for quality and service, which
the franchisees were required to meet. \ldots Although these characteristics produced goodwill in the operating
countries, that goodwill was embodied in the franchises and trademarks conveyed to Duskin.

Petitioner also transferred its Mister Donut System and trademarks for each of the nonoperating countries.
Duskin received the right to exploit--either by entering franchise agreements in these territories or by opening shops
itself--the Mister Donut System along with the accompanying trademarks, formulas, and other intangible assets. In the
nonoperating countries, there were no Mister Donut shops for customers to patronize at the time the purchase agreement
was executed. Goodwill is founded upon a continuous course of dealing that can be expected to continue
indefinitely. Goodwill is the expectancy of continued patronage. Petitioner concedes on brief that
\begin{quote}
in the operating countries where the franchises had been developed, the value to Duskin was in obtaining the
assets which comprised the goodwill. In contrast, there was no value, or negligible value, in the trademarks or trade names in the non--operating countries. * * * \textit{Thus, in the non-operating countries where the franchises
had not been developed, any value acquired by Duskin was merely for the right to do so.} [Emphasis
added.]
\end{quote}
Petitioner has failed to establish that it transferred any goodwill in the nonoperating countries other than what might have
been embodied in its trademarks.

We find that petitioner did not establish that it transferred any goodwill separate and apart from the goodwill inherent
in the franchisor's interest and trademarks that petitioner conveyed to Duskin. Pursuant to section 865(d)(1), income
attributable to the sale of a franchise or a trademark is sourced in the residence of the seller. The income petitioner received
upon the sale of these assets must, therefore, be sourced in the United States.

\begin{center} \textbf{2. Covenant Not To Compete}
\end{center}
The only remaining asset transferred to Duskin that could produce foreign source income is petitioner's covenant
not to compete. Respondent concedes that any amount allocated to the covenant constitutes foreign source income to
petitioner.

Respondent argues that the covenant (like goodwill) was inseverable from the franchisor's interest that petitioner
conveyed to Duskin. Respondent alleges that the franchise rights Duskin acquired provided it with the exclusive
right to use the know-how, trade secrets, trademarks, and other components of the Mister Donut System in the operating
and nonoperating countries. Any competition or disclosure of the Mister Donut System by petitioner in these countries,
respondent contends, would have deprived Duskin of the beneficial enjoyment of the rights it had acquired. Thus,
respondent maintains that petitioner's covenant should be viewed as an inseverable element of the franchisor's
interest acquired by Duskin. We disagree.

The covenant granted Duskin benefits in addition to those necessarily conveyed by petitioner's transfer of its
franchisor's interests and trademarks. The covenant prohibited petitioner from conducting any business similar to the
Mister Donut business in the operating or nonoperating countries or from otherwise selling doughnuts in any of these
countries. Since petitioner possessed expertise, knowledge, and contacts regarding the donut business, it was reasonable
for Duskin to preclude petitioner from reentering the donut business in Asia and the Pacific under a different name. We
conclude that the covenant not to compete possessed independent economic significance, as it did more than
simply preclude petitioner from depriving Duskin of rights which it had acquired in purchasing petitioner's franchise
rights and trademarks. As we stated in Horton v.\@ Commissioner, 13 T.C. 143, 147 (1949) (Court reviewed):
\begin{quote}It is well settled that if, in an agreement of the kind which we have here, the covenant not to compete can
be segregated in order to be assured that a separate item has actually been dealt with, then so much as is paid
for the covenant not to compete is ordinary income and not income from the sale of a capital asset. * * *
\end{quote}

It is necessary, therefore, to determine what portion of the \$2,050,000 sale price must be allocated to the covenant not
to compete. \ldots 
 
Petitioner urges us to uphold the allocation in the purchase agreement of \$820,000. Petitioner relies upon case law
indicating that an allocation in a purchase agreement to a covenant not to compete will be respected for Federal income tax
purposes if it was the intent of the parties to make such an allocation and the covenant possessed independent economic
significance. 

We decline to place reliance upon the allocation contained in the purchase agreement. The cases upholding the
contracting parties' allocation of a specific amount to a covenant not to compete are premised upon the assumption that
the competing tax interests of the parties will ensure that the allocation is the result of arm's-length bargaining. Where
the assumption is unwarranted, there is no reason to be bound to the allocation in the contract. In the instant case, Mr. Suess' memorandum of September 8, 1988, indicates that the interests of Duskin and petitioner were apparently not adverse as to the allocation of the sale price. No representatives from Duskin testified at trial regarding whether Duskin considered the allocation important, and, given Mr. Suess' statements,
we suspect that Duskin was unconcerned. Petitioner, on the other hand, was certainly cognizant of the potential tax
consequences of the allocation, because of the obvious impact on the calculation of petitioner's foreign tax credit, as well
as the possibility that the transfer of petitioner's trademarks to Duskin would generate a tax in several Asian and Pacific
nations.

Petitioner's expert witness, Robert F. Reilly, \ldots valued the covenant at \$620,000,\footnote[22]{Reilly's report contained the following allocation:
Asset Fair Market Value: Non-compete agreement \$620,000; Trade secrets and know-how \$50,000; Trademarks and trade names \$370,000; Existing franchise agreements \$200,000; and Goodwill \$810,000. Total \$2,050,000} almost \$200,000 less than the
amount allocated by petitioner in the purchase agreement with Duskin. Although expert opinions can assist the Court in
evaluating a claim, we are not bound by the opinion of any expert and may reach a decision based on our own
analysis of all the evidence in the record. 

Mr. Reilly computed the value of the covenant not to compete under the comparative business valuation method and a
discounted net cash-flow analysis. Utilizing this comparative approach, Mr. Reilly computed Mister Donut's discounted
net cash-flow under two scenarios. Scenario 1 assumed that the covenant was in place, and petitioner could
not reenter the Asian and Pacific donut market. Scenario 2 assumed that the purchaser did not receive a covenant, and
petitioner would reenter the market and compete. Mr. Reilly attributed the difference in the sum of Mister Donut's
discounted net cash-flows under these two scenarios to the covenant not to compete. Mr. Reilly then added the income
tax benefits of amortization over the covenant's estimated enforceable period of 5 years to determine the portion of the
\$2,050,000 sale price to be allocated to the covenant.

Mr. Reilly performed these calculations twice, once assuming the most likely competition scenario from petitioner in
the event it reentered the Asian and Pacific market, and a second time assuming the worst case competition scenario from
petitioner. \ldots  Mr. Reilly estimated the values of the covenant under the most likely competition scenario and the worst
case competition scenario at \$620,000 and \$630,000, respectively. He then reconciled these differences and arrived at a
final value of \$620,000.

We find two difficulties with Mr. Reilly's report and his calculations. First, we are unsure whether Mr. Reilly's
calculations and valuation of the covenant not to compete erroneously assumed that petitioner could reenter these
Asian and Pacific markets again as ``Mister Donut'', despite the fact that petitioner had conveyed its existing franchise
agreements, trademarks, and Mister Donut System to Duskin in the purchase agreement. For instance, Mr. Reilly testified
at trial that ``The value of the [Duskin's] business would be reduced by \$620,000, due to the most likely competition from
Mister Donut.'' But petitioner had already transferred its rights to Mister Donut in the operating and nonoperating countries. Assuming no covenant existed, and petitioner had chosen to reenter the donut market in these territories, it would have had to do so under a different name. \ldots

Second, Mr. Reilly computed the value of the covenant not to compete under both the most likely and the worst cases
of competition without factoring in the likelihood of petitioner's competition into his calculations. Although Mr. Reilly's
report stated that there existed a less-than-50-percent chance of petitioner's reentering the Asian and Pacific market for
such franchise operations, his calculations ignored the fact that competition was unlikely even without a covenant.

Based on our review of the record, we conclude that \$300,000 of the sale price should be allocated to the covenant
not to compete. Respondent concedes that the amount allocable to the covenant not to compete constitutes foreign source
income for purposes of computing petitioner's foreign tax credit limitation pursuant to section 904(a).


\ldots
\end{select}


\section{Source of Miscellaneous Income}

The source of an item of income that is not specifically addressed in sections 861 and 865 is determined by examining the underlying nature of the income and finding the statutory category in which it fits most closely.  In a settlement or judgment, for example, the nature of the item for which the settlement or judgment is paid determines the character of the item.  In cross-border litigation, it is necessary to be attuned to the nuances of international tax to ensure that any settlement is structured in the most tax-efficient manner possible.  

In PLR 200620016 (May 19, 2006), a nonresident alien (A) operated a sole proprietorship (SP) in Country X that imported from the United States sporting goods for sale abroad.  All of SP's employees  resided and worked abroad, and SP was  never engaged in a U.S. trade or business.  SP was the exclusive distributor of a U.S. corporation's sporting equipment in Country X.  A successor to the U.S. corporation terminated the distribution contract with SP, and SP brought suit.  A received a payment in settlement of his claims for breach of contract  from the former owners of the successor corporation and the  trustee of the bankruptcy estate of the successor corporation.  The IRS ruled that the payments would be foreign source and not subject to withholding:

	\begin{quote}
		With regard to the taxation of a settlement payment made to a nonresident alien individual, the nature of the item for which the settlement payment is substituted controls the characterization of the payment. U.S. v.\@ Gilmore, 372 U.S. 39 (1963). Similarly, the source of the item for which a settlement payment is substituted controls the source of the payment. Rev.\@\@ Rul.\@ 83-177, 1983-2 C.B. 112. In Rev.\@\@ Rul.\@ 83-177, a foreign partnership formed by two nonresident aliens, which was not engaged in a U.S. trade or business, filed suit for breach of contract against a domestic corporation. All of the services to be performed by the foreign partnership pursuant to the agreement were to be performed outside of the United States. Rev.\@\@ Rul.\@ 83-177 holds that the amount paid under the settlement agreement representing principal is foreign source income under section 862(a)(3) and is therefore neither subject to tax under section 871(a) nor withholding under section 1441(a).
		
		For purposes of determining the source of the settlement payment, the amount of principal received pursuant to the settlement agreement depends upon the nature of the item for which the bankruptcy claims settled. The bankruptcy claims settled the alleged wrongful breach of contract under which Sole Proprietorship B was the distributor of Corporation D's sporting equipment in Country X. The purchase of sporting equipment within the United States for sale and use in Country X would constitute foreign source income. I.R.C. \S 862(a)(6).

	When payments are made to Individual A in satisfaction of a breach of contract where the underlying income would be income from sources without the United States under section 862(a)(6), the principal payments made in settlement of that obligation will also be considered payments made from sources without the United States
		
		\end{quote}

The Tax Court recently addressed the vexing issue of the source of guarantee fees, which are fees paid to another party in exchange for that party guaranteeing the debt issued by the paying party.  Guarantee fees are typically paid by a subsidiary to the parent corporation in exchange for the parent's guarantee of the subsidiary's debt.  A guarantee fee lowers the interest rate paid by the borrower.  A guarantee fee may be advantageous if the guarantee fee is less than the increase in the interest rate the subsidiary would have to pay without the guarantee.

Although guarantee fees are ubiquitous, especially in cross-border financings, they raise many tax issues.  Although they arise in lending situations and are paid in connection with borrowing, they are clearly not interest because they are not paid from the borrower to the lender.  Many argued that guarantee fees should be treated as income from services.  In the international context, the distinction is paramount because of the different sourcing rules for interest and services.  Prior to 2010, section 861 did not specifically address the source of guarantee fees.  In \emph{Container Corp. v.\@ CIR}, 134 T.C. 122 (2010), excerpted below, the Tax Court ruled that guarantee fees are analogous to services and should therefore be sourced where the guarantee services are being performed.  The \emph{Container} opinion is very instructive for the analytical method it employs and sources it relies on to arrive at its holding.   


In P.L. 111-240 (Creating Small Business Jobs Acts of 2010), Congress overrode the decision in \emph{Container Corp.} and enacted new section 861(a)(9), which provides the guarantee fee income received directly or indirectly from a domestic corporation or noncorporate resident is U.S. source.  The following excerpt from the Joint Committee on Taxation, Technical Explanation of the Tax Provisions in Senate Amendment 4594 to H.R. 5297, the "Small Business Jobs Act of 2010," describes the scope of the new section.
	\begin{quote}
This provision effects a legislative override of the opinion in Container Corp. v.\@ Commissioner, supra, by 
amending the source rules of section 861 and 862 to address income from guarantees issued after the date 
of enactment. Under new section 861(a)(9), income from sources within the United States includes amounts 
received, whether directly or indirectly, from a noncorporate resident or a domestic corporation for the 
provision of a guarantee of indebtedness of such person. The scope of the provision includes payments that 
are made indirectly for the provision of a guarantee. For example, the provision would treat as income from 
U.S. sources a guarantee fee paid by a foreign bank to a foreign corporation for the foreign corporation's 
guarantee of indebtedness owed to the bank by the foreign corporation's domestic subsidiary, where the 
cost of the guarantee fee is passed on to the domestic subsidiary through, for example, additional interest 
charged on the indebtedness. 

Such U.S.-source income also includes amounts received from a foreign person, whether directly or 
indirectly, for the provision of a guarantee of indebtedness of that foreign person if the payments received 
are connected with income of such person which is effectively connected with conduct of a U.S. trade or 
business. A conforming amendment to section 862 provides that amounts received from a foreign person, 
whether directly or indirectly, for the provision of a guarantee of that person's debt, are treated as foreign 
source income if they are not from sources within the United States as determined under new section 861(a)(9). 

	\end{quote}


\addcontentsline{toc}{section}{\protect\numberline{}Container Corp. v.\@ CIR} 
\begin{select}
\caseart{Container Corp. v.\@ CIR}{ 134 T.C. 122 (2010) }{Holmes, Judge}
\ldots
\begin{center} \textbf{Background}
\end{center}

\textit{[Ed.: Vitro, S.A., a Mexican corporation was engaged in the manufacture of glass products.  In the late 1980's, it decided to expand into the U.S. market by acquiring two U.S. glass container producers.  Vitro had various U.S. marketing and distribution subsidiaries all owned by a U.S. holding company, Vitro International.  Vitro formed an acquisition  company, Container, which in turn, formed a shell corporation, THR Corp, which would hold the shares of the acquired companies.  THR eventually acquired all of the shares of the U.S. target companies, Anchor and Latchford, with the proceeds of a combination of debt and equity from Container and third parties.  Most of the acquisition debt was short term and was expected to be refinanced with junk bonds.  Unfortunately  this proved impossible when Drexel Burnham Lambert filed for bankruptcy.}

\textit{With the impending maturity of one of the Anchor acquisition loans, Vitro refinanced some of Anchor's debt and as part of the refinancing was required to contribute additional capital to THR to repay an acquisition loan and a portion of a bridge financing note issued by THR.  To make the bridge financing indebtedness more marketable, Vitro decided to move the notes outside of the Container group to Vitro International.  On May 2, 1990, the bridge note was restructured by International issuing \$151 of senior notes and loaning those proceeds to THR, which repaid the bridge loan.  As part of the restructuring, Vitro was required to guarantee the International debt.}

\textit{To make the first payment on the International loan, International borrowed \$31 million from Banca Serfin.  Vitro guaranteed this borrowing.  In 1991, International issued additional senior debt that was also guaranteed by Vitro.  The proceeds of this debt was used to retire the 1990 debt and the Banca Serfin loan.}

\textit{As compensation for the guarantee on the 1991 debt, International paid over \$6 million to Vitro from 1992 through 1994.  These amounts were calculated based on a 1.5\% fee applied to the outstanding balance and was rate Vitro charged all of its subsidiaries.  The fees were not tied to the amount of work Vitro did to negotiate or monitor the guaranty. }

\textit{International did not have the cashflow to make the interest payments on the International 1991 notes. To make those payments, Vitro and Container contributed almost \$80 million in capital to International from 1990 to 1994. But the money didn't help.  Anchor filed for bankruptcy in 1997.]}

\ldots

\begin{center} \textbf{Discussion}
\end{center}

The parties agree that the guaranty fees, paid regularly in fixed amounts, are FDAP income. The key question in this case is whether the second requirement is met--was the source of the guaranty fees the United States or Mexico?

We determine FDAP income's source by using the rules in sections 861 to 863. Two rules are especially important here. The first is for interest--the rule is that the source of interest is the residence of the obligor. Secs. 861(a)(1), 862(a)(1); sec. 1.861-2, Income Tax Regs. The Commissioner would like the guaranty fees to be treated as interest, because International  is a U.S. company.

The second rule that's especially important here is the rule on services--that rule is that the source of services is where the services are performed. Sec. 861(a)(3), 862(a)(3); sec. 1.861-4, Income Tax Regs. Container would like the guaranty fees to be treated as payments by International for a service performed by Vitro in Mexico.

The sourcing rules are not comprehensive. If a category of FDAP is not listed, caselaw tells us to proceed by analogy. In other words, if the guaranty fees were neither interest nor payment for services rendered, we would still have to figure out whether they were more like interest or more like payment for services rendered (or, possibly, some other category of FDAP that has a specific sourcing rule).\ldots 

\begin{center}
	\textbf{A. Guaranty Fees as Interest}
\end{center}

Interest is ``compensation for the use or forbearance of money."  \ldots We agree with the parties that Vitro's guaranty was not a loan to International, so the guaranty fees are not interest.

\begin{center}
	\textbf{B. Guaranty Fees as Payment for Services}
		\end{center}

Sections 861(a)(3) and 862(a)(3) specifically source ``labor or personal services," and Container argues that that is what Vitro performed for International. Under the Guaranty agreement, Vitro was required to maintain records and supply information to the note purchasers. It performed these acts using Corporativo personnel, facilities, equipment, and capital--all located in Mexico. Container asks us to find that the guaranty fees were compensation for these services and are therefore Mexican source income. See Commissioner v.\@ Piedras Negras Broad. Co., 127 F.2d 260 (5th Cir. 1942), affg. 43 B.T.A. 297 (1941); Dillin v.\@ Commissioner, 56 T.C. 228, 244 (1971) (explaining that where the benefits of the services are received or where a guaranty agreement was entered into does not affect the source of services).

The Commissioner does not challenge Container's assertion that Corporativo performed services, but argues that services were not the predominant feature of the guaranty and should be ignored for sourcing purposes. See Bank of Am., 230 Ct. Cl. at 690, 680 F.2d at 149. Container responds by arguing that providing services is not a possible feature of a guaranty, but that a guaranty is itself a service; indeed, that the Code and regulations actually refer to guaranties as services.

\ldots Container also asks us to look at transfer pricing of services under section 482.

This might be as a useful guide. Section 482's purpose ``is to ensure that taxpayers clearly reflect income attributable to controlled transactions, and to prevent the avoidance of taxes with respect to such transactions." \ldots. For example, if a U.S. corporation guarantees a loan made to its foreign subsidiary by a third party without receiving compensation from the foreign sub, it could avoid the income it would have incurred had it charged a fee. But the guaranty adds some value, and the section 482 regulations tell taxpayers that the U.S. parent should recognize the amount it would have charged had the transaction been made at arm's length with an uncontrolled third party. \ldots. But this is just a summary of a general rule. When it comes to deciding whether payments for a guaranty are services in particular transfer-pricing situations, the Commissioner has struggled.

In General Counsel Memorandum (GCM) 38499 (Sept. 19, 1980),\footnote[12]{Although GCMs have no precedential value, they are ``helpful in interpreting the Tax Code when `faced with an almost total absence of case law.' "} the Commissioner agreed with a proposed revenue ruling\footnote[13]{The proposed revenue ruling was never published. See Field Service Advice Memoranda, 1995 FSA LEXIS 135 at 16 (May 1, 1995).} concluding that the ``guarantee of the parent constitutes the performance of a service for the subsidiary." The Commissioner used section 1.482--2(b)(7)(v), Example (9), Income Tax Regs., to reach this result.\footnote[14]{At the time of the GCM's release, the section 482 regulations were in final form. In 1993, temporary regulations were issued. \ldots. The final regulations were issued in 1994, but didn't go into effect until tax years beginning after October 6, 1994. T.D. 8552, 1994-2 C.B. 93. Throughout the regulation's final-to-temporary-to-final journey, "Example (9)" remained unchanged. But that example was removed from section 1.482-2 by T.D. 9278, 2006-2 C.B. 256.}

The proposed revenue ruling also concluded that guaranty fees should be sourced to the country where the financing is secured and where the subsidiary resides because that is the situs of the risk of default. In the General Counsel Memorandum, the Commissioner expressed reservations about that conclusion and suspended further consideration. \ldots 15 GCM 38499 (Sept. 19, 1980)

We also have some caselaw. In Centel Commcns. Co. v.\@ Commissioner, 92 T.C. 612 (1989), affd. 920 F.2d 1335 (7th Cir. 1990), we decided that the guaranties were not a service, though in a very different context: A burgeoning telephone interconnect business got a loan to provide it with operating funds. Id. at 616. As a condition of the loan, the lender required guaranties from three of the company's shareholders. Id. The shareholders signed the agreements without compensation, but five years later they received stock warrants for their guaranties. Id. at 617-19. The issue we decided was whether the warrants were given for the performance of services under section 83(a). Id. at 626. We held that ``within the meaning of section 83" the shareholder had not performed a service. Id. at 633.

``[W]ithin the meaning of section 83" is the key. We did characterize the guaranties as "shareholder/investor actions to protect their investment * * * [that] as such do not constitute the performance of services." Id. at 632-33. But we  also stressed that our decision turned on a question of fact: whether the shareholders got the warrants in exchange for services rendered as employees or independent contractors. Id. at 629. The parties agreed the shareholders weren't employees, and we found that they were not independent contractors because they were not in the business of guaranteeing loans. Id. at 632. We did not hold that providing a guaranty is never a service, and noted that we were analyzing only the language of section 83. An analysis under that section is quite different from an analysis under the sourcing rules, but it nevertheless prompted the Commissioner to rethink his position when the problem came up in the transfer-pricing context again. This time he reasoned that

\begin{quote}
The Centel decision increases the litigating hazards * * *. However, we do not read this case as contradicting the position of the Service as established in * * * G.C.M. 38499. Guarantees do not fit comfortably within normal tax law concepts in a number of areas and, consequently, there are substantial arguments that can be made against any possible analysis of guarantees. * * *
\end{quote}
1995 FSA LEXIS 135, 1995 WL 1918236 (IRS FSA May 1, 1995).

All we can conclude from this detour through transferpricing law is that it will not help us reach a reasonable conclusion on whether guaranties are services under section 861.

So we'll fall back on the dictionary. The common meaning of ``labor or personal services'' implies the continuous use of human capital, ``as opposed to the salable product of the person's skill.'' \ldots Under this definition, we find that Container failed to prove that Corporativo performed sufficient ``labor or personal services" to justify the \$6 million International paid in guaranty fees over three years. Container presented very little evidence about the specific acts Corporativo performed and how much time it took to perform them. For example, Container's posttrial brief explains that the Guaranty agreement required Vitro to ``take certain actions, confirm certain facts, provide certain information, and create and supply certain documents." The Guaranty agreement required only minimal accountings and reporting to the note purchasers. In any event, the fees were not tied to the amount of work that Vitro did, but to the amount of the outstanding principal that Vitro was standing behind. This leads us to hold that International did not pay the guaranty fees to Vitro as compensation for services. The value of Vitro's guaranty stems ``from a promise made and not from an intellectual or manual skill applied." Bank of Am., 47 AFTR 2d at 81-657.


We therefore move on to reasoning by analogy, and ask whether guaranty payments are more like interest or more like services.

\begin{center}
	\textbf{C. Guaranty Fees as Analogous to Interest or Payments for Services}
		\end{center}

When we source FDAP income by analogy, our goal is to find the ``source of income in terms of the business activities generating the income or * * * the place where the income was produced. Thus, the sourcing concept is concerned with the earning point of income or, more specifically, identifying when and where profits are earned." Hunt, 90 T.C. at 1301 (citation omitted).

There are only a few examples in the caselaw of sourcing by analogy. Alimony was the first. The question of its source arose when a U.S. resident paid alimony to his British ex from an English bank. We held that the alimony's source was the ex-husband's residence, and not where the funds were deposited or where the divorce decree was entered. See Manning v.\@ Commissioner, 614 F.2d 815 (1st Cir. 1980), affg. T.C. Memo. 1979-146; Howkins, 49 T.C. at 694. Taking perhaps too modern a view of marriage, we reasoned that alimony, like interest, is not exchanged for property or services. And since interest is sourced to the residence of the obligor, so too would we source alimony. Howkins, 49 T.C. at 694.

Another example of sourcing by analogy came from the Court of Claims in Bank of America. In that case, the court sourced commissions received by Bank of America from foreign banks in connection with transactions involving commercial letters of credit. Bank of Am., 230 Ct. Cl. at 680-681, 680 F.2d at 143. The conflict in Bank of America, as in this case, was whether the commissions should be sourced by analogy to personal services or to interest. Id. at 686-687, 680 F.2d at 147.

To understand the holding in Bank of America requires some background in letters of credit. Such letters make trade easier by allowing a bank, rather than the seller, to examine a buyer's credit.   For example, when a U.S. exporter wants to sell goods to a foreign buyer, assessing the creditworthiness of the foreign buyer can be a problem. So, instead of having the seller do it, the buyer requests a letter of credit from a foreign bank and the foreign bank does the job. If the buyer is creditworthy, the foreign bank (sometimes called the opening bank) substitutes its credit for the buyer's and commits to pay the seller when certain conditions are met, e.g., presentment of an inspection certificate and a bill of lading to the opening bank. After the opening bank pays the seller, the buyer reimburses it. There are two types of commercial letters of credit: sight and time. A sight letter of credit obligates the opening bank to pay as soon as the seller meets the conditions in the letter of credit. A time letter of credit obligates the opening bank to pay on a specific future date if the conditions were met. See id. at 681, 680 F.2d at 144.

BofA performed four kinds of transactions involving letters of credit, and charged the opening bank commissions for three of them. It's these three, and how the Court of Claims sourced each of them that are useful here. The first kind was an acceptance, and BofA received acceptance commissions in two situations--if BofA determined that the conditions of a time letter of credit had been met it would stamp the letter accepted, obligating itself to pay any holder in due course when the letter came due; or, if an opening bank with an established line of credit with BofA wanted to refinance a letter of credit, it would accept a time draft at a discount to the face amount of the letter of credit.

The Court of Claims began its analysis by noting that both these types of acceptance transactions are similar to a loan and that the commissions ``include elements covered by the interest charges made on direct loans." Id. at 689, 680 F.2d at 148. The court also held that the predominant feature of an acceptance transaction was the substitution of BofA's credit for that of the opening bank and not the services BofA performed. Id. at 690, 680 F.2d at 149. These factors led the  Court of Claims to source acceptance commissions by analogy to interest, with the obligor being the opening bank. Id. at 689, 680 F.2d at 148.

BofA also received confirmation commissions. It confirmed sight letters of credit by advising the letter and committing to pay the letter's face amount after the seller met its conditions. The opening bank reimbursed BofA by either prepaying it or by keeping an account that BofA could debit. When the opening bank prepaid, BofA didn't charge a commission. Otherwise it charged a commission that reflected its assumption of the risk that the foreign bank could default. The Court of Claims again found that the performance of services was a part of the deal but that its predominant feature was BofA's substituting its credit for the opening bank's. Id. at 691, 680 F.2d at 149-50. The court also thus sourced confirmation commissions, as it had acceptance commissions, by analogy to interest and with the obligor being the opening bank. Id. at 691-92, 680 F.2d at 150.

Finally, the Court of Claims examined negotiation commissions. Negotiations took place when BofA determined if the seller met the conditions for payment in the letter of credit. After BofA performed a negotiation, it would forward the papers to the opening  bank, which would do an independent check. The Court of Claims found that negotiation commissions were paid for services performed in the United States and were distinguishable from the other two types of commission because the only risk that BofA assumed was that it might improperly determine that the seller met the conditions. Id. at 692, 680 F.2d at 150.

The Commissioner argues that Bank of America is controlling because acceptance and confirmation commissions, like guaranty fees, are uses of another's credit and are analogous to interest. But, as the Commissioner thoughtfully concedes, the ``use" of credit is different in guaranties compared to acceptance and confirmation of letters of credit. When BofA confirmed or accepted a letter of credit, it assumed an unqualified primary legal obligation to pay the seller--it stepped into the shoes of the opening bank and substituted its own credit for the opening bank's. It was, in effect, making a short-term loan and the commissions approximated interest. Id. at 688-91, 680 F.2d at 148-50.

Vitro's case is different. It was augmenting International's credit, not substituting its own. But should this distinction matter? We conclude that it  should, and begin our explanation by examining the effects of a default. When a debtor defaults on a loan, he is defaulting on an existing primary obligation. Default causes the creditor to lose the outstanding principal because he has already extended funds to the debtor. Interest is the creditor's compensation for putting his own money at risk. As in a loan, BofA put its money directly at risk when it paid the seller, and it charged for the risk-although it called that charge a ``commission" rather then ``interest". Vitro's obligation was, in contrast, entirely secondary. Unlike a lender, Vitro was not required to pay out any of its own money unless and until International defaulted. And Vitro's guaranty might not even put its money at risk after default, because if International de-faulted and Vitro paid the 1991 International senior notes, it would step into the note purchasers' shoes and acquire any rights that they had against International. \ldots Vitro loses only if International defaults and Vitro repays the 1991 International senior notes (which transfers International's obligation from the note purchasers to Vitro) and then International defaults on the transferred debt.

Vitro's guaranty therefore lacks a principal characteristic of a loan because Vitro did not extend funds to International. To find otherwise would require us to assume that at the time of the guaranty, the 1991 International senior notes was somehow a loan to Vitro. Neither party makes this argument. \ldots Vitro's later choice to subsidize International through capital contributions--instead of allowing International to default--does not affect our analysis. Capital contributions also lack a distinguishing characteristic of a loan--a promise to repay.

The Commissioner argues, however, that if guaranties are unlike loans because the guarantor does not have to hand over his money at the outset, guaranty fees may be like interest in some broader sense under Howkins. That case, the Commissioner argues, held that alimony is analogous to interest because it is not paid for property or services. Howkins, 49 T.C. at 694. Reading Howkins this way, however, is reading it less as a useful analogy than as creating a default rule. Property and services are listed in sections 861 and 862, so by definition, any unlisted type of income is not paid for property or services. And if we were to follow such reasoning without qualification, we would source all unlisted types of income by analogy to interest. But we read Howkins more narrowly; we reasoned there that alimony is analogous to interest because its source is the obligor. Howkins, 49 T.C. at 693. This logic also reminds us of the goal of sourcing by analogy: namely, find the location "of the business activities generating the income or * * * the place where the income was produced." Hunt, 90 T.C. at 1301. So we have to ask if there's a useful analogy to guaranty fees that would help us figure out, in some reasonable way, where they are produced.


International paid Vitro to guarantee the 1991 International senior notes. These fees compensated Vitro for incurring a contingent future obligation to either pay International's debt or make a capital contribution. Vitro was able to make this promise because it had sufficient Mexican assets--and its Mexican corporate management had a sufficient reputation for using those assets productively--to augment International's credit and enable the long and complex series of financings we charted at the beginning of this opinion to keep going as long as it did. So we conclude that it is Vitro's promise and its Mexican assets that produced the guaranty fees.\footnote[19]{The parties did not argue the point, but in this sense the guaranty fees were somewhat analogous to rents or royalties for the use of Vitro's goodwill, see sec. 862(a)(4), which would also source them to Mexico rather than the United States.}

We do not choose International as the source of the income because the guaranty fees were not like alimony: Alimony is only an obligation to pay, because once a court orders one spouse to pay alimony, nothing more is required of the other spouse. Guaranty fees are different--they are payments for a possible future action.

We think that makes guaranties more analogous to services. Guaranties, like services, are produced by the obligee and so, like services, should be sourced to the location of the obligee. \ldots  We realize that we are deciding a close question, but an analogy to interest has too many shortcomings: Guaranty fees do not approximate the interest on a loan; Vitro, not International, produced the guaranty fees; and Vitro's guaranty was not an obligation to pay immediately, but a promise to possibly perform a future act.

\begin{center}
\textbf{Conclusion} 
\end{center}
We hold that International was not required to withhold taxes on the guaranty fees that it paid Vitro because those fees are Mexican source income. \ldots

\end{select}
		
		
%Section 862(a)(6) of the Code provides that gains, profits, and income derived from the purchase of inventory property (within the meaning of section 865(i)(1)) within the United States and its sale or exchange without the United States shall be treated as an income from sources without the United States.

%For purposes of determining the source of the settlement payment, the amount of principal received pursuant to the settlement agreement depends upon the nature of the item for which the bankruptcy claims settled. The bankruptcy claims settled the alleged wrongful breach of contract under which Sole Proprietorship B was the distributor of Corporation D's sporting equipment in Country X. The purchase of sporting equipment within the United States for sale and use in Country X would constitute foreign source income. I.R.C. \S 862(a)(6).

%When payments are made to Individual A in satisfaction of a breach of contract where the underlying income would be income from sources without the United States under section 862(a)(6), the principal payments made in settlement of that obligation will also be considered payments made from sources without the United States.


%\addcontentsline{toc}{section}{\protect\numberline{}PLR 200620016}
%\begin{select}
%\revrul{PLR 200620016}{May 19, 2006}
%\ldots
%\begin{center}\textbf{FACTS}
%\end{center}
%Individual A is a nonresident alien, who operates and owns Sole Proprietorship B in Country X. Sole Proprietorship B imports sporting goods for sale and use solely in Country X. All of Sole Proprietorship B's employees reside in Country X. Taxpayer represents that Sole Proprietorship B has never engaged in a U.S. trade or business.
%Corporation C, a domestic corporation, entered into a contract with Sole Proprietorship B under which Sole Proprietorship B was the exclusive distributor of Corporation C's sporting equipment in Country X. Prior to the termination date of the contract, Corporation C and Sole Proprietorship B extended the contract for an additional four years. During the course of the contract period Corporation C's assets were acquired by Corporation D, a domestic corporation.
%Corporation D continued to supply Sole Proprietorship B with sporting equipment. Corporation D entered into an arrangement for the manufacture of sporting equipment for sale to Sole Proprietorship B.

%On Date 1, Corporation D terminated the distributorship agreement and stopped supplying sporting equipment to Sole Proprietorship B. Sole Proprietorship B filed suit in an *** State District Court against Corporation C and Corporation D, alleging that the termination of the distributorship agreement was wrongful. Following initiation of the suit, Sole Proprietorship B dismissed the case pursuant to Rule 41(A)(1)(a) of the *** Rules of Civil Procedure. A later suit was initiated by Sole Proprietorship B in a U.S. District Court against Corporation C and Corporation D.
%Subsequent to the U.S. District Court case, Corporation D filed a voluntary petition for relief under chapter 7 of the Bankruptcy Code. Individual A filed a claim in the bankruptcy case for breach of contract and incorporated by reference all the claims and causes of action asserted in his *** State Court and U.S. District Court cases.
%Under a settlement agreement entered into by all the parties involved in the bankruptcy case, Individual A will receive a cash settlement from the former owners of Corporation D and Trustee E, the trustee of the bankruptcy estate of Corporation D in complete satisfaction of their claims.

%Amounts owed to Individual A under the bankruptcy settlement agreement will be paid into Taxpayer's trust account. Taxpayer is the attorney representing Individual A in the bankruptcy settlement.

%\begin{center}\textbf{RULING REQUESTED}
%\end{center}
%Taxpayer is requesting a ruling that the payments to be distributed by Taxpayer to Individual A pursuant to the bankruptcy settlement agreement constitutes income from without the United States under section 862(a)(6) and that no withholding is required under section 1441.

%\begin{center}\textbf{LAW AND ANALYSIS}
%\end{center}
%\ldots
%%Section 871 of the Code generally imposes a tax of 30 percent on the amount received by a nonresident alien individual from sources within the United States as interest, dividends, rents, salaries, wages, premiums, annuities, compensations, remunerations, emoluments, and other fixed or determinable annual or periodical gains, profits, and income, but only to the extent the amount so received is not effectively connected with the conduct of a trade or business within the United States.

%%Section 1441(a) of the Code provides, in general, for a withholding of tax at a 30 percent rate on certain income from sources within the United States of a nonresident alien individual.

%With regard to the taxation of a settlement payment made to a nonresident alien individual, the nature of the item for which the settlement payment is substituted controls the characterization of the payment. U.S. v.\@ Gilmore, 372 U.S. 39 (1963). Similarly, the source of the item for which a settlement payment is substituted controls the source of the payment. Rev.\@\@ Rul.\@ 83-177, 1983-2 C.B. 112. In Rev.\@\@ Rul.\@ 83-177, a foreign partnership formed by two nonresident aliens, which was not engaged in a U.S. trade or business, filed suit for breach of contract against a domestic corporation. All of the services to be performed by the foreign partnership pursuant to the agreement were to be performed outside of the United States. Rev.\@\@ Rul.\@ 83-177 holds that the amount paid under the settlement agreement representing principal is foreign source income under section 862(a)(3) and is therefore neither subject to tax under section 871(a) nor withholding under section 1441(a).
%Section 862(a)(6) of the Code provides that gains, profits, and income derived from the purchase of inventory property (within the meaning of section 865(i)(1)) within the United States and its sale or exchange without the United States shall be treated as an income from sources without the United States.

%For purposes of determining the source of the settlement payment, the amount of principal received pursuant to the settlement agreement depends upon the nature of the item for which the bankruptcy claims settled. The bankruptcy claims settled the alleged wrongful breach of contract under which Sole Proprietorship B was the distributor of Corporation D's sporting equipment in Country X. The purchase of sporting equipment within the United States for sale and use in Country X would constitute foreign source income. I.R.C. \S 862(a)(6).

%When payments are made to Individual A in satisfaction of a breach of contract where the underlying income would be income from sources without the United States under section 862(a)(6), the principal payments made in settlement of that obligation will also be considered payments made from sources without the United States.

%Accordingly, based solely on the facts submitted and the representations made, we conclude that payments of principal received by Individual A under the settlement agreement are foreign source income under section 862(a)(6) and, therefore, are neither subject to tax under section 871(a) nor to withholding at source under section 1441(a). The interest portion of the payment must be sourced according to the source of interest rules.
%\ldots
%\end{select}

In 1994, the Treasury issued proposed regulations addressing the source of income from computer software.  These important regulations were finalized in 1998.  The issue the drafters of the regulations had to grapple with was whether a sale of computer software constituted a transfer of a copyright right (either sale or license) or the transfer (sale or lease) of a copyrighted article.  The regulations incorporate aspects of copyright law in making this determination.  Excerpted below is the preamble to the proposed regulations that explains the provisions. 

\addcontentsline{toc}{section}{\protect\numberline{}Preamble to Prop. Reg.\@ 1.861-18}
\begin{select}
\revrul{Preamble to the Proposed Regulations Addressing Income from Computer Software}{1994-1 C.B. 173}
\ldots

\begin{center} \textbf{I. INTRODUCTION}
\end{center}
Computer programs are generally protected by copyright law. Typically the protection afforded by copyright law is a principal source of the value of a computer program to the owner of the copyright. Conversely, the principal source of the value of a computer program to the purchaser of a copy of the program is not the protection afforded by copyright law, but the right to use or sell the copy. In this regard, computer programs are similar to other copyrighted works such as books, records, motion pictures, etc. For example, when a copy of a book is purchased, the purchaser does not thereby also acquire any copyright rights. Accordingly, the proposed regulations generally distinguish between transactions in a copyright and in the subject of the copyright.

In developing regulations addressing the treatment of computer programs, the IRS and Treasury generally have been guided by the following principles: (i) the rules should take into account the special features of computer programs, such as the ability to deliver copies electronically as well as physically, and to make perfect copies at little or no cost, and (ii) wherever possible, transactions that are functionally equivalent should be treated similarly. For example, a transaction that involves the transfer for internal use only of fifty copies of a computer program should generally be treated the same as a transfer of one copy (for internal use) with the right to make forty-nine other copies all for internal use. Similarly, if the right to use a computer program is limited in time, the transaction should generally be treated the same irrespective of whether, at the end of the period of permitted use, a disk containing the computer program must be returned, or the program automatically deactivates itself.

\begin{center} \textbf{II. COPYRIGHT LAW PRINCIPLES}
\end{center}

Distinguishing between transactions in a copyright and in the subject of the copyright requires an examination of U.S. and foreign copyright law (e.g. EC Directive on Legal Protection of Computer Programs, 1991 (91/250/EEC); and the Berne Convention (Paris Text, July 24, 1971)). An overview of U.S. copyright law as it relates to computer programs is set forth below. However, the IRS and the Treasury do not purport in these regulations to interpret U.S. copyright law and these proposed regulations should not be taken as an expression of the legal or policy views of the U.S. Copyright Office.

The Copyright Act of 1976, as amended (17 U.S.C. 101 et seq.), provides protection against infringement of the exclusive rights of the owner of a copyright in original works of authorship, fixed in any tangible medium of expression, including literary works. (17 U.S.C. 102.) The term LITERARY WORKS is defined to include: ``. . . numbers, or other verbal or numerical symbols or indicia, regardless of the nature of the material objects, such as books, periodicals, manuscripts, phonorecords, film, tapes, disks, or cards, in which they are embodied." (17 U.S.C. 101.) Thus, computer programs are literary works for purposes of the Copyright Act.

The Copyright Act grants five exclusive rights to a copyright owner. Of these, three are most relevant in the case of computer programs: the right to reproduce copies of the copyrighted work (17 U.S.C. 106(1)); the right to prepare derivative works, which may themselves be separately copyrighted, based upon the copyrighted work (17 U.S.C. 103 and 106(2)); and the right to distribute copies of the copyrighted work to the public by sale or other transfer of ownership, or by rental, lease or lending (17 U.S.C. 106(3)). Additionally, in certain circumstances, the right to publicly perform the copyrighted work (17 U.S.C. 106(4)) and the right to publicly display the copyrighted work may also be relevant (17 U.S.C. 106(5)).

Thus, under U.S. copyright law, the user of a computer program who does not possess any of those five rights (or parts of them) has obtained only rights to use the copyrighted article it possesses. Generally, that user is treated only as having received a copy of the copyrighted work. Under U.S. copyright law, a copy is a material object in which a work is fixed by any method now known or later developed, and from which the work can be perceived, reproduced, or otherwise communicated, either directly or with the aid of a machine or device (17 U.S.C. 101.). In these proposed regulations a copy is also referred to as a ``copyrighted article." The distinction between copies and copyrights is made most clearly in section 202 of the Copyright Act which provides:

\begin{quote}
Ownership of a copyright, or of any of the exclusive rights
under a copyright, is distinct from ownership of any material
object in which the work is embodied. Transfer of ownership of
any material object, including the copy or phonorecord in which
the work is first fixed, does not of itself convey any rights in
the copyrighted work embodied in the object; nor, in the absence
of an agreement, does transfer of ownership of a copyright or of
any exclusive rights under a copyright convey property rights in
any material object.
\end{quote}

Certain rights pass to the purchaser of a copy of a computer program. The most important of these is the right to sell (but not, without permission, to lease, rent or lend) the copy to another person. (17 U.S.C. 109.) Additionally, the owner of a copy of a computer program has the right to make a copy of that copy as an essential step in the utilization of the program (e.g., copying to the memory of the computer) and may also make a copy for archival purposes. (17 U.S.C. 117.) If, however, the owner of the copy sells that copy, all copies made pursuant to the 17 U.S.C. 117 right must be destroyed.

\begin{center} \textbf{III. THE PROPOSED REGULATIONS AND COPYRIGHT LAW PRINCIPLES}
\end{center}

Although the proposed regulations are guided by copyright law principles in determining whether a copyright right or copyrighted article has been transferred, the regulations depart in some cases from a strict reliance on copyright law in order to take into account the special nature of computer programs and to treat functionally equivalent transactions in the same way. For example, the proposed regulations do not treat the transfer of a right to copy as the transfer of a copyright right, unless it is accompanied by the right to distribute the copies to the public.

Thus, where a corporation obtains the right, under an agreement, to make fifty copies of a program for use by its employees at one location (a site license) the transaction is not, for all practical purposes, any different from a transaction in which fifty individual disks are purchased. Accordingly, the proposed regulations treat the transaction as the transfer of a copyrighted article, rather than of a copyright right, despite a copyright law requirement that the corporation receive a ``license" to make those fifty copies. Similarly, under the proposed regulations, the transfer of a computer program in perpetuity for internal use only on a single disk or set of disks in return for a one-time payment, in a transaction styled as a license of copyright rights (a so-called shrink wrap license), is treated as the sale of a copyrighted article and not the transfer of a copyright right. Therefore, such a transfer is classified solely as the sale of a copyrighted article for the purposes of the proposed regulations.

\begin{center} \textbf{IV. EXPLANATION OF PROVISIONS}
\end{center}
Section 1.861-18(a)(1) of the proposed regulations describes the scope of the proposed regulations. These proposed regulations provide rules for classifying transfers of computer programs for the purposes of subchapter N of chapter 1 of the Internal Revenue Code, sections 367, 404A, 482, 551, 679, 1057, 1059A, chapter 3, chapter 5, sections 842 and 845 (to the extent involving a foreign person), and transfers to foreign trusts not covered by section 679.

Section 1.861-18(a)(2) describes the categories of transactions relating to computer programs. In particular, a transfer of a copyright right may be either a sale or license of that right and a transfer of a copyrighted article may be either a sale or lease of that copyrighted article. Section 1.861-18(a)(3) defines the term COMPUTER PROGRAM.

Section 1.861-18(b)(1) provides that a transaction involving the transfer of a computer program will be classified as either the transfer of a copyright right, the transfer of a copyrighted article, the provision of services relating to the development of a computer program, or the provision of know-how.

Section 1.861-18(b)(2) provides that a transaction involving computer programs which consists of more than one of the categories in paragraph (b)(1), is treated as separate transactions. Any resulting transaction that is de minimis, however, taking into account all facts and circumstances, will not be treated as a separate transaction.

Section 1.861-18(c)(1)(i) provides that the transfer of a computer program will be classified as the transfer of a copyright right if the transferee acquires one or more of the rights set forth in paragraph (c)(2).


Section 1.861-18(c)(1)(ii) provides that if such rights are not transferred and the transaction does not involve, or involves to only a de minimis extent, the provision of services or know-how, then the transaction will be classified solely as the transfer of a copyrighted article.

Section 1.861-18(c)(2) identifies those rights that will be treated as copyright rights for purposes of the proposed regulations. This list differs from the list of rights set out in the Copyright Act to take into account the special nature of computer programs. Specifically, the copyright law right to copy will only be treated as a copyright right for the purposes of the proposed regulations if it is accompanied by the right to distribute such copies to the public. The copyright rights that apply for purposes of this section are, in addition to the right to copy and distribute to the public, the right to prepare derivative computer programs, the right to make a public performance of the computer program, and the right to publicly display the computer program. The list of rights contained in section 1.861-18(c)(2) rather than those contained in the Copyright Act will apply for the purposes of the proposed regulations.

Section 1.861-18(c)(3) defines a copyrighted article as a copy of a computer program from which the work can be perceived, reproduced or otherwise communicated.

Section 1.861-18(d) of the proposed regulations provides rules for determining whether a transaction involving a newly- developed or modified computer program will be treated as the provision of services or another transaction described in paragraph (b)(1) of this section. The determination is based on all facts and circumstances, including how risk of loss is allocated and the intent of the parties as to ownership of the copyright. See, e.g., Boulez v.\@ Commissioner, 83 T.C. 584 (1984); Rev.\@\@ Rul.\@ 74-555 (1974-2 C.B. 202); Rev.\@\@ Rul.\@ 84-78 (1984-1 C.B. 173).

Section 1.861-18(e) provides rules for determining whether a transfer of information related to a computer program will be considered the provision of know-how. A provision of know-how will not be considered to occur unless a party transfers information that (i) relates to computer programming techniques, (ii) is not capable of being copyrighted, and (iii) is protected by trade secret protection.

Under section 1.861-18(f)(1), if a transfer involves copyright rights, it will be further classified as either a sale or a license of copyright rights. This classification will be made by examining whether, taking into account all facts and circumstances, all substantial rights, under the principles of sections 1222 and 1235, have passed to the transferee.

Under section 1.861-18(f)(2), if a transfer involves a copyrighted article, it will be further classified as either a sale or a lease of a copyrighted article. This classification will be made by examining whether the benefits and burdens of ownership have passed to the transferee. See, e.g., Grodt \& McKay Realty, Inc. v.\@ Commissioner, 77 T.C. 1221, 1237-38 (1981); Torres v.\@ Commissioner, 88 T.C. 702, 720- 27 (1987); Estate of Thomas v.\@ Commissioner, 84 T.C. 412, 431-40 (1985).

Under section 1.861-18(f)(3), the determination of the classification of a transfer involving a copyright right or copyrighted article must appropriately consider the special nature of computer programs in transactions that take advantage of those characteristics. For example, a transaction in which a person acquires a copyrighted article on disk subject to a requirement that the disk be destroyed after a specified period is generally the equivalent of a requirement that the disk be returned after such period. Similarly, a transaction in which the program deactivates itself after a specified period may also be treated as the equivalent of returning the copy.

Section 1.861-18(g) of the proposed regulations provides certain additional rules of operation. Section 1.861-18(g)(1) provides that neither the form adopted by the parties to a transaction nor the classification of a transaction under copyright law are determinative for tax purposes. Therefore, as illustrated in Example 1, a transfer of a computer program on a disk subject to a shrink-wrap license will generally be a sale of a copyrighted article.

Section 1.861-18(g)(2) provides that the method of transferring the computer program, for example by disk or electronically, shall not be relevant in determining whether a copyright right or a copyrighted article has been transferred.
\end{select}

 \addcontentsline{toc}{section}{\protect\numberline{}Sale of Property Problems} 
	\begin{center}
		\textbf{Sale of Property Problems}
	\end{center}
	\begin{select}
	
	\begin{enumerate}
	
		\item USCo is a U.S. corporation that purchases trendy products from unrelated U.S. parties and exports them.  It purchases \$100 of trendy lunch boxes and sells them to a French importer with title passing to the French importer upon delivery to a common carrier in the United States.  What is the source of any gain realized by USCo?  [Sections 861(a)(6) and 865(b); Reg.\@ \S 1.861-7(c)]
		\item UKP, a UK citizen, realizes gain on a sale of stock in a Delaware corporation that does business only in the United States.  The stock is sold on the New York stock exchange. [Section 865(a) and (g)]
			\begin{enumerate}
				\item UKP has never been physically present in the United States.
				\item UKP is present in the United States for a continuous period of 185 days during the taxable year, but is not present in this country during any prior year or during the following year.  Assume alternatively:
					\begin{enumerate}
						\item The stock is sold while UKP is present in the United States.
						\item The stock sale occurs during the portion of the year when UKP is in the UK.
					\end{enumerate}
				\end{enumerate}
				
		\item Virgin, PLC, sells all of the stock of its U.S. subsidiary for a gain of \$1 billion.  
				\begin{enumerate}
					\item  What is the source of the gain?
					\item Is there any U.S. tax?
				\end{enumerate}
							
		\item USP, a US citizen and resident, sells at a loss stock of a UK corporation, which has never paid a dividend. [Reg.\@ \S 1.865-2(a)]  Why does USP care about the source of a loss?  [Section 904(a)]
		
		\item C is a student from the UK doing a masters program at Fordham Law School.  She is present in the US for all of 2008.  She sells stock of IBM in 2008 at a gain.  
			\begin{enumerate}  
				\item What is the source of the gain?
				\item Is there any U.S. tax due?	[Section 871(a)(2)]
			\end{enumerate} 
	
	\item BritCo, a UK corporation, transfers all of the U.S. rights in perpetuity to USCO, a U.S. corporation, of its soccer photo archives in exchange for a fixed payment of \$1 million and yearly payment of 10\% of the income earned by USCO from the archives for the next 10 years.
		\begin{enumerate}  
			\item Is the transaction a sale or license?
			\item What is the source of the fixed and variable payments? [Sections 865(d)]
			
			\item How are the income and gains taxed? [Section 871(a)(1)(D)]
			
			\item Does the Treaty change your answers?
		\end{enumerate}  
	
		\item Software (``SW''), a U.K. corporation, owns a hot new program for kids to use in managing their SpaceBook pages.  SW transfers the program onto a disk, which is covered by annoying plastic with tiny print on it (known as a shrink-wrap license).  The tiny print allows the holder of the disk to make some copies for personal use and to use the program.  The holder can sell the program, but only if he destroys old copies of the program.  Justine, a US citizen and resident, buys the program.  (Assume that the property is inventory in the hands of SW.)
			\begin{enumerate}
				\item Is the transfer a sale or lease of a copyrighted article, or license or sale of a copyright?
				\item Would it make a difference if Justine had downloaded the program from SW's home page?	
				\item What is the source of the gain in (a) \& (b) if the Justine buys the program in the U.K. or downloads it from the U.S?
				\item How would your answers change if the program lasted only one month after which time the U.S. person would have to pay another fee to use it for an additional month?  [Reg.\@ 1.861-18, Exs. 1 and 2]
				\end{enumerate}
				
					\item DC, a US corporation, develops software in the United States and burns it to CDs in its U.S. plant.  DC's production assets in the U.S. have a value of \$15 million and an average adjusted basis \$5 million.  DC sells the software through its Mexican branch sales office to retail customers at a unit price of \$100.  The value of the property used solely in the selling process in Mexico is \$5 million (average adjusted basis \$1 million).  DC also sells the same software to unrelated distributors (title passes in Mexico) at a unit price of \$70.  DC has gross sales in the US of \$20 million and in Mexico of \$80 million.  What is the source of DC's income?  Assume a cost of goods sold of zero. [863(b)(2) and Reg.\@ \S 1.863-3.]    
	
	\item Minnie, a UK citizen and resident, entered the U.S. on an F visa to study in New York and is present in the U.S. for the entire year.  Assume Minnie receives a scholarship from a UK foundation.  What is the source of the scholarship?  How would Minnie be taxed by the U.S.? [Regs. \S 1.863-1(d); Treaty, Article 20.]

	\item A, a US citizen and resident, wins a Nobel prize (medal plus \$1MM).  What's the source of the income?  [Regs. \S 1.863-1(d)]

			\end{enumerate}	
	\end{select}



\begin{framed}
Last modified: Sept 9, '15; source\_SaleProperty\_Sept9\_15
\end{framed}