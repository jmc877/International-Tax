\chapter{Passive Foreign Investment Companies}

	\section{Introduction}
	
	Prior to 1986, a U.S. person could transfer assets, such as cash, to a foreign corporation and avoid current U.S. taxation on the foreign corporation's earnings if the foreign corporation was not a foreign personal holding company or a controlled foreign corporation.  This could be accomplished by structuring the ownership of the foreign corporation so that: (1) it was not owned by five or fewer persons (FPHC rule); and (2) U.S. shareholders did not own more than 50\% of the vote or value (CFC rule).  Having 11 unrelated, equal U.S. shareholders, for example, was sufficient to avoid both regimes.  The benefits to investing in an offshore corporation was deferral of U.S. tax on the yearly earnings and potential conversion of ordinary income into capital gains when the shares of the foreign corporation were sold.  In addition, if the shares of the foreign corporation were bequeathed to U.S. heirs, because the basis would be stepped up at death, any pre-death appreciation would disappear.  
	
	In the late 1970's and early 1980's, there was a significant increase in the marketing of such offshore investment vehicles to wealthy Americans.  Congress responded by enacting in 1986 the passive foreign investment company (PFIC) provisions (\S\S 1291-1298).  Briefly, a PFIC is any foreign corporation that earns a significant amount of passive income or that owns a significant percentage of assets that produce passive income.  
	
	Once a foreign corporation is a PFIC, a U.S. shareholder, regardless of his ownership percentage, is subject to one of three tax regimes.  First, if a qualified electing fund (QEF) election is made, the U.S. shareholder is taxed currently on the PFIC's ordinary income and net capital gain for the years that the foreign corporation is a PFIC.  If the PFIC is not a QEF for a shareholder's entire holding period, the shareholder will be taxed when he receives a distribution from the foreign corporation (or sells the shares) and will pay interest on the distributions (or gains) to compensate the U.S. fisc for the deferral.  Finally, if a U.S. person owns shares of a publicly traded PFIC, the PFIC shares can be marked to market each year, and the shareholder will recognize annually ordinary gain or loss.
	
	\section{Definition of PFIC}
	
	     A PFIC is any foreign corporation if: (1) 75\% of the gross income for the taxable year is passive income; or (2) the average percentage of assets (by value) held by the corporation during the taxable year which produce passive income or which are held for the production of passive income is at least 50\%.  \S1297(a).  Passive income is defined to be income that would be foreign personal holding company income under the CFC rules.  \S1297(b)(1).  For asset valuation purposes, a corporation must use FMV, unless it elects to use adjusted bases.  \S 1297(f).
	     
	     If a foreign corporation owns 25\% or more of the stock of another corporation, for purposes of determining whether the upper-tier corporation is a PFIC, it will be treated as owning its proportionate share of the assets and as receiving its proportionate share of the income of the lower-tier subsidiary.  \S1297(c).
	     
	     There are two somewhat limited exceptions under which a foreign corporation will not be a PFIC for a start-up year and transition years.  \S1298(b)(2) and (3).  These rules are important for start-ups that would otherwise be PFICs because of the interest they earn on their working capital prior to the receipt of business income.  	
	     
	     \section{Tax Consequences of Owning Shares in a PFIC}
	     
	     There are three tax regimes for PFICs: the deferred tax regime; the QEF regime; and the mark-to-market regime.
	     \begin{center}
	     \textbf{Deferred Tax Regime}
	     \end{center}
	     If a shareholder is subject to the deferred tax regime and receives an \emph{excess distribution}, the distribution is allocated ratably to each day in the shareholder's holding period.  \S1291(a)(1)(A).  Amounts allocated to the current year or any post-'86 years for which the foreign corporation was a PFIC will be taxed as ordinary income.  In addition, the shareholder's tax lability for the current year is increased by the deferred tax amount.  \S1291(a)(1)(B) and (C).  This amount is the sum of the total taxes the shareholder would have paid had the amounts allocated to prior PFIC years been included in income in the years to which the income is allocated.  The rate used for this calculation is the highest tax rate under section 1 or 11 in that year.  Finally, an interest charge is imposed on the deferred taxes.  \S1291(c).
	     
	     A couple of points to note.  Under the the deferred tax regime, the income is deemed to be earned ratably over the shareholder's holding period.  If the PFIC actually earned the income only recently, for example, but the shareholder has a long holding period, the shareholder will be charged for deferral when no income tax was actually deferred.  Also, for purposes of the deferred tax regime, any gain from the disposition of the stock of a PFIC is treated as an excess distribution.  \S1291(a)(2).
	     
	     A shareholder is subject to the deferred tax regime if the PFIC has not been a QEF PFIC for the shareholder's entire post-'86 holding period.  Under the section 1291 regulations, a PFIC for which a QEF election has never been made is referred to as a nonqualified fund.  A foreign corporation that has been a PFIC but for which a QEF election didn't apply for the shareholder's entire holding period is a unpedigreed qualified fund.  Both nonqualified and unpedigreed qualified funds are subject to the deferred tax regime.  If a foreign corporation has been a QEF PFIC for the shareholder's entire holding period, it is a pedigreed fund.
	     
	     These rules can give rise to unpleasant surprises.  Assume, for example, a U.S. person buys stock in a foreign corporation that is not a PFIC, but for one year it satisfies the PFIC definition and no QEF election is made.  Under section 1298(b)(1)--also known as the once-a-PFIC-always-a-PFIC rule--the foreign corporation will always be a PFIC to the shareholder even though it never again is a PFIC.  The deferred tax regime will apply when the shares are sold or an excess distribution is received.  To avoid this rule, an election can be made to purge PFIC taint by marking to market the stock.  \S1298(b)(1).  It is generally advisable to avoid the deferred tax regime.  
	     
	  \begin{center}   
	     \textbf{QEF Regime}
	     \end{center}	       
	       If a QEF election is made, each electing shareholder must include in income its pro rata share of the PFIC's ordinary earnings and net capital gain regardless whether the shareholder receives any earnings.  \S1293(a).  Amounts taxed under section 1293(a) increase a shareholder's basis and are treated as PTI when distributed.  \S1293(d) and (c).  Under regulations, if the foreign corporation is not a PFIC for the current year, there is no QEF inclusion.  If a shareholder discovers he owns stock in a 1291 fund, he can turn the PFIC into a QEF PFIC if he makes a mark-to-market election under section 1291(d)(2)--also known as the purging election.  Any gain realized upon the sale of a pedigreed fund is capital.  Most shareholders make the QEF election.
	       
	       \begin{center}
	       \textbf{Mark to Market Regime} 
	          	\end{center}
	       A U.S. person who owns stock in a publicly traded PFIC or mutual fund that publishes net asset values can elect to mark to market annually the stock and recognize gain or loss.  \S 1296(a).  The gains are included in income as ordinary income.  \S1296(c)(1)(A).  If basis of the PFIC shares exceeds their FMV at year end, the shareholder can deduct the difference only to the extent that it exceeds net gains previously included in income.  \S1296(a)(2).  The deduction will be ordinary.  \S1296(c)(1).  Any loss recognized upon the sale of marketable stock is also ordinary to the extent of prior net gains.  \S1296(c)(1)(B).
	       
	       \addcontentsline{toc}{section}{\protect\numberline{}PFIC Problems} 
	\begin{center}
		\textbf{PFIC Problems}
	\end{center}
	\begin{select}
	
			\begin{enumerate}
	
	\item   USP, a U.S. person, owns 5 shares (5\%) of FC, a foreign corporation; the remaining stock is held by unrelated foreign persons.  USP purchased the shares on December 31, 2000 (thus, under general tax principles, USP's holding period for the stock begins on January 1, 2001) for \$100,000 when FC was a manufacturing company.  For tax years 2001-2004, FC earns each year \$100,000 of manufacturing income and no passive income (as defined in section 1297(b)).  For tax years 2005-2008, FC earns each year \$20,000 of manufacturing income and \$80,000 of passive income.  For 2009, FC earns \$100,000 of manufacturing income and no passive income.  On December 31, 2009, USP sells the stock of FC for \$1,000,000  FC makes no distributions during USP's holding period.  Assume that FC is publicly traded and appreciates each year in value by the amount of its earnings.
	
	What are the tax consequences each year to USP if (1) no QEF election is made; (2) a QEF election is made in the first year it is a PFIC; and (3) a MTM election is made?
	
	
	\end{enumerate}
	\end{select}
 