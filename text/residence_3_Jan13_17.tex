
\section{Corporations and Partnerships}
	\crt{11(d); 7701(a)(1)-(10); 7701(a)(30) and(31)}{1.881-1(a), (b), and (c); 301.7701-1(a)(1) and (b), -2(a), (b)(1)-(8)(i), -3(a) (skim), (b)(1) and (2), -5}{Articles 1(8); 3(1)(a)-(e); and 4}

A business entity in the United States is generally classified as either a partnership or a corporation.  \S 7701(a)(2) and (3).  Corporations are generally taxed separately from their owners.  \S 11(a).  In contrast, a partnership is not subject to tax, but the partners must take into account their share of the partnership's income, gain, or loss, etc.  \S\S 701 and 702.

As in the case of individuals, a tax demarcation exists between U.S. and foreign legal entities:  a U.S. corporation (including an association taxable as a corporation), trust, or estate, is subject to U.S. residence basis taxation, but a foreign corporation, trust, or estate is subject only to source basis taxation.  \margit{Whether a partnership is U.S. or foreign is relevant for other tax purposes, for example, in determining the source of interest paid by a partnership and certain U.S. reporting and withholding tax requirements.}   A partnership (including an LLC treated as a partnership) can be either U.S. or foreign.  Although a partnership is not subject to tax, a partner who is a U.S. person--citizen, resident alien, U.S. corporation, trust, or estate--will be taxed on a residence basis, and a foreign partner will be taxed on a source basis regardless of whether the partnership is U.S. or foreign.

Also remember that a partnership's activities are often imputed to its partners, both limited and general.  In particular, a foreign partner of a partnership that engaged in a U.S. trade or business will be taxed on his distributive share of the partnership's income that is connected with the U.S. trade or business as if he were directly engaged in the U.S. business.  Thus, when dealing with legal entities, you must determine the type of entity--partnership or corporation--and its nationality--U.S. or foreign--to know how the entity and its owners will be taxed, and the scope of any U.S reporting, filing, and withholding tax requirements.

Prior to 1997, whether a legal entity was a partnership (unincorporated entity) or corporation for tax purposes was determined by applying a four-factor test set out Old Reg.\@ \S 301.7701-2(a)(1).\footnote{There were six factors, but two of the factors, associates and profit motive, were common to both profit-oriented partnerships and corporations and were therefore irrelevant to distinguishing between them.} These factors, which derive from \textit{Morrissey v. CIR}, 296 U.S. 344 (1935), were continuity of life, centralized management, limited liability, and free transferability of interest.  An entity was a corporation if it possessed more corporate characteristics than noncorporate characteristics.  These tests were applied not only to U.S. entities, such as limited partnerships and limited liability companies (LLCs),  but also to foreign entities.  \emph{See} Rev.\@ Rul.\@ 88-8, 1988-1 C.B. 403 (all foreign entities were ``unincorporated organizations'' for purposes of the regulations requiring application of the four-factor test).  

The four-factor test generated much criticism.  The regulations required a detailed examination of the entity's organizational documents and local law.  Although taxpayers could apply for a ruling that an entity would be treated as a partnership or corporation, the ruling process was costly, especially for foreign entities, as local counsel was often required to be retained.  The IRS also had to devote resources to review and process the rulings and to draft, review, and issue guidance in the form of revenue rulings.  In addition, obtaining a ruling often took many months, which caused delays in transactions going forward because of tax risks of the entity classification issue.  

The emergence of new entities such as LLCs, limited liability partnerships (LLPs), and limited liability limited partnerships (LLLPs), placed additional strains on IRS resources.  With the enactment of new business regimes that granted limited liability to partnerships and other unincorporated entities, partnership status could be obtained for entities that were virtually identical to traditional corporations.  Entity classification was therefore becoming elective in most cases.  In Notice 95-14, 1995-14 IRB 1, the IRS announced that it was considering abandoning the four-factor approach in favor of a regime that permitted taxpayers to elect the tax status of unincorporated entities.  Regulations were proposed and finalized in 1996, and the so-called ``check the box'' regime became effective January 1, 1997.  

%	\addcontentsline{toc}{section}{\protect\numberline{}Check the Box Regulations}
%		\begin{center}
%			\textbf{Check the Box Regulations}
%				\end{center}

	\subsection{Check the Box Regulations}
	
Classifying an entity as either a partnership or corporation under the check-the-box (CTB) regulations requires first establishing that a separate entity exists for federal tax purposes.  In limited instances, a separate entity exists for state law purposes but not for federal tax purposes.  Reg.\@ \S301.7701-1(a).  If a separate entity exists, it is either a \emph{business entity} or trust, which generally does not have associates or an objective to carry on business for profit.  Special rules apply to trusts.  \emph{See} Reg.\@ \S301.7701-4.\footnote{The CTB regulations do not apply to certain legal entities, such as Qualified Settlement Funds (\S1.468B-1(b)) or Real Estate Mortgage Investment Conduits (REMICs) (\S 860A(a)).}

A business entity that formed pursuant to a state incorporation statute is classified as a corporation for federal tax purposes.  Also treated as corporations are associations, joint-stock companies, insurance companies, organizations that conduct certain banking activities, organizations wholly owned by a State, and organizations that are taxable as corporations under a specific provision of the IRC.  The regulations list entities formed under foreign law that are treated as corporations for federal tax purposes.  These entities, referred to as \emph{per se corporations}, generally possess attributes similar to U.S. corporations, \textit{e.g.}, limited liability, separation of ownership and management, free transferability of shares, and oftentimes are the entity of choice for public offering of interests.  Per se entities \margit{Corporations and per se corporations may \emph{not} elect their federal tax status.} include the Spanish Sociedad An\'{o}nima, the French Societe Anonyme, the German Aktiengesellschaft, and the U.K. Public Limited Company.  Reg.\@ \S 301.7701-2(b)(1); (b)(8).  

A business entity \margit{\emph{Eligible entities} may elect their tax status.  An entity with 2 or more members can be a partnership or an association.  A single member entity is either an association or disregarded entity.} that is not a corporation under Reg.\@ \S301.7701-2 may elect its tax status.  Reg.\@ \S301.7701-3(a).  An entity that can generally elect its tax status is referred to as an \emph{eligible entity}.  An eligible entity with two or more members can be classified as either a partnership or an association (taxed as a corporation).  An eligible entity with a single member--single member entity or SME--can be classified as an association (taxed as a corporation) or can be disregarded as an entity separate from its owner.  If the single owner of a disregarded entity is a corporation, the disregarded entity will be a branch of the owner; if the owner is an individual, the disregarded entity is a sole proprietorship.  It is important to note that the status of being disregarded applies only for federal tax purposes; for state law purposes, the entity continues to exist, \textit{i.e.}, it can hold property, sue, be sued, etc.

The regulations simplify the election process by providing default rules that assign a tax status to an entity in the absence of an explicit election.  \margit{Default classification rules for domestic and foreign entities}  For domestic entities, such as LLCs, with at least two members, the default classification is partnership; if the domestic entity has only one member, it is disregarded.  Reg.\@ \S301.7701-3(b)(1).   

The classification of a foreign eligible entity, such as a GmBH (Germany) or Private Limited Company (United Kingdom), turns on the limited liability\footnote{Limited liability is determined under foreign law or the entity's organizational documents.  Reg.\@ \S301.7701-3(b)(2)(ii).} of it owners.   Reg.\@ \S301.7701-3(b)(2). If all members of a foreign entity have limited liability, it will be an association.  For a foreign entity with more than one member, if at least one member does not have limited liability, it will be a partnership.  Finally, a foreign entity will be a disregarded entity if it has a single owner that does not have limited liability.  An entity's default classification continues until an election is made to change its classification.

The members of an eligible entity may check the box--affirmatively elect a different tax classification than the default classification--by filing Form 8832, Entity Classification Election.  \margit{A taxpayer can elect a different tax classification than the default classification. In tax argot, this is known as \emph{checking the box}.}If an election is made to change a classification, the entity cannot change its classification again for the succeeding sixty months.  Reg.\@ \S301.7701-3(c)(1)(iv).  Furthermore, a change in the tax classification may trigger unpleasant tax consequences. For example, if an association elects to be a partnership, the association is deemed to liquidate and distribute its assets and liabilities to its owners, who contribute the assets and liabilities to a new partnership.  \emph{See} Reg.\@ \S301.7701-3(g).

%{Relevance of foreign eligible entity.  301-7701-3(d)   
  


%\begin{select}
%\revrul{Explanation of the Check the Box Regulations}{Excerpt from T.D. 8697, 1997-1 C.B. 215}
%Section 7701(a)(2) of the Code defines a partnership to include a syndicate, group, pool, joint venture, or other unincorporated organization, through or by means of which any business, financial operation, or venture is carried on, and that is not a trust or estate or a corporation. Section 7701(a)(3) defines a corporation to include associations, joint-stock companies, and insurance companies.  

%The existing regulations for classifying business organizations as associations (which are taxable as corporations under section 7701(a)(3)) or as partnerships under section 7701(a)(2) are based on the historical differences under local law between partnerships and corporations. Treasury and the IRS believe that those rules have become increasingly formalistic. This document replaces those rules with a much simpler approach that generally is elective.  

%As stated in the preamble to the proposed regulations, in light of the increased flexibility under an elective regime for the creation of organizations classified as partnerships, Treasury and the IRS will continue to monitor carefully the uses of partnerships in the international context and will take appropriate action when partnerships are used to achieve results that are inconsistent with the policies and rules of particular Code provisions or of U.S. tax treaties.
%
%\begin{center}\textbf{A. Summary of the Regulations}
%\end{center}
%Section 301.7701-1 provides an overview of the rules applicable in determining an organization's classification for federal tax purposes. The first step in the classification process is to determine whether there is a separate entity for federal tax purposes. The regulations explain that certain joint undertakings that are not entities under local law may nonetheless constitute separate entities for federal tax purposes; however, not all entities formed under local law are recognized as separate entities for federal tax purposes. Whether an organization is treated as an entity for federal tax purposes is a matter of federal tax law, and does not affect the rights and obligations of its owners under local law. For example, if a domestic limited liability company with a single individual owner is disregarded as an entity separate from its owner under \S301.7701-3, its individual owner is subject to federal income tax as if the company's business was operated as a sole proprietorship.

%An organization that is recognized as a separate entity for federal tax purposes is either a trust or a business entity (unless a provision of the Code expressly provides for special treatment, such as the Qualified Settlement Fund rules (\S1.468B) or the Real Estate Mortgage Investment Conduit (REMIC) rules, see section 860A(a)). The regulations provide that trusts generally do not have associates or an objective to carry on business for profit. The distinctions between trusts and business entities, although restated, are not changed by these regulations.
%
%Section 301.7701-2 \margit{Corporations and per se corporations may not elect their federal tax status.}clarifies that business entities that are classified as corporations for federal tax purposes include corporations denominated as such under applicable law, as well as associations, joint-stock companies, insurance companies, organizations that conduct certain banking activities, organizations wholly owned by a State, organizations that are taxable as corporations under a provision of the Code other than section 7701(a)(3), and certain organizations formed under the laws of a foreign jurisdiction (including a U.S. possession, territory, or commonwealth).

%The regulations in \S301.7701-2 include a special grandfather rule, under which an entity described in the list of foreign entities treated as per se corporations will nevertheless be classified as other than a corporation. The regulations also list certain situations where a grandfathered entity would lose its grandfathered status.

%Any business entity \margit{Eligible entities may choose their tax status.  An entity with 2 or more members can be a partnership or an association.  A single member entity is either an association or disregarded entity.} that is not required to be treated as a corporation for federal tax purposes (referred to in the regulation as an eligible entity) may choose its classification under the rules of \S301.7701-3. Those rules provide that an eligible entity with at least two members can be classified as either a partnership or an association, and that an eligible entity with a single member can be classified as an association or can be disregarded as an entity separate from its owner. However, if the single owner of a business entity is a bank (as defined in section 581), then the special rules applicable to banks will continue to apply to the single owner as if the wholly owned entity were a separate entity.

%In order to provide most eligible entities \margit{Default classification rules.}with the classification they would choose without requiring them to file an election, the regulations provide default classification rules that aim to match taxpayers' expectations (and thus reduce the number of elections that will be needed). The regulations adopt a passthrough default for domestic entities, under which a newly formed eligible entity will be classified as a partnership if it has at least two members, or will be disregarded as an entity separate from its owner if it has a single owner. The default for foreign entities is based on whether the members have limited liability. Thus a foreign eligible entity will be classified as an association if all members have limited liability. A foreign eligible entity will be classified as a partnership if it has two or more members and at least one member does not have limited liability; the entity will be disregarded as an entity separate from its owner if it has a single owner and that owner does not have limited liability. Finally, the default classification for an existing entity is the classification that the entity claimed immediately prior to the effective date of these regulations. An entity's default classification continues until the entity elects to change its classification by means of an affirmative election.

%An eligible entity may affirmatively elect its classification on Form 8832, Entity Classification Election. The regulations require that the election be signed by each member of the entity or any officer, manager, or member of the entity who is authorized to make the election and who represents to having such authorization under penalties of perjury. An election will not be accepted unless it includes all of the required information, including the entity's taxpayer identifying number (TIN).

%Taxpayers are \margit{Changes in tax status may cause adverse tax consequences.}reminded that a change in classification, no matter how achieved, will have certain tax consequences that must be reported. For example, if an organization classified as an association elects to be classified as a partnership, the organization and its owners must recognize gain, if any, under the rules applicable to liquidations of corporations.
%\end{select} 

Once the tax classification--corporation or partnership--of entity is determined, the entity's residence or nationality must be determined.  A corporation or partnership is a U.S. person if it is created or organized as any type of entity in the United States.  \S7701(a)(4).  \margit{An entity's tax status is first determined and then its nationality or residence.} If an entity is not created or organized in the United States as any type of entity, it is foreign.  Reg.\@ \S301.7701-5(a).  The United States looks solely to the entity's  place of formation or where its charter was issued in determining residence.  Many other countries, however, determine a legal entity's residence by where it is managed and controlled--where its effective management is located.  Thus, a corporation formed in the United Kingdom will be treated as a Spanish corporation if it is managed and controlled in Spain.  

The U.S. regime, while virtually eliminating any dispute over a legal entity's residence or nationality, has been criticized as being easy to manipulate and played a key role in certain highly publicized inversion transactions.  In an inversion transaction, a U.S.-based multinational ``inverts" its corporate structure so that the parent of the corporate chain is now a foreign corporation rather than a U.S. corporation, but the shareholders of the foreign parent corporation are the same as the shareholders of the former U.S. parent.  The goal of an inversion transaction is to lower the entity's worldwide tax rate by removing the earnings of the foreign subsidiaries from any residual U.S. tax when the earnings are distributed.     

As a result of these inversion transactions, Congress enacted section 7874, which treats the new top-tier foreign corporation in an inversion transaction as a U.S. corporation for all federal tax purposes if at least 80\% of the stock is held by former shareholders.  Serious consideration has been given to amending the U.S. rules to treat any foreign corporation as a U.S. person if it is managed and controlled in the United States, thereby aligning the U.S. and European rules.  For example, in 2005, Congress considered, but ultimately rejected a bill that treated any publicly-traded foreign corporation as a U.S. person if its primary place of management and control was in the United States.  \emph{See} \emph{Options to Improve Tax Compliance and Reform Tax Expenditures}, Joint Committee on Taxation (JCS-02-05) (Jan. 27, 2005).



%\addcontentsline{toc}{section}{\protect\numberline{}Dual Chartered Entity Regulations}
%\begin{select}
%	\revrul{Entity Classification and Dual Chartered Entities}{Excerpt from T.D. 9153, 2004-2 C.B. 517}
%Several jurisdictions have recently enacted provisions (generally referred to as either continuance or 
%domestication statutes) that make it possible for a business entity to be treated as created or organized under the laws of more than one jurisdiction at the same time (a dually chartered entity). A dually chartered entity and the interest holders in the entity must determine for Federal tax purposes (1) the entity's classification (e.g., corporation or partnership) and (2) whether the entity is foreign or domestic. The regulations contained in this document are intended to clarify the rules for these determinations. 
% 
%Section 7701(a)(3) of the Internal Revenue Code of 1986 (Code) provides that the term corporation 
%includes associations, joint stock companies, and insurance companies. The definition of a corporation 
%under the tax statutes has not changed since the Revenue Act of 1918,\ldots
% 
%Section 7701(a)(4) of the Code provides that the term domestic when applied to a corporation or 
%partnership means ``created or organized in the United States or under the law of the United States or of any 
%State unless, in the case of a partnership, the Secretary provides otherwise by regulations.'' Section 
%7701(a)(5) of the Code provides that the term foreign when applied to a corporation or partnership means a 
%``corporation or partnership that is not domestic.'' This definition is significantly different than the 
%definition of foreign entity that preceded it. The Revenue Act of 1918 used the term foreign to mean a 
%corporation or partnership ``created or organized outside the United States.'' Thus, under that definition, a 
%dually chartered entity that was organized in the United States and in a foreign jurisdiction would have met 
%the definitions of both a domestic entity and a foreign entity, creating uncertainty as to the entity's status. 
%The Revenue Act of 1924 \ldots eliminated that potential for uncertainty by 
%providing the definition of a foreign entity that is currently reflected in section 7701(a)(5). This definition 
%of a foreign entity as ``a corporation or partnership that is not domestic'' makes it impossible for an entity to 
%meet the definitions of both a domestic entity and a foreign entity for Federal tax purposes at the same time. 
%As a result, \margit{If a DCE is organized in the U.S., it's a domestic entity.}If organized in the U.S. and a foreign jurisdiction, a dually chartered entity that is organized both in the United States and in a foreign jurisdiction 
%is a domestic entity.  
%\ldots
%Under the existing rules, the characterization of a business entity for Federal tax purposes is established in 
%two separate and independent steps. The first involves a determination of whether the entity is a corporation 
%or a non-corporate entity (e.g., a partnership). The second involves a determination of whether the entity is 
%foreign or domestic. 
% 
%The determination \margit{If a DCE is a corporation in any jurisdiction, it is a corporation for U.S. tax purposes.}
%of whether a business entity is classified as a corporation is made by applying the 
%definition in \S301.7701-2(b). If the entity is not a corporation under that definition, then it is a partnership 
%if it has more than one owner and it is a disregarded entity if it has only a single owner. The temporary 
%regulations in this document clarify that this same definition applies to dually chartered entities. Thus, to 
%determine whether a dually chartered entity is a corporation, it must first be determined if the entity's 
%organization in any of the jurisdictions in which it is organized would cause it to be treated as a corporation 
%under the rules of \S301.7701-2(b). If the entity would be treated as a corporation as a result of its formation 
%in any of the jurisdictions in which it is organized, it is treated as a corporation for Federal tax purposes 
%even though its organization in the other jurisdiction or jurisdictions would not have caused it to be treated 
%as a corporation. 
% 
%Once the classification of a \margit{A DCE is domestic if it created or organized in the U.S.}business entity has been determined, a determination will generally need to be 
%made regarding whether it is a domestic or foreign entity. It is a domestic entity if it is created or organized 
%in the United States or under the laws of the United States or of any state. It is a foreign entity only if it is 
%not domestic. The temporary regulations in this document revise \S301.7701-5 to clarify that a dually 
%chartered entity is domestic if it is organized as any form of entity in the United States, regardless of how it 
%is organized in any foreign jurisdiction. An entity that is classified as a corporation because of its form of 
%organization in a foreign country is considered a domestic corporation if it is also organized as some form 
%of entity in the United States, regardless of what form the entity takes in the United States (e.g., 
%corporation, limited liability company, or partnership). 
% 
%These temporary regulations also remove from \S301.7701-5 the definitions of resident foreign corporation, 
%nonresident foreign corporation, resident partnership and nonresident partnership because these terms have 
%become obsolete due to statutory changes since the final regulations were published in 1960. \margit{Is that correct?  See Section 861(a)(1) and Reg.\@ 1.861-2(a)(2).}
% 
%These regulations clarify current law and do not change the outcome that would result under a proper 
%application of the existing rules as they apply to dually chartered entities. For example, the temporary 
%regulations are consistent with the result in Rev. Rul. 88-25 (1988-1 C.B. 116). These regulations are also 
%not intended to affect the result under existing rules regarding whether an organization is a separate entity 
%for Federal tax purposes (e.g., whether, in a particular case, two sets of organizational documents constitute 
%different facets of a single entity or the foundations of two separate entities). In addition, if a business entity 
%undertakes a continuance, domestication, or other transaction that, upon application of these rules, changes 
%its entity classification or changes its foreign or domestic status, the tax effects of that transaction are 
%determined under the regular tax principles that apply to such changes. Finally, the regulations contained in 
%this document do not determine an entity's place of residence for the purpose of applying the provisions of 
%a tax treaty. 
% 
%\ldots
%Section 7701(a)(4) of the Code provides regulatory authority to define a domestic partnership other than 
%based on where the partnership is created or organized. The Treasury and the IRS are continuing to explore 
%whether, and under what circumstances, a different definition may be appropriate. If any change to the 
%definition of a domestic partnership were to be proposed, it would apply only to partnerships created or 
%organized after the issuance of regulations or other guidance substantially describing the change in 
%definition. 
%\end{select}

One sometimes overlooked consequence of the check-the-box regulations, especially among non-tax specialists, is that separate legal entities under state or foreign law, including tiers of entities, may not be separate entities for federal tax purposes.  Rev.\@ Rul.\@ 2004-77, 2004-2 C.B. 119, confirms that a state or foreign law partnership whose partners consist of a disregarded entity and the disregarded entity's sole member is not a partnership for federal tax purposes.  It is also important to remember that the foreign law treatment may be different than the U.S. tax treatment.

\addcontentsline{toc}{section}{\protect\numberline{}Rev. Rul. 2004-77}
\begin{select}
\revrul{Rev. Rul. 2004-77}{2004-2 C.B. 119}
\begin{center}\textbf{FACTS}
\end{center} 
\textbf{Situation 1.} X, a domestic corporation, is the sole owner of L, a domestic limited liability company 
(LLC). Under \S301.7701-3(b)(1), L is disregarded as an entity separate from its owner, X. L and X 
are the only members under local law of P, a state law limited partnership or LLC. There are no 
other constructive or beneficial owners of P other than L and X. L and P are eligible entities that do not elect under \S301.7701-3(c) to be treated as associations for federal tax purposes.\\ 

\textbf{Situation 2.} X is an entity that is classified as a corporation under \S301.7701-2(b). X is the sole 
owner of L, a foreign eligible entity. Under \S301.7701-3(c), L has elected to be disregarded as an 
entity separate from its owner. L and X are the only members under local law of P a foreign eligible 
entity. There are no other constructive or beneficial owners of P other than L and X. 
\begin{center}\textbf{LAW AND ANALYSIS}
\end{center}

Section 7701(a)(2) of the Internal Revenue Code provides that the term partnership includes a 
syndicate, group, pool, joint venture, or other unincorporated organization through or by means of 
which any business, financial operation, or venture is carried on, and which is not a trust, estate, or corporation. 

Section 301.7701-1(a)(1) provides that whether an organization is an entity separate from its 
owners for federal tax purposes is a matter of federal tax law and does not depend on whether the 
organization is recognized as an entity under local law.
 
Section 301.7701-2(a) provides that a business entity is any entity recognized for federal tax 
purposes (including an entity with a single owner that may be disregarded as an entity separate 
from its owner under \S301.7701-3) that is not properly classified as a trust under \S301.7701-4 or 
otherwise subject to special treatment under the Code. A business entity with two or more owners is 
classified for federal tax purposes as either a corporation or a partnership. A business entity with 
only one owner is classified as a corporation or is disregarded; if the entity is disregarded, its 
activities are treated in the same manner as a sole proprietorship, branch, or division of the owner. 
Section 301.7701-2(c)(1) provides that, for federal tax purposes, the term ``partnership" means a 
business entity that is not a corporation under \S301.7701-2(b) and that has at least two owners. 
Section 301.7701-2(c)(2)(i) provides, in general, that a business entity that has a single owner and 
is not a corporation under \S301.7701-2(b) is disregarded as an entity separate from its owner. 
Section 301.7701-3(a) provides that a business entity that is not classified as a corporation under \S301.7701-2(b)(1), (3), (4), (5), (6), (7), or (8) (an eligible entity) can elect its classification 
for federal tax purposes. An eligible entity with at least two owners can elect to be classified as 
either an association (and thus a corporation under \S301.7701-2(b)(2)) or a partnership, and an 
eligible entity with a single owner can elect to be classified as an association or to be disregarded as 
an entity separate from its owner. 

Section 301.7701-3(b)(1) provides generally that in the absence of an election otherwise, a domestic 
eligible entity is (a) a partnership if it has at least two members, or (b) disregarded as an entity 
separate from its owner if it has a single owner.
 
Section 301.7701-3(b)(2) provides generally that, in the absence of an election otherwise, a foreign 
eligible entity is (a) a partnership if it has two or more owners and at least one owner does not have 
limited liability, (b) an association if all its owners have limited liability, or (c) disregarded as an 
entity separate from its owner if it has a single owner that does not have limited liability. 

\textbf{Situation 1.} Under \S301.7701-2(c)(2), L is disregarded as an entity separate from its owner, X, and 
its activities are treated in the same manner as a branch or division of X. Because L is 
disregarded as an entity separate from X, X is treated as owning all of the interests in P. P is a 
domestic entity, with only one owner for federal tax purposes, that has not made an election to be 
classified as an association taxable as a corporation. Because P has only one owner for federal tax 
purposes, P cannot be classified as a partnership under \S7701(a)(2). For federal tax purposes, P is 
disregarded as an entity separate from its owner. 

\textbf{Situation 2.} Under \S301.7701-3(c), L is disregarded as an entity separate from its owner, X, and its 
activities are treated in the same manner as a branch or division of X. Because L is disregarded as 
an entity separate from X, X is treated as owning all of the interests in P. Because P has only one 
owner for federal tax purposes, P cannot be classified as a partnership under \S7701(a)(2). For 
federal tax purposes, P is either disregarded as an entity separate from its owner or an association 
taxable as a corporation.
\begin{center} \textbf{HOLDING}
\end{center}
If an eligible entity has two members under local law, but one of the members of the eligible entity 
is, for federal tax purposes, disregarded as an entity separate from the other member of the 
eligible entity, then the eligible entity cannot be classified as a partnership and is either disregarded 
as an entity separate from its owner or an association taxable as a corporation. 
\end{select}

\addcontentsline{toc}{section}{\protect\numberline{}Comments}
	\begin{center}
		\textbf{\emph{Comments}}
			\end{center}

	\begin{enumerate}

	\item Some commentators have suggested that the Treasury may have exceeded its regulatory authority in issuing the check-the-box regulations on the basis that the statutes that define corporation, association, and partnership do not permit an elective regime.  To date, courts, following the framework set forth in \emph{Chevron U.S.A. Inc. v. Natural Resources Defense Council, Inc.}, 467 U.S. 837 (1984), have rejected such arguments on the grounds that the relevant statutes are ambiguous and the check the box regulations are not arbitrary or capricious.  The Supreme Court, in \emph{Mayo Foundation v. United States}, 131 S. Ct. 704 (2011), affirmed that the validity of treasury regulations, both those issued under the Treasury's general regulatory authority in \S7805 and under a specific statutory grant, is determined under \emph{Chevron}.  Revenue rulings and revenue procedures, however, are probably not entitled to \emph{Chevron} deference.   The excerpt below from \emph{Littriello v. U.S.}, 484 F.3d 372 (6th Cir. 2007), demonstrates how the Chevron framework is applied. 
	\begin{quote}
	The first two arguments raised by Littriello are intertwined. He contends that the statute underlying the ``check-the-box" regulations is unambiguous and that the district court's invocation of Chevron was, therefore, erroneous. Under Chevron, a court reviewing an agency's interpretation of a statute that it administers must first determine ``whether Congress has directly spoken to the precise question at issue." 467 U.S. at 842. If congressional intent is clear, then ``that is the end of the matter; for the court, as well as the agency, must give effect to the unambiguously expressed intent of Congress." Id. at 842-43. However, ``if the statute is silent or ambiguous with respect to the specific issue, the question for the court is whether the agency's answer is based on a permissible construction of the statute." Id. at 843; see also Barnhart v. Thomas, 540 U.S. 20, 26 (2003) (when a statute is silent or ambiguous, the court must ``defer to a reasonable construction by the agency charged with its implementation").

Littriello argues, first, that Chevron has been modified by the Supreme Court's recent decision in National Cable \& Telecommunications Ass'n v. Brand X Internet Services, 545 U.S. 967 (2005), which ``seems to revise the Chevron formula by substituting as the second agency requirement `reasonableness' for `permissible construction of the statute.''' But this argument overlooks the fact that the Chevron opinion uses the terms ``reasonable" and ``permissible" interchangeably in reference to statutory construction. See, e.g., 467 U.S. at 843, 845. Second, and more substantially, he posits that the regulations run afoul of Morrissey, ``the seminal case on section 7701,'' which he reads to hold that the IRS is legally required to determine the classification of a taxpayer-business within the definitions set out in the statute and may not ``abdicate the responsibility of making that determination to the taxpayer itself" by permitting an election of classification such as a ``check-the-box'' option.

Although the plaintiff's Morrissey argument is not a model of clarity, it seems to depend on the proposition that the terms defined in section 7701 (``corporation,'' ``association,'' ``partnership,'' etc.) are not ambiguous but ``[have been] in common usage in Anglo American law for centuries" and, as a corollary, that ``Morrissey provides a test of identification [that is itself] unambiguous.'' Hence, the argument goes, it is the ``check-the-box" regulations that ``render whole portions of the Internal Revenue Code ambiguous" and are therefore ``in direct conflict with the decision of the Supreme Court in Morrissey'' in the absence of Congressional amendment to section 7701.

It is unnecessary, in our judgment, to engage in an exegesis of Chevron here. The perimeters of that opinion and its directive to courts to give deference to an agency's interpretation of statutes that the agency is entrusted to administer and to the rules that govern implementation, as long as they are reasonable, are clear, and are clearly applicable in this case. Moreover, the argument that Morrissey has somehow cemented the interpretation of section 7701 in the absence of subsequent Congressional action or Supreme Court modification is refuted by Chevron, in which the Court suggested that an agency's interpretation of a statute, as reflected in the regulations it promulgates, can and must be revised to meet changing circumstances. See Chevron, 467 U.S. at 863-64. Even more to the point, the Court in Morrissey observed that the Code's definition of a corporation was less than adequate and that, as a result, the IRS had the authority to supply rules of implementation that could later be changed to meet new situations. See 296 U.S. at 354-55. Finally, we note that our interpretation is buttressed by the opinion in National Cable, on which the plaintiff relies to support the proposition that the ``check-the-box" regulations are impermissible in light of Morrissey. In that case, the Supreme Court noted that ``[a] court's prior judicial construction of a statute trumps an agency construction otherwise entitled to Chevron deference only if the prior court decision holds that its construction follows from the unambiguous terms of the statute and thus leaves no room for agency discretion." Nat'l Cable, 545 U.S. at 982 (emphasis added).

In short, we agree with the district court's conclusions: that section 7701 is ambiguous when applied to recently emerging hybrid business entities such as the LLCs involved in this case; that the Treasury regulations developed to fill in the statutory gaps when dealing with such entities are eminently reasonable; that the ``check-the-box" regulations are a valid exercise of the agency's authority in that respect; that the plaintiff's failure to make an election under the ``check-the-box" provision dictates that his companies be treated as disregarded entities under those regulations, thereby preventing them from being taxed as corporations under the Internal Revenue Code; and that he is, therefore, liable individually for the employment taxes due and owing from those businesses because they constitute sole proprietorships under section 7701, and he is the proprietor.
		\end{quote}
		
		
	\item The contours of judicial review of the validity of Treasury regulations is still a moving hand.  In \textit{Altera v.\@ CIR}, 145 T.C. No. 3 (July 27, 2015), the Tax Court struck down the provisions in cost-sharing regulations under section 482 that require taxpayers to include stock-based compensation costs in the cost pool.  The Tax Court held that the regulations did not satisfy \S706(a)(A) of the APA, under which a court can set aside agency actions that are ``arbitrary, capricious, an abuse of discretion, or otherwise not in accordance with law.''  In particular, the Tax Court found that the regulations did not satisfy the ``reasoned decisionmaking standard'' set forth in \textit{Motor Vehicle Manufacturers Assoc.\@ of the United States v.\@ State Farm}, 463 U.S. 29 (1983).  The Tax Court determined that the Treasury  failed to conduct any factfinding, failed to support its position that unrelated parties would share stock-based compensation in the context of a cost-sharing arrangement, and failed to respond to significant comments.  This important case has been appealed to the 9th Circuit.

	\item During the early 2000's, questions arose regarding the classification of an entity that was organized in more than one country.  This could be accomplished pursuant to a domestication or continuance statute such as DGCL \S388.  An entity with more than one charter is referred to as a dual chartered entity.  The tax status of a dual chartered entity was not entirely certain because it could be a corporation in one country but a passthrough entity in another.  Which classification should prevail?  Did the order in which the charters were acquired matter?  

Amendments to Reg.\@ \S301.7701-2(b)(9) issued in 2004 clarify that the tax status and residence of a dual chartered entity is determined under a two-step process.  First, a dual chartered entity will be a corporation if it is a corporation under the entity classification rules of Reg.\@  \S301.7701-2(b) in any country, regardless of its status in another country and the order in which it acquires its charters.  Thus, a Spanish sociedad an\'{o}nima with two owners that is dually chartered as a Delaware LLC will be a corporation, even though the default classification for the LLC under U.S. law would be partnership.  The residence or nationality of the entity is then determined.  The entity will be a U.S. entity if it is created or organized in the United States or under the laws of the United States or of any state, again regardless of the order of formation.  Reg.\@ \S301.7701-5.  The preamble to the regulations states that these rules do not apply for determining an entity's residence for purposes of an income tax treaty.      		
	\end{enumerate}		
%such as the Sixth Circuit below in \textit{Littriello v. U.S.}, have rejected these arguments and have upheld the validity of the regulations.  When reading \textit{Littriello}, make sure that you can articulate the standard under which a court reviews the validity of a regulation. 

%\addcontentsline{toc}{section}{\protect\numberline{}Littriello v. U.S.}
%\begin{select}
%\caseart{Littriello v. U.S.}{484 F.3d 372 (6th Cir. 2007)}{Daughtrey, J.}\\
%\ldots
%The plaintiff, Frank Littriello, was the sole owner of several Kentucky limited liability companies (LLCs), the operation of which resulted in unpaid federal employment taxes totaling \$1,077,000. Because Littriello was the sole member of the LLCs and had not elected to have the businesses treated as ``associations'' (i.e., corporations) under Treasury Regulations sections 301.7701-3(a) and (c), the LLCs were ``disregarded" as separate taxable entities and, instead, were treated for federal tax purposes as sole proprietorships under Treasury Regulation section 301.7701- 3(b)(1)(ii).  [Littriello, as sole proprietor, failed to pay the outstanding employment taxes, the IRS filed notices of determination and, eventually, notified him of its intent to levy on his property to enforce previously filed tax liens. Littriello responded by initiating complaints for judicial review in district court, contending that the regulations in question (1) exceed the authority of the Treasury to issue regulatory interpretations of the Internal Revenue Code; (2) conflict with the principles enunciated by the Supreme Court in Morrissey v. Commissioner, 296 U.S. 344 (1935); and (3) disregard the separate existence of an LLC under Kentucky state law. He also argued in his motion for summary judgment that the regulations are not applicable to employment taxes. After the cases were consolidated for disposition, the district court held that the "check-the-box regulations" are "a reasonable response to the changes in the state law industry of business formation," upheld them under Chevron/1/ analysis, and held that the plaintiff was individually liable for the employment taxes at issue. We conclude that the district court's analysis was correct and affirm.
% 
% \begin{center} \textbf{PROCEDURAL AND FACTUAL BACKGROUND}\\ 
%\end{center}
%Frank Littriello was the owner of several business entities, including Kentuckiana Healthcare, LLC; Pyramid Healthcare Wisc. I, LLC; and Pyramid Healthcare Wisc. II, LLC. Each of these businesses was organized as a limited liability company under Kentucky law, with Littriello as the sole member. He did not elect to have them treated as corporations for federal tax purposes and, as a result, none of the LLCs was subject to corporate income taxation. For the tax years in question, Littriello reported his income from the three businesses on Schedule C of his individual income tax return -- the schedule on which the profits and losses of a sole proprietorship are reported. Because the LLCs were ``disregarded entities" under the pertinent tax regulations, and not corporate entities, the IRS assessed Littriello for the full amount of the unpaid employment taxes for 2000-2002.
%
%\ldots
% 
% \begin{center} \textbf{DISCUSSION}\\ 
%\end{center}                       
%
%The Treasury Regulations at the heart of this litigation, 26 C.F.R. sections 301.7701-1 - 301.7701-3, were issued in 1996 to clarify the rules for determining the classification of certain business entities for federal tax purposes, replacing the so-called ``Kintner regulations."\ldots The earlier regulations had been developed to aid in classifying business associations that were not incorporated under state incorporation statutes but that had certain characteristics common to corporations and were thus subject to taxation as corporations under the federal tax code. \ldots 
%Corporate income is, of course, subject to "double taxation" -- once at the corporate level under I.R.C. section 11(a) and again at the individual- shareholder level, pursuant to I.R.C. section 61(a)(7). In contrast, partnership income benefits from "pass-through" treatment -- it is taxed once, not at the business level but only after it passes through to the individual partners and is taxed as income to them, pursuant to I.R.C. sections 701-777. A sole proprietorship -- in which a single individual owns all the assets, is liable for all debts, and operates in an individual capacity -- is also taxed only once.

%The Kintner regulations built on an even earlier standard, set out by the Supreme Court in Morrissey, in which the Court addressed the tax code provision that included an ``association" within the definition of a corporation, in order to determine whether a ``business trust" qualified as an ``association" for federal tax purposes. 296 U.S. at 346. Morrissey identified certain characteristics as those typical of a corporation, including the existence of associates, continuity of the entity, centralized management, limited personal liability, transferability of ownership interests, and title to property. Id. at 359-61. However, the Court did not hold that a specific number of those characteristics had to be present in order to establish the business entity as a corporation, nor did it address the consequence of a partnership having some of those characteristics, leaving the distinctions between and among the various defined entities less than clear.
%
%Meant to clarify some of the confusion created in the wake of Morrissey, the Kintner regulations developed four essential characteristics of a corporate entity and provided that an unincorporated business would be treated as an ``association" -- and, therefore, as a corporation rather than a partnership -- if it had three of those four identifying characteristics. See former Treas. Reg.\@ sections 301.7701-2(a)(1) and (3). The Kintner regulations, adequate to provide a measure of predictability at the time of their promulgation in 1960 and for several decades afterward, proved less than adequate to deal with the new hybrid business entities -- limited liability companies, limited liability partnerships, and the like -- developed in the last years of the last century under various state laws. These unincorporated business entities had the characteristics of both corporations and partnerships, combining ease of management with limited liability, and were increasingly structured with the Kintner regulations in mind, in order to take advantage of whatever classification was thought to be the most advantageous. The ``Kintner exercise" required skillful lawyering by business entities and case-by-case review by the IRS; it quickly came to be seen as squandering of resources on both sides of the equation.
%
%As a result, the IRS undertook to replace the Kintner regulations with a more practical scheme, consistent with existing tax statutes \ldots  
%
%The district court noted that Littriello's unincorporated businesses had not elected to be treated as corporations under the new regulations and were, therefore, deemed by the IRS to be sole proprietorships. This result provided Littriello with a major tax advantage: his income from the healthcare facilities would be taxed to him only once. But, of course, it also meant that he would be responsible not only for taxes on business income but also for those federal employment taxes that were required by statute and that had not been paid for the years in question.
%
%The district court found that the regulations were a reasonable interpretation by the IRS of a tax statute (I.R.C. section 7701) that was otherwise ambiguous, upheld them under Chevron analysis, after noting that it was apparently the first court asked to review those regulations, and held Littriello individually liable for the amounts assessed by the IRS.  \ldots 
%
% \begin{center} \textbf{A. Chevron Analysis}\\ 
%\end{center}                       

%The first two arguments raised by Littriello are intertwined. He contends that the statute underlying the ``check-the-box" regulations is unambiguous and that the district court's invocation of Chevron was, therefore, erroneous. Under Chevron, a court reviewing an agency's interpretation of a statute that it administers must first determine ``whether Congress has directly spoken to the precise question at issue." 467 U.S. at 842. If congressional intent is clear, then ``that is the end of the matter; for the court, as well as the agency, must give effect to the unambiguously expressed intent of Congress." Id. at 842-43. However, ``if the statute is silent or ambiguous with respect to the specific issue, the question for the court is whether the agency's answer is based on a permissible construction of the statute." Id. at 843; see also Barnhart v. Thomas, 540 U.S. 20, 26 (2003) (when a statute is silent or ambiguous, the court must ``defer to a reasonable construction by the agency charged with its implementation").

%Littriello argues, first, that Chevron has been modified by the Supreme Court's recent decision in National Cable \& Telecommunications Ass'n v. Brand X Internet Services, 545 U.S. 967 (2005), which ``seems to revise the Chevron formula by substituting as the second agency requirement `reasonableness' for `permissible construction of the statute.''' But this argument overlooks the fact that the Chevron opinion uses the terms ``reasonable" and ``permissible" interchangeably in reference to statutory construction. See, e.g., 467 U.S. at 843, 845. Second, and more substantially, he posits that the regulations run afoul of Morrissey, ``the seminal case on section 7701," which he reads to hold that the IRS is legally required to determine the classification of a taxpayer-business within the definitions set out in the statute and may not ``abdicate the responsibility of making that determination to the taxpayer itself" by permitting an election of classification such as a ``check-the-box" option.

%Although the plaintiff's Morrissey argument is not a model of clarity, it seems to depend on the proposition that the terms defined in section 7701 (``corporation," ``association,'' ``partnership," etc.) are not ambiguous but ``[have been] in common usage in Anglo American law for centuries" and, as a corollary, that ``Morrissey provides a test of identification [that is itself] unambiguous." Hence, the argument goes, it is the ``check-the-box" regulations that ``render whole portions of the Internal Revenue Code ambiguous" and are therefore ``in direct conflict with the decision of the Supreme Court in Morrissey" in the absence of Congressional amendment to section 7701.

%It is unnecessary, in our judgment, to engage in an exegesis of Chevron here. The perimeters of that opinion and its directive to courts to give deference to an agency's interpretation of statutes that the agency is entrusted to administer and to the rules that govern implementation, as long as they are reasonable, are clear, and are clearly applicable in this case. Moreover, the argument that Morrissey has somehow cemented the interpretation of section 7701 in the absence of subsequent Congressional action or Supreme Court modification is refuted by Chevron, in which the Court suggested that an agency's interpretation of a statute, as reflected in the regulations it promulgates, can and must be revised to meet changing circumstances. See Chevron, 467 U.S. at 863-64. Even more to the point, the Court in Morrissey observed that the Code's definition of a corporation was less than adequate and that, as a result, the IRS had the authority to supply rules of implementation that could later be changed to meet new situations. See 296 U.S. at 354-55. Finally, we note that our interpretation is buttressed by the opinion in National Cable, on which the plaintiff relies to support the proposition that the ``check-the-box" regulations are impermissible in light of Morrissey. In that case, the Supreme Court noted that ``[a] court's prior judicial construction of a statute trumps an agency construction otherwise entitled to Chevron deference only if the prior court decision holds that its construction follows from the unambiguous terms of the statute and thus leaves no room for agency discretion." Nat'l Cable, 545 U.S. at 982 (emphasis added).

%In short, we agree with the district court's conclusions: that section 7701 is ambiguous when applied to recently emerging hybrid business entities such as the LLCs involved in this case; that the Treasury regulations developed to fill in the statutory gaps when dealing with such entities are eminently reasonable; that the ``check-the-box" regulations are a valid exercise of the agency's authority in that respect; that the plaintiff's failure to make an election under the ``check-the-box" provision dictates that his companies be treated as disregarded entities under those regulations, thereby preventing them from being taxed as corporations under the Internal Revenue Code; and that he is, therefore, liable individually for the employment taxes due and owing from those businesses because they constitute sole proprietorships under section 7701, and he is the proprietor.

% \begin{center} \textbf{B. Status Under State Law}\\ 
%\end{center} 

%Citing United States v. Galletti, 541 U.S. 114 (2004), Littriello argues that the IRS must recognize the separate existence of his LLCs as a matter of state law. We conclude that the opinion is inapplicable here. Galletti involved a partnership, not a disregarded entity, that was assessed as an employer for unpaid employment taxes. See id. at 117. The partners, who were liable for partnership debts under state law, contended that they should therefore also be assessed as ``employers," but the Court held as a matter of federal law that ``nothing in the Code requires the IRS to duplicate its efforts by separately assessing the same tax against individuals or entities who are not the actual taxpayers but are, by reason of state law, liable for payment of the taxpayer's debt." Id. at 123. Hence, the Court in Galletti was concerned with a business actually organized as a partnership and not a disregarded entity deemed a sole proprietorship for federal tax purposes. Of course, partnerships are recognized entities under federal tax law and explicitly included in section 7701's definitions, while single-member LLCs are not. See I.R.C. section 7701(a)(2).

%The same flaw prevents application of the ruling in People Place Auto Hand Carwash, LLC v. Commissioner, 126 T.C., 359 (2006), to the facts here. In this recent opinion, submitted as supplemental authority by Littriello, the Tax Court held that imposition of an employment tax on the LLC could not be viewed as equivalent to the imposition of an employment tax on its members. Again, however, the LLC in People Place had more than a single member and, because it had not opted to be treated as a corporation, it was perforce a disregarded entity treated as a partnership. But under no circumstances could Littriello's single-member LLCs be treated as partnerships for federal tax purposes -- his choice was to elect treatment of each of them as a corporation or, in the absence of an election, have them treated as sole proprietorships.

%The federal government has historically disregarded state classifications of businesses for some federal tax purposes. In Hecht v. Malley, 265 U.S. 144 (1924), for example, the United States Supreme Court held that Massachusetts trusts were ``associations" within the meaning of the Internal Revenue Code despite the fact they were not so considered under state law. As courts have repeatedly observed, state laws of incorporation control various aspects of business relations; they may affect, but do not necessarily control, federal tax provisions. See, e.g., Morrissey, 296 U.S. at 357-58 (explaining that common law definitions of certain corporate forms do not control interpretation of federal tax code). As a result, Littriello's single-member LLCs are entitled to whatever advantages state law may extend, but state law cannot abrogate his federal tax liability.

% \begin{center} \textbf{C. Proposed Amendments to the Regulations}\\ 
%\end{center} 
%
%In October 2005, after the notice of appeal in this case had been filed, the IRS circulated a notice of proposed rule-making that set out possible amendments to the entity-classification regulations that would shelter individuals similarly situated to Littriello for unpaid employment taxes. The proposed amendments would treat ``single-owner eligible entities that currently are disregarded as entities separate from their owners for federal tax purposes . . . as separate entities for employment tax and related reporting requirements." Disregarded Entities; Employment and Excise Taxes, 70 Fed. Reg.\@ 60475 (proposed Oct. 18, 2005) (to be codified at 26 C.F.R. pts. 1.301). Thus, if the amendments had been in place when the tax deficiencies in this case arose, singlemember LLCs such as Littriello's would be treated as separate entities for employment tax purposes, although not for other federal tax purposes.
%
%Littriello argues that the proposed amendments should be taken as reflecting current Treasury Department policy and applied to his case. But, it appears that the changes contemplated by the amendments are intended to simplify employment tax collection procedures and do not represent an endorsement of the position that Littriello has advocated in this litigation. As the Supreme Court noted in Commodity Futures Trading Commission v. Schor, 478 U.S. 833 (1986):
%\begin{quote} 
%    It goes without saying that a proposed regulation does not
%    represent an agency's considered interpretation of its statute
%    and that an agency is entitled to consider alternative
%    interpretations before settling on the view it considers most
%    sound. Indeed, it would be antithetical to the purposes of the
%    notice and comment provisions of the Administrative Procedure
%    Act, 5 U.S.C. section 553, to tax an agency with "inconsistency"
%    whenever it circulates a proposal that it has not firmly decided
%    to put into effect and that it subsequently reconsiders in
%    response to public comment.
%\end{quote}
%Id. at 845. As the IRS urges, we conclude that ``[b]ecause the further development of permissible alternatives is part of the administering agency's function under Chevron, the proposed regulations do not in any way undermine the District Court's determination that the current regulations are reasonable and valid." Plainly, an agency does not lose its entitlement to Chevron deference merely because it subsequently proposes a different approach in its regulations. \ldots 
% 
%  \begin{center} \textbf{CONCLUSION}\\ 
%\end{center} 
% For the reasons set out above, we reject the plaintiff's challenge to the ``check-the-box" regulations and AFFIRM the district court's grant of summary judgment to the defendant.
% \end{select}
%\addcontentsline{toc}{section}{\protect\numberline{}Treaties and Business Entities}
%	\begin{center}
%		\textbf{Treaties and Business Entities}
%			\end{center}

	\subsection{Treaties and Business Entities}

Under Article 4, a resident includes any corporation that is liable to tax because of its place of incorporation or its place of management.  Note, however, that for a corporation, merely being a resident under Article 4 is necessary but not sufficient to avail itself of treaty benefits.  It must also be a qualified resident under Article 23, which is discussed below in Chapter 6.  

Under U.K. law, a corporation is a U.K. resident if it is either formed or controlled and managed in the U.K.  A company is generally managed  and controlled where the board of directors meets.   A corporation formed in the United States but managed and controlled in the United Kingdom can also be a U.K. corporation.  When a corporation is a dual resident, its residence for treaty purposes must be determined by agreement of the competent authorities.  Article 4(5).  

The treatment of partnerships under tax treaties has historically presented many challenges.  In particular, since partnerships are generally not subject to tax, they would generally not be a resident under Article 4.  The partners of a partnership, however, are generally subject to tax on their distributive share of the partnership's income, but the partners may be residents of different countries than the partnership.  The issue thus is how should treaties apply to partnerships, at the partner level or at the partnership level?  


Under Article 1(8), the income of an entity that is treated as a partnership under the domestic laws of \emph{either} country is considered to be derived by a resident of a treaty country only if the resident is treated under the tax laws of the country of residence as deriving the income.  For example, if a U.K. corporation pays a dividend to an entity that is treated as a partnership under U.S. law and has a U.S. partner, since under U.S. law the U.S. partner is treated as receiving a portion (or all) of the dividend, the partner will be entitled to treaty benefits, provided that the partner is a U.S. resident.  The result would be the same even if the entity receiving the dividend were treated under U.K. law as a corporation instead of a partnership.  As another example, if IBM pays a dividend to a U.S. entity that is treated for U.K. purposes as a corporation, under the Treaty, the income is treated as derived by the U.S. entity and not the U.K. partner even if it is treated as fiscally transparent under U.S. law.  The Technical Explanation to Article 1(8) contains some helpful examples. 

As Rev.\@ Rul.\@ 2004-76 below illustrates, it is sometimes not sufficient to know how an entity is treated for U.S. purposes, but it is necessary to know how the entity is treated under the laws and treaties of other countries in order to determine how the entity will be taxed in the United States.

\addcontentsline{toc}{section}{\protect\numberline{}Rev.\@ Rul.\@ 2004-76}
\begin{select}
	\revrul{Rev.\@ Rul.\@ 2004-76}{2004-2 C.B. 111}
\begin{center}\textbf{ISSUE}
\end{center} 
If Corporation A, a resident of both Country X and Country Y under the laws of each country, is 
treated as a resident of Country Y and not of Country X for purposes of the X-Y Convention and, as 
a result, is not liable to tax in Country X by reason of its residence, is it entitled to claim the benefits of the U.S.-X Convention as a resident of Country X or of the U.S.-Y Convention as a resident of Country Y? 
\begin{center}\textbf{FACTS}
\end{center}
\textbf{Situation 1.} Corporation A is incorporated under the laws of Country X. Its place of effective management is situated in Country Y. Corporation A does not have a fixed place of business in Country X. Under the laws of Country X, prior to application of any income tax convention, Corporation A is liable to tax as a resident. Under the laws of Country Y, prior to application of any income tax convention, Corporation A is liable to tax as a resident. Corporation A receives U.S.-source income during the taxable year, with respect to which it seeks benefits under either the U.S. income tax convention with Country X (U.S.-X Convention) or the U.S. income tax convention with Country Y (U.S.-Y Convention). 

The relevant articles of the U.S.-X Convention and the U.S.-Y Convention each provide: 
\begin{quote}Except as provided in this paragraph, for the purposes of this Convention, the term ``resident of a Contracting State'' means any person who, under the laws of that State, is liable to tax therein by reason of his domicile, residence, citizenship, place of management, place of incorporation, or any other criterion of a similar nature. 
* * * \\
The term ``resident of a Contracting State'' does not include any person who is liable to tax in that State in respect only of income from sources in that State.
\end{quote}
There is in force an income tax convention between Country X and Country Y (the X-Y Convention) that contains the following article with respect to residence: 
\begin{quote}
For purposes of the Convention, the term ``resident of a Contracting State'' means any 
person who, under the laws of that State, is liable to tax therein by reason of his 
domicile, residence, place of management or any other criterion of a similar nature, and also includes that State and any political subdivision or local authority thereof. This term, however, does not include any person who is liable to tax in that State in respect only of income from sources in that State or capital situated therein. 
* * * \\
Where by reason of the above paragraph, a person other than an individual is a resident of both Contracting States, the person shall be deemed to be a resident only of the State in which its place of effective management is situated.
\end{quote}

\textbf{Situation 2.}  The facts are the same as in Situation 1 except that Corporation A has a fixed place of business in Country X, to which the income is attributable. 

\begin{center}\textbf{LAW AND ANALYSIS}
\end{center} 
In Situation 1, before application of the X-Y Convention, Corporation A would be a resident of both Country X and Country Y under the domestic laws of each of Country X and Country Y. After the application of the relevant article of the X-Y Convention, Corporation A is treated as a resident of Country Y and not a resident of Country X because its place of effective management is situated in Country Y. 

Accordingly, Corporation A continues to be liable to tax in Country Y by reason of residence. Therefore, under the relevant article of the U.S.-Y Convention, Corporation A is a resident of Country Y. Corporation A will be entitled to claim benefits under the U.S.-Y Convention as a resident of Country Y with respect to the U.S.-source income if it satisfies the requirements of the applicable limitation on benefits article, if any, and other applicable requirements in order to receive benefits under the U.S.-Y Convention. 

Because Corporation A is treated as a resident of Country Y for purposes of the X-Y Convention, Corporation A is not subject to comprehensive taxation in Country X as it would be if it were liable to 
tax by reason of residence. Therefore, Corporation A is not a resident of Country X under the relevant article of the U.S.-X Convention and is not entitled to claim benefits under the U.S.-X Convention as a resident of Country X. 

In Situation 2, after the application of the X-Y Convention, Corporation A continues to be liable to tax in Country Y by reason of residence. Therefore, under the relevant article of the U.S.-Y Convention, Corporation A is a resident of Country Y. Corporation A will be entitled to claim benefits under the U.S.-Y Convention as a resident of Country Y with respect to the U.S.-source income if it satisfies the requirements of the applicable limitation on benefits article, if any, and other applicable requirements in order to receive benefits under the U.S.-Y Convention. Because Corporation A is treated as a resident of Country Y for purposes of the X-Y Convention, Corporation A's fixed place of business in Country X is treated as a permanent establishment within the meaning of the X-Y Convention. Thus, Corporation A is liable to tax in Country X in respect of profits attributable to its permanent establishment, but is not subject to 
comprehensive taxation in Country X as it would be if it were liable to tax by reason of residence. Therefore, Corporation A is not a resident of Country X under the relevant article of the U.S.-X Convention and is not entitled to claim benefits under the U.S.-X Convention as a resident of Country X.

Rev. Rul. 73-354, 1973-2 C.B. 435, provided that a bank incorporated in Switzerland, managed and controlled in the United Kingdom, and engaged in the conduct of a business in both Switzerland and the United Kingdom, could choose to apply the provisions of either the United States-Swiss Confederation Income Tax Convention then in force or the United States-United Kingdom Income Tax Convention then in force to interest arising in the United States. Under those conventions, which are 
no longer in force, the determination of whether a corporation was a resident did not depend on whether the corporation was liable to tax in that country. 
\begin{center} \textbf{HOLDING}
\end{center} 
If Corporation A is treated as a resident of Country Y and not of Country X for purposes of the X-Y Convention and, as a result, is not liable to tax in Country X by reason of its residence, it is not entitled to claim benefits under the U.S.-X Convention, because it is not a resident of Country X under the relevant article of the U.S.-X Convention. However, Corporation A is entitled to claim benefits under the U.S.-Y Convention as a resident of Country Y, if it satisfies the requirements of the applicable limitation on benefits article, if any, and other applicable requirements in order to 
receive benefits under the U.S.-Y Convention. 

This holding is applicable in interpreting income tax treaties that contain provisions that are the same as or similar to the relevant articles of the U.S.-X Convention, the U.S.-Y Convention, and the X-Y Convention. \ldots \\
Rev. Rul. 73-354, 1973-2 C.B. 435, is obsolete. 
\end{select}

\addcontentsline{toc}{section}{\protect\numberline{}Comments}
	\begin{center}
		\textbf{\emph{Comments}}
			\end{center}

	\begin{enumerate}

	\item Since the enactment of the check-the-box rules, it has become relatively easy to create an entity that is treated as a passthrough (no tax at the entity level) for U.S. tax purposes and a corporation for foreign tax purposes.  These entities are referred to as \emph{hybrid entities}. In contrast, a \emph{reverse hybrid} is an entity treated as a passthrough under foreign law and a separate or opaque entity under U.S. law. For example, if an Irish  subsidiary of Google creates a wholly owned German company that is not a per se corporation, Google can elect to treat the German entity as a disregarded entity.  For U.S. tax purposes, interest or royalty payments between the two entities will have no tax significance because the United States views the two entities as a single Irish entity with a German division or branch, and the payments between the two entities are treated as intra-company transfers.  If the entity is treated as a corporation under German law, however, the interest or royalty payments will have significance for German tax purposes, \emph{i.e.}, the Germany entity may deduct them and reduce German tax, but for U.S. purposes will not be treated as income to the Irish company.  
	
	\item The use of hybrids and reverse hybrids in international tax planning has flourished over the last decade and is a significant part of the BEPs project.  \textit{See} OECD, \textit{Neutralising the Effects of Hybrid Mismatch Arrangements}, Oct. 5, 2015.  The voluminous report focuses on payments that generate a deduction for the payer but that are not included in the payee's income and payments that generate more than one deduction.   
	
	
	\item In response to perceived abuses of hybrids in tax planning, Congress enacted section 894(c) in 1997, which denies treaty benefits to certain hybrid entities.  This provision is discussed in Chapter 7.
	
			
	
	\item 
	The foreign law treatment of U.S. LLCs is unsettled.  Sometimes LLCs are viewed as passthroughs, but other times they are treated as separate (opaque) entities.  The tax consequences to foreign holders of these entities when two countries treat them differently for tax purposes can be catastrophic.  In  \emph{HMRC v. Anson}, (2015) UKSC 44, the U.K. Supreme Court ruled that a Delaware LLC was a passthrough for U.K. tax purposes.   Anson was a VC who set up a Delaware LLC to act as an investment manager to some VC funds.  The funds paid management fees to the Delware LLC, which were distributed to the members, including Anson.  Anson argued that he should be able to credit the U.S. tax (about 45\%) against his U.K. tax liability.  Under U.K. law, a credit is available if the U.K. and U.S. tax were computed on the same profits.  The Supreme Court found that a provision in the LLC agreement requiring all profits to be currently distributed was sufficient to ensure that U.S. and U.K. tax were computed on the same profits.  The Supreme Court had overturned a Court of Appeal decision that had treated the LLC as opaque, which would have subjected Anson to an additional U.K. of 22\% on the LLC's distributed profits (after-U.S. tax) for an effective tax rate of 57.1\%.  
	    
	
	\end{enumerate}


\section{Trusts and Estates}
	\crt{ 7701(a)(30) and (31)}{ 301.7701-4(a) and (b); 301.7701-7(c)(5), Ex. 2; (d)(1)(v), Ex. 2}{Articles 3 (definitions of person); and 4}

Under regulations, trusts are classified as either ordinary trusts or business trusts.  An ordinary trust, which is generally formed to protect or conserve property, is subject to the general Subchapter J rules for taxation of trusts.  \emph{See} \S641 \emph{et seq}.  Business trusts, in contrast, are generally created by the beneficiaries to carry on for-profit activities and are treated as eligible entities under the check-the-box regulations. Reg.\@ \S301.7701-4(a) and (c).  

Prior to 1997, the residence of a trust or estate was determined under (former) sections 7701(a)(30)(D) and (E), which basically provided that a trust or estate was a U.S. person unless it was taxed as a foreign person.  In particular, the residence of a trust or estate was determined by applying the former residence rules applicable to individuals under Reg.\@ 1.871-2(b).  These rules are virtually impossible to apply to legal entities as they require, \emph{inter alia}, an examination of physical presence and intent.  In response to some perceived abusive transactions involving U.S. persons and foreign trusts, Congress amended section 7701(a)(30) in 1996 to provide guidance for determining the residence of a trust.

Under section 7701(a)(E), a trust is a U.S. person if a U.S. court is able to exercise primary supervision over the administration of the trust, and one or more United States persons have the authority to control all substantial decisions of the trust.  A court has primary supervision if the court can determine ``substantially all issues regarding the administration of the entire trust," including maintaining the books and records, filing tax returns, managing and investing the assets of the trust, and defending the trust from suits by creditors, and determining distributions.  Reg.\@ \S301.7701-7(c)(3)(iv) and (v). Substantial decisions include whether and when to distribute income or corpus, the amount of any distributions, the selection of a beneficiary, and whether to terminate the trust.  Reg.\@ \S301.7701-7(d)(1)(ii). The scope of both of these tests are further fleshed out in Reg.\@ \S301.7701-7.

When Congress amended section 7701(a)(30)(E) to provide rules for determining the residence of a trust, it did not address the residence of an estate, which is still determined under the principles of the former residence regulations as interpreted by the courts and IRS.  In reading Revenue Ruling 81-112 below, what advice would you give to a client regarding the location of investment property, such as stocks and bonds?  Should these assets be held directly or indirectly by the estate? Is this sound policy?

Under the Treaty, both trusts and estates are treated as ``persons.''  Article 3(1)(a).  Thus, if a trust or estate is liable to tax as a resident of a treaty country, it will be a resident for treaty purposes.

\addcontentsline{toc}{section}{\protect\numberline{}Rev. Rul. 81-112}
\begin{select}
	\revrul{Rev. Rul. 81-112}{1981-1 C.B. 598}
	\begin{center}\textbf{FACTS}
	\end{center} 
A, a United States citizen by birth, was a resident of Country X for 20 years prior to dying in 1978.  At the time of A's death A's spouse, who was the primary beneficiary of A's estate, was a citizen of Country X.  A's children, who are equal residuary beneficiaries of A's estate, were citizens and residents of the United States.  A's last will and testament was executed in Country X.

Upon A's death, A left an estate that consisted of several businesses incorporated and operated in Country X.  The estate's assets also included certificates of deposit and accounts in foreign banks.  A had no business interests or assets in the United States.

  A company and a bank, both incorporated and operating under the laws of Country X, were granted letters of administration and letters testamentary and hold legal title to all the assets of A's estate.  The estate is not subject to ancillary administration in the United States or any other country.  The administrator and executor are each represented by local counsel. All the income of the estate is from foreign sources.

\begin{center}\textbf{LAW AND ANALYSIS}
\end{center} 

  Section 7701(a)(31) of the Code provides that the terms foreign estate and foreign trust mean an estate or trust, as the case may be, the income of which, from sources without the United States that is not effectively connected with the conduct of a trade or business within the United States, is not includible in gross income under subtitle A.

  Section 641(b) of the Code provides that the taxable income of an estate or trust shall be computed in the same manner as in the case of an individual.

  Section 872(a) of the Code provides that in the case of a nonresident alien individual, gross income includes only--(1) gross income that is derived from sources within the United States and which is not effectively connected with the conduct of a trade or business within the United States; and, (2) gross income which is effectively connected with the conduct of a trade or business within the United States.

  Section 1.871-2(a) of the Income Tax Regulations provides that the term nonresident alien individual means an individual whose residence is not within the United States, and who is not a citizen of the United States.

  In determining whether an estate is a foreign estate under section 7701(a)(31) of the Code, the question is whether the estate is comparable to a nonresident alien individual.  Thus, it must be decided whether the estate is alien and nonresident in the United States.  Rev. Rul. 62-154, 1962-2 C.B. 148, concludes that the standards that have been developed for making these determinations in the case of trusts are equally applicable to estates.  This ruling cites and relies on the case of B. W. Jones Trust v. Commissioner, 46 B.T.A. 531 (1942), aff'd, 132 F.2d 914 (4th Cir. 1943), which sets forth standards for determining the alienage and residency of a trust.

  B. W. Jones Trust concluded that the trust in question there was an alien entity.  In reaching this conclusion, the Board of Tax Appeals considered 1) the country under whose law the trust was created, and 2) the alienages of the settlor, the trustees, and the beneficiaries.

  Applying these standards in the instant case indicates that the estate is an alien entity.  The assets of the estate are located in country X and are administered under the laws of that country. The company and the bank that hold legal title to the assets of the estate are both incorporated and operating under the laws of country X.  Only the alienage of the decedent and the two residuary beneficiaries weigh against alien status for the estate.  These factors by themselves, however, do not prevent the estate from being considered an alien entity.

  With respect to the residency question, B. W. Jones Trust concluded that the trust in question there was a United States resident.  In reaching this conclusion the United States Court of Appeals relied upon the following facts: 1) 90\% of the trust property was securities of United States corporations, 2) these securities were held in the United States by a trustee who was a United States citizen, 3) these securities were traded by that trustee on United States exchanges, and 4) these securities returned income collected by the trustee in the United States and handled from an office maintained in the United States for that purpose.

  The estate in the instant case had none of the indicia of residency that were present in B. W. Jones Trust.  The assets of the estate are held in country X and their management involves no contact with the United States.

\begin{center}\textbf{HOLDING}
\end{center} 

  A's estate is a nonresident alien entity and, therefore, is a foreign estate for purposes of section 7701(a)(31) of the Code. Thus, the estate is only subject to federal income tax on income that is derived from sources within the United States or income that is effectively connected with the conduct of a trade or business within the United States.

  However, for federal estate tax purposes, since A was a United States citizen, the value of all of A's property situated in foreign countries is includible in A's gross estate.  See section 2001(a) and section 2031(a) of the Code.
\end{select}


\section{Residence Problems}
\begin{enumerate}
 
 	\item Stu P. Id, a U.S. citizen, goes to Cancun on spring break in 1980, and having ingested lots of peyote, performs an expatriating act.  The State Department issues a certificate of loss of nationality to Stu.  In 2008, after the effects of the peyote have long worn off, Stu applies to have his citizenship restored, claiming that he never intended to renounce his citizenship.  If it is restored in 2008, what are the U.S. tax consequences to Stu from 1980 to 2008?
	
	\item Ana is a citizen of the U.K.  She has a U.S. ``green card'' permitting her to live permanently in the U.S., but she chooses to live year round in the U.K.  Under the Code, is Ana a resident alien?
	
	\item Paul is a British citizen working for a law firm in London but spends some time working for his firm's New York office.  Under U.S. immigration law, Paul may work in the United States for temporary periods but may not establish permanent residence.  Paul owns a house in London; while in New York, Paul typically stays at a hotel.  Paul enjoys New York, but his family is in London, and he has no intention of applying for a green card.  In 2006, Paul spends 180 days in the U.S.; in 2007 he spends 30 days in the United States, and in 2008 he spends 143 days in the U.S.  Under the Code, is Paul a resident alien in 2006, 2007, or 2008?  Are there any procedural requirements Paul must satisfy? [Reg.\@ \S 301.7701(b)-8(a)(1), (d)]
	
	\item Same facts as previous question, except that Paul is present in the U.S. for 183 days in 2008.  Under the Code, is Paul a resident alien in 2008?
	
	\item Same facts as the previous question.  Under the Treaty, is Paul a resident alien in 2008, assuming that Paul is taxable by the U.K. on a residence basis? [Article 4 and Reg.\@ \S 301.7701(b)-7.]
	
	\item Ana's sister, Elizabeth, entered the U.S. on an F visa to study in New York and is present in the U.S. for the entire year.  Under the Code, is she a U.S. resident?  Are there any procedural requirements Elizabeth must satisfy?  What are the consequences of not complying with the procedural requirements? [Reg.\@ \S 301.7701(b)-8(a)(2) and (d).]
	
		\item Terrance and Phillip, two funny-looking Canadians, sneak over to Detroit to sell illegally copied DVDs every day (even on July 1, Canada Day) and return to their frigid homeland every night.  Under the Code, are they U.S. residents in 2008? [Reg.\@ \S 301.7701(b)-3(e).]
		
		\item An alien can elect section 7701(b)(4) under certain circumstances to be treated a resident alien even though he does not satisfy the day count test.  Under what circumstances would it be advantageous to be taxed on a residence rather than source basis?  \emph{Hint}: What deductions are available to a resident alien that are not available to a nonresident alien?  \emph{See} \S873 and Reg.\@ \S 1.873-1(a)(1)-(5).
		
		\item When a nonresident alien becomes a resident alien, the basis of any property acquired prior to becoming a resident is determined by treating the property as if it had always been subject to U.S. tax jurisdiction.  What tax planning strategies would you recommend for an alien owning property \emph{before} becoming a resident alien? 
	
	\item John, a U.S. citizen, resides in London and is a U.K. resident for tax purposes.  John receives interest on a bond from a U.S. corporation.  He examines the Treaty and discovers that U.K. residents (Article 4) are exempt from U.S. tax on U.S. source interest (Article 11).  John comes to you to confirm that he can use the Treaty to lower his U.S. tax on the interest under Article 11.  What do you tell John?  Is it possible that John will be subject to double taxation, assuming that the U.K. would tax John on a residence basis?  Under the Code and Treaty, would the U.S. grant relief?  [ \S 904(a); Articles 1(4); 11; and 24(6)(b)-(d) (skim very lightly the Technical Explanation for Article 24(6)) ]
		
	\item John, a U.S. citizen, resides in Argentina and receives a dividend from a U.K. corporation.  Assuming that the U.K. generally taxes dividends paid to a foreigner at 30\%, under the Treaty would the U.K. 30\% tax be reduced?  [Treaty, Articles 1(4); 4; and 10(2).]  
	
	\item Can IBM elect to be taxed as a partnership?  
	
	\item John owns an interest in Sodor, a U.K. Public Limited Company (PLC).  Can Sodor elect to be taxed as a partnership?  What if Sodor were a Private Limited Company (Ltd) with 100 members?  (The creditors of a private limited company can reach only the assets of the company to satisfy any unpaid debts.) [Reg.\@ \S 301.7701-1, -2, and -3]
	
	\item John forms a Delaware LLC and is its sole shareholder.  What's the default tax status of the LLC?  [Reg.\@ \S 301.7701-1, -2, and -3]
	
	\item X is organized in the U.K. as a public limited company and in Delaware as an LLC.  How is it taxed, and what's its residence?  What if X were a Ltd? [Reg.\@ \S\S 301.7701-2(b)(9) and 301.7701-5]
	
	\item X, owned by 2 U.S. persons, US1 and US2, is organized as a U.S. LLC and receives a dividend from UKCO.  Is the dividend treated as being received by LLC or US1 and US2 if the U.S. treats the LLC as a partnership?  What if the U.K. views LLC as a corporation?  What if the LLC is treated under U.S. law as a corporation?  [Read carefully and slowly the Technical Explanation to Article 1(8).]                                
	
	\item What are the basic tests under which the validity of a regulation is determined? [Littriello]
		
	\item Amendments to Reg.\@ \S301.7701-5 removed from the definitions of domestic and foreign business entities, the definition of resident foreign corporation, nonresident foreign corporation, resident partnership and nonresident partnership "because these terms have become obsolete due to statutory changes since the final regulations were published in 1960."  Is there still a need to know the \emph{residence} of a partnership?  See Section 861(a)(1) and Reg.\@ 1.861-2(a)(2).

	
\end{enumerate}


\begin{framed}
	Last revised Jan. 10, 2017; residence\_3\_Jan10\_17
	\end{framed}
%	Example 1
%	J, a US citizen (or green card holder), resides in the US and receives a dividend from Repsol, a Spanish corporation.  J is a US resident under Article 4 and entitled to the lower Treaty withholding rate because he has a "substantial presence" in the US.  Although this term is not defined in the Treaty, other treaties, e.g., the US-Switzerland treaty, contain an accompanying reference to section 7701(b)(3), and it should probably be interpreted similarly under the Treaty.

%Example 2
%	J, a US citizen (or green card holder), resides in France and receives a dividend from Repsol, a Spanish corporation.  J is not a US resident under Article 4 because (1) he does not have a "substantial presence" in the US, and (2) he would not be a resident of the US and not France under the principles of tie-breaker provisions of the Treaty. 

%	Example 3
%	J, a Spanish citizen, spends enough time in the US to be US resident alien and is also a Spanish resident under Spanish law.  J would be considered to be a dual resident, and his tax residency under the Treaty would be determined under the tie-breaker provisions of Article 4, � 2.  If J were determined to be a Spanish resident under the Treaty, he would be treated for US tax purposes as a nonresident alien�the Treaty determination of residence would trump the domestic law determination�and would be entitled to treaty benefits with respect to US source income, e.g., lower tax rates on income such as interest, dividends, and royalties.  In addition, the savings clause would not apply because J would not be a US resident; residency for purposes of the savings clause for non-citizens is determined under Article 4 of the Treaty.  

%	If J were determined to a be US resident under the Treaty, he would be entitled treaty benefits with respect to Spanish source income.  The savings clause would probably not apply because Spain does not impose taxes on the basis of citizenship.  See Technical Explanation to Article 1.

%	Example 4
%	J, a US citizen, resides in Spain and thus is a Spanish resident under Spanish law.  J would not be considered to be a US resident Article 4 because he does not have a "substantial presence" in the US.  Under the Treaty, J would be a Spanish resident.  Because J is a US citizen, however, the savings clause would apply, and J would be subject to US tax on a residence basis except for certain income listed in Article 1, � 4 of the Treaty.  If J were subject to both Spanish and US tax, the US would have to grant a credit for Spanish taxes under Article 24, � 3.  The same results would apply if J had a "substantial presence" in the US so that he was a dual resident who was determined to be a Spanish resident under the tie-breaker provision.

%	Example 5
%	J, a US green card holder, resides in Spain and thus is a Spanish resident under Spanish law.  J would not be considered to be a US resident Article 4 because he does not have a "substantial presence" in the US.  Under the Treaty, J would be a Spanish resident.  The savings clause would not apply because J would not be a US resident under Article 4.
	
