\chapter{Foreign Tax Credit}
\crt{78; 901(a) and (b); 902; 903; and 904(a) and (c)}{1.901-2(a)-(c)}{Article 24}

\section{Direct Foreign Tax Credit:  Sections 901 and 903}

A U.S. person transacting business abroad, performing services abroad, or earning an investment return from capital invested abroad may be subject to source basis taxation.  Because the United States taxes its residents, citizens, and corporations on a residence basis, double taxation will result if a U.S. person is also subject to tax on a source basis.  To mitigate double taxation, the United States permits U.S. taxpayers to elect to credit against their U.S. tax liability foreign taxes paid.  These rules are found in sections 901-909.

Section 901(a) and (b) permits a U.S. taxpayer to elect to credit against its U.S. tax liability foreign ``income, war profits, and excess profits taxed paid or accrued to foreign country or possession of the United States.''  The foreign tax credit allowed is subject to the limitations of section 904, under which the credit is limited to the same proportion of the U.S. tax liability as the proportion of foreign source taxable income bears to worldwide taxable income.  Any foreign taxes paid in excess of the amount allowed as a credit is nonrefundable, but can be carried back 1 year and forward ten years.  Section 904(c).  If a taxpayer does not elect the foreign tax credit, any foreign taxes paid can be deducted.  Section 164(a)(3).  A credit is almost always superior to a deduction, but if a foreign country imposes taxes on U.S. source income, a deduction may be the only alternative.  Under section 275, a taxpayer cannot take both a credit and a deduction for foreign taxes.

As a consequence of disputes that arose in the 1970s between oil producers and the government, the U.S. Treasury undertook to draft comprehensive regulations under sections 901 and 903 addressing the issue of income taxes and also establishing parameters for distinguishing between taxes on one hand and royalties and subsidies on the other.

A foreign levy is an income tax if and only if: (1) it is a tax; and (2) the \emph{predominant character} is that of an income tax in the U.S. sense. Reg. \S1.901-2(a).  In distinguishing between taxes and other payments, the regulations state that a foreign levy is not a tax if the taxpayer receives a \emph{specific economic benefit} from a foreign country in exchange for payment pursuant to the levy.  Reg. \S1.902-2(a)(2).  A specific economic benefit is a benefit not made available on substantially the same terms to substantially all persons who are subject to the income tax, for example, a concession to extract government owned petroleum.  Surprisingly, taxes to finance retirement, old age, and unemployment are generally \emph{not} considered payments for specific economic benefits. Reg. \S1.901-2(a)(ii)(C).  Compare this treatment with U.S. social security taxes, which are neither deductible nor creditable.

The predominant character of a foreign tax is that of an income tax in the U.S. sense if the foreign tax reaches net gain--that is, it satisfies requirements of realization, gross receipts and net income--and not a soak-up tax.  Obviously, the touchstone is the U.S. income tax, but there is flexibility in regulations for alternate tax regimes. Reg. \S1.901-2(a)(3).

A tax satisfies the realization requirement if it is imposed no earlier than it would be under U.S. tax rules, or if it is imposed earlier, it is imposed on the change in the value of property and any gain is only taxed once.  Some leeway is permitted. Reg. \S1.901-2(b)(2).  \emph{See below} Rev. Rul. 2002-16, 2002-1 C.B. 740.

A foreign tax is treated as reaching net income, if judged on its predominant character, the tax allows: (1) the recovery of significant costs and expenses attributable under reasonable principles to gross receipts; or (2) the recovery of costs and expenses computed under a method that approximates or exceeds the amount of actual costs and expenses.  Reg. \S1.902-2(b)(4).  

A foreign tax satisfies the gross receipts test if it is imposed on actual gross receipt or gross receipts computed under a method that is likely to produce an amount not greater than fair market value in transactions that gross receipts may not otherwise be clearly reflected, \emph{e.g.}, transactions between related parties.  Reg. \S1.902-2(b)(3).  

The IRS applied these tests to rule that a minimum tax of \pounds30,000 that the U.K. imposed on long-term non-domiciliaries as part of their remittance basis tax regime was a creditable income tax.  \emph{See below} Rev. Rul. 2011-19, 2011-2 C.B. 199.  

In \emph{PPL Corp. v. CIR}, 133 S. Ct. 1897 (2013), the U.S. Supreme Court held that a U.K. windfall profits tax imposed on privatized utility companies was creditable under \S901.  The U.K. statute determined the liability based on the difference between a company's profit-making value and its flotation value (the price at which the utility was sold by the U.K.).  The profit-making value was 9 times the utility's average annual profit over the 4-year period following the sale.  The IRS argued that none of the tests above were satisfied.  The Court rejected the IRS's arguments and found that the substance of the tax, which was determined by an algebraic manipulation of the U.K. statutory formula, revealed that it was a windfall tax of 51.75\% of the amount by which the utility's profits exceed 11\% of its flotation value.  Consequently, the tax satisfied the three regulatory requirements.   

A foreign tax is creditable only if it is not a soak up tax, which is a foreign tax that is imposed only if the residence country taxes the income as well.  For example, assume a country wants to provide a tax incentive for foreign investors to invest but realizes that if the residence country also taxes the income, the tax incentives may be for naught.  Consequently, it could tax only those investors whose home country taxes the source country income and gives a credit for foreign taxes paid.  The foreign investor is indifferent because the tax will be paid either to the country of investment or the home country.  \emph{See} Rev. Rul. 87-39, 1987-1 C.B. 180.

What about taxes on FDAP?  Don't they violate net income rule?  They probably don't qualify as a creditable tax, but may qualify as an \emph{in lieu} tax under section 903.  Under section 903, ``income war profits and excess profits taxes'' include a tax paid in lieu of a tax on income war profits or excess profits (income tax) otherwise generally imposed by any foreign country.   To qualify under 903, (1) the tax must be tax; and (2) the substitution requirement must be met; and (3) the tax cannot be a soak up tax.  Note that the base on which an in lieu tax is imposed need not approximate net income, and it may consist of gross income, gross receipts, or the number of units produced or exported.   
  
 Tax treaties typically provide for relief from double taxation.  If a country does not otherwise have a domestic tax credit provision, a treaty may be the only mechanism to avoid double taxation.  Double taxation under a treaty may be avoided by either a credit or exemption method.  Because the United States has a domestic foreign tax credit mechanism, the relief from double taxation article of U.S. tax treaties are focused on specialized application of the foreign tax credit or address situations in which both countries tax on a residence basis.  \emph{See, e.g.,} Article 24(6) of the Treaty, which provides rules for U.S. citizens who are residents of the U.K.  Treaties may also specify which foreign taxes are eligible to be credited.  This provides a measure of certainty in concluding that a foreign tax is a creditable income tax.  When the provision of a country's tax law are modified, the question arises whether the new tax is still eligible to be credited pursuant to a treaty.  \emph{See} Rev. Rul. 2002-16, 2002-1 C.B. 740.     
 
\addcontentsline{toc}{section}{\protect\numberline{}Rev. Rul. 2002-16}
\begin{select}
\revrul{Rev. Rul. 2002-16 }{2002-1 C.B. 740}
\ldots\\

\begin{center} \textbf{FACTS}
\end{center}

Prior to January 1, 2001, the Netherlands individual income tax (de inkomstenbelasting) was imposed on a single taxable income base, which comprised all the income that a taxpayer received in a year. Various deductions were allowed against this income. Additionally, a dividend and interest allowance could be used against income from savings and investments. 

Effective January 1, 2001, the Netherlands introduced a schedular system of individual taxation that applies to both residents and nonresidents. Under the new system, instead of a single taxable income base there are three separate 
bases. These are referred to as ``Boxes,'' and are organized as follows: 
\begin{quote}
Box 1 includes taxable income from work (including profits and losses from business and professions 
and sales of business property, wages, pensions, and income from partnerships), dividends received by 
security dealers, and imputed income from an owner--occupied home. Expenses related to business profits are 
deductible. Nonresidents are subject to tax on the income in this box from Dutch sources, including imputed 
income from a home within the Netherlands. 

Box 2 includes taxable income derived from a substantial business interest in corporations. Substantial is 
defined as a 5\% or more interest. Two types of income are taxed: dividends and capital gain realized on 
selling assets that form a substantial holding. Acquisition (margin) interest is deductible. Nonresidents are 
subject to tax on the income in this box with regard to substantial interests in Dutch companies. 

Box 3 includes taxable income from savings and investments. Taxable income is the fixed yield (imputed 
income) set at 4\% of the value of the investment assets reduced by certain liabilities. Taxpayers cannot 
reduce their tax burden by proving that their actual rate of return on investments was in fact less than 4\%. 
If income imputed from investment assets is subject to tax under Box 3, any actual interest, dividend, and 
rental income will not be taxed. The investment yield tax applies to assets such as real estate (other than the 
taxpayer's personal residence), stocks and shares, savings deposits, and non--exempt endowment insurance. 
If income from an asset is subject to tax under Box 1 or 2, income will not be imputed with respect to that 
asset for purposes of Box 3. Also, income is not imputed with respect to assets without yield capacity (such
as personal use property). Interest paid and other expenses relating to imputed income taxed under Box 3 are 
not deductible. However, debt that exceeds 2,500 EUR and that is not related to the assets included within 
Box 1 and Box 2 can be deducted from the tax base on which the imputed income of 4\% is computed; 
in addition, all taxpayers are entitled to a tax-free asset allowance of 17,600 EUR. Nonresidents are 
subject to tax on the income in this box from assets within the Netherlands minus related debt. Assets within 
the Netherlands include only immovable property, rights in immovable property, and rights in the profits of a 
company with a registered office within the Netherlands provided that the rights are not in the form of stock. 
\end{quote}

Under the new system, each form of income may be included in only one box. If there is a loss in one box, it may not 
offset positive income in the other two boxes. The loss may, however, be carried over and deducted against income in that 
box in a later year. In addition, if a taxpayer incurs a loss on the complete termination of his or her substantial interest that 
is subject to tax in Box 2, as much as 25\% of that loss may be applied against the Box 1 tax. 

Box 1 taxable income comprises approximately 95\% of the total income tax base. Boxes 2 and 3 comprise 
approximately 1.5\% and 3.5\% respectively of the total income tax base. 
Taxable income within the three boxes is reduced by personal deductions, such as medical expenses, educational 
expenses, donations, and alimony. Personal deductions are used to offset first income in Box 1, then income in Box 
3, and finally income in Box 2. Any excess personal deductions may be carried forward. 
After reduction by personal deductions, taxable income is subject to the following income tax rates in the three boxes 
as follows: 
\begin{quote}
Box 1 Progressive rates of up to 52\%\\ 
Box 2 25\% \\
Box 3 30\% \\
\end{quote}
Various credits are allowed against the taxes of the three boxes combined. Some of these credits, such as the general 
tax credit, child credit, old--age credit, and the handicapped credit, are nonrefundable. Two additional credits, a wage 
credit and a credit for the dividend tax, also are allowed, and may, in some cases, lead to a refund. 

\begin{center} \textbf{LAW AND ANALYSIS}
\end{center}

Under Article 25(4) of the Treaty, a credit may be allowed against U.S. tax liability for the Box 3 tax if, under Article 
2(2), the Box 3 tax is a substantially similar tax imposed in place of a tax that was in force at the time the Treaty was 
signed. Under the general rule of Article 25(4) of the Treaty, Methods of Elimination of Double Taxation, the United 
States treats as an income tax, for which a credit may be allowed under Article 25, the appropriate amount of income tax 
paid or accrued to the Netherlands by or on behalf of a resident or national of the United States. For purposes of 
Article 25(4), the taxes referred to in paragraphs 1(a) and 2 of Article 2, Taxes Covered, are considered income taxes. 

Article 2(1)(a) lists the Netherlands taxes that were in force at the time the Treaty was signed and that were covered 
under the Treaty. These included the Dutch individual income tax, de inkomstenbelasting. 

Under Article 2(2), Netherlands taxes that were not in force at the time the Treaty was signed are nonetheless covered 
taxes, and thus taxes for which a credit may be allowed under Article 25(4), if they are identical or substantially similar 
taxes imposed after the date of signature of the Treaty in addition to, or in place of, the existing taxes. 
In general, the purpose of a Taxes Covered Article is to ensure that tax treaties do not become obsolete due to changes 
in the tax systems of the parties to a treaty. Thus, if identical or substantially similar taxes are imposed in addition to, or in 
place of, the taxes that were in force and covered at the time a treaty was signed, it is appropriate to give effect to the intent 
of the Contracting States, and allow the treaty to continue to apply to the basic income tax structures of Contracting 
States. There is no definitive test for whether a tax is substantially similar to a covered tax; rather, the outcome rests on 
the facts and circumstances of each particular case. If it is concluded that a newly enacted tax is substantially similar to 
a covered tax, it also becomes a covered tax, but remains so only until such time as it is amended. When that occurs, a 
separate analysis must be made in order to determine whether the amended tax is substantially similar to the taxes in force 
at the time the treaty was signed. 

\begin{center} \textbf{HOLDINGS}
\end{center}

Considered in its entirety, the Netherlands Individual Income Tax Act of 2001 imposes taxes that are substantially 
similar to the income tax referred to in Article 2(1)(a) of the Treaty. Because the taxes imposed pursuant to the Netherlands 
Individual Income Tax Act of 2001 are substantially similar to the income tax referred to in Article 2(1)(a) of the Treaty, 
those taxes are covered under Article 2(2), and therefore treated as income taxes for which a credit may be allowed under 
Article 25(4). Accordingly, the tax imposed under Box 3, which forms a part of the Netherlands Individual Income Tax 
Act of 2001, is treated as an income tax for which a credit may be allowed under Article 25(4). 

Taxpayers generally may rely upon Revenue Rulings to determine the tax treatment of their own transactions, and need 
not request a ruling that would apply the principles of a published Revenue Ruling to their own particular cases. However, 
because each Revenue Ruling represents the conclusion of the Service as to the application of the law to the specific 
facts involved, taxpayers, Service personnel, and others concerned are cautioned against reaching the same conclusions 
in other cases unless those cases present facts and circumstances that are substantially the same as those in the Revenue 
Ruling. Treas. Reg. \S 601.601 (d)(2)(v)(e). Accordingly, because the provisions of the Netherlands Individual Income Tax 
Act of 2001 described in this Revenue Ruling are facts on which this Ruling bases its holding, a taxpayer must verify that 
the description is still accurate before relying on the Ruling. A taxpayer may not rely on the Ruling if the Netherlands 
Individual Income Tax Act of 2001 has been altered or changed in any material respect by subsequent Dutch law. 
\end{select}

 
 
 \addcontentsline{toc}{section}{\protect\numberline{}Rev. Rul. 2011-19}
\begin{select}
\revrul{Rev. Rul. 2011-19 }{2011-1 C.B. 199}
\ldots\\

\begin{center} \textbf{FACTS}
\end{center}

Effective as of April 6, 2008, \ldots [U.K. law] allows individuals who are U.K. residents but who are not domiciled in the United Kingdom (\emph{i.e.}, do not intend to live in the United Kingdom permanently) (non-domiciliaries) to elect each year to be taxed on an alternative basis. The default basis of taxation for non-domiciliaries--as it is for any U.K. resident individual--is to pay U.K. income and capital gains taxes on their worldwide income and capital gains (gains), which is known as the arising basis of taxation. The alternative method of taxation is the remittance basis, under which non-domiciliaries with non-U.K.-source income or gains can elect to be taxed on their non-U.K.-source income and gains only when they are remitted to the United Kingdom. Remittance basis taxpayers also are subject to tax on their U.K.-source income and gains, which are taxed in the year in which they arose, under separate statutory provisions applicable to all U.K. resident individuals.

Both the arising basis and the remittance basis of taxation are computed on the basis of realized gross receipts and permit recovery of significant costs and expenses. Losses and deductions allocated to one type of taxable U.K.-source or non-U.K.-source income or gains may be available to offset other categories of taxable income and gains, depending on the type of income or gains involved and the nature of the losses and deductions.

A special rule applies to U.K. residents who are non-domiciliaries age 18 years or older and who were U.K. residents in at least seven of the prior nine taxable years (long-term non-domiciliaries). Under this rule, a long-term non-domiciliary who elects the remittance basis of taxation is required to pay the ["remittance base charge"] RBC of \pounds30,000 in addition to the remittance basis tax. The RBC constitutes tax imposed on the arising basis on part (or all) of a long-term non-domiciliary's unremitted non-U.K.-source income or gains that arose or accrued in that year and were nominated (\emph{i.e.}, identified) by the long-term non-domiciliary, where the nominated income or gains taxed on the arising basis would give rise to an increase in tax of \pounds30,000, after taking into account any credit due for foreign tax paid. A minimum of \pounds1 of non-U.K.-source income or gains must be nominated in order for a long-term non-domiciliary's election to be taxed on the remittance basis to be valid.

A long-term non-domiciliary's required nomination schedule must show specifics concerning the nominated income or gains and the expenses or capital losses that are deducted in arriving at the taxable income or gains giving rise to the \pounds30,000 tax charge. U.K.-source, remitted non-U.K.-source, and nominated income or gains, although separately computed, are combined into a single base and taxed at the graduated rates of U.K. income or capital gains tax that are generally applicable in the relevant tax year, with the rate that applies to nominated income or gains being determined after the tax is calculated on U.K.-source income and gains and on remitted non-U.K.-source income and gains.

The U.K. statute provides ordering rules that treat nominated amounts remitted to the United Kingdom as paid first out of income or gains that have not previously been subject to U.K. tax. If and when nominated income or gains are treated as remitted to the United Kingdom, they will be regarded as previously taxed on the arising basis and the long-term non-domiciliary will not again be subject to U.K. tax with respect to such amounts.

If the long-term non-domiciliary realizes or accrues, but does not nominate, sufficient realized non-U.K.-source income or gains to result in a tax charge of \pounds30,000, the long-term non-domiciliary will be taxed as if sufficient additional realized income or gains had been nominated to result in a tax charge of \pounds30,000. The U.K. statute also provides a rule for when a long-term non-domiciliary elects the remittance basis but does not have sufficient realized income or gains to result in a tax charge of \pounds30,000. In that case, the long-term non-domiciliary is deemed to have sufficient realized income or gains and to have nominated an amount necessary to make the tax charge equal \pounds30,000. Income or gains, whether realized but not nominated, or imputed, which were deemed to have been nominated in order to generate a tax charge of \pounds30,000 are not treated as previously-taxed income and therefore will be subject to tax if and when such income or gains are actually remitted.

\begin{center}
\textbf{LAW AND ANALYSIS}
\end{center}

\ldots A foreign levy is an income tax if and only if (i) it is a tax and (ii) the predominant character of that tax is that of an income tax in the U.S. sense. Reg. \S1.901-2(a)(1).

\begin{center}
\textbf{A. Single Levy or Separate Levies}
\end{center}

Each levy must be analyzed separately to determine if it is an income tax under section 901. Reg. \S1.901-2(d)(1). For purposes of section 901, whether a single levy or separate levies are imposed by a foreign country depends on U.S. principles and not on whether foreign law imposes the levy or levies in a single or separate statutes. Where the base of a levy is different in kind, and not merely in degree, for different classes of persons subject to the levy, the levy is considered for purposes of section 901 to impose separate levies for such classes of persons. For example, regardless of whether they are contained in a single or separate foreign statutes, a foreign levy identical to the tax imposed under section 871(b) on a U.S. nonresident alien individual's income that is effectively connected with the conduct of a U.S. trade or business is a separate levy from a foreign levy identical to the tax imposed by section 1 on the income of a U.S. citizen or resident, as the tax on nonresidents has a more limited scope and therefore is different in kind from the tax on the worldwide income of U.S. citizens and residents.

Where foreign law imposes a levy that is the sum of two or more separately computed amounts, and each such amount is computed by reference to a separate base, separate levies are considered, for purposes of section 901, to be imposed. Amounts are not separately computed if they are computed separately merely for purposes of a preliminary computation and are then combined as a single base. Reg. \S1.901-2(d)(1). For example, where excess deductible expenses allocated to one type of income are applied to reduce other types of income, a single levy exists, since despite a separate preliminary computation the bases are combined before computing the tax due. See  Reg. \S1.901-2(d)(3), Examples (3), (4), and (5).

\begin{center}
\textbf{1. The Remittance Basis and the Arising Basis of Taxation Are Separate Levies}
\end{center}

Under both the arising basis and the remittance basis of taxation, a non-domiciliary is subject to tax on U.K.- source and non-U.K.-source income and gains. However, under the arising basis, a non-domiciliary is subject to tax on worldwide income and gains that arise or accrue in a particular taxable year; while under the remittance basis, a non- domiciliary is subject to tax only on U.K.-source income and gains and on non-U.K.-source income or gains that are remitted in a particular taxable year, whether the non-U.K.-source income or gains arise or accrue in the year remitted or in an earlier year. Thus, the bases of these levies are different in kind, and not merely in degree; therefore, the arising basis and remittance basis of taxation are considered for purposes of section 901 to impose separate levies.

\begin{center}
\textbf{2. The RBC, in Combination with the Remittance Basis of Taxation, is a Single Levy that is a Separate Levy}
\end{center}

All non-domiciliaries who elect the remittance basis of taxation are subject to tax on their U.K.-source and remitted non-U.K.-source income and gains. In addition, long-term non-domiciliaries must pay the RBC on nominated but unremitted non-U.K.-source income and gains (and, if the amount nominated generates a tax charge of less than \pounds30,000, on income or gains realized but not nominated and on imputed income such that the tax charge equals \pounds30,000). Losses and deductions allocated to U.K.-source income or gains, remitted non-U.K.-source income or gains, or nominated but unremitted non-U.K.-source income or gains may offset income or gains in another category in determining the amount of the long-term non-domiciliary's taxable income. Therefore, under Reg. \S1.901-2(d)(1), despite a separate preliminary computation, the long-term non-domiciliary's U.K.-source income or gains, remitted non-U.K.-source income or gains, and unremitted non-U.K.-source income or gains giving rise to the RBC are combined in determining the long-term non-domiciliary's taxable income; therefore, the tax imposed on the sum of the long-term non-domiciliary's three separately computed amounts of income constitute a single levy (the Long-Term Non-Domiciliary (LTND) Levy).

\begin{center}
\textbf{B. The LTND Levy Is a Tax in the U.S. Sense}
\end{center}

A foreign levy is an income tax if and only if it is a tax and the predominant character of that tax is that of an income tax in the U.S. sense. Reg. \S1.901-2(a)(1). A foreign levy is a tax if it requires a compulsory payment pursuant to the authority of a foreign country to levy taxes. Reg. \S1.901-2(a)(2)(i). The LTND Levy is a tax because it is required to be paid pursuant to the authority of the government of the United Kingdom to levy taxes.

\begin{center}
\textbf{C. The Predominant Character of the LTND Levy Is that of an Income Tax in the U.S. Sense}
\end{center}

The predominant character of a foreign tax is that of an income tax in the U.S. sense if the tax is likely to reach net gain in the normal circumstances in which it applies, and liability for the tax is not dependent, by its terms or otherwise, on the availability of a credit for the tax against income tax liability to another country. Reg. \S1.901-2(a)(3). Liability for the LTND Levy is not dependent on the availability of a credit for the LTND Levy against income tax liability to another country.

Thus, whether the LTND Levy has the predominant character of an income tax depends on whether it is likely to reach net gain in the normal circumstances in which it applies. A foreign tax is likely to reach net gain in the normal circumstances in which it applies if and only if the tax, judged on the basis of its predominant character, satisfies each of the realization, gross receipts, and net income requirements set forth in Reg. \S1.901-2(b)(2), (b)(3), and (b)(4), respectively.

\begin{center}
\textbf{1. REALIZATION}
\end{center}

A foreign tax satisfies the realization requirement if, judged on the basis of its predominant character, it is imposed upon or subsequent to the occurrence of events that would result in the realization of income under the income tax pro- visions of the Code. Reg. \S1.901-2(b)(2)(i)(A). A foreign tax that, judged on the basis of its predominant character, is imposed upon the occurrence of realization or pre-realization events described in Reg. \S1.901-2(b)(2)(i) satisfies the realization requirement even if it is also imposed in some situations upon the occurrence of events not described in that paragraph. For example, a foreign tax that, judged on the basis of its predominant character, is imposed upon the occurrence of realization events satisfies the realization requirement even though the base of that tax also includes imputed rental income from a personal residence used by the owner. Reg. \S1.901-2(b)(2)(i).

The U.K.-source and remitted non-U.K.-source income and gains of a long-term non-domiciliary electing the remittance basis of taxation generally are computed on the basis of amounts that satisfy the realization requirement. In addition, the income or gains nominated for purposes of the RBC must be part (or all) of the non-U.K.-source income and gains arising or accruing in that taxable year. Income tax is charged on nominated income, and capital gains tax is charged on nominated gains, as if the arising basis applied for the relevant taxable year. In other words, the nominated income and gains are subject to tax, even though they have not been remitted in the taxable year. Thus, the RBC is imposed on nominated income or nominated gains that have been realized or accrued in the taxable year.

If the long-term non-domiciliary realizes or accrues, but fails to nominate, sufficient income or gains, with the result that the tax charge on nominated income would be less than \pounds30,000, an amount of income or gains is deemed to be nominated so as to make the tax charge equal \pounds30,000. Thus, income or gains that have been realized or accrued but not nominated will be subject to the RBC. If a long-term non-domiciliary elects the remittance basis but does not have sufficient realized or accrued income or gains to make the tax charge equal \pounds30,000, the long-term non-domiciliary is deemed to have sufficient realized or accrued income or gains and to have nominated such imputed income or gains to make the tax charge equal \pounds30,000.

Section 1.901-2(b)(2)(i) states that, as provided in Reg. \S1.901-2(a)(1), a tax either is or is not an income tax, in its entirety, for all persons subject to the tax; therefore, a foreign tax on a base that includes imputed rental income will satisfy the realization requirement even though some persons subject to the tax will on some occasions not be subject to the tax except with respect to such imputed income. However, a foreign tax based only or predominantly on such imputed income would not satisfy the realization requirement. Although it is possible for a long-term non-domiciliary to elect the remittance basis without having sufficient non-U.K.-source income or gains to support a \pounds30,000 tax charge, in which case the base of the tax would include imputed income or gains, it is highly unlikely that substantial numbers of long-term non-domiciliaries in this situation would elect to be taxed on the remittance basis. Accordingly, it is reasonable to conclude that the RBC is not based only or predominantly on such imputed income or gains. The RBC in general is imposed on realized income or gains, whether nominated by the long-term non-domiciliary or considered by the U.K. statute to have been nominated. Thus, judged on its predominant character, the LTND Levy meets the realization test.

\begin{center}
\textbf{2. GROSS RECEIPTS}
\end{center}

A foreign tax satisfies the gross receipts requirement if, judged on the basis of its predominant character, it is imposed on the basis of gross receipts or gross receipts computed under a method that is likely to produce an amount that is not greater than fair market value. Reg. \S1.901-2(b)(3)(i). A foreign tax that, judged on the basis of its predominant character, is imposed on the basis of amounts described in Reg. \S1.901-2(b)(3)(i) satisfies the gross receipts requirement even if it is also imposed on the basis of some amounts not described in that paragraph. As is the case with income and gains that are taxed under either the arising basis or the remittance basis, nominated income and gains subject to the RBC generally are based on gross receipts. Therefore, the LTND Levy meets the gross receipts test.

\begin{center}
\textbf{3. NET INCOME}
\end{center}

A foreign tax satisfies the net income requirement if, judged on the basis of its predominant character, the base of the tax is computed by reducing gross receipts to permit recovery of the significant costs and expenses (including significant capital expenditures) attributable, under reasonable principles, to such gross receipts; or recovery of such significant costs and expenses computed under a method that is likely to produce an amount that approximates, or is greater than, recovery of such significant costs and expenses. Reg. \S1.901-2(b)(4)(i). Since the LTND Levy is imposed on U.K.-source income and gains, remitted non-U.K.-source income and gains, and nominated but unremitted non-U.K.- source income and gains, all of which consist of gross receipts less costs and expenses, the LTND Levy satisfies the net income requirement.

\begin{center}
\textbf{4. Conclusion}
\end{center}

Because the LTND Levy is likely to reach net gain in the normal circumstances in which it applies, it has the predominant character of an income tax in the U.S. sense.

\begin{center}
\textbf{HOLDING}
\end{center}

Because the LTND Levy is a tax (within the meaning of Reg. \S1.901-2(a)(2)), and its predominant character is that of an income tax in the U.S. sense, the LTND Levy, including the Remittance Basis Charge (RBC) of \pounds30,000, is an income tax for which a credit is allowable under section 901. However, a credit for the LTND Levy will be available only if the other legal requirements for obtaining a foreign tax credit are satisfied. For example, an amount paid is treated as a compulsory payment of income tax only to the extent the taxpayer applies the substantive and procedural provisions of foreign law, including elective provisions such as those available under U.K. law relating to the LTND Levy, in such a way as to reduce, over time, the taxpayer's reasonably expected liability under foreign law for income tax. Reg. \S1.901-2(e)(5).

\ldots 
 
 
 \end{select}
 
 

\addcontentsline{toc}{section}{\protect\numberline{}PPL Corp. v. CIR} 
\begin{select}
\caseart{PPL Corp. v. CIR}{133 S. Ct. 1897 (2013)}{Justice Thomas delivered the opinion for a unanimous Court.}


\ldots 


\begin{center}
 I \\
 A
\end{center}

\ldots 
During the 1980's and 1990's, the U. K.'s Conservative Party controlled Parliament and privatized a number of government-owned companies. These companies were sold to private parties through an initial sale of shares, known as a ``flotation." As part of privatization, many companies were required to continue providing services at the same rates they had offered under government control for a fixed period, typically their first four years of private operation. As a result, the companies could only increase profits during this period by operating more efficiently. Responding to market incentives, many of the companies became dramatically more efficient and earned substantial profits in the process.

The U. K.'s Labour Party, which had unsuccessfully opposed privatization, used the companies' profitability as a campaign issue against the Conservative Party. In part because of campaign promises to tax what it characterized as undue profits, the Labour Party defeated the Conservative Party at the polls in 1997. Prior to coming to power, Labour Party leaders hired accounting firm Arthur Andersen to structure a tax that would capture excess, or ``windfall," profits earned during the initial years in which the companies were prohibited from increasing rates. Parliament eventually adopted the tax, which applied only to the regulated companies that were prohibited from raising their rates. \ldots 

In the proceedings below, the parties stipulated that the following formula summarizes the tax imposed by the Labour Party: 

\begin{displaymath}
 \textrm{Tax} = 23\%\;  [ (365\; \textrm{x}\;  (\frac{P}{D}) \; \textrm{x}\; 9)-FV]
 \end{displaymath}


D equals the number of days a company was subject to rate regulation (also known as the ``initial period"), P equals the total profits earned during the initial period, and FV equals the flotation value, or market capitalization value after sale. For 27 of the 32 companies subject to the tax, the number of days in the initial period was 1,461 days (or four years). Of the remaining five companies, one had no tax liability because it did not earn any windfall profits. Three had initial periods close to four years (1,463, 1,456, and 1,380 days). The last was privatized shortly before the Labour Party took power and had an initial period of only 316 days.

The number 9 in the formula was characterized as a price-to-earnings ratio and was selected because it represented the lowest average price-to-earnings ratio of the 32 companies subject to the tax during the relevant period. 1   The statute expressly set its value, and that value was the same for all companies.  The only variables that changed in the windfall tax formula for all the companies were profits (P) and flotation value (FV); the initial period (D) varied for only a few of the companies subject to the tax. The Labour government asserted that the term [365 x (P/D) x 9] represented what the flotation value \emph{should have been given} the assumed price-to-earnings ratio of 9. Thus, it claimed (and the Commissioner here reiterates) that the tax was simply a 23 percent tax on the difference between what the companies' flotation values \emph{should have been} and what they actually were. 

\begin{center}
 B
\end{center}
Petitioner PPL Corporation (PPL) was an owner, through a number of subsidiaries, of 25 percent of South Western Electricity plc, 1 of 12 government-owned electric companies that were privatized in 1990 and that were subject to the tax.  South Western Electricity's total U. K. windfall tax burden was \pounds90,419,265. In its 1997 federal income-tax return, PPL claimed a credit under \S901 for its share of the bill. The [CIR] rejected the claim, but the Tax Court held that the U. K. windfall tax was creditable for U.S. tax purposes under \S901. The Third Circuit reversed. We granted certiorari to resolve a Circuit split concerning the windfall tax's creditability under \S901. 

\begin{center}
II
\end{center}


Internal Revenue Code \S901(b)(1) provides that ``[i]n the case of ... a domestic corporation, the amount of any income, war profits, and excess profits taxes paid or accrued during the taxable year to any foreign country or to any possession of the United States" shall be creditable.  Under relevant Treasury Regulations, ``[a] foreign levy is an income tax if and only if ... [t]he predominant character of that tax is that of an income tax in the U.S. sense." Reg. \S1.901-2(a)(1). The parties agree that Reg. \S1.901-2 applies to this case. That regulation codifies longstanding doctrine dating back to \emph{Biddle v. Commissioner}, 302 U.S. 573, 578-579 (1938), and provides the relevant legal standard.

The regulation establishes several principles relevant to our inquiry. First, the ``predominant character" of a tax, or the normal manner in which a tax applies, is controlling.  (``We are here concerned only with the `standard' or normal tax"). Under this principle, a foreign tax that operates as an income, war profits, or excess profits tax in most instances is creditable, even if it may affect a handful of taxpayers differently. Creditability is an all or nothing proposition. As the Treasury Regulations confirm, ``a tax either is or is not an income tax, in its entirety, for all persons subject to the tax." Reg. \S1.901-2(a)(1). 

Second, the way a foreign government characterizes its tax is not dispositive with respect to the U.S. creditability analysis. See Reg. \S1.901-2(a) (1)(ii) (foreign tax creditable if predominantly ``an income tax in the U.S. sense"). \ldots Instead of the foreign government's characterization of the tax, the crucial inquiry is the tax's economic effect.  In other words, foreign tax creditability depends on whether the tax, if enacted in the U.S., would be an income, war profits, or excess profits tax.

Giving further form to these principles, Reg. \S1.901-2(a) (3)(i) explains that a foreign tax's predominant character is that of a U.S. income tax ``[i]f ... the foreign tax is likely to reach net gain in the normal circumstances in which it applies." The regulation then sets forth three tests for assessing whether a foreign tax reaches net gain. A tax does so if, ``judged on the basis of its predominant character, [it] satisfies each of the realization, gross receipts, and net income requirements set forth in paragraphs (b)(2) , (b)(3) and (b)(4) , respectively, of this section."  Reg. \S1.901-2(b) (1). The tests indicate that net gain (also referred to as net income) consists of realized gross receipts reduced by significant costs and expenses attributable to such gross receipts. A foreign tax that reaches net income, or profits, is creditable. 

\begin{center}
III \\
A
\end{center}

It is undisputed that net income is a component of the U. K.'s ``windfall tax" formula.  Indeed, annual profit is a variable in the tax formula.  It is also undisputed that there is no meaningful difference for our purposes in the accounting principles by which the U. K. and the U.S. calculate profits.  The disagreement instead centers on how to characterize the tax formula the Labour Party adopted.

The Third Circuit, following the Commissioner's lead, believed it could look no further than the tax formula that the Parliament enacted and the way in which the Labour government characterized it. Under that view, the windfall tax must be considered a tax on the difference between a company's flotation value (the total amount investors paid for the company when the government sold it) and an imputed ``profit-making value," defined as a company's ``average annual profit during its `initial period' ... times 9, the assumed price-to-earnings ratio." 665 F. 3d , at 65 . So characterized, the tax captures a portion of the difference between the price at which each company was sold and the price at which the Labour government believed  each company \emph{should have been} sold given the actual profits earned during the initial period. Relying on this characterization, the Third Circuit believed the windfall tax failed at least the Treasury Regulation's realization and gross receipts tests because it reached some artificial form of valuation instead of profits. See id ., at 67 , and n. 3.

In contrast, PPL's position is that the substance of the windfall tax is that of an income tax in the U.S. sense. While recognizing that the tax ostensibly is based on the difference between two values, it argues that every ``variable" in the windfall tax formula except for profits and flotation value is fixed (at least with regard to 27 of the 32 companies). PPL emphasizes that the only way the Labour government was able to calculate the imputed ``profit-making value" at which it claimed companies should have been privatized was by looking after the fact at the \emph{actual profits} earned by each company. In PPL's view, it matters not how the U. K. chose to arrange the formula or what it \emph{claimed} to be taxing, because a tax based on profits above some threshold is an excess profits tax, regardless of how it is mathematically arranged or what labels foreign law places on it. PPL, thus, contends that the windfall taxes it paid meet the Treasury Regulation's tests and are creditable under \S901 .

We agree with PPL and conclude that the predominant character of the windfall tax is that of an excess profits tax, a category of income tax in the U.S. sense. It is important to note that the Labour government's conception of ``profit-making value" as a backward-looking analysis of historic profits is not a recognized valuation method; instead, it is a fictitious value calculated using an imputed price-to-earnings ratio. At trial, one of PPL's expert witnesses explained that `` `9 is not an accurate P/E multiple, and it is not applied to current or expected future earnings.' "  Instead, the windfall tax is a tax on realized net income disguised as a tax on the difference between two values, one of which is completely fictitious.  

The substance of the windfall tax confirms the accuracy of this observation. As already noted, the parties stipulated that the windfall tax could be calculated as follows: 
\begin{displaymath}
 \textrm{Tax} = 23\%\;  [ (365\; \textrm{x}\;  (\frac{P}{D}) \; \textrm{x}\; 9)-FV]
 \end{displaymath}
This formula can be rearranged algebraically to the following formula, which is mathematically and substantively identical:
\begin{displaymath}
 \textrm{Tax} =[\frac{(365\; \textrm{x}\; 9\; \textrm{x}\; 23\%)}{\textrm{D}}] \; \textrm{x}\; \{ \textrm{P}-[FV \; \textrm{x} \; \frac{D}{(365\; \textrm{x}\; 9)}] \}
  \end{displaymath}

The next step is to substitute the actual number of days for D. For 27 of the 32 companies subject to the windfall tax, the number of days was identical, 1,461 (or four years). Inserting that amount for D in the formula yields the following: 
\begin{displaymath}
 \textrm{Tax} =[\frac{(365\; \textrm{x}\; 9\; \textrm{x}\; 23\%)}{\textrm{1,1461}}] \; \textrm{x}\; \{ \textrm{P}-[FV \; \textrm{x} \; \frac{1,461}{(365\; \textrm{x}\; 9)}] \}
  \end{displaymath}
Simplifying the formula by multiplying and dividing numbers reduces the formula to:
\begin{displaymath}
 \textrm{Tax} =51.71\% \; \textrm{x}\; [\textrm{P}\;-(\frac{FV}{9}) \; \textrm{x} \; 4.0027]
  \end{displaymath}

As noted, FV represents the value at which each company was privatized. FV is then divided by 9, the arbitrary ``price-to-earnings ratio" applied to every company. The economic effect is to convert flotation value into the profits a company \emph{should} have earned given the assumed price-to-earnings ratio. See 135 T. C., at 327 (`` `In effect, the way the tax works is to say that the amount of profits you're allowed in any year before you're subject to tax is equal to one-ninth of the flotation price. After that, profits are deemed excess, and there is a tax'" (quoting testimony from the treasurer of South Western Electricity plc)). The annual profits are then multiplied by 4.0027, giving the total ``acceptable" profits (as opposed to windfall profit) that each company's flotation value entitled it to earn during the initial period given the artificial price-to-earnings ratio of 9. This fictitious amount is finally subtracted from \emph{actual} profits, yielding the excess profits, which were taxed at an effective rate of 51.71 percent. 

The rearranged tax formula demonstrates that the windfall tax is economically equivalent to the difference between the profits each company \emph{actually} earned and the amount the Labour government believed it \emph{should} have earned given its flotation value. For the 27 companies that had 1,461-day initial periods, the U. K. tax formula's substantive effect was to impose a 51.71 percent tax on all profits earned above a threshold. That is a classic excess profits tax. See, e.g., Act of Mar. 3, 1917, ch. 159, Tit. II, §201, 39 Stat. 1000 (8 percent tax imposed on excess profits exceeding the sum of \$5,000 plus 8 percent of invested capital).

Of course, other algebraic reformulations of the windfall tax equation are possible. See 665 F. 3d , at 66 ; Brief for Anne Alstott et al. as \emph{Amici Curiae} 21-23 (Alstott Brief). The point of the reformulation is not that it yields a particular percentage (51.75 percent for most of the companies). Rather, the algebraic reformulations illustrate the economic substance of the tax and its interrelationship with net income. 

The Commissioner argues that any algebraic rearrangement is improper, asserting that U.S. courts must take the foreign tax rate as written and accept whatever tax base the foreign tax purports to adopt.  As a result, the Commissioner claims that the analysis begins and ends with the Labour government's choice to characterize its tax base as the difference between ``profit-making value" and flotation value. Such a rigid construction is unwarranted. It cannot be squared with the black-letter principle that ``tax law deals in economic realities, not legal abstractions." \emph{Commissioner v. Southwest Exploration Co.}, 350 U.S. 308, 315 (1956). Given the artificiality of the  U. K.'s method of calculating purported ``value," we follow substance over form and recognize that the windfall tax is nothing more than a tax on actual profits above a threshold. 

\begin{center}
B
\end{center}

We find the Commissioner's other arguments unpersuasive as well. \ldots 

The Commissioner contends that the U. K. was not trying to establish valuation as of the 1997 date on which the windfall tax was enacted but instead was attempting to derive a proper flotation valuation as of each company's flotation date.  The Commissioner asserts that there was no need to estimate future income (as in the case of the gift or estate recipient) because actual revenue numbers for the privatized companies were available.  That argument also misses the mark. It is true, of course, that the companies might have been privatized at higher flotation values had the government recognized how efficient-and thus how profitable-the companies would become. But, the windfall tax requires an underlying concept of value (based on actual ex post earnings) that would be alien to any valuer. Taxing actual, realized net income in hindsight is not the same as considering past income for purposes of estimating future earning potential.

The Commissioner's reliance on Example 3 to the Treasury Regulation's gross receipts test is also misplaced. Reg. \S1.901-2(b)(3)(ii), Ex. 3. That example posits a petroleum tax in which ``gross receipts from extraction income are deemed to equal 105 percent of the fair market value of petroleum extracted. This computation is designed to produce an amount that is greater than the fair market value of actual gross receipts."  Under the example, a tax based on inflated gross receipts is not creditable.

The Third Circuit believed that the same type of algebraic rearrangement used above could also be used to rearrange a tax imposed on Example 3. It hypothesized: 
\begin{quote}
``Say that the tax rate on the hypothetical extraction tax is 20\%. It is true that a 20\% tax on 105\% of receipts is mathematically equivalent to a 21\% tax on 100\% of receipts, the latter of which would satisfy the gross receipts requirement. PPL proposes that we make the same move here, increasing the tax rate from 23\% to 51.75\% so that there is no multiple of receipts in the tax base. But if the regulation allowed us to do that, the example would be a nullity. \emph{Any} tax on a multiple of receipts or profits could satisfy the gross receipts requirement, because we could reduce the starting point of its tax base to 100\% of gross receipts by imagining a higher tax rate." 665 F. 3d , at 67 .
\end{quote}

There are three basic problems with this approach. As the Fifth Circuit correctly recognized, there is a difference between imputed and actual receipts. ``Example 3 hypothesizes a tax on the extraction of petroleum where the income value of the petroleum is deemed to be ... deliberately greater than actual gross receipts." \emph{Entergy Corp.}, 683 F. 3d , at 238 . In contrast, the windfall tax depends on actual figures. \emph{Ibid}. (``There was no need to calculate imputed gross receipts; gross receipts were actually known"). Example 3 simply addresses a different foreign taxation issue. 

The argument also incorrectly equates imputed \emph{gross receipts} under Example 3 with \emph{net income}. See 665 F. 3d , at 67 (``[a]ny tax on a multiple of receipts or profits"). As noted, a tax is creditable only if it applies to realized gross receipts \emph{reduced by significant costs and expenses attributable to such gross receipts}. Reg. \S1.901-2(b)(4)(i). A tax based solely on gross receipts (like the Third Circuit's analysis) would be noncreditable because it would fail the Treasury Regulation's \emph{net income} requirement.

Finally, even if expenses were subtracted from imputed gross receipts before a tax was imposed, the effect of inflating only gross receipts would be to inflate revenue while holding expenses (the other component of net income) constant. A tax imposed on inflated income minus actual expenses is not the same as a tax on net income.

For these reasons, a tax based on imputed gross receipts is not creditable. But, as the Fifth Circuit explained in rejecting the Third Circuit's analysis, Example 3 is ``facially irrelevant" to the analysis of the U. K. windfall tax, which is based on true net income.  

The economic substance  of the U. K. windfall tax is that of a U.S. income tax. The tax is based on net income, and the fact that the Labour government chose to characterize it as a tax on the difference between two values is not dispositive under Reg. \S1.901-2 . Therefore, the tax is creditable under \S901 . 

The judgment of the Third Circuit is reversed. 

 \end{select}
 
 
%In taxing extraction operations, countries often allow deductions based on statutory formulas rather than actual expenses.  The issue arises whether the taxes imposed on such notional profit are income taxes under section 901.  This issue is explored below in the context of the Ontario Mining Tax in \emph{Texasgulf, Inc. v. CIR}.
% 
% \addcontentsline{toc}{section}{\protect\numberline{}Texasgulf, Inc. v. CIR}
%\begin{select}
%\caseart{Texasgulf, Inc. v. CIR}{72 F.3d 209 (2nd Cir. 1999), \emph{aff'g} 107 T.C. 51 (1998)}{STRAUB, Circuit Judge}
%
%\emph{Ed.: Texasgulf, a U.S. corporation, was the common parent of an affiliated group of corporations that included Texasgulf Canada Ltd. (``Canada"), which owned and operated the Kidd Creek Mine, a copper, zinc, lead, and silver deposit near Timmins, Ontario.  Canada was subject to the Ontario Mining Tax (``OMT") for which a foreign tax credit was claimed.}
%
%\ldots
%
%Essential to the resolution of this appeal is an understanding of the nature and operation of the OMT during the relevant taxable years. The versions of the OMT at issue impose a graduated tax on every mine in the Province of Ontario to the extent that the mine's ``profit," as defined for OMT purposes, exceeds a statutory exemption. The tax is generally imposed upon the mine ``operator,"' defined as the party with the right to produce and sell the mine's minerals. \ldots The versions of the OMT in effect between 1978 and 1980 define ``profit" as:    
%	
%	\begin{quote}
%	the difference between, 
%	
%	(a) where the mineral substances raised, taken or gained from the mine are sold as such, the amount of the gross receipts from the output during the taxation year; 
%	
%	(b) where the mineral substances or a part thereof are not sold as such, the amount of the actual market value at the pit's mouth of the mineral substances raised, taken or gained from the mine that are fed into a treatment plant at any mill, smelter or refinery and the product thereof is sold in the taxation year; or 
%	
%	(c) if there is no means of ascertaining the actual market value at the pit's mouth of the mineral substances referred to in clause b, the amount at which the mine assessor appraises the value of such mineral substances, provided that the mine assessor in appraising such value shall deduct, 
%	
%	(i) the processing costs incurred as prescribed or determined by the regulations, and 
%	
%	(ii) an allowance for profit in respect of processing at a rate or rates prescribed by the regulations or determined by the mine assessor, 
%	
%	from the proceeds of the processed mineral substances sold during the taxation  year, and [certain specified] expenses, payments, allowances and deductions \ldots
%	
%	\end{quote}
%	
%	 In defining the ``expenses, payments, allowances and deductions" that may be recovered, the OMT as enacted in 1978, 1979, and 1980 includes mine--related salaries, operating expenses, a depreciation allowance, certain development costs, and expenditures for scientific research conducted in Canada and related to mining in Ontario, but specifically excludes investment interest, cost depletion,\footnote[4]{Specifically, the OMT provision at issue excludes ``depreciation in the value of the mine, mining land or mining property by reason of exhaustion or partial exhaustion of the ore or mineral."}  and royalties paid for production of a mine on privately owned land (``non-Crown royalties").
%
%
%Although the versions of the OMT at issue establish three modes of calculating the appropriate tax, most OMT taxpayers, including Texasgulf Canada, used the third method, described in section (c) of the portion of the OMT quoted above. According to regulations in effect during the relevant period, the ``allowance for profit in respect of processing" for purposes of this method is computed on the basis of a sliding scale, varying based on which processing assets the taxpayer owns and operates and where they are situated.
%
%Specifically, the processing allowance is:
% 
% \begin{quote}
% 
%   8\% of the capital cost of the processing assets for miners who own and operate a concentrator in Canada, but no smelter or refinery;
%   
%   16\% of the capital cost of the processing assets for miners who own and operate a concentrator and a smelter in Canada, but no refinery;
%   
%   20\% of the capital cost of the processing assets for miners who own and operate a processor, a smelter, and a refinery in Canada;
%   
%   30\% (or 25\% after April 10, 1979) of the capital cost of the processing assets for miners who own and operate a processor, a smelter, and a refinery located in Northern Ontario; or
%   
%	35\% (or 30\% after April 10, 1979) of the capital cost of the processing assets for miners who own and operate a processor, a smelter, and a refinery located in Northern Ontario and a semi-fabricating plant in Northern Ontario, where a significant proportion of the input to the plant originates from a mine in Ontario owned and operated by the person liable to pay the tax.
%
%	\end{quote}
% 
%The relevant OMT regulations also provide that the allowance may not be less than 15\% nor more than 65\% of the OMT taxpayer's combined profits. 
%
%\ldots According to the stipulated figures, Texasgulf's processing allowance exceeded its total nonrecoverable expenses in ten of the thirteen years between 1968 and 1980, and Texasgulf's total processing allowance for the thirteen-year period exceeded its total nonrecoverable expenses by approximately \$223 million (Canadian) or 91\%. 
%
%With respect to the industry-wide operation of the OMT, the stipulated data derive from a study conducted by Texasgulf's expert Robert B. Parsons, C.A., a partner in the Toronto office of the accounting firm Price Waterhouse and the author of a textbook on Canadian mining taxation. Parsons examined 213 OMT returns, which collectively account for roughly 80\% of the total OMT paid between 1968 and 1980. The Commissioner has conceded that Parsons' study includes a representative cross-section of OMT taxpayers. Of the 213 returns studied, 70 returns or 32.9\% show nonrecoverable expenses in excess of the processing allowance claimed. Only 145 of the 213 returns show OMT liability. Of the 145, 23 returns or 15.9\% show nonrecoverable expenses in excess of the processing allowance claimed. Over the period from 1968 to 1980, the aggregate processing allowance claimed on the 213 returns exceeds aggregate nonrecoverable expenses incurred by a ratio of 2.7 to 1. For the 145 returns showing OMT liability, the aggregate ratio for 1968-80 is even higher.
%
%The stipulated data also show the magnitudes of the various nonrecoverable expenses as compared to the mine operators' gross receipts. In the aggregate for all  returns studied, interest, non-Crown royalty, and cost depletion expenses constitute 2.8\%, 0.7\%, and less than 0.1\%, respectively, of gross receipts. All other nonrecoverable expenses incurred by the mine operators constitute in the aggregate approximately 1\% of gross receipts.
%
%The Tax Court held a two-day trial in June 1995, at which Texasgulf argued that the version of the OMT at issue meets the net income requirement for creditability under \S 901 on two grounds: first, because the processing allowance effectively compensates for the nonrecoverable expenses, and second, because any non-recoverable expenses are insignificant. In an opinion entered September 9, 1996, the Tax Court adopted Texasgulf's first argument, after concluding that Texasgulf had ``shown that the processing allowance exceeded nonrecoverable expenses both in the aggregate and for the vast majority of OMT taxpayers" and that therefore ``the OMT processing allowance was likely to approximate or exceed the nonrecoverable expenses for the years in issue." Texasgulf Inc. v. CIR, 107 T.C. 51, 67 (1996). The Tax Court did not reach the issue of the significance of the nonrecoverable expenses. \emph{Id.}
%
%\begin{center} \textbf{DISCUSSION}
%\end{center}
%
%The Commissioner contends that the Tax Court erred in holding that the version of the OMT at issue is an income tax creditable under \S901 because the tax neither allows recovery of nor effectively compensates for significant expenses--specifically, interest expense, non-Crown royalties, and cost depletion.\ldots  
%
%
%Section 1.901-2 provides that as a general rule, ``a tax either is or is not an income tax, in its entirety, for all persons subject to the tax." Treas. Reg. \S 1.901-2(a)(1). Under \S 1.901-2, ``[a] foreign levy is an income tax if and only if--(i) it is a tax; and (ii) the predominant character of that tax is that of an income tax in the U.S. sense." \emph{Id.} As a general matter, the predominant character of a foreign tax is that of an income tax in the U.S. sense if ``the foreign tax is likely to reach net gain in the normal circumstances in which it applies."  Treas. Reg. \S 1.901-2(a)(3)(i).  Further, ``[a] foreign tax is likely to reach net gain in the normal circumstances in which it applies if and only if the tax, judged on the basis of its predominant character, satisfies each of the realization, gross receipts, and net income requirements set forth [in \S 1.901-2]." Treas. Reg. \S 1.901-2(b)(1).
%
%The Commissioner concedes that the versions of the OMT at issue are taxes rather than royalties and meet the realization and gross receipts requirements. The only dispute therefore is whether the OMT as enacted and interpreted during the relevant period satisfies the net income requirement.
%
%Section 1.901-2 provides that:
%
%	\begin{quote}
%	 
%   A foreign tax satisfies the net income requirement if, judged on the basis of its predominant character, the base of the tax is computed by reducing gross receipts [as computed under \S 1.901-2] to permit--
%
%(A) Recovery of the significant costs and expenses (including significant capital expenditures) attributable, under reasonable principles, to such gross receipts; or
%
%(B) Recovery of such significant costs and expenses computed under a method that is likely to produce an amount that approximates, or is greater than, recovery of such significant costs and expenses.
%
%	\end{quote}
%
%Treas. Reg. \S 1.901-2(b)(4)(i). Additionally, ``[a] foreign tax law that does not permit recovery of one or more significant costs or expenses, but that provides allowances that effectively compensate for nonrecovery of such significant costs or expenses, is considered to permit recovery of such costs or expenses." \emph{Id.} Finally, ``[a] foreign tax whose base, judged on the basis of its predominant character, is computed by reducing gross receipts by items [described in (A) or (B) above] satisfies the net income requirement even if gross receipts are not reduced by some such items." \emph{Id.}
%
%As it did before the Tax Court, Texasgulf advances two theories for why the OMT meets the net income requirement: first, that the processing allowance effectively compensates for all nonrecoverable expenses, and second, that the nonrecoverable expenses are not significant. In support of its first theory, Texasgulf cites the empirical evidence presented to the Tax Court as to the relative magnitudes of processing allowances and nonrecoverable expenses as reported on OMT returns for Texasgulf and other OMT taxpayers. In weighing this evidence, we focus our attention on how the OMT operates with respect to the entire industry rather than just Texasgulf because under \S 1.901-2, ``a tax either is or is not an income tax, in its entirety, for all persons subject to the tax." Treas. Reg. \S 1.901-(2)(a)(1). At the same time, we are reluctant to rely upon aggregate figures as opposed to return-by-return statistics in determining whether an allowance effectively compensates for nonrecoverable expenses, as aggregate figures may well be skewed by aberrant taxpayers or taxable years.
%
%Without considering the aggregate data, we are able to resolve this appeal based on the return-by-return industry data to which the parties have stipulated. Only 32.9\% of the returns studied show nonrecoverable expenses in excess of the processing allowance claimed. Moreover, among the returns that show an OMT liability, only 15.9\% indicate nonrecoverable expenses that exceed the processing allowance claimed. Given the large size and representative nature of the sample considered, these statistics suffice to show that the Tax Court did not clearly err in finding that the processing allowance was likely to exceed nonrecoverable expenses for the tax years at issue. Texasgulf has therefore met its burden of proving that the predominant character of the OMT as enacted and interpreted during the relevant taxable years is such that the processing allowance effectively compensates for any nonrecoverable costs.
%
%The Commissioner raises two substantial objections to this conclusion. The Commissioner first argues that Texasgulf has not shown that the OMT satisfies the net income requirement because Texasgulf has not shown anything more than an accidental relationship between the processing allowance and the nonrecoverable expenses. At bottom, the Commissioner's argument is that the type of quantitative, empirical evidence presented in this case is not relevant to the creditability inquiry. However, the language of \S 1.901-2-specifically, ``effectively compensate" and ``approximates, or is greater than"--suggests that quantitative empirical evidence may be just as appropriate as qualitative analytic evidence in determining whether a foreign tax meets the net income requirement. We therefore hold that empirical evidence of the type presented in this case may be used to establish that an allowance effectively compensates for nonrecoverable expenses within the meaning of \S 1.901-2(b)(4). 
%
%The Commissioner's second objection is that holding the versions of the OMT at issue in this case to be creditable would run afoul of two precedents: \emph{Inland Steel Co. v. United States}, 230 Ct. Cl. 314, 677 F.2d 72 (Ct. Cl. 1982) (per curiam), and Revenue Ruling 85-16, 1985-1 C.B. 180. In \emph{Inland Steel}, the then-Court of Claims held that the OMT, as enacted in 1964 and 1965, was not eligible for the foreign tax credit under \S901. See  677 F.2d at 87. If for no other reason, \emph{Inland Steel} can be distinguished because it was decided before the promulgation of \S 1.901-2 in 1983. Although \S 1.901-2's preamble reaffirms \emph{Inland Steel}'s general focus upon the extent to which a tax reaches net gain, both the preamble and \S 1.901-2 introduce three detailed tests for conducting the net gain inquiry. The last of the tests includes at least one principle that was not contemplated by \emph{Inland Steel} or the cases cited therein--that is, the idea that ``[a] foreign tax law that does not permit recovery of one or more significant costs or expenses, but that provides allowances that effectively compensate for nonrecovery of such significant costs or expenses, is considered to permit recovery of such costs or expenses." Treas. Reg. \S 1.901-2(b)(4)(i). As explained above, this principle, in conjunction with \S 1.901-2's clear mandate that a tax is or is not an income tax for all persons subject to it, leads us to a different result than that reached in \emph{Inland Steel}
%
%The Commissioner also cites Revenue Ruling 85-16, in which the Internal Revenue Service ruled that OMT payments are not royalties and therefore may not be excluded from the taxpayer's gross income from mining in computing percentage depletion under I.R.C. \S 613. \ldots  While Revenue Rulings are entitled to great deference,\ldots we are not swayed in this instance because the comment on the OMT's creditability is not part of the Revenue Ruling's holding and because we read \S 1.901-2 to dictate a contrary conclusion. 
%
%Having dismissed the Commissioner's objections, we conclude that the OMT as enacted and interpreted in 1978, 1979, and 1980 provides an allowance that effectively compensates for the nonrecoverable expenses and therefore meets the net income requirement for creditability under \S 901. Because the processing allowance effectively compensates for nonrecoverable expenses, we do not reach Texasgulf's alternate argument that the version of the OMT at issue meets the net income requirement on the ground that the nonrecoverable expenses are not significant.
%
%\ldots
% 
%\end{select}   
% 
 
 Rev. Rul. 87-39 addresses soak-up taxes, which are taxes levied only on foreign investors whose home country grants a credit for foreign income taxes. 
 
 
  \addcontentsline{toc}{section}{\protect\numberline{}Rev. Rul. 87-39}
\begin{select}
\revrul{Rev. Rul. 87-39 }{1987-1 C.B. 180}
\ldots\\


\begin{center} \textbf{FACTS}
\end{center}
Law No. 15.767 of September 13, 1985, published in the Uruguayan Official Gazette of September 24, 1985, page 
755--A (effective October 4, 1985), amended article 2, Title 2 of the 1982 Consolidated Tax Law by adding a new letter 
D. The new law imposes a 30 percent tax, withheld at the source of payment, on Uruguayan source dividends and profits 
paid or credited to non-Uruguayan shareholders. However, the tax is imposed only if the country of the recipient's 
domicile allows the tax to be credited against the recipient's domestic income tax liability. The Uruguayan tax authorities 
require any dividend recipient who claims to be exempt from the tax to present a translated certification from its country 
of domicile that the tax will not be creditable in that country.
\begin{center} \textbf{LAW AND ANALYSIS}
\end{center}
 Section 901 of the Code provides that a credit is allowed against United States income tax for the amount of any 
income, war profits, and excess profits taxes paid or accrued to any foreign country. Section 1.901-2(a)(3) of the Income 
Tax Regulations provides that a foreign levy is an income tax for this purpose only if it is a tax, and if the predominant 
character of that tax is an income tax in the United States sense. Section 1.901-2(a)(3)(ii) provides that the predominant 
character of a foreign tax is that of an income tax in the United States sense only to the extent that liability for the tax is 
not dependent on the availability of a credit for the tax against income tax liability to another country. Section 1.901-2(c) 
provides that liability for foreign tax is dependent on the availability of a credit for the tax against income tax liability to 
another country to the extent that the foreign tax would not be imposed but for the availability of such a credit. 

Section 903 of the Code extends the credit available under section 901 to taxes paid in lieu of income taxes. Section 
1.903--1(a)(1) of the regulations provides that a foreign levy is a tax in lieu of an income tax only if it is a tax, and if 
it meets the ``substitution requirement'' of section 1.903-1(b). Section 1.903-1(b)(2) provides that a foreign tax meets the 
substitution requirement only to the extent that liability for the tax is not dependent on the availability of a credit for the 
tax against income tax liability to another country. Accordingly, if a foreign country imposes a withholding tax only if a 
credit for the tax is available from the recipient's country of domicile, the tax is not creditable under section 901 or 903. 
\begin{center} \textbf{HOLDING}
\end{center}
The Uruguayan withholding tax on dividends and profits imposed by Law No. 15.767 of September 13, 1985, is not a 
creditable tax under section 901 or 903 of the Code, since it is imposed only if a credit for the tax is available from the
recipient's country of domicile. This ruling is the official certification by the Internal Revenue Service that the Uruguayan 
withholding tax on dividends and profits is not creditable in the United States. The United States shareholder must supply 
any translations required by the Uruguayan authorities.
\end{select}

\section{Indirect Foreign Tax Credit:  Section 902}

Sections 901 and 903 permit U.S. taxpayers to credit against their U.S. tax liability foreign income taxes that they \emph{directly} pay.  If a U.S. person owns a foreign corporation, foreign taxes paid by the foreign corporation are not creditable under sections 901 and 903 because the U.S. person did not directly pay the taxes.  Recognizing that there are often legitimate business reasons to operate abroad using a foreign subsidiary instead of a branch, Congress enacted section 902, which permits a \emph{U.S. corporation} to treat foreign taxes paid by 10\%--owned subsidiaries as having been paid by the U.S. corporation when the U.S. corporation receives a dividend from the foreign corporation. \margit{Individuals and other entities, such as S corporations, do not qualify for the deemed paid credit under section 902.} Note that section 902 merely deems the U.S. corporation to have paid the foreign taxes actually paid by the foreign subsidiary; to be creditable, the taxes deemed paid must also pass muster under section 901 (or section 903) and are further subject to the limitations of section 904.  

Section 902 applies only if a U.S. corporation owns 10\% or more of the voting stock of the dividend paying foreign corporation when the dividend is paid.  Voting interest is determined by the power to elect members of the board of directors.  Reg.\@ \S 1.902-1(a)(2).   \margit{The focus solely on voting power is questionable.   A 1\% ownership of a large, publicly traded foreign corporation is certainly more significant than a 90\% ownership of a small, privately held company.}  Also, section 902 requires a U.S. corporation to directly own the stock of the dividend paying foreign corporation; there are no stock ownership attribution rules under section 902, even among corporations that file a consolidated return.   \emph{See} First Chicago Corp. v. CIR, 96 TC 421 (1991), \emph{aff'd}, 135 F3d 457 (7th Cir.\@ 1998).  Under section 902(c)(7), however, stock owned by a partnership is treated as being owned proportionately by its partners.  

Foreign investments are often made through chains of foreign subsidiaries due to foreign legal, business, or tax exigencies.  To accommodate these structures, section 902(b) permits under some circumstances a foreign corporation to have deemed paid the foreign taxes of a lower-tier foreign subsidiary when it receives a dividend from the subsidiary.  Very generally, dividends paid by a lower-tier foreign subsidiary bring with them foreign taxes as they are distributed up the ownership chain.  For example, if a third-tier subsidiary pays a dividend to a second-tier foreign subsidiary, the second-tier subsidiary is treated as having paid the foreign taxes of third-tier subsidiary.  When the second-tier subsidiary pays a dividend to first-tier subsidiary, the first-tier subsidiary is treated as having paid taxes of the second-tier subsidiary,  and when the first-tier subsidiary pays a dividend to the U.S. parent, the U.S. parent is deemed to have paid the taxes paid (and deemed paid) by first-tier subsidiary under 902(a).  In general, section 902(b) permits foreign taxes from sixth-tier and higher foreign subsidiaries to be deemed paid by a U.S. corporation.  

Under section 902(b)(1), if a foreign corporation is a member of a \emph{qualified group} and owns 10\% or more of the voting stock of another member of the group, it is treated as having paid its proportionate share of the taxes paid by the lower-tier subsidiary when it receives a dividend from the subsidiary.  A \emph{qualified group} includes: (1) the first tier 10\% subsidiary; and (2) any other foreign corporation \emph{not below the sixth tier}, provided that the U.S. parent owns indirectly at least 5\% of the voting stock of each foreign subsidiary, and each intermediate foreign corporation owns at least 10\% of the voting stock of each foreign corporation immediately below it.  \S902(b)(2).  Furthermore, for foreign corporations below the third tier, they must be controlled foreign corporations (CFCs), \emph{i.e.}, U.S. persons must own more than 50\% of the vote and value, and the domestic parent corporation must own at least 10\% of the vote of the foreign corporation.   

\begin{quote}
\emph{Example}: DC is a U.S. corporation and FC a foreign corporation.  DC owns 30\% of FC1, which owns 40\% of FC2, which owns 50\% of FC3.  Taxes paid by FC1 will be treated as deemed paid by DC under section 902(a) when FC1 pays a dividend to DC.  Taxes paid by FC2 will be treated as deemed paid by FC1 under section 902(b)(1) when FC2 pays a dividend to FC1, because FC1 is a member of a \emph{qualified group}---it is a foreign corporation described in section 902(a)---and it owns more than 10\% of the voting stock of FC2.  In addition, FC2 is also a member of the same group because DC indirectly owns 12\% (30\% * 40\%) and there is a 10\% chain of ownership connecting FC1 and FC2.  Taxes paid by FC3 will also be deemed paid by FC2 when FC3 pays a dividend to FC2, because DC owns indirectly 6\% (30\% * 40\% * 50\%) of FC3, and there is a 10\% chain of ownership connecting FC1, FC2, and FC3.
\end{quote}

The amount of foreign taxes deemed paid under section 902(a) is determined as follows:  \
	\begin{displaymath}
		DPT =  Post '86 \ Foreign Taxes\  \frac{Dividend}{Post '86 E\&Ps}
	\end{displaymath}

Post-'86 foreign income taxes are the foreign income taxes paid for the year of the dividend plus foreign taxes paid for all post-'86 years, to the extent the taxes were not already deemed paid with respect to dividends distributed by the foreign corporation in prior taxable years.  If, however, the foreign corporation did not satisfy the stock ownership requirements under section 902 for a taxable year starting after 1986, the periods used for computing post-'86 foreign income taxes include only periods starting on or after the date that such ownership existed.

Post-'86 undistributed earnings are the earnings and profits (E\&Ps) accumulated in years starting after 1986 through the close of the taxable year in which the dividend is paid.  E\&Ps attempt to measure funds available to be distributed to shareholders.  They are computed by adjusting a corporation's taxable income for certain items that increase cash but don't affect taxable income, \emph{e.g.}, municipal bond interest, and certain items that decrease cash but don't reduce taxable income, \emph{e.g.}, income taxes (foreign or U.S.).  Note that dividends paid during the year in question don't reduce post-'86 undistributed earnings for these purposes.  In computing the post-'86 pool of undistributed earnings, the E\&Ps include the E\&Ps for all taxable years starting after 1986, or a shorter period, if the ownership requirements under section 902 are not met.

When a U.S. corporation receives a dividend from a first-tier foreign corporation, and the dividend brings with it deemed paid credits for foreign taxes paid by the foreign corporation, under section 78, the dividend received is grossed up (increased) by the amount of foreign taxes deemed paid.  The section 78 gross up has the effect of increasing the taxable income of the U.S. parent to equal the pretax income of the subsidiary before the credit is taken.  This prevents the U.S. parent from receiving the benefit of a deduction \emph{and} credit for the same foreign taxes.  It also places the U.S. corporation roughly in the same position it would have been had the U.S. corporation directly earned the foreign income and paid the foreign taxes.  

\begin{quote}
\emph{Example}:  Assume that DC operates a branch in the U.K. and earns 100 and pays 35 of U.K. tax.  DC would have income of 100 and be entitled to a credit for 35, subject to the limitations of section 904.  Now assume that DC owns 100\% of FC1, a U.K. corporation, which earns 100, pays 35 of taxes, and pays a dividend of 65 to DC.  If DC only included 65 and was deemed to have paid 35 of FC1's taxes, it would be in essence deducting the 35 of taxes and claiming a credit as well.  Under section 78, the 65 of dividend would be grossed up by 35 to 100, and DC would be entitled to a credit for the 35 of deemed paid taxes.
\end{quote}

Section 902 applies when dividends are actually paid by lower tier foreign corporations.  It can also apply when a U.S. corporation receives a deemed distribution under the CFC or PFIC rules.  \emph{See} \S\S960 and 1293(f) (section 902 applies to inclusions of subpart F income and current inclusion of income of a qualified electing fund). 

Pre-'87 taxes and E\&Ps are subject to rules of former section 902, prior to its amendment in 1986.  These rules could potentially apply three centuries from now.

In Chief Counsel Advice AM2013-006, the IRS addressed whether post-'86 foreign income taxes must be reduced when a minority shareholder's stock in the foreign corporations is redeemed under \S301(a), which causes a reduction in the post-'86 undistributed E\&Ps of the corporation under \S312(a).

\addcontentsline{toc}{section}{\protect\numberline{}Chief Counsel Advice AM2013-006}
\begin{select}
\revrul{Chief Counsel Advice AM2013-006}{9/30/13 (release date)}
\ldots\\

\begin{center} \textbf{Facts}
\end{center}

U.S. Parent (USP) owns 60\% of the stock of a controlled foreign corporation (CFC). The other 40\% of the stock is owned by an unrelated foreign party (FP). CFC does not earn any subpart F income.

In Year 1, CFC redeems all of the stock owned by FP by way of a distribution of cash.  Pursuant to section 302(b)(3), the redemption is treated as a section 302(a) distribution in full payment in exchange for the stock.

After the redemption, USP owns 100\% of the stock of CFC.

The distribution of cash in the redemption results in a reduction of the post-1986 undistributed earnings of CFC pursuant to section 312(a). Section 312(n)(7) limits this reduction to an amount equal to FP's pro-rata share of CFC's post-1986 undistributed earnings.

\begin{center} \textbf{Law and Analysis}
\end{center}

\begin{center} \textbf{1.  Statute and Regulations}
\end{center}

Section 902 provides that a domestic corporation that owns at least 10\% of the voting stock of a foreign corporation will be deemed to have paid a portion of the foreign income taxes paid by that foreign corporation (the ``section 902 corporation") when it receives a dividend from the section 902 corporation. The amount of the section 902 corporation's foreign income taxes deemed to have been paid is equal to the same proportion of the section 902 corporation's post-1986 foreign income taxes  that the amount of the dividend bears to the section 902 corporation's post-1986 undistributed earnings. The purpose of the section 902 deemed paid credit is to avoid double taxation of a U.S. corporation on income earned through a foreign subsidiary as well as to eliminate the disparity that would otherwise exist between foreign branches and foreign subsidiaries of U.S. corporations by allowing the U.S. corporation a credit corresponding to the direct credit it would have been allowed had the U.S. corporation earned the foreign-taxed income through a foreign branch.  

Section 902(c) defines post-1986 undistributed earnings as the earnings and profits of a section 902 corporation (computed in accordance with sections 964(a) and 986) accumulated in taxable years beginning after 1986 as of the end of the section 902 corporation's taxable year, without diminution by reason of dividends paid during the year. Section 964 and Treas. Reg.\@ \S1.964-1(a)(1) generally provide that the earnings and profits of a foreign corporation are computed substantially as if such corporation were a domestic corporation, that is, in accordance with the rules of section 312. See Treas. Reg.\@ \S1.902-1(a)(9).

Section 902 (c) defines post-1986 foreign income taxes as the foreign income taxes paid by a section 902 corporation with respect to the taxable year in which the dividend is paid as well as foreign income taxes paid with respect to prior taxable years beginning after December 31, 1986, to the extent the foreign taxes are not attributable to dividends distributed by the section 902 corporation in prior taxable years.

Section 902(c)(8) provides authority for regulations as may be necessary or appropriate to carry out the provisions of section 902. 

Treas. Reg.\@ \S1.902-1(a)(8)(i) provides, in part:

\begin{quotation}
Except as provided in paragraphs (a)(10) and (13) of this section, the term post-1986 foreign income taxes of a foreign corporation means the sum of the foreign income taxes paid, accrued, or deemed paid in the taxable year of the foreign corporation in which it distributes a dividend plus the foreign income taxes paid, accrued, or deemed paid in the foreign corporation's prior taxable years beginning after December 31, 1986, to the extent the foreign taxes were not attributable to dividends distributed to, or earnings otherwise included (for example, under section 304, 367 (b), 551, 951 (a), 1248, or 1293) in the income of, a foreign or domestic shareholder in prior taxable years. Except as provided in paragraph (b)(4) of this section, foreign taxes paid or deemed paid by a foreign corporation on or with respect to earnings that were distributed \underline{or otherwise removed} from post-1986 undistributed earnings in prior post-1986 taxable years shall be removed from post-1986 foreign income taxes regardless of whether the shareholder is eligible to compute an amount of foreign taxes deemed paid under section 902, and regardless of whether the shareholder in fact chose to credit foreign income taxes under section 901 for the year of the distribution or inclusion. \underline{Thus, if an amount is distributed or deemed distributed} by a foreign corporation to a United States person that is not a domestic shareholder within the meaning of paragraph (a) (1) of this section (for example, an individual or a corporation that owns less that 10\% of the foreign corporation's voting  stock), or to a foreign person that does not meet the definition of an upper-tier corporation under paragraph (a)(6) of this section, then although no foreign income taxes shall be deemed paid under section 902, foreign income taxes attributable to the distribution or deemed distribution that would have been deemed paid had the shareholder met the ownership requirements of paragraphs (a)(1) through (4) of this section shall be removed from post-1986 foreign income taxes. (Emphasis added.)
\end{quotation}

	\begin{center}
		\textbf{2. History of Statute and Regulations}
	\end{center}

The current post-1986 multi-year ``pooling rules" under section 902(c) revised prior law under which deemed paid credits were determined based on annual accounts of earnings and foreign income taxes. The pooling rules were enacted in 1986 in part to prevent the loss of deemed-paid foreign tax credits in situations in which a section 902 corporation had earnings and profits deficits in one or more years but paid foreign taxes in those years. The pooling rules also were enacted to limit foreign tax credit planning that timed repatriations to take advantage of foreign subsidiaries' varying effective foreign tax rates from one year to another by accumulating earnings in years in which the effective foreign tax rate was low and distributing earnings in years in which the effective tax rate was high. The pooling rules therefore represent Congress' attempt to ensure that the amount of a taxpayer's indirect foreign tax credit is determined based on the average effective rate of foreign tax associated with foreign earnings over time. If there is no corresponding reduction to post-1986 foreign income taxes for taxes associated with post-1986 undistributed earnings that have been eliminated, there will be a distortion in the average effective rate of tax associated with the remaining earnings, thereby frustrating a principal goal of the pooling rules. In addition, the result would run counter to the broader policies of section 902 concerning the avoidance of double taxation and achieving branch/subsidiary parity if USP is allowed a credit for foreign income taxes that relate to earnings of CFC that USP does not take into account as income.

After Treas. Reg.\@ \S1.902-1(a)(8) was issued in proposed form, the Service became aware that some taxpayers were taking the position that a section 902 corporation's post-1986 foreign income taxes were required to be reduced only for taxes attributable to distributions with respect to which a shareholder both was eligible to claim a credit for taxes deemed paid under section 902(a) and in fact elected to credit, rather than deduct, foreign taxes for the taxable year of the distribution.\footnote[6]{T.D. 8708, January 6, 1997. At the time the proposed and final regulations were issued, the language of section 902(c) only referred to reductions for foreign taxes ``deemed paid with respect to dividends distributed," which some taxpayers argued only meant taxes for which a shareholder claimed a deemed paid credit. Section 902(c) was amended later in 1997 (Public Law 105-34) to refer to foreign taxes ``attributable to dividends distributed," so as to preclude the argument that only foreign taxes attributable to earnings distributed to or included in income by a shareholder that was eligible for, and in fact claimed, a deemed-paid foreign tax credit were removed from the pool of post-1986 foreign income taxes.} 

In response to this argument, the text of the proposed regulations was revised in the final regulation by, among other changes, adding the second sentence of the quoted text above. The preamble to the final regulations explains the revisions as follows:

\begin{quotation}
The IRS has not changed its position . . . that the foreign taxes pool must be reduced to account for foreign taxes attributable to all distributions and deemed distributions or inclusions to all shareholders. However, the text of the final regulations has been amended to clarify the rule. The requirement that the foreign taxes pool must be reduced proportionately as the earnings pool is reduced is consistent with the legislative history of the Tax Reform Act of 1986 (Public Law 99-514). 
\end{quotation}

Under the facts set forth above, it is clear that as a result of the Year 2 dividend distribution from CFC, USP will be deemed to have paid an amount of foreign income taxes equal to 100\% of the balance in CFC's pool of post-1986 foreign income taxes as of the end of Year 2. In order to determine the amount of that balance, it is necessary to determine what effect the Year 1 redemption of the stock held by FP has, if any, on CFC's pool of post-1986 foreign income taxes.

Although the distribution of cash by CFC in redemption of the stock held by FP is treated as a sale or exchange transaction under section 302(a) rather than a dividend, section 312(a) and (n)(7) provides that CFC's earnings and profits are decreased by the amount of the distribution to the extent of the ratable share of the earnings and profits attributable to the stock so redeemed. Accordingly, CFC's post-1986 undistributed earnings are reduced to take into account the Year 1 redemption.

At issue is whether there is also a corresponding reduction in CFC's post-1986 foreign income taxes by an amount of taxes attributable to the Year 1 distribution made by CFC in redemption of the stock held by FP. As noted, this distribution is treated as a sale or exchange transaction under section 302 rather than a dividend. The statutory language of section 902(c)(2)  specifically refers to reductions to post-1986 foreign income taxes for foreign taxes attributable to dividends distributed. The regulatory grant of section 902(c)(8) authorizes regulations further to provide such regulations as may be necessary or appropriate to carry out the provisions of this section. The regulation language under Treas. Reg.\@ \S1.902-1 (a) (8) also refers to foreign taxes attributable to earnings that are ``otherwise removed" from post-1986 undistributed earnings. The language of the regulation is sufficiently broad to cover reductions of earnings under section 312(a) related to redemptions that are treated as a sale or exchange transaction under section 302. Accordingly, CFC's post-1986 foreign income taxes are reduced as a result of the Year 1 redemption of the stock held by FP. As discussed below, to interpret the regulation instead as limited to dividend distributions (or deemed distributions or inclusions ) would frustrate the purpose of the statute and regulations evidenced in the history of the pooling rules to avoid distortions in the average effective rate of the remaining foreign taxes in the taxes pool associated with the remaining earnings in the earnings pool.

Taxpayers have argued that this is not a proper interpretation of the language ``otherwise removed" in Treas. Reg.\@ \S1.902-1(a)(8), quoted above, taking into consideration the context of the paragraph.

Under this view, the reference to income ``otherwise included" in the first sentence should be read only to refer to deemed distributions or inclusions out of earnings and profits, given that the parenthetical only includes examples of deemed distributions or inclusions, with no reference to section 302(a) redemptions. The consequence is also urged that the third sentence beginning ``thus, if an amount is distributed or deemed distributed," further narrows the application to dividends or deemed distributions or inclusions. Taking these two sentences into account, it is asserted that the language in the second sentence about earnings that are ``otherwise removed" refers only to earnings removed by dividend or deemed distributions or inclusions.

We disagree with the latter interpretation. Neither the first nor the third sentence requires that reading of the second sentence. The parenthetical in the first sentence describing earnings ``otherwise included" begins with the language ``for example," which indicates it is not intended as an exclusive list. The parenthetical does not include an express reference to section 302 or section 312, but neither does it specifically exclude those Code sections, under which a stock redemption clearly results in a reduction under section 312(a) in earnings and profits, and so creates the potential for distorting the effective rate associated with the remaining pools of foreign taxes and earnings, absent a reduction in the taxes pool for foreign taxes attributable to the earnings being ``otherwise removed" from the earnings pool. Similarly, the third sentence plainly should be read as an example of the application of the rule provided in the second sentence, and not as limiting the scope of that rule.

The suggested interpretation is contrary to the intent and the text of the final regulation and the preamble to the regulations, and more generally to the principles and purpose of the section 902 rules, as discussed above. Interpreting the regulations to require reductions in foreign taxes corresponding to section 302 reductions in earnings resulting from stock redemptions serves the purpose of the section 902 rules, while the interpretation  limiting foreign tax pool reductions to dividend or deemed distributions or inclusions would frustrate the purpose of the pooling rules by distorting the effective tax rate associated with the pools of foreign taxes and earnings remaining after a stock redemption. Under the recognized canons of construction, preference must be given to an available interpretation that best serves the purpose of the statute or regulation being construed. 


\end{select}



\addcontentsline{toc}{section}{\protect\numberline{}Indirect Foreign Tax Credit Problems} 
	\begin{center}
		\textbf{Foreign Tax Credit Problems}
	\end{center}
	\begin{select}
	
\emph{For each of the questions below, consult sections 78, 901, 902, and 903.  ``DC'' refers to a U.S. corporation; ``FC1'' to a U.K. corporation wholly owned by DC; and  ``FS2'' to a U.K. corporation wholly owned by FC1.  Assume that DC's tax rate is 35\%, and that all income falls into the same section 904(d) basket.}  

	\begin{enumerate}
	
	\item For 2005-2007, FC1 has annual income of \$1,000, pays annually U.K. tax of \$300, and in 2007, distributes \$1050 to DC.  DC also earns \$1,000 of U.S. source dividend income in 2007.  Calculate DC's foreign tax credit. [Don't forget about section 78]    
	
	\item For 2005-2007, FC1 has annual income of \$1,000, pays annually UK tax of \$500, and in 2007, distributes \$750 to DC on which the U.K. imposes a 10\% withholding tax.  DC also earns \$1,000 of U.S. source dividend income in 2007.  Calculate DC's foreign tax credit. [Don't forget about section 78] 
	
	\item DC owns 100\% of FC1 and 100\% of FC2---FC1 and FC2 are brother-sister corporations.  FC1 has \$900 of post-'86 E\&Ps and \$900 of post-'86 foreign taxes; FC2 has \$1,000 of post-`86 E\&Ps and \$200 of post-`86 foreign taxes.  In 2007, FC1 distributes \$20 to DC and FC2 distributes \$30 to DC.  With respect to the FC2 distribution, a \$5 withholding tax is imposed.  Calculate DC's foreign tax credit.  [Hint: First determine the total foreign taxes DC has paid or is deemed to have paid.  Then calculate DC's worldwide income, taking into account section 78.  Finally, determine any limitation under section 904.]
	 
	\item In 2007, FS2 (\$600 of post-'86 E\&Ps and \$400 of post-'86 foreign taxes) distributes \$30 to its parent, FS1 (\$800 of post-'86 E\&Ps and \$200 of post-'86 foreign taxes).  There is a \$5 withholding tax levied on the distribution.  In the same year, FS1 distributes \$20 to DC.  There is a \$2 withholding tax levied on the distribution.  DC has \$50 of U.S. source income.  Calculate DC's foreign tax credit. 
	
	%%For problems 3 and 4, add some US source income; for 3, have FC1 rate be greater than 35% to show averaging
	\end{enumerate}
\end{select}

 \begin{framed}
 Last modified: Mar. 24, '17; FTC\_Mar24\_17.tex
 \end{framed}
	
