 
\chapter{Residence, Nationality, and U.S. Tax Jurisdiction}
	\section{Citizens and Residence Basis Taxation}
		\crt{1}{1.1-1(b) and (c)}{1(1), 1(4), and 1(5); 4; and 23(1) and (2) (skim only)}

\intro{This chapter discusses the tax residence of individuals and legal entities.  It focuses first on U.S. citizens and explores the long-standing U.S. position of taxing its citizens (and resident aliens) on their worldwide income, regardless of actual physical residence or economic contacts with the United States.  It then addresses \S 7701(b), which determines when a foreign national is treated as a U.S. resident.  Next, the rules regarding the tax residence of legal entities, such as partnerships and corporations, are covered, including the check-the-box regulations (Reg.\@ \S\S 301.7701-1, 2, and 3), which are without doubt one of most important developments in the U.S. international tax regime in the last twenty years.  Finally, the residence of trusts and estates is briefly addressed.}

%\addcontentsline{toc}{section}{\protect\numberline{}Taxation of Citizens and Residents under U.S. Law} 
%	\begin{center}
			\subsection{Taxation of Citizens and Residents under U.S. Law}
		%\textbf{Taxation of Citizens and Residents under U.S. Law}
%			\end{center}

The notion of residence is one of the cornerstones of the U.S. international tax regime.  U.S. citizens, including dual citizens, and resident aliens are generally taxed on a residence basis, regardless of their actual physical residence, domicile, or economic contacts with the United States. Thus, \margit{The United States is the only country that taxes its non-domiciled citizens and resident aliens on a residence basis.}  all income, regardless of its geographic origin, is subject to U.S. income tax.  \emph{Cook v.\ Tait}, below, recognizes that the constitutional power to levy income taxes on U.S. citizens (and by extension resident aliens) is not tethered at the U.S. border.  Nonresident aliens, in contrast, are taxed on a source basis, and income of a nonresident that does not have any nexus to the United States (foreign source income) is not taxed by the United States.  Because of the fundamental distinction between residence and source basis taxation, one of the first determinations you must make as a tax advisor is your client's tax residence.

A notable exception to residence basis taxation is found in  \S 911, which permits a citizen or resident alien who resides and earns income abroad to elect to exclude from U.S. tax a portion of his foreign earned income (up to \$100,800 for 2015) and other non-cash benefits.  

Aware that the lure of source basis taxation may be an irresistible inducement to well-heeled citizens and resident aliens to renounce their U.S. citizenship or residence, Congress has enacted special income, gift, and estate provisions intended to discourage persons from renouncing their U.S. citizenship or long-term residency solely for tax purposes.  Under \S 877A, certain citizens and long-term resident aliens who renounce their citizenship or abandon their U.S. residency are subject to tax on the unrecognized gain in their property.  In addition, they are also subject to a modified U.S. estate and gift tax regime.  Sections 911 and 877A are discussed below in Chapter 8.   

The Constitution imposes virtually no limits on Congress's power to tax income.  Article I, Section 8, Clause 1 of the Constitution grants Congress the ``the power To Lay and collect Taxes, Duties, Imposts, and Excises..."  Although ``direct taxes" must be apportioned among the states in proportion to their population (Article I, Section 9, Clause 4), the Sixteenth Amendment abolished the apportionment requirement for ``taxes on incomes, from whatever source derived..."

The word ``source" in the Sixteenth Amendment refers to the economic origin or source of the income, \textit{e.g.}, wages or property, and not to geographic source.  Early Treasury regulations extended the income tax to encompass the income of U.S. citizens and resident aliens arising from any geographic source.  The validity of this regulation, and the U.S. constitutional power to tax the worldwide income of its citizens and resident aliens, even those with a foreign domicile, was affirmed in \textit{Cook v. Tait} 

% tie breaker whats subjective what easy to change
\addcontentsline{toc}{section}{\protect\numberline{}Cook v.\ Tait} \begin{select}
\caseart{Cook v.\ Tait}{ 265 U.S. 47 (1924)}{Justice McKennna} delivered the opinion of the Court.

\ldots The tax was imposed under the Revenue Act of 1921, which provides by \S210 (42 Stat. 227, 233): ``That, in lieu of the tax imposed by section 210 of the Revenue Act of 1918, there shall be levied, collected, and paid for each taxable year upon the net income of every individual a normal tax of 8 per centum \margit{Note the maximum federal tax rate.}of the amount of the net income in excess of the credits provided in section 216: Provided, That in the case of a citizen or resident of the United States the rate upon the first \$4,000 of such excess amount shall be 4 per centum.''\footnote[1]{\ldots [R]egulation, No. 62 \ldots provides in Article 3: ``Citizens of the United States except those entitled to the benefits of section 262 [\ldots] wherever resident, are liable to the tax. It makes no difference that they may own no assets within the United States and may receive no income from sources within the United States. Every resident alien individual is liable to the tax, even though his income is wholly from sources outside the United States. Every nonresident alien individual is liable to the tax on his income from sources within the United States.''} 

Plaintiff is a native citizen of the United States and was such when he took up his residence and became domiciled in the City of Mexico.\ldots\ 

The question in the case \ldots\ [is] whether Congress has power to impose a tax upon income received by a native citizen of the United States who, at the time the income was received, was permanently resident and domiciled in the City of Mexico, the income being from real and personal property located in Mexico.
 
Plaintiff assigns against the power not only his rights under the Constitution of the United States but under international law, and in support of the assignments cites many cases. It will be observed that the foundation of the assignments is the fact that the citizen receiving the income, and the property of which it is the product, are outside of the territorial limits of the United States. \margit{Is this what Cook argued?}These two facts, the contention is, exclude the existence of the power to tax. Or to put the contention another way, as to the existence of the power and its exercise, the person receiving the income, and the property from which he receives it, must both be within the territorial limits of the United States to be within the taxing power of the United States. The contention is not justified, and that it is not justified is the necessary deduction of recent cases. \ldots 
%%In United States v.\ Bennett, 232 U.S. 299, the power of the United States to tax a foreign built yacht owned and used during the taxing period outside of the United States by a citizen domiciled in the United States was sustained. The tax passed on was imposed by a tariff act, \ldots\ but necessarily the power does not depend upon the form by which it is exerted.
 \ldots
%%It will be observed that the case contained only one of the conditions of the present case, the property taxed was outside of the United States. In United States v.\ Goelet, 232 U.S. 293, the yacht taxed was outside of the United States but owned by a citizen of the United States who was ``permanently resident and domiciled in a foreign country.'' It was decided that the yacht was not subject to the tax---but this as a matter of construction. Pains were taken to say that the question of power was determined ``wholly irrespective'' of the owner's ``permanent domicile in a foreign country.'' And the Court put out of view the situs of the yacht. That the Court had no doubt of the power to tax was illustrated by reference to the income tax laws of prior years and their express extension to those domiciled abroad. The illustration has pertinence to the case at bar, for the case at bar is concerned with an income tax, and the power to impose it. 

We may make further exposition of the national power as the case depends upon it. It was illustrated at once in United States v.\ Bennett by a contrast with the power of a State. It was pointed out that there were limitations upon the latter that were not on the national power. The taxing power of a State, it was decided, encountered at its borders the taxing power of other States and was limited by them. There was no such limitation, it was pointed  out, upon the national power; and the limitation upon the States affords, it was said, no ground for constructing a barrier around the United States ``shutting that government off from the exertion of powers which inherently belong to it by virtue of its sovereignty.''
 
The contention was rejected that a citizen's property without the limits of the United States derives no benefit from the United States. The contention, it was said, came from the confusion of thought in ``mistaking the scope and extent of the sovereign power of the United States as a nation and its relations to its citizens and their relations to it.'' And that power in its scope and extent, it was decided, is based on the presumption that government by its very nature benefits the citizen and his property wherever found, and that opposition to it holds on to citizenship while it ``belittles and destroys its advantages and blessings by denying the possession by government of an essential power required to make citizenship completely beneficial.'' In other words, the principle was declared that the government, by its very nature, benefits the citizen and his property wherever found and, therefore, has the power to make the benefit complete. \margit{The benefits and burden rationale.}Or to express it another way, the basis of the power to tax was not and cannot be made dependent upon the situs of the property in all cases, it being in or out of the United States, and was not and cannot be made dependent upon the domicile of the citizen, that being in or out of the United States, but upon his relation as citizen to the United States and the relation of the latter to him as citizen. The consequence of the relations is that the native citizen who is taxed may have domicile, and the property from which his income is derived may have situs, in a foreign country and the tax be legal---the government having power to impose the tax.
 
Judgment affirmed. 
\end{select}

	\addcontentsline{toc}{section}{\protect\numberline{}Comments} 
			\begin{center}
		\textbf{\textit{Comments}}
			\end{center}

\begin{enumerate}
	\item

The  \textit{Cook} court invokes the benefits and burden rationale to support its holding.  Under the benefits principle, a person is taxed (the burden) in order to pay for the services (the benefits) provided by the government.  The court did not consider the question of whether the benefits provided by the U.S. government to nonresident citizens are the same as those provided to resident citizens.  A moment's reflection should be sufficient to answer that question in the negative.  Logically extended, the benefits rationale would at least require different tax rates for foreign income and U.S. income.  Perhaps the benefits rationale can be salvaged if one views the minimum benefit provided to all citizens and resident aliens is the right return to and live in the United States.  Finally, the benefits principle of taxation is incompatible with the notion that an aim of government is to redistribute goods and services to those who do not have the means to purchase them in the market.    

Under the more modern ability-to-pay principle, the tax burden should be borne in relation to a person's ability to pay as measured by his income.  Since \$100 of foreign income and \$100 of U.S. income both equally increases a person's ability to pay, both should be included in the tax base.  In addition, including foreign income in the tax base ensures that capital is allocated efficiently.  If foreign income were exempt from tax, U.S. persons would have a tax incentive to shift capital abroad.  

The U.S. tax system departs significantly from the ability-to-pay principle by deferring tax on the business income of the foreign subsidiaries of U.S. multinationals until the income is remitted to the U.S. parent.  This issue, and the U.S. anti-deferral regimes that have been enacted to prevent abusive use of foreign corporation by U.S. persons, is explored more fully below in Chapters 7, 10, and 11.

The benefits and burden principle may still have relevance if it is viewed as jurisdiction principle.  A wealthy foreigner clearly has more ability to pay than a U.S. pauper, but if the foreigner has no economic nexus to the United States, he pays no U.S. tax.  Could the United States tax wealthy foreigners with no nexus to the United States?  Such a regime would certainly raise due process concerns.  And if the United States implemented such a regime, it is certain that other countries would follow, potentially leading to a tax war. But more fundamentally, we do not tax such persons because they have not received any economic benefit from the United States.  

	\item
	Some scholars have questioned U.S. citizenship taxation on the basis that since it imposes tax and compliance barriers it may undermine the important value of free movement.  It may discourage the emigration of talented foreigners and thereby place the United States at a competitive disadvantage.  \textit{See} Ruth Mason, \textit{Citizenship Taxation}, 89 S. Cal. L. Rev. 169  (2016).  
\end{enumerate}

%	\addcontentsline{toc}{section}{\protect\numberline{}Treaties and Citizens} 
%	\begin{center}
%		\textbf{Tax Treaties and U.S. Citizens and Residents}
%			\end{center}
\subsection{Tax Treaties and U.S. Citizens and Residents}

A tax treaty bestows tax benefits--generally in the form of reduced source basis taxation--only to a treaty \emph{resident}. Article 1(1).  To qualify for treaty benefits, a person (including legal persons) must be a \emph{resident} as determined in Article 4; a legal person, such as a corporation, must also be a \emph{qualified person} under Article 23.  Article 23(1) and (2).  \margit{Residence is defined in Article 4.  Legal entities must also be qualified persons under Article 23.} 

An individual is a treaty resident if he is subject to tax by one of the contracting states ``by reason of his domicile, residence, citizenship \ldots or any other criterion of a similar nature.''  Article 4(1).  Although a U.S. citizen or resident alien will generally qualify as a U.S. treaty resident, Article 4(2), however, requires a U.S. citizen or resident alien with a ``green card'' to satisfy two additional requirements to be a Treaty resident.  First, he must have a ``substantial presence, permanent home, or habitual abode in the United States''; and second, he must not be treated as a treaty resident under any other U.K. treaty with a third country.  Article 4(2).  Consequently, a U.S. citizen or green card holder with minimal physical presence or economic connections to the United States is not a resident under the Treaty.  The Technical Explanation to Article 4(2) states that the second requirement prevents a citizen or resident alien from choosing the (potentially superior) benefits of the Treaty over those of the treaty between the United Kingdom and his foreign country of residence.  

The United States generally negotiates to extend treaty benefits to U.S. citizens and green card holders wherever resident.  This is beneficial to the United States as reducing source basis taxation generally increases the tax revenues of the residence country.  

To illustrate, assume a U.S. citizen whose marginal tax rate is 35\%, receives \$100 of interest from a U.K. corporation that would be taxed at 30\% by the United Kingdom but is taxed at 0\% under Article 11(1) of the Treaty.  If the Treaty did not apply, the United States would also tax the \$100 but would grant a credit for the 30\% U.K. tax paid leaving the U.S. fisc with a residual \$5--\$35 U.S. tax liability less a credit of \$30 of U.K. tax.  As a result of the Treaty, the U.K. tax rate is 0\%, and United States now collects the entire \$35 for an increase in U.S. tax revenues of \$30.  See Example 1.

	\begin{framed}
		\begin{center}
				\textsc{\textbf{Example 1:  Treaties Shift Revenues from Source to Residence Countries}}\\
		\end{center}
P, a U.S. citizen whose marginal tax rate is 35\%, receives  \$100 of interest from a U.K. corporation.  The U.K. tax rate in the absence of the Treaty is 30\%.  Assuming that P can credit the U.K. tax against his (pre-credit) U.S tax liability of \$35, P pays an additional \$5 to the United States, which receives only \$5.  If the Treaty applies, however, the U.K. tax rate is 0\%, and P pays \$35 to the United States.  P pays of total of \$35 in either case, but the Treaty shifts \$30 of revenue from the United Kingdom (source country) to the United States (residence country). 

 	\begin{center}
	  \begin{tabular}{l c c}
  		& No Treaty & Treaty\\
  	\hline
 		 Taxable Income & 100 & 100 \\
  		\ \ \ US Tax (Pre-credit) & 35 & 35 \\
 		 Less credit for U.K. Tax & (30) & (0) \\
  		Residual U.S. Tax & 5 & 35 \\
  		\hline
    		\end{tabular}
   	\end{center} 
   		\end{framed}
Our treaty partners, such as the United Kingdom, however, are generally not so keen to extend treaty benefits to resident aliens or U.S. citizens residing in third countries.  Because most of our treaty partners generally do not tax the worldwide income of their nonresident citizens, any source basis tax concession given by the United States to, for instance, a U.K. citizen residing in Mexico and not subject to U.K. tax on his worldwide income, would not affect U.K. tax revenues.  Assume that a U.K. citizen residing in Mexico receives a royalty for the use of a patent in the United States that is subject to a 30\% U.S. withholding tax.  If the U.K. citizen were able to use the Treaty to reduce the U.S. tax rate to 0\%, the United States would forego \$30 of revenue, but because the United Kingdom does not tax the non-U.K. income of its non-domiciled citizens, U.K. tax revenues would remain unchanged.  Thus, if the United Kingdom were to agree to give up source basis taxes on U.S. citizens and residents residing in third countries, its tax revenues from U.S. persons would decrease, but its tax revenues from its nonresident citizens would remain unchanged even with a reciprocal U.S. concession. 

	
	\section{Dual Citizens}
	
U.S. citizenship can be acquired in many ways: being born in the United States, becoming a naturalized citizen through marriage or residence in the United States, or being born outside of the United States to parents who are U.S. citizens.  A citizen retains his citizenship regardless where he subsequently resides, unless it is renounced.  Many citizens who were born abroad, have resided abroad their entire lives, and possess citizenship of another country may not be aware of their US. citizenship and the U.S. fiscal responsibilities that accompany it.  

A dual citizen of the United States and another country is also subject to residence basis taxation by the United States, unless he renounces his citizenship.  In Rev. Rul. 75-82, 1975-1 C.B. 5, the IRS ruled that a naturalized U.S. citizen who was born in Canada and eventually reestablished Canadian residence did not lose his U.S. citizenship solely by returning to Canada.  In addition, he continued to remain subject to U.S. tax.
\begin{quote}
Since the mere act of returning to and residing in Canada is not one of the acts described in 8 
U.S.C. section 1481 by which United States nationality is lost, and since the individual in the instant case had never performed any of the acts by which United States nationality is lost, he remained a United States citizen when he returned to Canada after attaining majority. Accordingly, he is not relieved of the duty incumbent on United States citizens of filing Federal income tax returns.
\end{quote}

Dual citizens are subject to overlapping residence tax claims by both countries.  The domestic law of each country rarely will provide complete relief against overlapping dual residence taxation, and in the absence of a tax treaty, double taxation will inevitably arise. 

Tax treaties \margit{To eliminate residence basis taxation by two countries, treaties provide for a single tax residence.}mitigate the problem of dual residence taxation by employing a series of tie breaker rules that generally result in the determination of a \emph{single} country of tax residence.  Under Article 4(4), a dual resident is considered to be a resident of the country in which he has a permanent home, where his personal and economic relations are closer, where he maintains a habitual abode, or where he is a national.  These tests are applied in order, so for example, if a dual resident has a permanent home in only one country, he will be a resident of that country regardless of his economic nexus with either country or his nationality.       

To protect residence basis taxation of its citizens residing abroad, the United States generally reserves the right pursuant to the so-called ``savings clause''--Article 1(4)--to tax its citizens and resident aliens regardless of any treaty benefits to which they otherwise may be entitled.  \margit{Under the savings clause, a U.S. citizen cannot generally use the Treaty to reduce U.S. tax.} Thus, even if a U.S. citizen is treated as a U.K. resident under Article 4(4), the savings clause would prevent him from using the Treaty to lower U.S. tax.  As there are almost no rules without exceptions, Article 1(5)(a) and (b) exempt certain narrow categories of income and individuals from the savings clause.\footnote{For instance, a U.S. citizen who is a U.K. resident is not subject to U.S. tax on U.S. social security benefits. Articles 1(5)(a) and 17(3).} But, you may ask yourself, wouldn't a U.S. citizen who's also a U.K. resident potentially be subject to double taxation? The answer is yes, but Article 24(6) of the Treaty coordinates overlapping fiscal claims to ameliorate possible double taxation.     

  \addcontentsline{toc}{section}{\protect\numberline{}Citizenship Regained} 
  \begin{center}
		\textbf{Citizenship Regained}
			\end{center}

If a U.S. citizen has lost or renounced his citizenship and has it restored retroactively, how should he be taxed during the period he was not treated as a U.S. citizen and did not reside in the United States or avail himself of any benefits of citizenship?  Resolving this issue raises questions about the underlying basis on which U.S. residence basis tax is levied.  

In \textit{Felix Benitez Rexach v.\ U.S.}, 390 F.2d 631 (1st. Cir. 1968), \emph{cert.\ denied}, 393 U.S. 833 (1968), Rexach, a U.S. citizen who resided in the Dominican Republic and worked on large scale construction projects, renounced his citizenship in 1958.  When then-Dictator Trujillo was assassinated in 1961, Rexach had a change of heart.  He successfully argued that his renunciation was coerced and had his U.S. citizenship restored \textit{ab inicio}.  After restoring his citizenship, the United States then sued Rexach for income taxes during these years.  Rexach argued that  ``since the United States `owed' him, or apparently owed him, no citizen's protection, he, in turn, owed no tax."  The court rejected Rexach, stating:

\begin{quote} 

While there is language in Cook v.\ Tait, supra, indicative that these are reciprocal obligations, the Court also observed that ``government by its very nature benefits the citizen * * *.'' \ldots We cannot agree that the reciprocal obligations are mutual, at least in the sense that taxpayer contends.  It is sufficient that the government's stem from its de jure relationship without regard to the subjective quid pro quo in any particular case. We will not hold that assessment of benefits is a prerequisite to assessment of taxes.\ldots\footnote{\textit{Felix Benitez Rexach v.\ U.S.}, 390 F.2d 631, 632 (1st. Cir. 1968), \emph{cert.\ denied}, 393 U.S. 833 (1968)}

\end{quote}

 A related case, \textit{U.S. v.\ Lucienne d'Hotelle de Benitez Rexach}, 558 F.2d 37 (1st.\ Cir.\ 1977), involved Lucienne, Felix's wife.  Lucienne was born in France and become a naturalized citizen in 1942.  She returned to France in 1946 and remained a French resident until May 20, 1952. During that time, \S 404(b) of the Nationality Act of 1940 provided that naturalized citizens who returned to their country of birth and resided there for three years lost their American citizenship. Her U.S. passport was renewed in 1947 and 1949, but her citizenship was stripped on May 20, 1952 pursuant to \S 404(b).  The successor statute to \S404(b) was held to be unconstitutional in \textit{Schneider v. Rusk}, 377 U.S. 163 (1964), and its holding was applied retroactively.  Because the Dominican Republic was a community property state, Lucienne legally owned one-half of Felix's income, and the U.S. government sued to collect tax on her share.  Lucienne had accepted her loss of citizenship and never applied to have it reinstated.
 
The First Circuit upheld the government's position that she was liable to U.S. taxes for the years 1949 (the date her citizenship was lost under \S 404(b)) through 1952 (the date a certificate of loss of nationality was issued to her) stating that ``...the balance of the equities mandates that back income taxes be collectible for periods during which the involuntarily expatriated persons affirmatively exercised a specific right of citizenship."  In Lucienne's case, the specific right of citizenship was her possession and use of an American passport.  For the post-1952 years, however, the court said \textit{in dicta} that the government should not be allowed to tax her:  

\begin{quote}
Although estoppel is rarely a proper defense against the government, there are instances where it would be unconscionable to allow the government to reverse an earlier position. \ldots This is one of those instances. Lucienne cannot be dunned for taxes to support the United States government during the years in which she was denied its protection. \ldots Here, Lucienne severed her ties to this country at the direction of the State Department. The right hand will not be permitted to demand payment for something which the left hand has taken away.\footnote{\textit{U.S. v.\ Lucienne d'Hotelle de Benitez Rexach}, 558 F.2d 37, 43 (1st.\ Cir.\ 1977).}
\end{quote}

Why was Felix taxed during his period of non-citizenship but Lucienne was not?  Should the basis on which citizenship was lost and restored matter if it is restored retroactively? If such persons should not be taxed because they did not receive any benefits of citizenship from the United States during the period of non-citizenship, then could it be argued that the foreign source income of U.S. persons residing abroad should also not be taxed or taxed at a lower rate?  Does a nonresident citizen receive the same benefits and protections as a resident citizen, especially with respect to property that is located abroad?  Can \S911 be construed as a partial attempt to implement a modified benefits principle for nonresident citizens? 

In addition to income taxes, the United States also subjects its residents and citizens to U.S. gift, estate, and generation skipping taxes on the worldwide transfers of property and worldwide estates.  The international implications of these taxes are discussed below in Chapter (  ).  Nonresidents, as specially defined for gift and estate tax purposes, are also subject to U.S. gift and estate taxes but generally only with respect to transfers of U.S. situs property.  Thus, a former citizen who regains his U.S. citizenship must not only determine whether he will be subject to income tax on a residence basis while an expatriate, but also whether he will be subject to U.S. gift or estate tax on a residence basis while an expatriate.

%\addcontentsline{toc}{section}{\protect\numberline{}Felix Benitez Rexach v.\ U.S.}
%\begin{select} \caseart{Felix Benitez Rexach v.\ United States}{ 390 F.2d 631 (1st. Cir. 1968), \emph{cert.\ denied}, 393 U.S. 833 (1968)}{Aldrich, Chief Judge.}\\
%\ldots 
%Felix Benitez Rexach, \ldots, an American citizen,\ldots left Puerto Rico [in 1944] and became a resident of the Dominican Republic, where he remained until 1961. In July 1958 he executed a written renunciation of his American citizenship before a United States consulate official in the Dominican Republic pursuant to the Immigration and Nationality Act of 1952, 8 U.S.C. \S 1481(a)(6). A certificate of loss of nationality was duly approved by the Department of State. On July 26 taxpayer was decreed to be a citizen of the Dominican Republic. Thereafter, he naturally suffered certain losses of status and benefits as a consequence of being declared a non-resident alien of the United States. 

%Taxpayer was engaged in large scale contracting activities in the Dominican Republic in connection with the then dictator, Trujillo. In 1961 Trujillo was assassinated. \margit{Another example of the importance of choosing wisely your friends and business associates.}The following year taxpayer applied for an American passport, claiming that his 1958 renunciation was not voluntary but had been compelled, against his will, by economic pressure and physical threats that he feared to resist. The United States Consul denied his application, and taxpayer appealed to the Department of State. The Board of Review on the Loss of Nationality took taxpayer's testimony and accepted it, as a result of which his certificate of loss of nationality was cancelled, and his passport application granted. There followed the present chapter. The [CIR] assessed taxpayer with an income tax on account of income earned in the Dominican Republic during the years following his renunciation of citizenship, alleged to be due because of his continued American citizenship. Cook v.\ Tait\ldots 

%Taxpayer concedes that as a matter of law he is precluded by the record from claiming that he ever ceased to be a United States citizen, and concedes that during the period in question he was a de jure citizen. However, he says that he was not a ``de facto'' citizen.  
%\begin{quote}``Appellant does not claim that his citizenship was lost as a result of the renunciation, but that as a result of the determination of the Secretary of State and consequent issue of the Certificate of Loss of Nationality, the United States was freed of its obligations to him as a citizen and he in fact lived and existed as an alien to the United States during the period in question.''
%\end{quote}
%He concludes that since the United States ``owed'' him, or apparently owed him, no citizen's protection, he, in turn, owed no tax. 

%While there is language in Cook v.\ Tait, supra, indicative that these are reciprocal obligations, the Court also observed that ``government by its very nature benefits the citizen * * *.'' \ldots We cannot agree that the reciprocal obligations are mutual, at least in the sense that taxpayer contends.  \margit{What's left of the benefit and burden rationale?}It is sufficient that the government's stem from its de jure relationship without regard to the subjective quid pro quo in any particular case. We will not hold that assessment of benefits is a prerequisite to assessment of taxes.\ldots
%\end{select}

%\begin{select}
%\addcontentsline{toc}{section}{\protect\numberline{}U.S. v.\ Lucienne Benitez Rexach}
%\caseart{United States v. Lucienne D'Hotelle de Benitez Rexach }{558 F.2d 37 (1st.\ Cir.\ 1977)}{Ingraham, Circuit Judge.}\\
%\ldots
%\begin{center}
%\textbf{FACTS} \\
%\end{center}
%Lucienne D'Hotelle was born in France in 1909. She became Lucienne D'Hotelle de Benitez Rexach upon her marriage to Felix in San Juan, Puerto Rico in 1928. She was naturalized as a United States citizen on December 7, 1942. The couple spent some time in the Dominican Republic, where Felix engaged in harbor construction projects. Lucienne established a residence in her native France on November 10, 1946 and remained a resident until May 20, 1952. During that time \S 404(b) of the Nationality Act of 1940 \ldots provided that naturalized citizens who returned to their country of 
%birth and resided there for three years lost their American citizenship. On November 10, 1947, after Lucienne had been in France for one year, the American Embassy in Paris issued her a United States passport valid through November 9, 1949. Soon after its expiration Lucienne applied in Puerto Rico for a renewal. By this time she had resided in France for three years. Nevertheless, the Governor of Puerto Rico renewed her passport on January 20, 1950 for a two year period beginning November 10, 1949. Three months after the expiration of this passport, Lucienne applied to the United States 
%Consulate in Nice, France for another one. On May 20, 1952, the Vice-Consul there signed a Certificate of Loss of Nationality, citing Lucienne's continuous residence in France as having automatically divested her of citizenship under \S 404(b). Her passport from the Governor of Puerto Rico was confiscated, cancelled and never returned to her. The State Department approved the certificate on December 23, 1952. Lucienne made no attempt to regain her American citizenship; neither did she affirmatively renounce it.
 
%In October 1952 the Dominican Republic (then controlled by the dictator Rafael Trujillo) extended citizenship to Lucienne retroactive to January 2, 1952. \margit{How old were Felix and Lucienne when they married?} Trujillo was assassinated in May 1961. The provisional government which followed revoked Lucienne's citizenship on January 20, 1962. On June 5, 1962 the French government issued her a passport. 

%For the years 1944 to 1958, Felix earned millions of dollars from harbor construction in the Dominican Republic. He was aided by Trujillo's favor and by his own undeniable skills as an engineer. Felix, an American citizen since 1917,\footnote[3]{Felix was born in Puerto Rico on March 27, 1886.\ldots\ [, and he lost his U.S. citizenship in 1958.]  However, the Board of Review on the Loss of Nationality later determined that the events which led to denaturalization were the result of coercion by Trujillo. It adjudged the denaturalization to be void \emph{ab initio}.\ldots} was sued by the United States for income taxes. The court held that Lucienne had a vested one-half interest in Felix's earnings under Dominican law, which established that such income was community property. Since the law of the situs where the income was earned determined its character, Felix could be sued only for his half of the earnings. \ldots
% 
%Predictably, the United States eventually sought to tax Lucienne for her half of that income. Whether by accident or design, the government's efforts began in earnest shortly after the Supreme Court invalidated  the successor statute \ldots to \S 404(b). In Schneider v.\ Rusk, 377 U.S. 163 (1964), the Court held that the distinction drawn by the statute between naturalized and native-born Americans was so discriminatory as to violate due process. In January 1965, about two months after this suit was filed, the State Department notified Lucienne by letter that her expatriation was void under Schneider and that the State Department considered her a citizen. Lucienne replied that she had accepted her denaturalization without protest and had thereafter considered herself not to be an American citizen. 

%Lucienne died on January 18, 1968. During her lifetime, Felix, as administrator of the marital community, retained and administered the community property, including Lucienne's share of the income earned in the Dominican Republic. \margit{Nice guy.  An example of the importance of choosing wisely your spouse.} Upon her death Felix did not return her share to the estate, but retained it. \ldots 

%The district court found that Lucienne was liable for taxes on her half of Felix's income from 1944 through November 9, 1949\ldots.  

%The United States appealed the denial of liability for the period November 10, 1949 to May 20, 1952.\ldots

%\begin{center}
%\textbf{LUCIENNE'S CITIZENSHIP} \\
%\end{center}

% The government contends that Lucienne was still an American citizen from her third anniversary as a French resident until the day the Certificate of Loss of Nationality was issued in Nice. This case presents a curious situation, since usually it is the individual who claims citizenship and the government which denies it. But pocketbook considerations occasionally reverse the roles.\ldots\ The government's position is that under either Schneider v.\ Rusk, supra, or Afroyim v.\ Rusk, 387 U.S. 253 (1967), the statute by which Lucienne was denaturalized is unconstitutional and its prior effects should be wiped out. Afroyim held that Congress lacks the power to strip persons of citizenship merely  because they have voted in a foreign election. The cornerstone of the decision is the proposition that intent to relinquish citizenship is a prerequisite to expatriation. 

%Section 404(b) would have been declared unconstitutional under either Schneider or Afroyim. The statute is practically identical to its successor, which Schneider condemned as discriminatory. \ldots 

%We think the principles governing retrospective application dictate that either Schneider or Afroyim apply to this case. \ldots\ This circuit has applied Afroyim retroactively.\ldots

%\ldots However, the district court went too far in viewing the equities as between Lucienne and the government in strict isolation from broad policy considerations which argue for a generally retrospective application of Afroyim and Schneider to the entire class of persons invalidly expatriated. \ldots\ The rights stemming from American citizenship are so important that, absent special circumstances, they must be recognized even for years past. Unless held to have been citizens without interruption, persons wrongfully expatriated as well as their offspring might be permanently and unreasonably barred from important benefits.\footnote[6]{For example, if expatriation was void \emph{ab initio}, the reinstated citizen will have the satisfaction of knowing that children born in the interim will have the right to become citizens. 8 U.S.C. \S\S1431, 1433, 1434.\ldots\ } Application of Afroyim or Schneider is generally appropriate. 

%Of course, American citizenship implies not only rights but also duties, not the least of which is the payment of taxes. Cook v.\ Tait.\ldots\ And were Schneider or Afroyim used to compel payment of taxes by all persons who mistakenly thought themselves to have been validly expatriated, the calculus favoring retrospective application might shift markedly. We do think that the balance of the equities mandates that back income taxes be collectible for periods during which the involuntarily expatriated persons affirmatively exercised a specific right of citizenship. This is precisely the position taken by the [IRS in Rev. Rul. 75-357].  As to such periods, neither the government nor the expatriate can be said to have relied upon the constitutionality of \S404. Since the expatriate in fact received benefits of citizenship, the equities favor the imposition of federal income tax liability. Cf.\ Benitez Rexach v.\ United States, \ldots. 
% 
%We now focus upon Lucienne's status. The years for which the government sought to collect taxes 
%can be divided into three discrete periods: 1944 through November 9, 1949; November 10, 1949 
%through May 20, 1952; and May 21, 1952 through 1958. The district court's ruling that 
%Lucienne was liable for taxes during the first period is not appealed. The district court refused to 
%distinguish between the two remaining periods. 
%During the interval from late 1949 to mid-1952, Lucienne was unaware that she had been
%automatically denaturalized. In fact, she applied for, obtained and used an American passport for 
%most of that period. On the passport application she stated that her travel outside the United States 
%had consisted of ``vacations," and her signature appeared below an oath that she had neither been 
%naturalized by a foreign state nor declared her allegiance to a foreign state. Her subsequent 
%application on February 11, 1952, which was eventually rejected, included an affidavit in which she 
%stated that her mother's death and other business obligations caused her to remain in France. 
%Ironically, on that same application, the following line appears: 
%\begin{quote}``I (do/do not) pay the American Income Tax at \_\_\_\_\_ ." 
%\end{quote}
%Lucienne scratched out the words ``do not" and filled in the blank with ``San Juan, Puerto Rico." 
%As late as February 1952 Lucienne regarded herself as an American citizen and no one had 
%disabused her of that notion. The Vice Consul reported that Lucienne had told him ``she was 
%advised (by the State Department) that she could remain in France without endangering her 
%American citizenship." 

%Fairness dictates that the United States recover income taxes for the period November 10, 1949 to 
%May 20, 1952. Lucienne was privileged to travel on a United States passport; she received the 
%protection of its government. 

%Although the government has not appealed the decision with respect to taxes from mid-1952 
%through 1958, the district court was presented with the issue. We wish to explain why the 
%government should be allowed to collect taxes for the two and one-half year interval but not for the 
%subsequent period. The letter from Lucienne to the Department of State official in 1965, which 
%appears in English translation in the record, states that after the Certificate of Loss of Nationality, ``I 
%have never considered myself to be a citizen of the United States." We think that in this case this 
%letter can be construed as an acceptance and voluntary relinquishment of citizenship. We also find 
%that in this particular case estoppel would have been proper against the United States. 
%Although estoppel is rarely a proper defense against the government, there are instances where it 
%would be unconscionable to allow the government to reverse an earlier position. \ldots This is one of those 
%instances. \margit{Is this consistent with the treatment of Felix?  Does the benefits and burden rationale still have some life?}Lucienne cannot be dunned for taxes to support the United States government during the 
%years in which she was denied its protection. \ldots Here, Lucienne severed her 
%ties to this country at the direction of the State Department. The right hand will not be permitted to 
%demand payment for something which the left hand has taken away. However, until her citizenship 
%was snatched from her, Lucienne should have expected to honor her 1952 declaration that she was 
%a taxpayer. \ldots
%\end{select}


%\addcontentsline{toc}{section}{\protect\numberline{}Rev. Rul. 75-82}
%\begin{select}
%\revrul{Rev. Rul. 75-82}{1975-1 C.B. 5}
%\ldots\\
%An individual born in Canada in 1951 of British parents came to the United States with his parents in 
%1953 and remained here until 1970. In 1958 his parents became naturalized citizens of the United 
%States, thereby conferring United States citizenship upon the child under 8 U.S.C. section 1432 
%(1970). The individual traveled and lived in other parts of the world from 1970 to 1973, and then he 
%went to Canada where he registered with the United States consul in 1974 as a United States citizen. 
%The individual had never performed any of the acts described in 8 U.S.C. section 1481 (1970) by 
%which nationality is lost. 

%\ldots

%Since the mere act of returning to and residing in Canada is not one of the acts described in 8 
%U.S.C. section 1481 by which United States nationality is lost, and since the individual in the instant 
%case had never performed any of the acts by which United States nationality is lost, he remained a 
%United States citizen when he returned to Canada after attaining majority. Accordingly, he is 
%not relieved of the duty incumbent on United States citizens of filing Federal income tax returns.

%\ldots
%\end{select}
\addcontentsline{toc}{section}{\protect\numberline{}Comments} 
			\begin{center}
		\textbf{\textit{Comments}}
			\end{center}

\begin{enumerate}
	\item
As a result of a series of Supreme Court decisions in the 1960's and 1970's that struck down certain provisions of prior U.S. immigration and nationality laws, many former U.S. citizens were entitled to have their citizenship restored retroactively.  To provide guidance for the tax consequences of the period of non-citizenship, the IRS issued Rev. Rul. 92-109, 1992-2 C.B. 3, which considers four situations: (1) A citizen performed an expatriating act in 1981 and had his citizenship restored retroactively in 1990; (2) A citizen performed an expatriating act in 1979, but has not applied to have his citizenship restored; (3) A citizen performed an expatriating act in 1980, but did not report this act to the Department of State and never lost his citizenship; and (4) A citizen resides outside of the United States and has never performed an expatriating act or filed tax returns.     

For persons in Situation 1, the IRS ruled that they would not be liable for U.S. taxes during the period prior to the restoration of their citizenship.  For persons in Situation 2 whose citizenship is eventually restored, the IRS ruled that they would not be liable for U.S. taxes from the time of expatriation until their first tax year beginning after December 31, 1992.  For person in Situation 3 who believed erroneously they had lost their citizenship, the IRS ruled that they may be eligible for administrative relief to be treated similarly to persons in Situations 1 and 2, provided ``they acted in a manner consistent with a good faith belief that they had lost United States citizenship by, among other things, \margit{The benefits and burden rationale once again.}not affirmatively exercising any rights of United States citizenship in the period when they did not file federal tax returns as United States citizens."  Finally, for persons in Situation 4, no special relief is granted under the ruling.  

Which of the \textit{Rexach} cases does the IRS follow in Situation 1?  In Situation 2?  What is the carrot the IRS holds out for fence sitters, \emph{i.e.}, those persons who are considering applying to have their citizenship restored?

	\item 
	When reading a particular provision treaty, you should remember that the saving clause is generally separately stated and will apply to a U.S. citizen or resident unless the income falls under a particular exception. Treaty provisions cannot always be read in isolation.  
	
	In \emph{LeTourneau v.\@ CIR}, T.C. Memo.\@ 2012-45 (2012), the taxpayer, LeTourneau, was a U.S. citizen and French resident under the U.S.-France Treaty who worked for United Airlines.  She argued that her income was exempt under Art.\@ 15(3) of U.S.-France Treaty [Art.\@ 14(3) of the Treaty], which prohibits source basis taxation of income derived in respect of an employment exercised as a member of a regular complement of a ship or aircraft operated in international traffic.  The Tax Court gave short shrift to LeTourneau's argument: 
	 
	 \begin{quotation}
	 
	 Although this provision on its face seems to favor petitioner's position, it cannot be read in isolation. Unlike many foreign countries, the United States taxes its citizens on their worldwide income. To reserve its right to tax its citizens on the basis of the provisions of the Internal Revenue Code without regard to the provisions of a treaty or convention, the United States typically includes a so-called saving clause in its tax treaties and conventions.  The Convention contains such a saving clause in article 29, paragraph 2, which provides in relevant part: ``Notwithstanding any provision of the Convention except the provisions of paragraph 3, the United States may tax its residents, as determined under Article 4 (Resident), and its citizens as if the Convention had not come into effect."	
	
	Although paragraph 3 of article 29 of the Convention provides that certain articles of the Convention take precedence over the saving clause, article 15, upon which petitioner relies, is not among those provisions. Accordingly, notwithstanding the provisions of article 15, paragraph 3 of the Convention, petitioner is subject to U.S. taxation on her wages earned while residing in France. 
	 
	 \end{quotation}
The court further reminded LeTourneau that the Technical Explanation specifically states that the saving clause permits the United States to tax its citizens under the Code, and that the exemption for crew members operating in international traffic is subject to the saving clause.  Busted.
 
	\item 
	Many U.S. citizens, dual citizens, and resident aliens residing abroad may not be aware of (or intentionally neglect) their U.S. tax filing and reporting obligations.  They do so at considerable risk to their financial well being (and at considerable benefit to the financial well being of their tax advisors).  For example, to exclude foreign earned income under \S911, a U.S. person must make a specific election on his tax return.  
	
	Under \S6038D, a U.S. person that hold interests in foreign financial assets, such as foreign bank accounts or stock or securities in foreign corporations, must disclose annually certain information about these holdings or risk significant, confiscatory financial penalties, and not to mention the separate annual disclosure of any interest in foreign financial accounts with a value in excess of \$10,000 (the so-called FBAR filing).  For some inexplicable reason, the FBAR must be filed separately from a taxpayer's tax return.  In addition, there are myriad reporting requirements covering such events as receiving large gifts from foreign persons (\S6039F) and transferring property to a foreign trust (\S6048).  Finally, pursuant to section 7345, a U.S. citizen can be denied a passport or have his passport revoked if he has \emph{seriously delinquent tax debt}, which is defined to be an unpaid tax liability of greater than \$50,000.    
	
\end{enumerate}




\begin{framed}
	Last updated on Jan. 9, 2019; residence\_1\_Jan9\_17
	\end{framed}

%\addcontentsline{toc}{section}{\protect\numberline{}Rev. Rul. 92-109}
%\begin{select}
%\revrul{Rev. Rul. 92-109}{1992-2 C.B. 3}
%\ldots

%\textbf{\textit{Situation 1.}}
%A is a United States citizen.  On June 17, 1981, A performed an expatriating act, as defined in the 
%Immigration and Nationality Act, section 349, 8 U.S.C. section 1481 (1976 \& Supp. III 1977-1980) 
%(amended 1981, 1986, and 1988).  A's expatriating act did not have for one of its principal purposes 
%the avoidance of federal income, estate, or gift taxes. 

%A's expatriating act was reported to the United States Department of State (``Department of State''). 
%Following review, the Department of State determined that A had lost her United States citizenship, 
%and, on November 16, 1981, approved a certificate of loss of nationality for A. In 1989 A applied to 
%have her loss of United States citizenship administratively reviewed. The Department of State 
%reviewed A's loss of United States citizenship, and determined that A did not intend to relinquish her 
%United States citizenship when she performed her expatriating act. As a result, in 1990 the 
%Department of State vacated A's certificate of loss of nationality, and retroactively restored 
%her United States citizenship. 

%A filed federal income and gift tax returns for 1981, the year she lost her United States citizenship. A 
%has not filed federal income or gift tax returns for 1982 through 1989, the period after the year she 
%lost her United States citizenship and before the year it was retroactively restored. A computes her 
%taxable income on the basis of a calendar year taxable year. 
% 
%\textbf{\textit{Situation 2.}} 
%B is a former United States citizen. On May 24, 1979, B performed an expatriating act, \ldots  B's expatriating act did not have for one of its principal purposes the avoidance of federal income, estate, or gift taxes. 

%B's expatriating act was reported to the Department of State. Following review, the Department of 
%State determined that B had lost his United States citizenship, and, on October 19, 1979, approved a 
%certificate of loss of nationality for B. B has not applied to have his loss of United States citizenship 
%administratively reviewed. 

%B filed federal income and gift tax returns for 1979, the year he lost his United States citizenship. B 
%has not filed federal income or gift tax returns since the 1979 returns. B computes his taxable 
%income on the basis of a calendar year taxable year. 
% 
%\textbf{\textit{Situation 3.}}
%C is a United States citizen. On August 25, 1980, C performed an expatriating act, \ldots C's expatriating act did not have for one of its principal purposes 
%the avoidance of federal income, estate, or gift taxes. 

%C's expatriating act was not reported to the Department of State. As a result, the Department of 
%State did not review C's citizenship status, did not review C's citizenship status, did not determine 
%that she had lost her United States citizenship, and did not approve a certificate of loss of nationality 
%for C. C did not intend to relinquish her United States citizenship when she performed her 
%expatriating act. As a result, if the Department of State had determined that C lost her United
%States citizenship, C would now be eligible to have her citizenship retroactively restored. 

%C filed federal income and gift tax returns for 1980, the year she performed the expatriating act. C 
%has not filed federal income or gift tax returns since the 1980 returns. C computes her taxable 
%income on the basis of a calendar year taxable year. 
% 
%\textbf{\textit{Situation 4.}} 
%D is a United States citizen who resides outside the United States. D has never performed an 
%expatriating act,\ldots. D has not filed federal income or gift tax returns during the period of his foreign 
%residence. \\
%\ldots \\
%%\begin{center}
%%\textbf{LAW}  \\
%%\end{center}
%%\ldots \\
%%Section 2501 of the Code imposes a tax for each calendar year on the transfer of property by gift 
%%during the calendar year by any individual. For gifts made after December 31, 1970, and 
%%before January 1, 1982, the tax imposed by section 2501 is applicable for each calendar quarter. 
%%Section 2511 provides that in the case of a nonresident not a citizen of the United States the gift 
%%tax imposed by section 2501 shall apply to a transfer only if the property is situated within the 
%%United States.

%%Section 25.2501-1(b) of the Gift Tax Regulations provides that, for purposes of the gift tax, an 
%%individual is a United States resident if the individual's domicile is in the United States at the time of 
%%the gift. All other individuals are nonresidents of the United States for purposes of the gift tax. \ldots
%\begin{center}
%\textbf{ANALYSIS AND HOLDINGS} \\
%\end{center}
%\ldots\\
%\textbf{ \textit{Situation 1.}} 
%Individuals who lost their United States citizenship and had (or have) it retroactively restored before 
%January 1, 1993, will not be held liable for federal income taxes as United States citizens between 
%the date they lost their United States citizenship and the beginning of the taxable year when their 
%citizenship was (or is) restored, and will not be held liable for federal gift taxes as United States 
%citizens between the date they lost their United States citizenship and January 1 of the calendar year 
%when their citizenship was (or is) restored. 

%As a result, A is not liable for federal income or gift taxes as a United States citizen between June 
%17, 1981, the date she lost her United States citizenship, and December 31, 1989, the end of the 
%year preceding the year in which her United States citizenship was retroactively restored. A is liable 
%for federal income and gift taxes as a United States citizen for taxable years beginning on or after 
%January 1, 1990, the year in which her United States citizenship was retroactively restored. 
% 
%\textbf{\textit{Situation 2.}} 
%B is not taxable as a United States citizen, and has not been taxable as a United States citizen since 
%May 24, 1979, the date he lost his United States citizenship. B is considered an alien 
%individual under the Code, either a nonresident alien under section 7701(b)(1)(B) or a resident alien 
%under section 7701(b)(1)(A). If B qualifies as a nonresident alien, he is taxable under section 871. 
%Alternatively, if B is considered a resident alien, he is taxable under section 1. 

%%For purposes of the gift tax, B's United States residency status is determined under section 25.2501- 
%%1(b) of the gift tax regulations. If B is considered a nonresident under section 25.2501-1(b), he is 
%%taxable under section 2511. If B is considered a resident under section 25.2501-1(b), he is taxable 
%%under section 2501. 

%B may apply to the Department of State to have his certificate of loss of nationality administratively 
%reviewed. If B applies for this review, and if his certificate of loss of nationality is vacated, B's United 
%States citizenship will be retroactively restored. 

%Individuals who lost their United States citizenship and have it retroactively restored after December 
%31, 1992, will not be held liable for federal income taxes as United States citizens between the 
%date they lost their United States citizenship and the beginning of their first taxable year 
%beginning after December 31, 1992, and will not be held liable for federal gift taxes as United States 
%citizens between the date they lost their United States citizenship and January 1, 1993. 

%As a result, if B has his United States citizenship retroactively restored after December 31, 1992, B 
%will not be liable for federal income or gift taxes as a United States citizen between May 24, 1979, 
%and December 31, 1992. B will be liable for federal income and gift taxes as a United States citizen 
%for taxable years beginning on or after January 1, 1993. 
% 
%\textbf{\textit{Situation 3.}}
%C is, and always has been since birth or naturalization, a United States citizen, taxable under 
%sections 1 and 2501 of the Code. The Department of State never determined that C lost her United 
%States citizenship, and never approved a certificate of loss of nationality for C. As a result, C never 
%lost her United States citizenship. Therefore, C is not eligible for the relief granted in situations 1 
%and 2 of this revenue ruling.
% 
%Pursuant to policy statement P-5-133, the Internal Revenue Service has designated for 
%special consideration individuals who did not file federal income and gift tax returns as United States 
%citizens because they had a reasonable, good faith belief that they had lost their United States 
%citizenship. These individuals performed expatriating acts (as defined in the Immigration and 
%Nationality Act as in effect at the time the acts were committed) but were not determined by the 
%Department of State to have lost United States citizenship, and certificates of loss of nationality 
%were not approved on their behalf. As a result, these individuals did not lose their United States 
%citizenship. Furthermore, these individuals did not intend to relinquish their United States citizenship 
%when they performed these acts. Under current law the acts these individuals performed are no 
%longer considered expatriating, absent proof of intent to relinquish United States citizenship. As a 
%result, if the Department of State had determined that these individuals lost their United States 
%citizenship, these individuals would now be eligible to have their citizenship retroactively restored. 

%Pursuant to policy statement P-5-133, the Assistant Commissioner (International)  and 
%District Directors may grant relief similar to the relief granted in situations 1 and 2 of this revenue 
%ruling. Among the circumstances that will be considered by the Assistant Commissioner 
%(International) and District Directors when evaluating requests for relief from the individuals 
%described in this situation 3 is whether they acted in a manner consistent with a good faith belief 
%that they had lost United States citizenship by, among other things, \margit{The benefits and burden rationale once again.}not affirmatively exercising any 
%rights of United States citizenship in the period when they did not file federal tax returns as United 
%States citizens.

%%As a result, pursuant to policy statement P-5-133, C may apply to the Assistant Commissioner 
%%(International) or to the appropriate District Director for relief based on the particular circumstances 
%%of her case, and may be eligible for special consideration. Following review, the Assistant 
%%Commissioner (International) or the appropriate District Director may grant C relief similar to the 
%%relief granted in situations 1 and 2 of this revenue ruling. \ldots
%% 
%\textbf{\textit{Situation 4.}} 
%D is, and always has been since birth or naturalization, a United States citizen, taxable under 
%sections 1 and 2501 of the Code. D is not eligible for any relief from federal income or gift taxes 
%based on this revenue ruling.
% 
%If extenuating circumstances prevented D from filing federal income and gift tax returns during the 
%period of his foreign residence, D may apply to the Assistant Commissioner (International) and 
%attempt to show that the extenuating circumstances justify relief under policy statement P-5-133. 
%However, D is not eligible for any special consideration based on this revenue ruling. D may also 
%attempt to show that he is eligible to settle his tax liabilities pursuant to an installment agreement 
%or an offer in compromise. \ldots

%\addcontentsline{toc}{section}{\protect\numberline{}Tax Leads Americans to Renounce U.S.} 
%\begin{select}
%\caseart{Tax Leads Americans to Renounce U.S}{ Doreen Carvajal, New York Times (12/18/06)}
%She is a former marine, a native Californian and, now, an ex-American who prefers to remain discreet about abandoning her citizenship. After 10 years of warily considering options, she turned in her United States passport last month without ceremony, becoming an alien in the view of her homeland.

%``It's a really hard thing to do,'' said the woman, a 16-year resident of Geneva who had tired of the cost and time of filing yearly United States tax returns on top of her Swiss taxes. ``I just kept putting this off. But it's my kids and the estate tax. I don't care if I die with only one Swiss franc to my name, but the U.S. shouldn�t get money I earned here when I die.''

%Historically, small numbers of Americans have turned in their passports every year for political and economic reasons, with the numbers reaching a high of about 2,000 during the Vietnam War in the early 1970s.

%But after Congress sharply raised taxes this year for many Americans living abroad, some international tax lawyers say they detect rising demand from citizens to renounce ties with the United States, the only developed country that taxes it citizens while they live overseas. Americans abroad are also taxed in the countries where they live.

%``The administrative costs of being an American and living outside the U.S. have gone up dramatically,'' said Marnin Michaels, a tax lawyer with Baker \& McKenzie in Zurich.

%So far this year, the Internal Revenue Service has tallied 509 Americans who have given up their citizenship, said Anthony Burke, an I.R.S. spokesman in Washington. He said complete figures were still being calculated.

%Applications to renounce citizenship are on the rise at the American Embassy in Paris, according to an official who spoke on condition of anonymity. At the embassy in London, the number of applications was reported to be fairly stable over the past two years, though it would be hard to spot a recent surge because applications are taking longer to process there than in past years. Neither embassy would disclose exact figures. A spokeswoman for the American Embassy in London, Karen Maxfield, said Americans living abroad usually took the step ``because they do not have strong ties to the United States and do not believe that they will ever live there in the future.''

%``All have two citizenships and generally say they would like to simplify their lives by giving up a citizenship they are not using,'' she said.

%Andy Sundberg, a director of the Geneva-based American Citizens Abroad, has been tracking renunciations dating back to the 1960s through annual Treasury Department figures. He considers the numbers low compared with some stretches in the past, like the early 1970s. But he has also noticed a recent increase in interest among Americans in renouncing their citizenship.

%``I think the cup is boiling over for a number of people living abroad,'' Mr. Sundberg said. ``With the Internet and the speed and the ubiquity of information, people are more aware of what's happening.'' With the changes in the tax laws, he said, some Americans living abroad fear ``they're heading toward a real storm.''

%He cited a survey by the American Chamber of Commerce in Singapore, which polled its members in October and November and found that many were considering returning to the United States because of the higher taxes.

%Concern about taxes among Americans living abroad has surged since President Bush signed into law a bill that sharply raises tax rates for those with incomes of more than \$82,400 a year. The legislation also increases taxes on employer-provided benefits like housing allowances.

%The changes, enacted in May, apply retroactively to Jan. 1, 2006.

%Matthew Ledvina, an international tax lawyer in Geneva, said demand for legal counsel on the citizenship issue was coming largely from American citizens who held second passports and who had minimal ties to the United States.

%``There are incentives to do it before the end of the year so that you can minimize your future reporting,'' he said.

%Mr. Ledvina said the waiting period for appointments at the American Embassy in London had increased from a few days to more than three and a half months. He said he had recently approached embassies in Vienna, Bern, London, Paris and Brussels before finally getting an appointment in Amsterdam for a client's renunciation application.

%The legal ritual of renunciation is largely unique to the United States because other countries base taxation on residency, not citizenship, according to Ingmar Dorr, a tax lawyer with Lovells in Munich.

%``We don't have that issue,'' he said. ``We only have the problem that rich people who don't want to pay taxes in Germany just move to a lower-tax country in Switzerland.''

%For some Americans abroad, motivations for renunciation are mixed and complex, involving social concerns, political displeasure with their government and other reasons. But it is clear that taxation plays a large role for many, even though few are willing to admit that because of penalties enacted a decade ago.

%In 1996, Congress tried to address a wave of tax-driven expatriation by the wealthy by requiring former citizens to file tax returns for a decade and forbidding Americans who renounced their passports for tax reasons from visiting the United States.

%But in practice, the government is mainly interested in wealthier ex-citizens with a net worth of more than \$2 million, few of whom pay further United States taxes because they generally avoid making American financial investments after giving up citizenship, Mr. Ledvina said. As for the rule barring entry to tax refugees, he said, it has not been enforced by the authorities.

%Still, that possibility prompts ex-citizens to tread carefully and remain discreet about their choices.

%``I didn't give up my citizenship with a sense of hostility,'' said an importer in Geneva who renounced her citizenship as President Bush was taking office in 2001. ``I gave it up with a sense of fairness.''
%\end{select}
